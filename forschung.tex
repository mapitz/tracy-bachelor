\part{Die empirische Studie}
\chapter{Die qualitative Untersuchung}
\section{Forschungsfrage und Begründung der Entscheidung für ein qualitatives Design}
Das Erkenntnisinteresse der vorliegenden Arbeit besteht darin, Aspekte und Zusammenhänge aufzudecken, die Inklusion im Kindergartenalltag gelingen lassen beziehungsweise erschweren, so dass notwendige, hemmende und fördernde Bedingungen für die Umsetzung von Inklusion gesammelt werden können. Der Gewinn im Identifizieren solcher Mechanismen ist vielfältig. Erstens ist denkbar, dass die befragten Leitungen durch eine Bestandsaufnahme ihre inklusiven Prozesse kritisch hinterfragen und verbessern. Zweitens können weitere Kindergärten aus den empirischen Ergebnissen Schlüsse ziehen, die ihnen Orientierung auf dem Weg zur inklusiven Einrichtung geben. Drittens geben die Ergebnisse darüber Auskunft, was konkret von Seiten der Verantwortungsträger erwartet und gebraucht wird, um das Inklusionskonzept erfolgreich im Kindergarten etablieren zu können.

Die Forschungsfrage, „Unter Berücksichtigung welcher Aspekte ist Inklusion im Kindergarten umsetzbar?“, zielt auf das Aufdecken von sozialen Mechanismen innerhalb der Strukturen des Kindergartens ab. Vielfältige Bezüge fanden im Leitfaden Beachtung, zum Beispiel der Einfluss der Rahmenbedingungen auf Inklusion, des Konzepts, des Inklusionsverständnisses, der pädagogischen Haltung, der mitgebrachten Kompetenzen der Fachkräfte, der Organisationsstruktur der Einrichtung und der Zusammenarbeit im Team. 
Die Forschungsfrage erfordert folglich eine mechanismusorientierte Strategie, die auf der Rekonstruktion von Fällen basiert, keine  standardisierte, die mittels der Aussagen von vielen individuellen Akteuren erhobenen wird, da Qualität und nicht Quantität erhoben werden soll. Als geeignete Vorgehensweise für die Beantwortung der Forschungsfrage wird ein  
qualitatives Forschungsdesign angedacht, da selbiges nicht nur der Rekonstruktion von Fällen oder Prozessen dient, sondern zudem auch immer dort empfohlen wird, „wo es um die Erschließung eines bislang wenig erforschten Wirklichkeitsbereichs“ geht (Flick, von Kardoff und Steinke 2000, 25) -- Inklusion im Untersuchungsfeld Kindergarten stellt einen solchen Wirklichkeitsbereich dar -- und wo es hinsichtlich der Theorieorientierung auf Entdeckung abzielt.
Laut Flick,von Kardoff und Steinke (2000, 45) bedeutet das, dass Theorien und Hypothesen aus den unmittelbar gesammelten Daten entwickelt werden. Diese Annahmen können an der Realität scheitern, weshalb diese an der Realität in einem weiteren Schritt geprüft werden müssen, um bestätigt werden zu können. Der Schritt der Überprüfung kann nicht Inhalt dieser Arbeit sein, da er den Rahmen sprengen würde.  

Um solcherart komplexe Wissensbestände zu rekonstruieren empfehlen Meuser und Nagel (1997, 481) das Experteninterview als eine geeignete Methode. Dieser Empfehlung folgend fiel die Wahl der Methode zur Datengewinnung auf das Experteninterview. 


\section{Die Erhebungsmethode: Experteninterview}
Im Folgenden wird zunächst der Expertenbegriff diskutiert, um darzulegen, wer für das Vorhaben der vorliegenden Arbeit überhaupt als Experte angesprochen werden konnte.
Daran anschließend wird die Auswahl der befragten Experte aufgezeigt. 

\subsection{Der Expertenbegriff}
Gläser und Laudel (2010, 13) verweisen darauf, dass der Begriff ’Experteninterview’ in der sozialwissenschaftlichen Literatur in der Regel an die Expertenrolle des Interviewten im untersuchten sozialen Feld, das heißt, an seine gehobene berufliche Position, gebunden ist (vgl. Meuser und Nagel, 2005, Bogner, Littig und Menz 2005). Entgegen dieser weit verbreiteten Meinung findet sich bei Gläser und Laudel (2010, 11) der bereichernde Gedanke, dass Experten nicht zwingend in einer gehobenen Position zu sein haben, um über ein besonderes Expertenwissen zu verfügen. Jeder Mensch, der einen Erfahrungsbereich für sich erschließt, wird zum Experten. So wird der Musiker, der einen bestimmten Musikstil aufgreift und alles darüber in Erfahrung bringt oder der von einer seltenen Krankheit Betroffene zum Experten für diesen Musikstil oder jene Krankheit. Die Autoren verweisen darauf, dass schließlich jeder Mensch über besonderes Wissen verfügt, nämlich das Wissen über die sozialen Kontexte in denen er agiert, an denen er unmittelbar Anteil hat. Jeder verfügt zum Beispiel über besonderes Wissen über die eigenen Arbeitsprozesse oder über die Organisation, in der er arbeitet. Ein Experte ist im Sinne von Gläser und Laudel (2010, 12) als „Quelle von Spezialwissen über die zu erforschenden sozialen Sachverhalte“ zu verstehen. „Experteninterviews [wiederum] sind eine Methode dieses Wissen zu erschließen.“ Ausgehend von diesem Expertenbegriff kommen für das Untersuchungsfeld Kindergarten sowohl die Leitung der Einrichtung als auch die Erzieherinnen als zu befragende Experten in Betracht. 
 
\subsection{Die befragten Experten}
Die wenigsten Kindergärten weisen sich als inklusiv aus. Das heißt, der Name der Einrichtung ist nicht leitend. Ein Ausschlusskriterium für die Auswahl der Experten war, dass die Einrichtung sowohl von Kindern mit besonderen Bedürfnissen als auch von Kindern ohne besondere Bedürfnisse besucht wird, so dass die Gruppenmischung der gesellschaftlichen Vielfalt entspricht und somit als inklusiv anzusehen ist. Um fündig zu werden, waren sozial schwierige Einzugsgebiete mit einem hohen Anteil an Familien mit einem Migrationshintergrund im Blickfeld, da die Kindergärten zumeist auf die soziale und kulturelle Vielfalt des Einzugsgebietes und somit auf die unterschiedlichen Bedarfslagen der Familien vor Ort Antworten zu finden versuchen und solche Antworten inklusive Prozesse anregen. Da bei diesen Einrichtungen der Anteil der sozial benachteiligten Familien hoch ist, kann nur bedingt von gesellschaftlicher Vielfalt gesprochen werden, da die sozial besser gestellten Familien in ’sozialen Brennpunkten’ gegebenenfalls als Minderheit vertreten sind.    
Weiterhin wurde bei der Auswahl der Einrichtungen berücksichtigt eine gewisse Trägerbreite abzubilden, da der Einfluss des Trägers, den politischen Willen zur Inklusion einzulösen, als entscheidendes Merkmal ausgewiesen wurde. Es konnten Einrichtungen gefunden werden, deren Träger erstens die Diakonie, zweitens die Arbeiterwohlfahrt (AWO) und drittens die katholische Kirchengemeinde sind. Der Erstkontakt per Telefon erfolgte stets mit der Leiterinnen der jeweiligen Einrichtungen, die sich in allen drei Fällen als Interviewpartner zur Verfügung stellten.

\section{Datenerhebung}
\subsection{Methode der Datenerhebung: Interviewleitfaden}
Um sich ein Bild von dem Arbeitsfeld der Expertinnen zu machen, sah die Gliederung des Gesprächsleitfadens vor, dass die Expertinnen als Einstieg gefragt wurden, wie sich die strukturellen Rahmenbedingungen der Einrichtung beschreiben lassen und welche Kinder die Einrichtung besuchen. Daran schlossen sich Leitfragen zu deren konzeptionellen Überlegungen, deren Inklusionsverständnis, deren Haltung, zum beschrittenen Prozess der Inklusion und seinen Stolpersteinen, zur Kooperation sowohl im Team als auch mit anderen Institutionen und zuletzt zur Partnerschaft mit den Eltern an (vgl. Gesprächsleitfaden im Anhang A).

\subsection{Durchführung der Datenerhebung}
Im Rahmen der Bachelorthesis wurden drei Experteninterviews im Zeitraum vom 15. bis 27.10.2012 durchgeführt. 
Die Anfragen erfolgten durch telefonische Kontaktaufnahme. Im Erstkontakt wurde kurz das Forschungsvorhaben erklärt sowie begründet, warum ein Interesse an der jeweiligen Einrichtung beziehungsweise an der Person vorliegt. Als ’Türöffner’ erwies sich die Information,  wodurch die Aufmerksamkeit auf die Person gelenkt wurde, zum Beispiel durch Presseberichte oder die Empfehlung des Gutachters, der in Kontakt zu der Einrichtung steht. 
Des weiteren wurde angeboten auf Wunsch den Interviewleitfaden zuzusenden, so dass die Fragen von der Leitung vorab eingesehen werden konnten und eine Vorbereitung auf den umfangreichen Leitfaden stattfinden konnte. Davon machten zwei von drei Expertinnen Gebrauch. Das hatte zur Folge, dass die Expertinnen unterschiedliche Voraussetzungen mitbrachten. Einerseits gab es die sehr gut vorbereitete Expertin, die Stichpunkte zu jeder Frage formuliert und die Fragen so verinnerlicht hatte, dass sie bei der Beantwortung der Fragen berücksichtigte, welche Antworten an welcher Stelle ihren Platz finden sollten. Andererseits gab es die Expertin, die unvorbereitet angetroffen, mit der Beantwortung einer Frage viele weitere Bezüge zu anderen Fragen herstellte und in ihren Aussagen eine hohe Spontanität zeigte. Die Durchführung der Interviews zeigte, dass die Auseinandersetzung mit den Fragen vorab Auswirkungen darauf hatte, wie reflektiert, aber auch spontan und damit verbunden emotional beteiligt geantwortet wurde.     
Die Vergleichbarkeit der Interviews wird laut Meuser und Nagel (2005, 81) durch die Nutzung des Leitfadens in der Interviewführung und den „gemeinsam geteilten institutionell-organisatorischen Kontext“ sicher gestellt, weshalb die erwähnten unterschiedlichen Ausgangssituationen, keinen Einfluss auf die Vergleichbarkeit haben.

Der Ort der Befragung war der Arbeitsplatz des Experten, konkret das Büro der Leitung. Die Befragungsdauer variierte zwischen 50 Minuten und zweieinhalb Stunden, wobei ohne Ausnahme alle Fragenkomplexe angesprochen wurden. Die unterschiedliche Interviewdauer ergab sich aus dem Verlauf des jeweiligen Interviews und der zusätzlichen Themen, die von Seiten der Experten angesprochen wurden. In dem zeitlich intensivsten Interview wurde die Aufzeichnung des Interviews verweigert, jedoch darauf hingewiesen, dass Bereitschaft besteht, ausreichende Zeit für das Interview zur Verfügung zu stellen, so dass entsprechende Notizen angefertigt werden konnten, um das Gespräch anschließend in einem Gedächtnisprotokoll zu rekonstruieren. Nach Anfertigung des Gedächtnisprotokolls wurde dieses der Expertin zugesandt, so dass für sie die Möglichkeit bestand, Veränderungen oder weitere Ergänzungen vorzunehmen. Auch den anderen Experten wurden zur Überprüfung die vollständig transkribierten Interviews zugesandt. 
Die Experteninterviews wurden mit Hilfe des Aufnahmegeräts Zoom H2 technisch problemlos auf Tonband aufgenommen und anschließend transkribiert. Parallel zu den Tonaufnahmen wurden Notizen zu den Interviews angefertigt. 

Insgesamt stieß die Befragung bei allen Interviewten auf große Bereitschaft, Neugierde und Interesse für das Thema.

\section{Das Auswertungsverfahren: Qualitative Inhaltsanalyse}

\paragraph{Transkription} Nach Abschluss der Interviewführung erfolgte als Vorbereitung der Qualitativen Inhaltsanalyse die Transkription entsprechend der Vorgaben von Meuser und Nagel (2005, 83). Das bedeutet, die Interviews wurden unter der Wahrung der Anonymität -- inhaltlich vollständig -- verschriftlicht und geglättet und somit von Zwischenlauten, Dialektfärbungen, überflüssigen Redundanzen und Floskeln befreit. Diese Methode fand Anwendung, weil sich die „Auswertung von Experteninterviews an thematischen Einheiten, an inhaltlich zusammengehörigen, über die Texte verstreute Passagen [und] nicht an der Sequenzialität von Äußerungen je Interview“ (Meuser und Nagel 2005, 81) orientiert und somit auf vergleichbare Datengewinnung ausgerichtet ist. 
Ergänzend wurden die Empfehlungen von Mayring (2010, 55) in zwei Punkten berücksichtigt. Er schlägt vor, unverständliche Passagen sowie abgebrochene Sätzen mit Hilfe von drei Punkten (...) und Stockungen, Unterbrechungen oder Gedankensprünge durch einen Gedankenstrich (--) zu kennzeichnen. 
Für das Formatieren wurde außerdem der Hinweis von Gläser und Laudel (2010, 194) berücksichtigt, inhaltlich zusammengehörende Aussagen -- Sinneinheiten -- durch Absätze erkenntlich zu machen. Das heißt, dass eine Antwort des Interviewpartners, vorausgesetzt ein neuer Gedanke kommt hinzu, mehrere Absätze umfassen kann. Eine bessere Lesbarkeit wurde außerdem durch das Kursivsetzen der Interviewfragen erreicht (vgl. Transkripte der Interviews im Anhang C). 

\paragraph{Extraktion} Anschließend wurden die Transkripte mit Hilfe der Qualitativen Inhaltsanalyse entsprechend der Vorgaben von Gläser und Laudel (2010) ausgewertet. Sie (2010, 200 ff.) beschreiben als  Ziel dieses Verfahrens eine neue Informationsbasis zu schaffen, die nur noch die Aussagen enthält, die für die Beantwortung der Forschungsfrage relevant sind. Der Kern dieses Verfahrens ist die Extraktion, die Reduktion des Datenmaterials durch die Fokussierung auf die Forschungsfrage. Extraktion bedeutet konkret, den Textabsatz zu lesen, diesen auf der Grundlage der theoretischen Vorüberlegungen zu interpretieren und zu entscheiden, welche der in ihm enthaltenen Informationen für die Beantwortung der Forschungsfrage relevant sind. Diese Informationen werden in zusammengefasster Form unter  entsprechenden Kategorien und Unterkategorien abgelegt, welche durch die theoretischen Vorüberlegungen entwickelt wurden. Das heißt, die Vorüberlegungen, welche Einflussfaktoren die Umsetzung von Inklusion im Kindergarten begünstigen oder auch behindern -- Gedanken, ohne die der Interviewleitfaden nicht hätte erstellt werden können -- leiten die Extraktion an. Sie ergeben das Suchraster, welches auf den Text gelegt wird. Das Kategoriensystem hat durch die Vorüberlegungen einerseits erste Festlegungen erfahren, wird aber gleichzeitig aus dem Textmaterial heraus neu geformt, was dem Prinzip der Offenheit dieses Verfahrens entspricht. Das Kategoriensystem wird während der Extraktion fortwährend an die Besonderheiten des Materials angepasst. Laut Gläser und Laudel (2010, 262 f.) ermöglicht die Offenheit des Auswertungsverfahrens, dass die empirischen Phänomene nicht unter theoretischen Annahmen subsumiert werden, sondern mit aufgenommen werden, wenn sie theoretischen Vorannahmen widersprechen oder über diese hinausgehen.

\paragraph{Kategoriensystem}
Das ’Kategoriensystem’ als Auswertungsraster beruht laut der Autoren (2010, 206) auf einer Zusammenstellung der theoretisch angestellten Einflussfaktoren und Kausalmechanismen, die Inklusion bedingen und somit die Forschungsfrage repräsentieren. Um das Auswertungsraster zu konstruieren, werden die fünf zentralen Aspekte, die in Kapitel~\ref{sec:Wie} erläutert werden, als so genannte Auswertungskategorien übernommen, die Merkmalsausprägungen dieser zentralen Aspekte bilden im Kategoriensystem die so genannten Dimensionen. 

Die Extraktionstabelle zur Auswertungskategorie 'Strukturelle Rahmenbedingungen' (vgl. Anhang ??) soll als Illustration dienen. 
Die Definition im Theorieteil in Kapitel~\ref{Strukturelle Rahmenbedingungen} mit einem Verweis auf Kapitel~\ref{subsec:Strukturquali} enthält die Dimensionen, die für das Kategoriensystem angenommen werden. „Strukturelle Rahmenbedingungen sind die erwähnten Aspekte, die unter Strukturqualität (vgl. Kapitel~\ref{subsec:Strukturquali}) fallen und somit laut Garai, Kerekes, Schiffer, Tamás, Trócsányi, Weiszburg und Zászkaliczky (2010, 47) im entscheidenden Maße von gesetzlichen, ökonomischen und regierungsamtlichen Entscheidungen abhängen“ (vgl. Kapitel~\ref{Strukturelle Rahmenbedingungen}).
„Laut Tietze (2004, 407) werden unter Strukturqualität Aspekte zusammengefasst, die situationsunabhängig, zeitlich stabil und politisch regulierbar sind. Solche strukturellen Rahmenbedingungen können wiederum unterschieden werden in einerseits der Einrichtung zur Verfügung stehende materielle Ressourcen und andererseits personelle Ressourcen. Unter materiellen Ressourcen werden die vorhandenen Räume und deren Ausstattung, die Finanzierung des Personals, der Betreuungsschlüssel, die Gruppengröße und die Gruppen- und Personalstruktur sowie die vorgesehenen Zeiten für mittelbare pädagogische Arbeit zusammengefasst, unter personelle Ressourcen zählt die Qualifikation der Fachkräfte, festgelegt durch Curriculum und Standards in der Erzieherinnen-Ausbildung (vgl. Kapitel~\ref{subsec:Strukturquali})“ 
„Jerg (2011, 54) sieht die systemerneuernden strukturellen Voraussetzungen vor allem in Veränderungen hinsichtlich des Finanzierungsmodells, der Angebots- und Infrastruktur, verlässlicher Unterstützungsleistung sowie Durchlässigkeit und Demokratisierung der Strukturen. [...] Zudem spielt der Träger bei der Umsetzung von Inklusion eine wesentliche Rolle. In Zusammenarbeit mit der Kommune löst er den politischen Willen zur Inklusion ein und schafft die dafür notwendigen Strukturen (vgl. Kapitel~\ref{Strukturelle Rahmenbedingungen}). 
Ausgehend von diesen theoretischen Vorüberlegungen werden nach Anpassung an die Expertenaussagen folgende Dimensionen in das Kategoriensystem der Auswertungstabelle integriert:  
Räume und Ausstattung, Finanzierung, der Einfluss des Trägers, Gruppenzusammensetzung und Bedarfslagen der Kinder, Personalstruktur, Qualifikation der Fachkräfte, Zeiten für unmittelbare pädagogische Arbeit, Fachkraft-Kind-Relation, notwendige Voraussetzungen von regierungsamtlicher Seite, Blick auf die aktuelle politische Situation, Qualitätsentwicklung.  


\paragraph{Extraktionsregeln und Kausalketten} Information können nicht immer eindeutig einer Auswertungskategorie zugeordnet werden. So konnte es in einigen Fällen nicht vermieden werden, dass in einem Absatz enthaltene Informationen mehrfach unter verschiedenen Kategorien abgelegt wurden. Um das Vorgehen bei Abgrenzungsproblemen transparent zu machen, wurden Extraktionsregeln formuliert. Diese stellen sicher, dass während der gesamten Extraktion einheitlich und somit für den Leser nachvollziehbar und nicht willkürlich vorgegangen wurde. 

Zudem wurden in der Auswertung so genannte Kausalketten berücksichtigt. Diese stellen Verbindungen zwischen den Kategorien her, welche für die Beantwortung der Forschungsfrage von Bedeutung sind, da von einer großen Verwobenheit der Aspekte auszugehen ist. Wenn in einem Absatz Kausalketten berichtet werden, erschwert das die Zuordnung erheblich. Deshalb ist das Kategoriensystem um die Spalten 'Ursache' und 'Wirkung' ergänzt. Dort werden Querverweise zu anderen Kategorien hergestellt oder auch innerhalb derselben Kategorie die Ursachen und Wirkungen vermerkt, die von dem Experten angesprochen wurden.

\paragraph{Aufbereitung der Daten}
Nachdem die Zuordnung aller Sinnabschnitte in das Kategoriensystem erfolgte, wurden die Daten in einem zweiten Schritt aufbereitet, verstreute Informationen wurden entsprechend der Empfehlungen von Gläser und Laudel (2009, 229)  
nach inhaltlichen Gesichtspunkten zusammengefasst und Redundanzen beseitigt, mit dem Ziel das Rohmaterial wiederholt zu reduzieren.
Die Zusammenfassung erfolgte hinsichtlich der Gesichtspunkte: notwendige und hemmende beziehungsweise fördernde Bedingungen für Inklusion. 

\paragraph{Auswertungsstrategien}
In der Auswertung wird laut ihnen (2009, 246 ff.) das Ziel verfolgt, die Forschungsfrage zu beantworten, indem die Kausalmechanismen identifiziert werden, die zwischen Ursachen und Wirkungen vermitteln. Für unsere Forschungsfrage ist zum Beispiel von Relevanz herauszufiltern, welche spezifischen Ausgangsbedingungen -- Ursachen -- dazu geführt haben, dass das Team entschieden hat, dass ein Kind mit besonderem Förderbedarf für die Einrichtung nicht länger tragbar sei -- Wirkung -- und was die Einrichtungen im Umgang mit solchen 'Grenzfällen' voneinander unterscheidet.
Die Autoren empfehlen bei einer geringen Zahl von Fällen, zunächst den Kausalmechanismus jedes Falls zu identifizieren und anschließend diese Mechanismen, die gewirkt haben, vergleichend zu analysieren.

Es gilt Kausalmechanismen auf unterschiedlichen Ebenen aufzugreifen.
Auf der ersten Ebene ist der Fokus auf die individuelle Perspektive der Leitung gerichtet, die ihre Ansicht von den wirkenden Kausalmechanismen erklärt. Dabei handelt es sich um die Interpretation der Beobachterin. Da das Experteninterview die Rekonstruktion von komplexen Wissensbeständen bezweckt, ist die Erklärung der Expertin als  ernstzunehmende Äußerung zu behandeln, trotzdem erhebt sie keinen Anspruch auf Wahrheit und 
Auf der zweiten Ebene wird der Kausalmechnismus des Falles rekonstruiert, der zeigt, 'wie es wirklich war', das heißt, „welche Ursachen in diesem konkreten Fall auf welche Weise welche Wirkung hervorgebracht haben“ (Gläser und Laudel 2009, 248). Hierbei werden auch die Informationen berücksichtigt, die im Widerspruch zueinander stehen.
Die dritte Ebene bezieht sich auf alle Fälle und auf alle Fälle anwendbare Kausalmechanismen, die der Beantwortung der Forschungsfrage dienen. Die unterschiedlichen Verläufe in den Fällen müssen durch das Auftreten oder das Fehlen bestimmter Bedingungen erklärt werden können, die bei allen Fällen zu beobachten sind. 

\section{Auswertungsergebnisse und ihre Interpretation}
Da es sich bei Experteninterviews um die Rekonstruktion von komplexen Wissensbeständen handelt, sollen die weiteren Ausführungen auch in diesem Sinn verstanden werden, als
ernstzunehmende Äußerungen der Experten, die dazu genutzt werden, die notwendigen Aspekte für die Umsetzung von Inklusion genauer zu bestimmen. Allerdings ist dieses Verfahren unter dem Vorbehalt zu betrachten, dass es sich hier um eine Rekonstruktion von
Wissensbeständen handelt, die keine Wahrheit beansprucht, sondern nur eine mögliche
Erklärung beziehungsweise eine mögliche theoretische Generalisierung darstellt.
Insgesamt gliedert sich die Ergebnisdarstellung in fünf Auswertungspunkte. 
Der Schwerpunkt der Auswertung liegt auf den gemeinsam geteilten Wissensbeständen, wobei dies nicht
bedeutet, dass gegensätzliche Positionen der Experten bei der Ergebnisdarstellung ausgeklammert werden.
 
\subsection{Kategorien in der Auswertung}
\paragraph{Strukturelle Voraussetzungen}

Für die Umsetzung von Inklusion wurde als begünstigend erfahren, ausreichend Räume zur Verfügung zu haben, konkret einen zusätzlichen Raum für Sprachförderangebote sowie einen Raum für die heilpädagogische Förderung. Auch wenn die Integrationshilfe, durch Heilpädagogen umgesetzt, das Ziel verfolgt, die Kinder mit besonderen Bedürfnissen in gemeinsame Gruppenprozesse einzubeziehen, wurde die Erfahrung gemacht, dass ein Schonraum als Rückzugmöglichkeit eine Bereicherung für das Kind darstellt, da sich ihm bei Bedarf im Eins-zu-Eins-Kontakt genähert und Belastungen und Nöte besser aufgefangen werden können und zudem die Kinder den intensiven Kontakt genießen würden. 

Als räumliche Hindernisse für Kinder mit Bewegungseinschränkung werden Türschwellen, Räume, die nur über Treppen zu erreichen sind und zu enge Gänge oder Türen für das Passieren mit dem Rollstuhl benannt. Weiterhin wird das Einrichten einer Behindertentoilette als notwendig erachtet.
Die bauliche Barrierefreiheit wird als Voraussetzung für die notwendige Offenheit für alle Kinder begriffen.. Weiterhin wurde ausreichend Platz als Antwort auf Kinder mit einem hohen Bewegungsdrang und Kochmöglichkeiten als Antwort auf den Elternwunsch, dass das Mittagessen in der Einrichtung eingenommen werden kann, benannt. Ausreichend finanzielle Mittel werden auch für die Ausstattung mit einer Therapieschaukel, einem Bälle-Bad oder entsprechenden Nassstellen, um auch im Innenbereich mit Wasser spielen zu können und Körperwahrnehmung zu fördern, gewünscht.

Die räumliche Ausstattung ist abhängig von den zur Verfügung stehenden finanziellen Ressourcen. Der bestehende Baubestand müsste hinsichtlich der gewünschten Barrierefreiheit verändert werden. „Wenn man das Ernst nehmen will, muss man richtig viel Geld in die Hand nehmen“ (C1 \ref{C1_25}).

Die finanzielle Situation wird hinsichtlich verschiedener Aspekten als unbefriedigend erlebt. 
Neben baulichen Veränderungen, ist die mittelbare pädagogische Arbeit, konkret die Qualitätsentwicklung der Leitung, nicht dotiert, weiterhin wird der Bedarf an einer stellvertretenden Leitung benannt. Verwaltungshürden werden angeführt und im Widerspruch gesehen zu inklusiven pädagogischen Arbeit, die gefragt ist, zu Deutschland weiter Anerkennung verholfen hat und zur Zusammenarbeit mit Hochschulen führte. Eine Kompensationsmöglichkeit ist in einem eigenen Förderverein gefunden, durch eine eigene Spendenakquise werden alle Kleingruppenprojekte und Eltern-Kind-Gruppen finanziert sowie der Speicherausbau, der zu zusätzlichen Räumen verhilft. Ohne Spendeneinnahmen „könnten wir diese Arbeit nicht machen“ (C1\ref{C1_37}).


     
 
