\part{Die empirische Studie -- Unter Berücksichtigung welcher Aspekte ist Inklusion im Kindergarten umsetzbar?}
\chapter{Die qualitative Untersuchung}
\section{Forschungsfrage und Begründung des qualitativen Designs}
Das Erkenntnisinteresse der vorliegenden Arbeit besteht darin, Aspekte und Zusammenhänge aufzudecken, die Inklusion im Kindergartenalltag gelingen lassen beziehungsweise erschweren, so dass notwendige Bedingungen für die Umsetzung von Inklusion gesammelt werden können. Der Gewinn im Identifizieren solcher Mechanismen ist vielfältig. Erstens ist denkbar, dass die befragten Leitungen durch eine Bestandsaufnahme ihre inklusiven Prozesse kritisch hinterfragen und verbessern. Zweitens können weitere Kindergärten aus den empirischen Ergebnissen Schlüsse ziehen, die ihnen Orientierung auf dem Weg zur inklusiven Einrichtung geben. Drittens geben die Ergebnisse darüber Auskunft, was konkret von Seiten der Verantwortungsträger erwartet und gebraucht wird, um das Inklusionskonzept erfolgreich im Kindergarten etablieren zu können.

Die Forschungsfrage, \emph{Unter Berücksichtigung welcher Aspekte ist Inklusion im Kindergarten umsetzbar?}, zielt auf das Aufdecken von sozialen Mechanismen innerhalb der Strukturen des Kindergartens ab, zum Beispiel welche vorhandenen strukturellen Voraussetzungen Inklusion fördern. 
Die Forschungsfrage erfordert folglich eine mechanismusorientierte Strategie, die auf der Rekonstruktion von Fällen basiert, keine  standardisierte, die mittels der Aussagen von vielen individuellen Akteuren erhobenen wird. Als geeignete Vorgehensweise für die Beantwortung der Forschungsfrage wird das  
qualitative Forschungsdesign ausgewählt, da selbiges nicht nur der Rekonstruktion von Fällen oder Prozessen dient, sondern zudem auch immer dort empfohlen wird, „wo es um die Erschließung eines bislang wenig erforschten Wirklichkeitsbereichs“ geht (Flick, von Kardoff und Steinke 2000, 25) -- Inklusion im Untersuchungsfeld Kindergarten stellt einen solchen Wirklichkeitsbereich dar -- und wo es hinsichtlich der Theorieorientierung auf Entdeckung abzielt.
Laut Flick, von Kardoff und Steinke (2000, 45) werden aus den unmittelbar gesammelten Daten Theorien und Hypothesen entwickelt, die  an der Realität scheitern können und deshalb in einem weiteren Schritt überprüft werden müssen, um bestätigt werden zu können. Das Forschungsanliegen der Überprüfung kann nicht Inhalt dieser Arbeit sein, da dieses den Rahmen sprengen würde.  

Um solcherart komplexe Wissensbestände zu rekonstruieren empfehlen Meuser und Nagel (1997, 481) das Experteninterview als eine geeignete Methode. Dieser Empfehlung folgend fiel die Wahl der Methode zur Datengewinnung auf das Experteninterview. 


\section{Die Erhebungsmethode: Experteninterview}
Im Folgenden wird zunächst der Expertenbegriff diskutiert, um darzulegen, wer für das Vorhaben der vorliegenden Arbeit überhaupt als Experte angesprochen werden konnte.
Daran anschließend wird die Auswahl der befragten Experte aufgezeigt. 

\subsection{Der Expertenbegriff}
Gläser und Laudel (2010, 13) verweisen darauf, dass der Begriff ’Experteninterview’ in der sozialwissenschaftlichen Literatur in der Regel an die Expertenrolle des Interviewten im untersuchten sozialen Feld, das heißt, an seine gehobene berufliche Position, gebunden ist. Entgegen dieser weit verbreiteten Meinung findet sich bei Gläser und Laudel (2010, 11) der bereichernde Gedanke, dass Experten nicht zwingend in einer gehobenen Position zu sein haben, um über ein besonderes Expertenwissen zu verfügen. Jeder Mensch, der einen Erfahrungsbereich für sich erschließt, wird zum Experten. So wird der Musiker, der einen bestimmten Musikstil aufgreift und alles darüber in Erfahrung bringt oder der von einer seltenen Krankheit Betroffene zum Experten für diesen Musikstil oder jene Krankheit. Die Autoren verweisen darauf, dass schließlich jeder Mensch über besonderes Wissen verfügt, nämlich das Wissen über die sozialen Kontexte, in denen er agiert und an denen er unmittelbar Anteil hat. Ein Experte ist im Sinne von Gläser und Laudel (2010, 12) als „Quelle von Spezialwissen über die zu erforschenden sozialen Sachverhalte“ zu verstehen. „Experteninterviews [wiederum] sind eine Methode dieses Wissen zu erschließen.“ Ausgehend von diesem Expertenbegriff kommen für das Untersuchungsfeld Kindergarten sowohl die Leitung der Einrichtung als auch die Erzieherinnen als zu befragende Experten in Betracht. Beide verfügen über besonderes Wissen im Hinblick auf die Organisation, in der sie arbeiten und die eigenen Arbeitsprozesse.
 
\subsection{Die befragten Experten}
Die wenigsten Kindergärten weisen sich als inklusiv aus, das heißt, der Name der Einrichtung war bei der Expertensuche nicht leitend. Ein Ausschlusskriterium für die Auswahl der Experten war, dass die Einrichtung sowohl von Kindern mit besonderen Bedürfnissen als auch von Kindern ohne besondere Bedürfnisse besucht wird, so dass die Gruppenmischung der gesellschaftlichen Vielfalt entspricht und somit als inklusiv anzusehen ist. Um fündig zu werden, waren sozial schwierige Einzugsgebiete mit einem hohen Anteil an Familien mit einem Migrationshintergrund im Blickfeld, da die Kindergärten zumeist auf die soziale und kulturelle Vielfalt des Einzugsgebietes und somit auf die unterschiedlichen Bedarfslagen der Familien vor Ort Antworten zu finden versuchen und solche Antworten inklusive Prozesse anregen. Da bei diesen Einrichtungen der Anteil der sozial benachteiligten Familien hoch ist, kann nur bedingt von gesellschaftlicher Vielfalt gesprochen werden, da berücksichtigt werden muss, dass die sozial besser gestellten Familien in ’sozialen Brennpunkten’ als Minderheit vertreten sein können.    
Weiterhin wurde bei der Auswahl der Einrichtungen darauf Wert gelegt eine gewisse Trägerbreite abzubilden, da die inklusive Ausrichtung der Einrichtung erheblich von der Unterstützung und Steuerung, die sie durch den Träger erfährt, abhängt (vgl. Kapitel~\ref{sec:kitaSelbst} und~\ref{Strukturelle Rahmenbedingungen}), was nicht automatisch heißt, dass dies der Wahrnehmung der Träger entspricht. Es konnten in der Stadt Freiburg im Breisgau\footnote{Da einige Angebote zur Förderung der Chancengleichheit von der Stadt Freiburg oder von der AWO-Freiburg initiiert wurden, würde das Anonymisieren der Stadt einen großen Informationsverlust bedeuten.} drei Einrichtungen gefunden werden, deren Träger erstens die Diakonie, zweitens die Arbeiterwohlfahrt (AWO) und drittens die katholische Kirchengemeinde sind. Der Erstkontakt per Telefon erfolgte stets mit der Kindergartenleitung, die sich in allen drei Fällen als Interviewpartner zur Verfügung stellte. 

\section{Datenerhebung}
\subsection{Methode der Datenerhebung: Interviewleitfaden}
Um sich ein Bild vom Arbeitsfeld der Kindergartenleitung zu machen, sah die Gliederung des Interviewleitfadens vor, dass die Expertinnen gefragt wurden, wie sich die strukturellen Rahmenbedingungen der Einrichtung und das Konzept im Zusammenhang mit dem Inklusionsverständnis beschreiben lassen und welche Kinder die Einrichtung zum gegenwärtigen Zeitpunkt der Befragung besuchten. Die ausgewählten Leitfragen orientieren sich an den theoretischen Vorüberlegungen in Kapitel~\ref{sec:Wie} und berücksichtigen den stattgefundenen Inklusionsprozess, die damit verbundenen Herausforderungen an das Team, die Beteiligung der Eltern sowie die persönliche Haltung zu Grenzen der Inklusion und stattgefundenen Veränderungen (vgl. Kapitel~\ref{Interviewleitfaden}).

\subsection{Durchführung der Datenerhebung}
Im Rahmen der Bachelorthesis wurden drei Experteninterviews im Zeitraum vom 15. bis 27.10.2012 durchgeführt. 
Die Anfragen erfolgten durch telefonische Kontaktaufnahme. Im Erstkontakt wurde kurz das Forschungsvorhaben erklärt sowie begründet, warum ein Interesse an der jeweiligen Einrichtung beziehungsweise an der Person vorliegt und wodurch die Aufmerksamkeit auf die Person gelenkt wurde, zum Beispiel durch Presseberichte oder die Empfehlung des Gutachters. 
Des weiteren wurde angeboten auf Wunsch den Interviewleitfaden zuzusenden, so dass die Fragen von der Leitung vorab eingesehen werden konnten und eine Vorbereitung auf den umfangreichen Leitfaden stattfinden konnte. Davon machten zwei von drei Expertinnen Gebrauch. Das hatte zur Folge, dass die Expertinnen unterschiedliche Voraussetzungen mitbrachten. Einerseits gab es die sehr gut vorbereitete Expertin, die Stichpunkte zu jeder Frage formuliert und die Fragen so verinnerlicht hatte, dass sie bei deren Beantwortung berücksichtigte, welche Antworten an welcher Stelle ihren Platz finden sollten. Andererseits gab es die Expertin, die unvorbereitet angetroffen, mit der Beantwortung einer Frage viele weitere Bezüge zu anderen Fragen herstellte. Die Durchführung der Interviews zeigte, dass die Auseinandersetzung mit den Fragen vorab Auswirkungen darauf hatte, wie reflektiert, aber auch spontan und damit verbunden emotional beteiligt geantwortet wurde.     
Die Vergleichbarkeit der Interviews wird laut Meuser und Nagel (2005, 81) durch die Nutzung des Leitfadens in der Interviewführung und den „gemeinsam geteilten institutionell-organisatorischen Kontext“ sicher gestellt, weshalb die erwähnten unterschiedlichen Ausgangsbedingungen keinen Einfluss auf die Vergleichbarkeit haben.

Der Ort der Befragung war der Arbeitsplatz der Expertinnen, konkret deren Büro. Die Befragungsdauer variierte zwischen 50 Minuten und zweieinhalb Stunden, wobei ohne Ausnahme alle Fragenkomplexe angesprochen wurden. Die unterschiedliche Interviewdauer ergab sich aus dem Verlauf des jeweiligen Interviews und der zusätzlichen Themen, die von Seiten der Experten angesprochen wurden. In dem zeitlich intensivsten Interview wurde die Aufzeichnung des Interviews verweigert, jedoch darauf hingewiesen, dass Bereitschaft besteht, ausreichend Zeit für das Anfertigen entsprechender Notizen zur Verfügung zu stellen, so dass das Gespräch anschließend in einem Gedächtnisprotokoll rekonstruiert wurde. Nach Anfertigung des Gedächtnisprotokolls wurde dieses der Expertin zugesandt, so dass für sie die Möglichkeit bestand Veränderungen vorzunehmen. Auch den anderen Experten wurden zur Überprüfung die vollständig transkribierten Interviews zugesandt. 
Die Experteninterviews wurden mit Hilfe des Aufnahmegeräts Zoom H2 technisch problemlos aufgezeichnet und anschließend transkribiert. Parallel zu den Tonaufnahmen wurden Notizen zu den Interviews angefertigt.  

Insgesamt stieß die Befragung bei allen Interviewten auf Neugierde und Interesse für das Thema.

\section{Das Auswertungsverfahren: Qualitative Inhaltsanalyse}

\paragraph{Transkription} Nach Abschluss der Interviewführung erfolgte als Vorbereitung der Qualitativen Inhaltsanalyse die Transkription entsprechend der Vorgaben von Meuser und Nagel (2005, 83). Das bedeutet, die Interviews wurden unter der Wahrung der Anonymität -- inhaltlich vollständig -- verschriftlicht und geglättet und somit von Zwischenlauten, Dialektfärbungen, überflüssigen Redundanzen und Floskeln befreit. Diese Methode fand Anwendung, weil sich die „Auswertung von Experteninterviews an thematischen Einheiten, an inhaltlich zusammengehörigen, über die Texte verstreute Passagen [und] nicht an der Sequenzialität von Äußerungen je Interview“ (Meuser und Nagel 2005, 81) orientiert und somit auf vergleichbare Datengewinnung ausgerichtet ist. 
Ergänzend wurden die Empfehlungen von Mayring (2010, 55) in zwei Punkten berücksichtigt. Er schlägt vor, unverständliche Passagen sowie abgebrochene Sätzen mit Hilfe von drei Punkten (...) und Stockungen, Unterbrechungen oder Gedankensprünge durch einen Gedankenstrich (--) zu kennzeichnen. 
Für das Formatieren wurde außerdem der Hinweis von Gläser und Laudel (2010, 194) berücksichtigt, inhaltlich zusammengehörende Aussagen -- Sinneinheiten -- durch Absätze erkenntlich zu machen. Das heißt, dass eine Antwort des Interviewpartners, vorausgesetzt ein neuer Gedanke kommt hinzu, mehrere Absätze umfassen kann. Eine bessere Lesbarkeit wurde außerdem durch das Kursivsetzen der Interviewfragen erreicht (vgl. Kapitel~\ref{Transkripte}). 

\paragraph{Extraktion} Anschließend wurden die Transkripte mit Hilfe der Qualitativen Inhaltsanalyse entsprechend der Vorgaben von Gläser und Laudel (2010) ausgewertet. Als Ziel dieses Verfahrens wird von ihnen (2010, 200 ff.) benannt eine neue Informationsbasis zu schaffen, die nur noch die Aussagen enthält, die für die Beantwortung der Forschungsfrage relevant sind. Der Kern dieses Verfahrens ist die Extraktion, die Reduktion des Datenmaterials durch die Fokussierung auf die Forschungsfrage. In der Umsetzung bedeutet das konkret, den Textabsatz zu lesen, diesen auf der Grundlage der theoretischen Vorüberlegungen zu interpretieren und zu entscheiden, welche der in ihm enthaltenen Informationen für die Beantwortung der Forschungsfrage relevant sind. Diese Informationen werden in zusammengefasster Form unter entsprechenden Kategorien und Dimensionen abgelegt, welche durch die theoretischen Vorüberlegungen entwickelt wurden. Das heißt, die Vorüberlegungen, welche Einflussfaktoren die Umsetzung von Inklusion im Kindergarten begünstigen oder auch behindern -- Gedanken, ohne die der Interviewleitfaden nicht hätte erstellt werden können -- leiten die Extraktion an. Sie ergeben das Suchraster, welches auf den Text gelegt wird. Das Kategoriensystem hat durch die Vorüberlegungen einerseits erste Festlegungen erfahren, wird aber gleichzeitig aus dem Textmaterial heraus neu geformt, was das Prinzip der Offenheit bewahrt. Das Kategoriensystem wird während der Extraktion fortwährend an die Besonderheiten des Materials angepasst. Laut Gläser und Laudel (2010, 262 f.) ermöglicht die Offenheit des Auswertungsverfahrens, dass die empirischen Phänomene nicht unter theoretischen Annahmen subsumiert werden, sondern Widersprüche oder über die Theorie hinausgehende Inhalte berücksichtigt werden.

\paragraph{Kategoriensystem}
Das ’Kategoriensystem’ als Auswertungsraster beruht laut ihnen (2010, 206) auf einer Zusammenstellung der theoretisch angestellten Einflussfaktoren und Kausalmechanismen, die Inklusion bedingen und somit die Forschungsfrage repräsentieren. Um das Auswertungsraster zu konstruieren, wurden die fünf zentralen Aspekte aus Kapitel~\ref{sec:Wie} als sogenannte Auswertungskategorien übernommen, die Merkmalsausprägungen dieser zentralen Aspekte bilden im Kategoriensystem die sogenannten Dimensionen. 

Die Extraktionstabelle zur Auswertungskategorie 'Strukturelle Rahmenbedingungen' (vgl. Anhang~\ref{Kategoriensys}) soll als Illustration dienen. 
Die Definition im Theorieteil in Kapitel~\ref{Strukturelle Rahmenbedingungen} enthält die Dimensionen, die für das Kategoriensystem vorab festgelegt werden. 

\begin{zitat}
Laut Tietze (2004, 407) werden unter [strukturelle Rahmenbedingungen] Aspekte zusammengefasst, die situationsunabhängig, zeitlich stabil und politisch regulierbar sind. Solche strukturellen Rahmenbedingungen können wiederum unterschieden werden in einerseits der Einrichtung zur Verfügung stehende materielle Ressourcen und andererseits personelle Ressourcen. Unter materiellen Ressourcen werden die vorhandenen Räume und deren Ausstattung, die Finanzierung des Personals, der Betreuungsschlüssel, die Gruppengröße und die Gruppen- und Personalstruktur sowie die vorgesehenen Zeiten für mittelbare pädagogische Arbeit zusammengefasst, unter personelle Ressourcen zählt die Qualifikation der Fachkräfte, festgelegt durch Curriculum und Standards in der Erzieherinnen-Ausbildung (vgl. Kapitel~\ref{subsec:Strukturquali}). 

Jerg (2011, 54) sieht die systemerneuernden strukturellen Voraussetzungen vor allem in Veränderungen hinsichtlich des Finanzierungsmodells, der Angebots- und Infrastruktur, verlässlicher Unterstützungsleistung sowie Durchlässigkeit und Demokratisierung der Strukturen. Zudem spielt der Träger bei der Umsetzung von Inklusion eine wesentliche Rolle. Im besten Falle löst er den politischen Willen zur Inklusion ein und schafft die dafür notwendigen Strukturen (vgl. Kapitel~\ref{Strukturelle Rahmenbedingungen}). 
\end{zitat}

Ausgehend von diesen theoretischen Vorüberlegungen werden nach Anpassung an die Expertenaussagen folgende Dimensionen in das Kategoriensystem der Auswertungstabelle integriert:  
Räume und Ausstattung, Finanzierung, der Einfluss des Trägers, Personalstruktur, Qualifikation der Fachkräfte, Fachkraft-Kind-Relation, notwendige Voraussetzungen von regierungsamtlicher Seite, Blick auf die aktuelle politische Situation.
 
\paragraph{Extraktionsregeln und Kausalketten} Information können nicht immer eindeutig einer Auswertungskategorie zugeordnet werden. So konnte es in einigen Fällen nicht vermieden werden, dass in einem Absatz enthaltene Informationen mehrfach unter verschiedenen Kategorien abgelegt wurden. Um das Vorgehen bei Abgrenzungsproblemen transparent zu machen, wurden Extraktionsregeln formuliert. Diese stellen sicher, dass während der gesamten Extraktion einheitlich und somit für den Leser nachvollziehbar und nicht willkürlich vorgegangen wurde. 

Die für dir Auswertung festgelegten Extraktionsregeln sind:
\begin{enumerate}
\item Vor dem Extrahieren die zum Fall dazugehörigen Gesprächsprotokolle~\ref{Gedachtnisprotokolle} lesen!
\item Querverweise zu anderen Kategorien werden im Textfeld 'Ursachen' und 'Wirkungen' in eckige Klammern [] geschrieben.
\item Aussagen bezüglich politisch festgelegter Standards innerhalb der Erzieherinnen-Ausbildung werden unter der Kategorie 'Strukturelle Rahmenbedingungen' abgelegt. Sobald persönliche, nicht-regulierbare Aspekte wie die Haltung der Fachkräfte oder individuelle Kompetenzen erwähnt werden, wird der Fakt unter der Kategorie 'Die Rolle der pädagogischen Fachkraft' und die entsprechende Dimension abgelegt.
\item Aussagen über die Gruppenzusammensetzung als eine eigentlich strukturelle Rahmenbedingung wird unter 'Bedarfslagen der Kinder' in der Auswertungskategorie 'Antworten auf die unterschiedlichen Bedarfslagen der Kinder' abgelegt.
\item Angebote, die beschrieben werden und die die Beteiligung der Eltern beinhalten, werden  unter 'Antworten auf Bedarfslagen' abgelegt.
\end{enumerate}

Zudem wurden in der Auswertung sogenannte Kausalketten berücksichtigt. Diese stellen Verbindungen zwischen den Kategorien her, welche für die Beantwortung der Forschungsfrage von Bedeutung sind, da von einer großen Verwobenheit der Aspekte auszugehen ist. Wenn in einem Absatz Kausalketten berichtet werden, erschwert das die Zuordnung erheblich. Deshalb ist das Kategoriensystem um die Spalten 'Ursache' und 'Wirkung' ergänzt. Dort werden Querverweise zu anderen Kategorien hergestellt oder auch innerhalb derselben Kategorie die Ursachen und Wirkungen vermerkt, die von dem Experten erwähnt wurden.

\paragraph{Auswertungsstrategien}
Die Situationen in den Einrichtungen sind heterogen. Sie unterscheiden sich durch die unterschiedlichen Bedingungen, die gegebenen Möglichkeiten und Anforderungen, mit denen der Kindergarten im jeweiligen Stadtteil konfrontiert ist. Zudem handelt es sich bei den Forschungsdaten um die individuelle Perspektive der Leitung, die ihre Ansicht von den wirkenden Bedingungen und Kausalmechanismen erklärt. Ihre Interpretation der Beobachtungen wird von ihrer individuellen Persönlichkeit, ihrer Haltung und ihrem Fokus beeinflusst. Die übergeordneten Werte und Ziele der Einrichtung, die einerseits durch die Träger, andererseits durch das Team unter der Führung der Leitung festgelegt werden, prägen wiederum die Entwicklung der Einrichtung, die beschritten wurde. All diese erwähnten Merkmale tragen zur Verschiedenheit der Kindergärten bei.   
Die Heterogenität der Kindergärten zeigte sich auch bei der Anpassung des Kategoriensystems an das Datenmaterial. 
Aus dieser Erkenntnis ist die Schlussfolgerung zu ziehen, die Auswertung nicht an dem gemeinsam geteilten Wissensstand auszurichten, sondern den Einzelfall nachzuzeichnen, weshalb bei der Darstellung der Auswertungsergebnisse die Ansichten der Interviewpartnerinnen im Hinblick auf fördernde, hemmende, notwendige oder auch verhindernde Bedingungen für Inklusion gesammelt und in Bezug auf die jeweilige Einrichtung zusammengefasst dargestellt wurden. Anschließend wurden die Einzelfälle einer vergleichenden Analyse unterzogen, in der Gemeinsamkeiten und Unterschiede zwischen den Fällen erklärt wurden. 
Dieses an die Forschungsdaten angepasstes Vorgehen entspricht der Empfehlung von Gläser und Laudel (2009, 247), nämlich bei einer geringen Anzahl von Fällen zunächst den Kausalmechanismus jedes Falls zu identifizieren und anschließend diese Mechanismen, die gewirkt haben, vergleichend zu analysieren.  

Da inklusive Erziehungs- und Bildungsarbeit mit ihren individuellen Antworten an den vorhandenen Bedarfen ausgerichtet ist, wurden diese --die Gruppenzusammensetzung und die Bedarfe der Kinder sowie die Bedarfe der Eltern -- in der folgenden Darstellung der Auswertungsergebnisse in einem ersten Schritt fallübergreifend dargestellt. Sie bildeten die Basis für die gefunden Antworten. Die unterschiedlichen Voraussetzungen und gefundenen Antworten der jeweiligen Einrichtungen wurden in einem  zweiten Schritt entlang des Kategoriensystems mit seinen Dimensionen interpretiert und aufgezeigt. In einem dritten Schritt erfolgte die vergleichende Analyse der wirkenden Kausalmechanismen.  

\section{Forschungsergebnisse}
 
\subsection{Auswertungskategorie 1: Gegebene strukturelle Voraussetzungen in den jeweiligen Einrichtungen}

\paragraph{Gruppenzusammensetzung und Bedarfe der Kinder und der Eltern}
Die bestehende Gruppenzusammensetzung wird von den Kindergartenleitungen durch Attribute wie 'hohe soziale und kulturelle Vielfalt' oder 'gute Mischung' beschrieben. Die Heterogenität lässt sich erstens dadurch definieren, dass sowohl Kinder aus bildungsorientierten als auch Kinder aus bildungsfernen Elternhäusern gemeinsam die Einrichtung besuchen, zweitens der Anteil an Kindern mit Migrationshintergrund zwischen 63 und 73 \,\% liegt und drittens, dass aufgrund der Einzugsgebiete, in denen soziale Problemlagen gehäuft vorkommen und der Kindergarten sich für diese unterschiedlichen Problemlagen öffnet, zum gegenwärtigen Zeitpunkt der Datenerhebung Kinder mit sehr unterschiedlichen Voraussetzungen die jeweiligen Einrichtung besuchten: 
Kinder mit diagnostiziertem Förderbedarf aufgrund einer seelischen oder geistigen Behinderung oder einer drohenden Behinderung, ein Kind, das an Krebs erkrankt war und sich zum gegenwärtigen Zeitpunkt in Remission befand und große Entwicklungsverzögerungen, insbesondere im sprachlichen Bereich zeigte, Kinder mit Verhaltensauffälligkeiten und Problemen im sozial-emotionalen, motorischen oder sprachlichen Bereich, drei Kinder mit diagnostizierter Lernbehinderung, zwei Kinder mit Hörbeeinträchtigung, die Hörgeräte trugen, ein Kind mit einer Sehbehinderung sowie Kinder mit Wahrnehmungsproblematiken, bei denen noch eine diagnostische Abklärung bevorstand.

\paragraph{Räume und Ausstattung}
Fördernde Bedingungen für die Umsetzung von Inklusion werden im Vorhandensein zusätzlicher Räume gesehen. Konkret wurden das Einrichten eines zusätzlichen Raums für Sprachförderangebote sowie ein Raum für die heilpädagogische Förderung als hilfreich empfunden. Auch wenn die Integrationshilfe, durch Heilpädagogen umgesetzt, das Ziel verfolgt, die Kinder mit besonderen Bedürfnissen in gemeinsame Gruppenprozesse einzubeziehen, wurde die Erfahrung gemacht, dass ein Schonraum als Rückzugmöglichkeit eine Bereicherung für das Kind darstellt, da sich ihm bei Bedarf im Eins-zu-Eins-Kontakt genähert, Nöte besser aufgefangen werden konnten und die Kinder den intensiven Kontakt zu einer Bezugsperson genießen würden. Ausreichend verfügbarer Raum wurde zudem als Antwort auf einen hohen Bewegungsdrang gesehen. Als Wunsch wurde formuliert, einen zusätzlichen Raum für Elterngespräche zur Verfügung zu haben, um Störungen wie das Klingeln des Telefons im Büro zu vermeiden. 

Für die Umsetzung von Inklusion wirkt demnach hemmend, wenn die baulichen Voraussetzungen nur eine begrenzte Anzahl an Räumen zuließen oder wenn der Umbau aus Kostengründen bisher nicht bewilligt wurde. 

Eine notwendige Voraussetzung für das Gelingen von Inklusion wurde im  barrierefreien Bewegen des Rollstuhls gesehen. Enge Flure und Türen, die für das Passieren mit dem Rollstuhl nicht breit genug sind, wurden als Inklusion verhindernde Bedingungen betrachtet, dass heißt, dass bauliche Voraussetzungen dieser Art dazuführen, dass Kinder, die auf einen Rollstuhl angewiesen sind, nicht aufgenommen werden können. Als weitere hemmende Bedingungen wurden die Erreichbarkeit des Turnraums nur über Treppen, das Fehlen einer Behindertentoilette und das Vorhandensein von Türschwellen genannt, letztere wurden durch das Einfügen eines Holzkeils zu kompensieren versucht, der jedoch beim Schließen der Gruppenraumtür wieder entfernt werden musste. %Dieses Beispiel ist signifikant dafür, dass Fachkräfte kaum und nur unter einem erheblichen Mehraufwand in der Lage sind Defizite innerhalb der strukturellen Rahmenbedingungen zu kompensieren.
Inklusion verlangt danach, dass bestehender Baubestand entsprechend der Bedarfe verändert werden muss und dass finanzielle Mittel ausreichend zur Verfügung gestellt werden. Das Einrichten von Kochmöglichkeiten, um den Wunsch der Eltern nach einem gemeinsamen Mittagessen in der Einrichtung zu beantworten, im Innenbereich Möglichkeiten zu schaffen, dass Kinder mit Wasser spielen können oder die Anschaffung von Materialien, die zur Förderung der Körperwahrnehmung eingesetzt werden können, wie 'Bällebad' oder Therapieschaukel, wurden als Wünsche benannt, deren Umsetzung in Ermanglung finanzieller Ressourcen noch nicht erfolgte. 
 
\paragraph{Finanzierung}
Gelder fehlten nicht nur für die notwendigen räumlichen Veränderungen, sondern auch für die Dotierung von Leistungen. So wurde zum Beispiel die Leitung für die kontinuierliche Qualitätsentwicklung, die sie im Rahmen des Familiennetzwerks leistet, nicht bezahlt. Außerdem ist dem Bedarf einer stellvertretenden Leitung noch nicht entsprochen worden. 
Durch Spendeneinnahmen konnten fehlende finanzielle Mittel kompensiert und Antworten auf Bedarfslagen gefunden werden, wie zum Beispiel der Ausbau des oberen Stockwerks, neue Projekte wie die Konstruktion Naturpädagogik, die Eltern-Kind-Gruppen und Eltern-Kind-Ausflüge beinhaltet, Bildungsausflüge wie Theater- oder Museumsbesuche oder Bildungsangebote wie musikalische Früherziehung, der Tanzkurs oder der Schwimmkurs – alles Angebote, die in den Kindergartenalltag eingebunden wurden; nicht zuletzt konnte durch Spenden die Teilnahme am Mittagessen für Familien, die den Elternbeitrag von einen Euro am Tag nicht leisten können, unterstützt werden.
Je etablierter Spenden aus dem Haus heraus oder durch die Unterstützung des Trägers akquiriert wurden, desto mehr Bedarfe konnten gedeckt werden. „[...] alle Projekte, die wir haben, bezahlt der Förderverein. Sonst könnten wir diese Arbeit nicht machen.“ (C1,\ref{C1_37})

\paragraph{Einfluss des Trägers}
Die Aufgabe des Trägers als finanzieller 'Unterstützer' wurde durch die Kinderarmutskampagne 2012 der AWO-Freiburg: „Wenn ich groß bin, werde ich arm.“ transparent, wodurch dem Kindergarten für das Jahr 2012 Spendeneinnahmen in Höhe von 10000 Euro zur Verfügung gestellt wurden, so dass Familien, die sich das Mittagessen für ihre Kinder in der Einrichtung nicht leisten konnten, über einen Gutschein unterstützt wurden. Weitere Spenden wurden für die bereits erwähnten Bildungsangebote und -ausflüge bereitgestellt. Durch den Austausch zwischen Leitungs- und Trägerebene und der konkreten Nachfrage: „Was braucht ihr jetzt vor Ort, welche Bildungsausflüge wären wichtig?“ (C2,~\ref{C2_601}) gab die Kindergartenleitung an, dass ihr Bedürfnis nach Unterstützung auf Seiten erfüllt sei. Sie gab zu verstehen, dass sie gern bei der AWO angestellt sei und Wertschätzung durch den Träger erfahren und an ihr Team weitergeben würde. 

Als fördernde Bedingungen für die Steuerung der pädagogischen Arbeit wurde die Möglichkeit benannt autonom Entscheidungen treffen zu können. 
Der zur Verfügung gestellte 'Spielraum' am Beispiel der Diakonie bemisst sich daran, dass die Fortbildungsthemen leitungsintern entschieden werden können: „Wir können uns den Themen widmen, die wirklich dran sind. Da steht dieser Träger hinter seinen Einrichtungen, weil die sagen, wir sind hier im Brennpunkt und wenn unsere Leute hier mit diesen Themen kommen, dann sind die dran.“ (C1,~\ref{C1_36}) Der Träger regelt lediglich den Finanzrahmen für die Fortbildungen. Unter Voraussetzung dieser Bedingungen konnte das Team am Bedarf orientiert entscheiden, welchen Themen sie sich widmen, was von Seiten der Leitung als Voraussetzung für konzeptionelle Weiterentwicklung und hohe fachliche Kompetenz gesehen wurde. 
Die Offenheit des Trägers wurde auch beim flexiblen Umgang der katholischen Kirchgemeinde mit der Gruppengröße erkennbar. Die Gruppengröße wurde, nachdem sich in einer Gruppe drei Kinder mit diagnostiziertem besonderen Förderbedarf befanden, von den üblichen 22 beziehungsweise 23 Kindern auf 20 reduziert. 

\paragraph{Fachkraft-Kind-Relation}
Die Personalbemessung richtete sich nach der Anzahl der Kinder und den Hauptbetreuungszeiten, das heißt, wenn eine Einrichtung verlängerte Öffnungszeiten bis 18 Uhr anbot, dann war der Personalschlüssel höher als bei Einrichtungen, deren Betreuungszeiten 13:30, 14:30 oder 15 Uhr endeten. Die Personalschlüssel differierten von Kindergarten zu Kindergarten zwischen 3,2 Fachkräften auf 20 Kinder, 2,5 Fachkräften auf 18 beziehungsweise 19 Kinder und 2 Fachkräften auf 22 beziehungsweise 23 Kinder. Ab einer Fachkraft-Kind-Relation von 3,2 Stellen auf 20 Kinder wurde die Situation als gut befunden, alle Relationen unter diesem Wert waren mit dem Bedarf nach 'mehr Personal' verbunden.     

\paragraph{Personalstruktur}
Zudem erfuhr das Personal eine Aufstockung, wenn die Einrichtung von einem hohen Anteil an Kindern mit Migrationshintergrund besucht wurde und diese Kinder wiederum einen erhöhten Förderbedarf hatten oder die Elternarbeit erschwert war, was zum Beispiel Unterstützung der Eltern bei der Antragstellung erforderlich machte. Unter Voraussetzung dieser Bedingungen gewährt die Stadt Freiburg seit 2011 eine sogenannte Migrationsanteilsaufstockung, die im vorliegenden Fall zur Freistellung der Leitung um weitere 20\,\% führte, sodass für diese Leitung die Freistellung von der mittelbaren Arbeit am Kind 50\,\% betrug, während die anderen Leitungen 100\,\% freigestellt waren. 
Auf insgesamt 12 Gruppen in den drei Einrichtungen verteilten sich 33 Erzieherinnen und ein Erzieher. Zusätzlich sind Heilpädagogen als Integrationshilfe in den Gruppen. Wenn die Kinder in der Einrichtung kommen, die einen erhöhten Unterstützungsbedarf haben, wird abhängig von der Art der Behinderung nach § 35 a oder § 53 Integrationshilfe und bei entsprechendem Bedarf eine 'begleitende Hilfe' beantragt. Wird dem Kind diese zugesprochen, kommt die Heilpädagogin an zwei Tagen in der Woche jeweils zwei Stunden in die Einrichtung, um das jeweilige Kind innerhalb der Gruppe zu unterstützen. Bei Bewilligung wird die begleitende Hilfe mit 300 Euro monatlich finanziert, weshalb diese durch Vorpraktikanten oder junge Menschen im Freiwilligen Soziale Jahr umgesetzt wird. Unter dem Zeitaspekt kann das Kind auf diese Weise neben den Erzieherinnen eine tatsächliche Begleitung im Alltag durch die begleitende Hilfe erfahren, jedoch mit der Einschränkung versehen, dass diese durch qualifiziertes Personal angeleitet werden muss, was wiederum Mehraufwand bedeutet. Die Anleitung wurde in dem konkreten Fall von der Heilpädagogin übernommen. 

Im Bereich der Sprachförderung habe alle Einrichtungen qualifiziertes Personal vorzuweisen, zum Beispiel umgesetzt durch die Heilpädagogin der Sprachheilschule, die stundenweise in die Einrichtung kommt und Sprachförderung in Kleingruppen anbietet. Eine Einrichtung hat bedingt durch die Entwicklung zum Familiennetzwerk zusätzliche Fachkräfte mit spezifischer Qualifikation im Haus, wodurch der Personalschlüssel aufgewertet wird, konkret 'Bücherwurmfrauen' im Ehrenamt, die in der Bibliothek arbeiten und vorlesen, eine Musikpädagogin, zwei Logopäden, eine Sprachförderkraft, wodurch Logopädie und Sprachförderung als integrierte Hilfen angeboten werden können, um die Kinder in dem sozialen Umfeld des Kindergartens zu belassen, Sportler im Ehrenamt, die die 'Ringergruppe' durchführen sowie eine Natur- und Umweltpädagogin, die Projekte zur Natur- und Umweltpädagogik anbietet. Durch die unterschiedlichen Fachkräfte wird der interdisziplinäre Austausch angeregt. 

\paragraph{Blick auf die aktuelle politische Situation sowie Schlussfolgerungen}

Die Schlussfolgerungen beziehen sich auf die Wahrnehmungen der aktuellen politischen Situation und zeigen, welche strukturellen Voraussetzungen für gelingende Inklusion, die von regierungsamtlicher Seite garantiert werden sollen, als notwendig angenommen werden.
 
„Deutschlandweit schätze ich das so ein, dass Kindergärten bereits viel Inklusionsarbeit leisten, das aber in der Schule steil abfällt. Die Kitas sind in ihrer Gesamtstruktur anders aufgestellt, sie sind näher an den Familien, bekommen Inklusion eher hin und wollen es eher hin bekommen.“ (C1, ~\ref{C1_24}) Das gemeinsam benannte Ziel  Inklusion in der Schule weiter zu führen, zeigt, dass die Bedeutung von Inklusion in ihrer gesellschaftlichen Tragweite erkannt wurde. Unter Weiterführung in der Schule wird zum Beispiel verstanden, dass sich die Schule für neue Strukturen öffnet, Bildungsangebote für Eltern einbezieht und Strategien entwickelt, Familien und deren Kinder zu stärken. Zudem wird daraufhin gewiesen Inklusion als Trend kritisch zu begegnen und zu überprüfen, „wo Inklusion drauf steht, ist da auch Inklusion drin?“ (C1,~\ref{C1_27}). 
Da Inklusion im Stadtteil stattfindet, setzt diese eine dezentrale Politik voraus, das heißt, dass Bildungsmöglichkeiten zur Partizipation aller Menschen, die in diesem Stadtteil wohnen, entwickelt werden müssen. Dafür bedarf es Konzepte, die von 'oben' auferlegt, aber nicht vorgegeben, sondern vom jeweiligen Kindergartenteam formuliert werden und Fragen zur Haltung einbeziehen sollten.

Das Amt für Kinder, Jugend und Soziales bewilligt den Förderbedarf und die entsprechende Hilfen, Integrations- und begleitende Hilfen. In der Wahrnehmung der Kindergartenleitung arbeitet selbiges unter der Vorgabe genau zu überprüfen, ob der Bedarf nicht auch mit dem vorhandenen Personal bewältigt werden kann und somit möglichst einzusparen. Der Umfang von zwei mal zwei Stunden in der Woche, in denen die Heilpädagogin die Integrationshilfe leistet, entspricht dem  bewilligten Höchstmaß. Hierbei Überprüft das Amt in regelmäßigen Abständen, ob der Bedarf noch vorhanden ist oder reduziert werden kann. Die begleitende Hilfe, die zusätzlich zur Integrationshilfe kommt, wird nach Feststellung der Leitung ungern finanziert. Die Befunde rufen nach Verbesserung. Es wird nicht nur nach mehr Zeit und Unterstützung durch die Heilpädagogin verlangt, sondern darauf hingewiesen, dass als notwendige Voraussetzung für Inklusion angesehen wird, dass die Heilpädagogin in jeder Einrichtung mit einem Stellendeputat vertreten ist. 

%Die Situation in den Kindergärten kann im Bezug auf die finanzielle Ausstattung nicht befriedigen. %Trotzdem kann gezeigt werden, dass der kompensatorische Umgang unterschiedlich ausfällt und entsprechende Auswirkungen auf das gesamte System Kindergarten, die Strukturen, Angebote und das Selbstbild hat. 
Der katholische Kindergarten zeigt in Bezug auf den Personalschlüssel, die räumliche und finanzielle Ausstattung die niedrigsten Werte für Strukturqualität. Die anderen beiden Einrichtungen zeigen positivere Werte bedingt durch deren Kompensationsmöglichkeiten. Die Einrichtung der Diakonie verfügt über einen eigenen Förderverein und die Strukturen im Familiennetzwerk, wodurch zusätzliche Fachkräfte gewonnen werden können, die vielfältige Projektarbeit in Kleingruppen anbieten, weshalb der Personalschlüssel aufgewertet wird. In der Einrichtung der AWO erfolgt eine Kompensation durch die umfangreichen Spenden des Trägers zur Finanzierung von Bildungsangeboten. Die Kindergartenleitung des katholischen Kindergartens mit der vergleichsweise niedrigsten Strukturqualität äußert den Wunsch nach gesamtgesellschaftlicher Anerkennung der Rolle der Erzieherin in ihren Aufgaben und in ihrer Bedeutung für das Kind. Unter ihre Aufgaben, die sich angesichts der politischen Ideen zur inklusiven Erziehung und Bildung erweitert haben, zählt sie Entwicklungsberichte schreiben, Gespräche mit Eltern und Kooperationspartnern führen und übersetzen. Sie erlebt den Anspruch, dem Orientierungsplan entsprechen und darüber hinaus den Kindern mit besonderen Bedürfnissen gerecht werden zu müssen, als Druck von außen, der in Anbetracht der Fülle der Aufgaben mit Überforderung, dem Gefühl des Allein-Gelassen-Seins und dem Bedarf nach Unterstützung verbunden wird. Die kritische Wahrnehmung der Leitung spiegelt sich in dem Satz wieder: „Wir {[die Politik]} schreiben uns das {[Inklusion]} auf die Fahne, aber nachgedacht haben wir nicht. Ach so, dafür brauchen wir Geld. Aber das haben wir nicht!“ (C3,~\ref{C3_53}) Konkret hinterfragt sie zum Beispiel an, wer den Dolmetscher bezahlt, wenn die Eltern kein Deutsch sprechen. Die Regelgruppen dieser Einrichtung arbeiten am Nachmittag aufgrund der geringeren Kinderzahl verstärkt gruppenübergreifend, so dass der Personalschlüssel positiver ausfällt. Hierbei wird die Erfahrung gemacht, dass auf den Bedarf der Kinder besser eingegangen werden kann, so dass Eltern die Empfehlung ausgesprochen wird, dem Kind, das nach Einschätzung der Mutter „noch viel lernen muss“ (C3,~\ref{C3_12}), in diese Nachmittagsbetreuung zu geben. Die Gruppensituation wird mitunter als nicht tragbar beschrieben, weshalb Kinder in andere Einrichtungen verwiesen werden.
Um das zu vermeiden, werden kleinere Gruppen als notwendige Bedingung benannt. Den Bedürfnissen eines Kind mit ausgeprägten Verhaltsauffälligkeiten im Sinne eines Aufmerksamkeitsdefizitsyndrom hätte nach der Wahrnehmung der Kindergartenleitung bei einer Gruppengröße von acht Kindern und drei Fachkräften entsprochen werden können. 
Die Gruppengröße zu senken wird als wichtigstes Merkmal eingeschätzt, da ein erhöhter Personalschlüssel mit der Erfahrung verbunden wird, dass zu viele Erwachsene im Raum Unruhe bringen und das Bedürfnis der Kinder nach einer ruhigen Atmosphäre gefährden.

Übereinstimmend mit den anderen Einrichtungen wird als zentrale Voraussetzung für die Umsetzung von Inklusion die Gruppengröße und entsprechend die positivere Fachkraft-Kind-Relation hervorgehoben. 

Positiv wird wahrgenommen, dass der Blick der Gesellschaft auf die Bedürfnisse der Kinder und ihre Entwicklungserschwernisse ein veränderter ist und durch Frühe Hilfen Erfolge beim Kind erzielt werden können. 

Inklusive Erziehungs- und Bildungsarbeit in jeder Einrichtung setzt voraus, dass die Rahmenbedingungen von vornherein verändert werden, so dass die Gruppensituation bewältigt werden kann und Kinder nicht in andere Einrichtungen empfohlen werden müssen. In allen  benannten Fällen wird als Motiv Kinder 'wegzuschicken' Hilflosigkeit und Überforderung benannt, dem Kind mit besonderen Bedürfnissen nicht gerecht zu werden. In der Einrichtung der Diakonie wird der Fall, dass ein Kind in eine andere Einrichtung verwiesen wird, nicht beschrieben. Auch die negative Zuschreibung überfordert zu sein bleibt aus. %Die Annahme besteht, dass ein möglicher Grund in der Ausstattung mit zusätzlichen Fachkräften zu suchen ist, die in Projekten tätig sind, die wiederum in Kleingruppen im Kindergartenalltag durchgeführt werden und dadurch die Stammgruppen entlasten.   

\subsection{Auswertungskategorie 2: Die Rolle der pädagogischen Fachkraft}

\paragraph{Erforderliche Kompetenzen}
Fördernde Bedingungen für Inklusion liegen vor, wenn die Fachkräfte über die Kompetenz verfügen, sich selbst sowie die stattfindenden Prozesse im Alltag zu reflektieren. Das setzt voraus, dass die Erzieherin sich mit ihrer Persönlichkeit und ihrer eigenen Biografie auseinandersetzt, da unverarbeitete familiäre Inhalte auf andere Menschen projiziert werden und auf diese Weise Missverständnisse und schwer wieder aufzulösende Konflikte entstehen können. Die selbstreflexiven Auseinandersetzung mit der eigenen pädagogischen Arbeit wird von der Einrichtung unter dem Träger der Diakonie als Voraussetzung für eine Weiterentwicklung der pädagogischen Qualität angesehen. 
Die Bereitschaft Konflikte zu lösen, in Zusammenarbeit mit den Eltern zu treten und kommunikationsfähig im Team zu sein, werden neben den pädagogischen Kenntnissen als notwendige Kompetenzen genannt.

Geeignete Erzieherinnen-Persönlichkeiten werden durch Merkmale wie respektvoll, wertschätzend, einfühlsam, flexibel im Umgang mit den Kindern und ihrer Spontanität, 'mit beiden Beinen auf dem Boden stehend' und offen beschrieben.
Im Umgang mit den Eltern wird eine Metapher des Reiters auf dem Pferd und eines Menschen, der nebenher läuft, beschrieben:  
„Um den da unten [die Eltern] zu erreichen, muss ich [die pädagogische Fachkraft] vom Pferd steigen, nicht ihn mit hinauf nehmen wollen.“ (C3,~\ref{C3_72}) 

\paragraph{Zusatzqualifikationen}

Die Forschungsergebnisse zeigen, dass Bildung als lebenslanger Prozess verstanden und somit der Bedarf an Fort- und Weiterbildung als selbstverständlich angesehen wird. Der gesehene Weiterbildungsbedarf ist an den Kindern und ihren Lebenslagen orientiert.
Zwischen der Bereitschaft zur Teilnahme an Fortbildungen und dem zur Verfügung gestellten finanziellen Rahmen wird ein Kausalzusammenhang gesehen. Zudem wird gesagt, dass um so mehr Personal zur Verfügung steht, das Wissen um so spezialisierter im Team angelegt werden kann. Entsprechend der Interessen und Ressourcen der einzelnen Teammitglieder werden Überlegungen angestellt, wer sich in welchen Bereichen weiterbilden kann. Zudem wurde die Teilnahme des Gesamtteams an Fortbildungen positiv bewertet, da beim Transport von Informationen in das Team keine Verluste verzeichnet werden. 

Das Inklusionsverständnis, die übergeordneten Ziele und Werte der Einrichtung, die Haltung und die benannten Leitungsaufgaben werden folgend im Zusammenhang und fallbezogen dargestellt. 

\paragraph{Die Einrichtung der Diakonie} 
Inklusion bedeutet für die Kindergartenleitung, die der Diakonie zugeordnet ist, dass die Einrichtung sich in einem Prozess der Veränderung befindet mit dem gemeinsam Ziel Bildungsangebote zu ermöglichen, mit Hilfe derer die unterschiedlichen Themen der Kinder beantwortet werden können. Die Herausforderung werden dabei in den unterschiedlichen Bedarfen und Ausgangsbedingungen, die die Kinder mitbringen, gesehen. 
Egal, aus welcher sozialen Lage die Kinder kommen, das angestrebte übergeordnete Ziel ist die Persönlichkeitsbildung in Vorbereitung auf die Schule. Das bedeutet, dass erstens das Sprechen der deutschen Sprache gefördert wird, sodass die Kinder in der Schule gut mitkommen können, zweitens, dass die Kindern im sozial-emotionalen Bereich so gestärkt werden, dass sie sich gegenseitig in ihrem schulischen Lernprozess unterstützen können und dementsprechend ihre Gefühle regulieren können und wissen, was ihnen gut tut. „Kinder stark machen, ist hier Programm!“ (C1,~\ref{C1_69})
Die Haltung der Fachkräfte wird im Zusammenhang mit der Frage gesehen, ob diese sich dem genannten gemeinsamen Ziel verpflichten wollen. Inklusion erfordert eine offene Haltung gegenüber den Fachkräften, worunter verstanden wird, offen und ohne Überheblichkeit in den Kontakt zu anderen zu treten und von deren Fachkompetenz zu profitieren, da jede Perspektive auf das Kind die Arbeit als Ganzes bereichert.
Da die Haltung der Fachkräfte als entscheidendes Kriterium für gelingende Inklusion angenommen wird, sind alle Fort- und Weiterbildungen daran ausgerichtet. Die Persönlichkeitsentwicklung der Erzieherin wird ernst genommen und als Voraussetzung für pädagogische Qualität gesehen, konkret die Erzieherin zu befähigen mit den komplexen Herausforderungen im Alltag authentisch umzugehen. 
Die Aufgaben der Kindergartenleitung werden darin gesehen, die Bildungs- und Entwicklungsprozesse der ihr anvertrauten Fachkräfte und Familien zu fördern und gemeinsam mit ihnen zu ermitteln, was deren Bedarfe sind sowie Hilfestellungen zu geben, so dass Antworten auf die Bedarfe gefunden werden können. Die Kindergartenleitung leitet die Fachkräfte über den Dialog zur Selbstreflexion an. „Wir sind alle im Werden. Wir entwickeln uns alle. Ich meine, wenn ich diese Tür zu mache -- es geht nicht um funktionieren hier drin.“ (C1,~\ref{C1_55}) Der Dialog ist 'auf Augenhöhe' und beidseitig, dass heißt, die Leitung ist ebenfalls offen in Austausch zu treten und in Situationen, in denen sie an ihre Grenzen stößt, Unterstützung von ihren Kollegen anzufragen. 
Zudem hat die Leitung eine Beratungsfunktion für die Eltern. Sie steht als Ansprechpartnerin jederzeit zur Verfügung und stabilisiert in Krisen. Die Notwendigkeit der Niederschwelligkeit des Beratungsangebots erklärt sich durch den Bedarf der Eltern: „Die Menschen, die hier sind, tun sich eher schwer in die Stadt zu laufen zur Beratungsstelle, das würden die nie machen.“ (C1,~\ref{C1_38}) Dass Eltern sich in der Einrichtung wohlfühlen, ist das Wichtigste, was es zu leisten gilt, denn nur unter dieser Voraussetzung besteht Offenheit ein Bildungsangebot zu machen. „Kinder und Eltern stark machen, auch die Mitarbeiter, das ist hier Programm!“ C1,~\ref{C1_69}
Grenzen der Inklusion werden bei Kindern, die die so stark beeinträchtigt sind, dass die Inklusion im Alltag massiv erschwert ist, in finanziellen und personellen Ressourcen gesehen. Ein Kind, das eine sehr hohen Pflegeaufwand benötigt, bedarf eines Extraraums für die Pflege und spezieller Pflegekräfte, was unter den bestehenden Rahmenbedingungen als undenkbar angesehen wird. Eine Zuführung in den Kindergarten aber wird mit positiver Wirkung für das betroffene Kind verbunden.

\paragraph{Die Einrichtung der AWO}
Das Inklusionsverständnis der Kindergartenleitung der AWO impliziert den uneingeschränkten Zugang zu Bildungsmöglichkeiten und benennt die uneingeschränkte Teilhabe als Ziel, womit das Schaffen von Erfahrungsräumen verbunden wird, die die Kinder von sich aus nicht hätten. 
Mit der Haltung wird wiederholt benannt, dass die Inklusionsidee vom Team getragen werden muss und das Inklusion -- verstanden als Prozess -- wächst und beim Umgang miteinander und der Offenheit sich weiter zu bilden ansetzt. In der Wahrnehmung der Kindergartenleitung entsteht Inklusion in einer Atmosphäre der Wertschätzung und des gegenseitigen Respekts. Diese herzustellen sieht die Leitung als ihre Hauptaufgabe an, das heißt, dafür Sorge zu tragen, dass sich alle wohl und angenommen fühlen sowie ein wertschätzender Umgang im Team und eine gute Zusammenarbeit herrschen. Das Wohlbefinden im Team wird in Zusammenhang mit dem der Eltern und schließlich dem der Kinder gebracht. 
Grenzen der Inklusion werden bei der Teilgabe gesehen und erfahren, im Bestreben alle Kinder zu integrieren, ohne dass einzelne das Gefühl haben in einer besonderen Situation zu sein.    
Die Leitung beschreibt, dass es trotz der vorangestellten Überlegungen im Team und entsprechender Bemühungen Situationen gibt, in denen Kinder mit besonderen Bedürfnissen erfahren, dass sie im Vergleich zu den anderen Kinder eingeschränkt sind, zum Beispiel beim Waldausflug oder Schwimmkurs.
Das Kind mit einer motorischen Einschränkung wurde beim Waldausflug im Buggy geschoben und ist mit den Barrieren konfrontiert, dass es nicht in den Wald hinein laufen kann, um mit den Naturmaterialien in Kontakt zu kommen und dass es nicht so schnell und autonom wie die anderen unterwegs sein kann, da es auf das Schieben durch eine Betreuerin angewiesen ist. Ideen zur Veränderung dieser Situationen fehlen. Gefahren auszuschließen wie das Umkippen des Kinderwagens wird als Grund benannt, warum das Kind nicht über die unwegsame Wege geschoben wurde oder ein anderes Kind das Schieben des Buggys übernehmen konnte. Als Grenzen werden im Zusammenhang mit der Teilhabe und Teilgabe die eigenen Barrieren im Kopf benannt. Es wird die Situation beschrieben, dass in Teambesprechungen viel Zeit investiert wird, um zu überlegen, wie das Angebot gestaltet werden kann, um Ausgrenzung zu vermeiden. In der jeweiligen Situation wurde dann die Erfahrung gemacht, dass die Kinder unkompliziert und offen im Umgang mit einander sind. So erklärt zum Beispiel ein Kind einem anderen Kind, das neu in der Gruppe ist: „Weißt du, der kann jetzt mit der Hand das nicht rüber machen, das musst du jetzt machen“ (C2,~\ref{508}).  

\paragraph{Die Einrichtung der katholischen Kirchgemeinde}
Inklusion verstanden als Prozess sowohl für den Kindergarten als auch für die Familien bedeutet für die Leitung, dass Kinder nicht ausgegrenzt, sondern respektiert werden, am 'normalen' Leben teilhaben können und ihnen etwas zugetraut wird. Es bedeutet für sie auch, ein gegenseitiges Lernen zu ermöglichen und den nicht behinderten Kindern zu zeigen, dass es nicht selbstverständlich ist, dass sie ohne Handicap leben und dass die Kinder mit Handicap dafür andere Kompetenzen mitbringen. Die Haltung gegenüber Inklusion kommt in den folgenden Sätzen zum Ausdruck: „Wir sind offen für alle Kinder und haben die Position, wir probieren es aus und wenn wir es nur zwei Wochen probieren, aber wir probieren es. Erst im Tun, im Leben und Erleben sehen wir oft, was das bringt. Manchmal stoßen wir dann auch an unsere Grenzen und haben Angst, diesem Kind nicht gerecht werden zu können.“ C3,~\ref{C3_28} Grenzen der Inklusion werden entsprechend der Haltung in Ermanglung von Zeit und Zuwendung und abhängig von dem Grad der Behinderung gesehen.  
Die Leitungsaufgaben sind auf das zur Verfügung stehen als Ansprechpartnerin  und auf die zwischenmenschliche Kommunikation zischen den Fachkräften, den Eltern und Kindern sowie anderen Institutionen bezogen. Um fehlende Informationen zu beschaffen, wird Fachpersonal von außen dazu geholt.
„Manchmal sagen meine Kollegen zu mir: 'Du hast doch immer für alles Verständnis und sagst, wir schaffen das.' Dann ist es besser, wenn jemand von außen kommt und aus der Praxis Beispiele bringt, so dass die Erzieherinnen sich verstanden fühlen und noch einmal eine andere Autorität zu Wort kommt.“ (C3,~\ref{C3_68}) 

\subsection{Auswertungskategorie 3: Antworten auf die individuellen Bedarfslagen}

\paragraph{Sprachförderung}
Die Sprachförderung ist in jeder der untersuchten Einrichtungen als zentrale Zielsetzung verankert.


\subsection{Auswertungskategorie 4: Partnerschaft mit den Eltern}

\paragraph{Bedarfslagen der Eltern}
Die gebündelten Expertenaussagen zeigen, dass viele Familien aufgrund schlechter beruflicher Perspektive staatliche Unterstützung durch Harz IV beziehen und mit Themen wie Armut, Ausgrenzung und Beschämung konfrontiert sind. Die Kindergartenleitung der AWO konstatiert, dass bildungsferne Eltern im Vergleich zu bildungsorientierten Eltern weniger Interesse an der pädagogischen Arbeit im Kindergarten zeigen würden, wobei die fehlende elterliche Präsenz auch durch die hohe familiäre Belastung durch beispielsweise mehrere Kinder erklärt wird.
Die Elternhäuser werden von der Kindergartenleitung der katholischen Kirchgemeinde durch Unruhe, fehlende Strukturen und eine Verarmung hinsichtlich der Lernanreize beschrieben. Der Zulauf zu Helfersystemen wie städtischen Beratungsstellen ist erschwert, wobei im diakonischen Kindergarten der Bedarf angezeigt wird, Krisen wie Suchtkonflikte, Trennung und Scheidung und Todesfälle aufzufangen. Zudem beschreibt die Kindergartenleitung des katholischen Kindergartens, dass elterliche Ängste und Sorgen nicht nur in Bezug auf die Kinder mit besonderen Bedürfnissen bestünden, sondern auch in Bezug auf die gemeinsame Erziehung und Bildung. In der Wahrnehmung der Eltern führt selbige dazu, dass die Kinder zu wenig Zuwendung und Lernanreize bekämen, weil die Erzieherinnen aufgrund des Mehrbedarfs bedingt durch die Kinder mit besonderen Bedürfnissen nicht die nötige Zeit zur Verfügung haben würden. Gründe sieht die Kindergartenleitung in dem gesellschaftlichen Erwartungsdruck, der mit Angst in Verbindung steht: „Mein Kind kommt zu kurz. Unser Kind fällt durch das Raster. Die Erzieherinnen haben nicht die nötige Zeit. Was passiert mit meinem Kind, wenn es nicht da 'hoch' kommt?“ (C3,~\ref{C3_44}). Demgegenüber stehen die -- von der Leitung der AWO beschriebenen -- finanziell gut situierten und bildungsinteressierten Eltern, die bewusst eine Einrichtung gewählt haben, in der Kinder unterschiedlicher sozialer Herkunft vertreten sind.

\paragraph{Elternabende}
Um die Ängste der Eltern ernstzunehmen, wurden von der Leitung des katholischen Kindergartens Referentinnen eingeladen, konkret die Heilpädagogin und die  Sprachtherapeutin, die von ihrer pädagogisch-therapeutischen Arbeit mit den Kindern berichteten und als Einblick eine Videoaufnahme des Kindes mit Down-Syndrom  zeigten. Die Leitung beschreibt folgende Resonanz: „Da gab es ein Aha-Erlebnis bei den Eltern -- es ist nicht selbstverständlich, dass wir gesund sind. Mir kann heute etwas passieren, dann bin ich behindert und will genauso geschätzt werden wie gestern noch.“ C3,~\ref{C3_38} Zudem wurde die Mutter, deren Kind eine spastische Lähmung hat, zu einem Elternabend eingeladen. Sie berichtete aus ihrem Leben, von ihren Erfahrungen und Ängsten. Auch hierbei stellte die Leitung eine positive Wirkung fest, die sie durch eine größere Offenheit der Eltern im Umgang mit der referierenden Mutter beschrieb. Diese Öffnung hätte sich zudem auf die Kinder übertragen. 
Als Antwort auf die unterschiedlichen kulturellen Hintergründe der Familien veranstaltete der diakonische Kindergarten einen Elternabend, in dem zum Austausch über die unterschiedlichen Feste in den jeweiligen Kulturen eingeladen wurde. 

\paragraph{Eltern- und Entwicklungsgespräche}
Im Katholischen Kindergarten wurden die Elternabende zugunsten einer höheren Frequenz von Elterngesprächen reduziert, da in einem kleineren Rahmen die Entwicklung des einzelnen Kindes besser in den Blick genommen werden kann und auch die elterlichen Ängste besser beantwortet werden können. Diesen wird mit der Frage begegnet: „Auf welchen Gebieten haben Sie Angst, dass ihr Kind zu kurz kommt?“ (C3,~\ref{C3_63}), woraufhin den Eltern die Möglichkeit eingeräumt wird, einen Vormittag in der Einrichtung zu verbringen, um erfahrbar zu machen, ob es den Kindern an Zeit und Zuwendung mangelt. 
Entwicklungsgespräche finden im katholischen Kindergarten nach Bedarf, jedoch mindestens einmal im Jahr statt. Sie werden anberaumt, wenn ein erhöhter Unterstützungsbedarf beobachtet wurde. Wenn die Wahrnehmungen des Kindes gegenüber den Eltern mitgeteilt werden, reagieren die Eltern sehr unterschiedlich. Manche sagen, dass sie das beschriebene Verhalten von zuhause kennen würden, andere wiederum reagieren sehr erschrocken und äußern die Angst, dass das Kind aus der Einrichtung verwiesen werden würde. In einem solchen Fall wurde als sinnvoll erlebt, den Eltern Zeit zu geben und im zeitlichen Abstand von einer Woche ein weiteres Elterngespräch durchzuführen. Entwicklungsgespräche finden unter Einsatz der Methoden zur Dokumentation kindlicher Entwicklung statt. 
Zudem wird als wichtig beschrieben, dass die Eltern sich nicht zu schämen brauchen, wenn sie zusätzliche Hilfe benötigen. Persönliche und bestärkende Elterngespräche werden als Antwort auf elterliche Überforderung und Entmutigung gesehen, die Durchführung wird jedoch als anstrengend beschrieben. 

\paragraph{Information und Transparenz}
Im Aufnahmegespräch wird zu einem Austausch angeregt und die Eltern werden informiert, dass Kinder mit besonderem Förderbedarf aufgenommen werden und der Anteil an Kindern mit Migrationshintergrund erhöht ist. Zudem werden zentrale Aspekte der Arbeitsweise erklärt, in der Einrichtung der AWO, dass die Fachkräfte einen ressourcen-orientierten Blick auf das Kind haben, in der Einrichtung der Diakonie, wie wichtig das 'Zusammenleben' und die 'Gemeinschaft' sind. „Wenn die aufgenommen werden, wird denen das natürlich auch gesagt, wie wir hier zusammenleben und wie wichtig die Gemeinschaft uns ist und die haben wirklich ganz schönes Vertrauen. Die kommen auch mit allem. Das wissen die auch, wenn etwas ist, dann darf man kommen. Wenn nicht hier, wo sonst.“ (C1,~\ref{C1_59})
Möglichkeiten Eltern zu informieren sehen verschieden aus. Der Kindergarten der AWO bringt zum Beispiel ein Blatt Papier an der Tür zum Gruppenraum an, auf dem die Bilder und Namen der neuen Kinder, die in die Gruppe kommen, festgehalten werden.
Der Kindergarten der katholischen Kirchgemeinde nutzt ein Gruppentagebuch, das die Eltern täglich über die Aktivitäten, die in der Gruppe stattgefunden haben, informiert und Liedtexte oder Geschichten bereithält, die die Eltern kopieren können. 

\paragraph{Beteiligungsmöglichkeiten}
Eine Möglichkeit für Eltern sich zu beteiligen und auszutauschen ist das Elterncafé, das von den Kindergärten der Diakonie und der AWO eingerichtet wurde. 
Die Kindergartenleitung der AWO berichtet, dass das Elterncafé einmal im Monat stattfindet und sich unter pädagogischer Leitung verschiedenen Themen widmet. Das Thema zum Zeitpunkt der Befragung bezog sich wie auch schon bei dem thematischen Elternabend auf die kulturellen Hintergründe, die in der Einrichtung vertreten sind – die Länder stellen sich vor. Ein Elterncafé wurde zum Beispiel von den russisch stämmigen Eltern gestaltet. Sie erzählten, wie sie 
nach Russland gekommen sind, wieso es diese Rückbewegung gab, stellten ihre Kultur und ihre Religion vor und hatten ein traditionelles Gericht vorbereitet. 
Die Bedeutung in solchen Treffen wird darin gesehen, dass Austauschmöglichkeiten für die Eltern bereitgestellt werden, wodurch das Bedürfnis einander zu verstanden und mitzufühlen beantwortet werden kann. Das Elterncafé ist auf Initiative der Eltern entstanden, die eine Möglichkeit des Austauschs suchten. 
In der Einrichtung unter dem Träger der Diakonie findet das Elterncafé aller 14 Tage statt. Auch hier finden thematische Gesprächsrunden mit kurzem fachlichen Input statt. Offen ist dieses nicht nur für die Eltern des Kindergartens, sondern auch für interessierte Eltern mit Kindern aus dem Stadtteil. 
Neben dem Elterncafé gibt es weitere Veranstaltungen, wie Elternaturtage oder Feste, an denen auch Eltern teilnehmen können, die ihre Kinder nicht in der Einrichtung haben, so dass die Arbeit im Kindergarten in das Gemeinwesen hinaus strahlt. Selbiges wird auch durch die Veranstaltung von Flohmärkten oder das Einrichten eines eigenen Standes auf dem Weihnachtsmarkt erreicht.
Die Einrichtung der Diakonie betont eine enge Interaktion in die Familie und gibt Raum für Begegnung und Beziehungsaufbau: „Wer mit den Kindern arbeitet, muss mit den Familien arbeiten.“ (C1,~\ref{C1_16}) Sie bietet Eltern-Kind-Gruppen im Bereich Naturpädagogik, Familienausflüge am Wochenende, die Möglichkeit der Beteiligung im Förderverein, im Elternbeirat oder in der 'Zukunftswerkstatt' an. Die Leitung spricht sich in Bezug auf die Elternarbeit Kompetenz. Sie beschreibt, dass sie Konflikte zwischen Eltern und Kollegen durch die Wahrung des Außenblicks in der Regel auflösen kann und dass sie aufgrund der langjährigen Leitungstätigkeit vielfältige Erfahrungen gesammelt habe, wie sie Eltern unterstützen kann.  
Die Leitung des katholischen Kindergartens beschreibt als elterliche Beteiligungsmöglichkeiten, die in den Gruppenalltag integriert sind,  Feste feiern, Bastelaktion, zum Beispiel Kinder und Eltern basteln gemeinsam Laternen oder Schultüten, das gemeinsame Frühstück zwei Mal im Jahr, was sehr dankbar von den Eltern angenommen wird, die mögliche Begleitung an Wandertagen oder Spieltagen, an denen die Eltern Anreize bekommen, was für unterschiedliche Spiele für welche Altersgruppe interessant ist.  
Die Kindergartenleitung unter dem Trägerdach der AWO beschreibt ein Projekt, das der Kindergarten in Kooperation mit der Hochschule durchführt -- \emph{Gesund aufwachsen in der Kita}. Hierbei wurden die Eltern in der Auswahl der Projekte von Anfang an einbezogen. So wurde entschieden, dass ein Waldausflug gemeinsam mit den Eltern stattfinden sollte, der an einem Freitagnachmittag zeitlich möglichst spät gelegt wurde, um einer großen Elternschaft die Teilnahme zu ermöglichen. Die Leitung berichtet von der Erfahrung, dass erstmals alle Elternteile, vollzählig erschienen seien. Diese Erfahrung verhilft der Leitung dazu ihren Fokus von dem Desinteresse der Eltern auf die Frage zu richten, wie die Eltern noch erfolgreicher beteiligt werden können.  

\subsection{Auswertungskategorie 5: Kooperation}

\paragraph{Kooperation mit Fachdiensten und Institutionen}

Als notwendige Bedingung für Inklusion formulieren die Kindergartenleitungen übereinsimmend ein „breite Bündnis“ (C1,~\ref{C1_14}), das mit den gegebenen Bedürfnissen der Kinder in der Gruppe korreliert. Das heißt, bei Gefährungssituationen werden Institutionen wie \emph{Wildwasser} oder \emph{Wendepunkt} zur Fallsupervision eingeladen, bei Suchtkonflikten wird der kommunale Suchtbeauftragte angefragt oder wenn Kinder mit einer Seh- oder Hörbeeinträchtigung in die Einrichtung aufgenommen werden, wird der Kontakt zur Beratungsstelle der Sehbehindertenschule oder zu dem Bildungs- und Beratungszentrum für Hörgeschädigte hergestellt. Bei der Einrichtung der AWO besteht eine enge Kooperation mit der Frühförderstelle der AWO, da beide Institutionen unter einem Trägerdach zusammengefasst sind. Da die Heilpädagogin der Frühförderstelle in die Einrichtung kommt, wurde eine niedrige Hemmschwelle für das Anbahnen des Erstkontakts zwischen Kind und Heilpädagogin als fördernd erlebt. Bei wahrgenommenen Bedarf wird die Heilpädagogin gebeten, in der Gruppe das Kind wahrzunehmen. Die Hemmschwelle Hilfen in Anspruch zu nehmen kann durch die einfachen Zugangswege und das Wahrnehmen vor Ort, dass viele Kinder heilpädagogische Förderung bekommen, abgebaut werden.
Auch im katholischen Kindergarten kommt ein bis zwei Mal im Jahr eine Fachkraft aus der Sprachheilberatungsstelle in die Einrichtung, ihr werden in Absprache mit den Eltern Kinder vorgestellt, bei denen Klärungsbedarf besteht. Die niedrige Hemmschwelle wird auch hier als positiv erlebt, da die Eltern entlastete werden, was mit der Wahrnehmung der Leitung korrespondiert: „Die meisten unserer Eltern würden es nicht schaffen einen Termin einzuhalten.“ (C3,~\ref{C3_21}) 
Wenn die Kinder bereits mit bestehenden Therapeuten-Kontakten, zum Beispiel Ergotherapie oder Spieltherapie, in den Kindergarten kommen, werden diese Kontakte aufgenommen.     
Die außenstehenden Helfer in Verbindung zueinander zu bringen, wird von der Leitung des katholischen Kindergartens als schwierig erlebt, was zum Beispiel durch bestehende lange Wartezeiten für eine Entwicklungsdiagnostik zum Ausdruck gebracht wird. 
  
\paragraph{Kooperation mit der Schule}
Die intensive Zusammenarbeit zwischen Kindergarten und Grundschule wird zum Beispiel durch das Modellprojekt \emph{'Bildungshaus'} unterstützt -- der diakonische Kindergarten gehört zu den Modellstandorten. Das Bildungshaus-Projekt wurde 2007 vom Ministerium für Jugend und Sport Baden-Württemberg initiiert und wird durch das Zentrum für Neurowissenschaften und Lernen wissenschaftlich untersucht und begleitet. 
Kinder im Alter von drei bis zehn Jahren lernen und spielen gemeinsam in der Schule und im Kindergarten, wodurch der Übergang in die Schule, erleichtert und die Bildungsbiografie des Kindes nahtlos fortgeführt werden kann sowie die Ängste und Vorbehalten seitens der Eltern gegenüber Schule abgebaut werden können.

Durch das Projekt \emph{'Schulreifes Kind'}, an dem der Kindergarten der AWO teilnimmt, ist der Kontakt zu der nahe gelegenen Grundschule intensiviert worden. 
Das Kooperationskonzept zwischen Kindergarten und Grundschule sieht vor förderbedürftige Kinder in einer eigenen Gruppe in unterschiedlichen Bereichen rechtzeitig vor Schuleintritt zu fördern, so dass diese Kinder keine Zurückstellung erhalten müssen.  
Die dortige Musiklehrerin kommt zwei Mal in der Woche in den Kindergarten, um mit den Kinder vor Ort zu arbeiten und einmal pro Woche gehen die Kinder in die Schule und erfahren musikalische Früherziehung im Musiksaal der Grundschule. Die Kinder haben die Schule dadurch bereits kennenlernen können, wodurch ihre Hemmschwelle abgebaut werden konnte.  
Als positive Veränderung wird von der Kindergartenleitung der AWO die aufgehobene Trennung zwischen Schule und Kindergarten genannt, die in gemeinsam besuchten Fortbildungen und gegenseitigen Einladungen zur Weihnachtsfeier ihren Ausdruck findet. Im Team wurde der Wunsch benannt: „Eigentlich wäre es schön, wenn die Einrichtung daneben wäre, dass man wie so ein Kinderhaus hätte, wo die Übergänge noch besser wären.“ (C2,~\ref{C2_600}) 

\paragraph{Kooperation im Team}

In allen Einrichtungen gibt es Teamsitzungen, in denen das Verhalten von Kindern diskutiert wird. Die Unterschiede bestehen in dem zur Verfügung gestellten zeitlichen Rahmen: Die Einrichtung der Diakonie
misst diesem Aspekt die vergleichsweise größte Bedeutung bei und stellt dementsprechend die meiste Zeit zur Verfügung. Die Leitung beschreibt, dass bedingt durch die vielfältige Projektarbeit der interdisziplinäre Austausch gewährleistet ist, der in der Gesamtteamsitzung für alle Beteiligten transparent gemacht wird. Die Kinder sind den Projektgruppen zugeordnet und die erwachsenen Verantwortlichen bringen ihre Beobachtungen, Wahrnehmungen und Erlebnisse ein, so dass bei jedem Gesamtteam-Treffen über die Kinder gesprochen wird und der Blick auf deren Entwicklung von der Leitung als 'pointiert' und 'wie mit der Lupe' beschrieben wird. Neben diesem Treffen gibt es noch sogenannte Kleingruppenteams, in diesen treffen sich die Fachkräfte von jeweils zwei Gruppen und bilden ein eigenes Team, das sich wöchentlich trifft, um die Kinder zu besprechen und Überlegungen anzustellen, ob die Förderungsangebote die individuellen Bedürfnisse der Kinder treffen und zuletzt gibt es noch ein weitres Treffen, in dem das erwähnte Kleingruppenteam mit der Heilpädagogin zusammen kommt, um speziell die Kinder, die Integrationshilfe erhalten, anzuschauen.  
Auch in dem Kindergarten der AWO finden Kleingruppentreffen statt, die jedoch mit einer Stunde pro Woche vor allem, wenn Kinder mit besonderem Förderbedarf in der Gruppe sind als nicht ausreichend beschrieben werden.    
Einen Freitag im Monat schließt die Einrichtung früher, so dass drei Stunden für Teambesprechungen zur Verfügung stehen, in denen auch die Heilpädagogin, die Sprachförderung anbietet, eingeladen wird. Vor allem diese Zeit wird für den kollegialen Austausch zur Unterstützung der Selbstreflexion zur Verfügung gestellt.  


 

  



  
 




   

 
 
 
 
