\part{Die empirische Studie}
\chapter{Die qualitative Untersuchung}
\section{Problemstellung und Begründung der Entscheidung für ein qualitatives Design}
Das Erkenntnisinteresse der vorliegenden Arbeit besteht darin, Aspekte und Zusammenhänge aufzudecken, die Inklusion im Kindergartenalltag gelingen lassen beziehungsweise erschweren. Der Gewinn im Identifizieren solcher Mechanismen ist zum einen im Nutzen für andere Kindergärten zu sehen, die aus den Ergebnissen Schlüsse ziehen können, die ihnen auf dem Weg zu einem inklusiven Kindergarten als Orientierung dienen. Zum anderen können die Ergebnisse darüber Auskunft geben, was konkret von Seiten der Verantwortungsträger erwartet und gebraucht wird, um das Inklusionskonzept erfolgreich etablieren zu können.

Die Forschungsfrage, „Unter Berücksichtigung welcher Aspekte ist Inklusion im Kindergarten umzusetzen?“, zielt auf das Aufdecken von sozialen Mechanismen innerhalb der Strukturen des Kindergartens ab. Diese betreffen sowohl die Rahmenbedingungen und das Konzept, die  Einfluss auf das Handeln und die Interaktionen im Kindergartenalltag nehmen, als auch die Haltung, Kompetenzen und die Zusammenarbeit im Team, die den Formen des Umgangs eine bestimmte Färbung geben. Die Forschungsfrage erfordert folglich eine mechanismusorientierte Strategie, die auf der Rekonstruktion von Fällen basiert. Solche eben genannten spezifischn Systemqualiäten des Kindergartens können nicht standartisiert und mit Hilfe der Aussagen von vielen individuellen Akteuren erhobenen werden, sondern beanspruchen als geeignete Vorgehensweise ein  
qualitatives Design. Das qualitative Forschungsdesign dient nicht nur der Rekonstruktion von Fällen oder Prozessen, sondern wird zudem auch immer dort empfohlen, „wo es [zum einen] um die Erschließung eines bislang wenig erforschten Wirklichkeitsbereichs“ geht (Flick,von Kardoff~\& Steinke 2000, 25) -- Inklusion im Untersuchungsfeld Kindergarten stellt einen solchen Wirklichkeitsbereich dar -- und wo es [zum anderen] hinsichtlich der Theorieorientierung auf Entdeckung abzielt.
Laut Flick,von Kardoff~\& Steinke 2000, 45 bedeutet das, dass Theorien und Hypothesen aus den unmittelbar gesammelten Daten entwickelt werden. Da es sich hierbei um unbestätigte Annahmen handelt, die an der Realität scheitern können, bedarf es einer sich anschließenden  quantitativen Überprüfung. 
Da jedoch die quantitative Bestätigung den Rahmen dieser Arbeit sprengen würde, muss auf die Überprüfung der aus den qualitativ gewonnenen Daten verzichtet werden.

Um solcherart komplexe Wissensbestände zu rekonstruieren empfehlen Meuser~\& Nagel (1997, 481) das Experteninterview als eine geeignete Methode vor. Entsprechend dieser Empfehlung fiel die Wahl der Methode zur Datengewinnung auf das Experteninterview. 


\section{Die Erhebungsmethode: Das Experteninterview}
Im Folgenden wird zunächst der Expertenbegriff diskutiert, um darzulegen, wer für das Vorhaben der vorliegenden Arbeit überhaupt als Experte angesprochen werden konnte.
Daran anschließend wird die Auswahl der befragten Experte aufgezeigt. 

\subsection{Der Expertenbegriff}
Gläser~\& Laudel (2010, 13) verweisen darauf, dass der Begriff ’Experteninterview’ in der sozialwissenschaftlichen Literatur in der Regel an die Expertenrolle des Interviewten im untersuchten sozialen Feld gebunden ist (vgl. Meuser~\& Nagel, 2005, Bogner, Littig~\& Menz 2005), das heißt, dass sich das besondere Wissen, über das der Experte verfügt, aus seiner herausgehobenen Position erklärt. Entgegen dieser weit verbreiteten Meinung findet sich bei Gläser~\& Laudel (2010, 11) der bereichernde Gedanke, dass Experten nicht zwingend in einer gehobenen Position zu sein haben, um über ein besonderes Expertenwissen zu verfügen. Jeder Mensch, der einen Erfahrungsbereich für sich erschließt, wird zum Experten. So wird der Musiker, der einen bestimmten Musikstil aufgreift und alles darüber in Erfahrung bringt oder der von einer seltenen Krankheit Betroffene zum Experten für diesen Musikstil oder diese Krankheit. Die Autoren verweisen darauf, dass schließlich jeder Mensch über besonderes Wissen verfügt, das Wissen über die sozialen Kontexte in denen er agiert, an denen er unmittelbar Anteil hat, zum Beispiel das besondere Wissen über die Organisation, in der er arbeitet und über die eigenen Arbeitsprozesse. Ein Experte ist im Sinne von Gläser~\& Laudel (2010, 12) als „Quelle von Spezialwissen über die zu erforschenden sozialen Sachverhalte“ zu verstehen. „Experteninterviews [wiederum] sind eine Methode dieses Wissen zu erschließen.“ Ausgehend von diesem Expertenbegriff kommen für das Untersuchungsfeld Kindergarten sowohl die Leitung der Einrichtung als auch die Erzieherinnen als zu befragende Experten in Betracht. 
 
\subsection{Die befragten Experten}
Die wenigsten Kindergärten weisen sich als inklusiv aus. Das heißt, der Name der Einrichtung ist nicht leitend. Ein Ausschlusskriterium für die Auswahl der Experten war, dass die Einrichtung sowohl von Kindern mit besonderen Bedürfnissen als auch von Kindern ohne besondere Bedürfnisse besucht wird, so dass die Gruppenmischung der gesellschaftlichen Vielfalt entspricht und somit als inklusiv anzusehen ist. Um fündig zu werden, waren sozial schwierige Einzugsgebiete mit einem hohen Anteil an Familien mit einem Migrationshintergrund im Blickfeld, da die Kindergärten zumeist auf die soziale und kulturelle Vielfalt des Einzugsgebietes und somit auf die unterschiedlichen Bedarfslagen der Familien vor Ort Antworten zu finden versuchen und solche Antworten inklusive Prozesse anregen. Da bei diesen Einrichtungen der Anteil der sozial benachteiligten Familien hoch ist, kann nur bedingt von gesellschaftlicher Vielfalt gesprochen werden, da die sozial besser gestellten Familien in ’sozialen Brennpunkten’ als Minderheit vertreten sind.    
Weiterhin wurde berücksichtigt bei der Auswahl der Einrichtungen eine gewisse Trägerbreite abzubilden, da der Einfluss des Trägers, den politischen Willen zur Inklusion einzulösen, als entscheidendes Merkmal ausgewiesen wurde. Es konnten Einrichtungen gefunden werden, deren Träger erstens die Diakonie, zweites die Arbeiterwohlfahrt (AWO) und drittens die katholische Kirchengemeinde sind. Der Erstkontakt per Telefon erfolgte stets mit der Leitung der jeweiligen Einrichtungen, die sich in allen drei Fällen als Interviewpartner zur Verfügung stellten, weshalb die Leiterinnen der Einrichtungen beschränken.

\section{Datenerhebung}
\subsection{Methode der Datenerhebung: der Gesprächsleitfaden}
Um sich ein Bild von dem Arbeitsfeld der Expertinnen zu machen und den Einstieg zu erleichtern, sah die Gliederung des Gesprächsleitfaden vor, dass die Experten als Einstieg gefragt wurden, wie sich die strukturellen Rahmenbedingungen der Einrichtung beschreiben lassen und welche Kinder die Einrichtung besuchen. Daran schlossen sich  Leitfragen zu den konzeptionellen Überlegungen, dem Inklusionsverständnis, der Haltung, zur individualisierten Unterstützung, zum Prozess und seinen Stolpersteinen, der Kooperation im Team und nach außen und der Partnerschaft mit den Eltern an (vgl. Gesprächsleitfaden im Anhang A).

\subsection{Durchführung der Datenerhebung}
Im Rahmen der Bachelorthesis wurden drei Experteninterviews im Zeitraum vom 15. bis 27. Oktober durchgeführt. 
Die Anfragen an die Experten erfolgten durch telefonische Kontaktaufnahme. Im Erstkontakt wurde kurz das Forschungsvorhaben erklärt sowie begründet, warum ein Interesse an der jeweiligen Einrichtung beziehungsweise Person vorliegt. Zudem erwies sich die Mitteilung als sinnvollen ’Türöffner’, wodurch die Aufmerksamkeit auf die Person gelenkt wurde. 
Des weiteren wurde angeboten auf Wunsch den Interview-Leitfaden zuzusenden, so dass die Fragen von der Leitung vorab eingesehen werden konnten. Da der Leitfaden sehr umfangreich ist und reflexive Betrachtung erfordert, wurde es als sinnvoll erachtet diese Möglichkeit einzuräumen. Davon machten zwei von drei Expertinnen Gebrauch. Das hatte zur Folge, dass die Expertinnen unterschiedliche Voraussetzungen mitbrachten. Einerseits gab es die sehr gut vorbereitete Expertin, die Stichpunkte zu jeder Frage formuliert und die Fragen so verinnerlicht hatte, dass sie bei der Beantwortung der Fragen berücksichtigte, welche Antworten an welcher Stelle ihren Platz finden sollten. Andererseits gab es die Expertin, die unvorbereitet angetroffen, mit der Beantwortung einer Frage viele weitere Bezüge zu anderen Fragen hergestellte und in ihren Aussagen eine hohe Spontanität zeigte. Die Durchführung der Interviews zeigte, dass die Auseinandersetzung mit den Fragen vorab Auswirkungen darauf hatte, wie reflektiert, aber auch spontan und damit verbunden emotional beteiligt geantwortet wurde.     
Diese durchaus als groß empfundenen Unterschiede, die letztlich vor allem auf die verschiedenen Temperamente der Expertinnen zurückzuführen sind, haben keinen Einfluss auf die Vergleichbarkeit der Interviewtexte. Diese wird nach Meuser~\& Nagel (2005, 81) durch die Nutzung des Leitfadens in der Interviewführung und den „gemeinsam geteilte institutionell-organisatorische Kontext“ sicher gestellt.

Der Ort der Befragung war der Arbeitsplatz des Experten. Die Befragungsdauer variierte zwischen 50 Minuten und zweieinhalb Stunden, wobei ohne Ausnahme alle Fragenkomplexe angesprochen wurden. Die unterschiedliche Interviewdauer ergab sich dabei aus dem Verlauf des jeweiligen Interviews und der zusätzlichen Themenansprechung von Seiten der Experten. In einem Interview wurde die Aufzeichnung des Interviews verweigert, jedoch darauf hingewiesen, dass große Bereitschaft besteht, ausreichende Zeit für das Interview zur Verfügung zu stellen, so dass entsprechende Notizen angefertigt werden konnten, um das Gespräch anschließend in einem Gedächtnisprotokoll zu rekonstruieren. Nach Anfertigung des Gedächtnisprotokolls wurde dieses der Expertin zugesandt, so dass die Möglichkeit gegeben wurde, Veränderungen oder weitere Ergänzungen vorzunehmen. Auch den anderen Experten wurde zur Überprüfung die vollständig transkribierten Interviews eingereicht. 
Die Experteninterviews wurden mit Hilfe des Aufnahmegeräts Zoom H2 technisch problemlos auf Tonband aufgenommen und anschließend transkribiert. Parallel zu den Tonaufnahmen wurden Notizen zu den Interviews angefertigt. 

Insgesamt stieß die Befragung bei allen Interviewten auf große Bereitschaft, Neugierde und Interesse für das Thema.

\section{Das Auswertungsverfahren: Die qualitative Inhaltsanalyse (unvollständig)}

Nach Abschluss der Interviewführung erfolgte die Transkription entsprechend der Vorgaben von Meuser~\& Nagel (2005, 83). Das bedeutet, die Interviews wurden unter der Wahrung der Anonymität inhaltlich vollständig verschriftlicht und geglättet und somit von Zwischenlauten, Dialektfärbungen, überflüssigen Redundanzen und Floskeln befreit. Diese Methode fand Anwendung, weil sich die „Auswertung von Experteninterviews an thematischen Einheiten, an inhaltlich zusammengehörigen, über die Texte verstreute Passagen [und] nicht an der Sequenzialität von Äußerungen je Interview“ (Meuser~\& Nagel 2005, 81) orientiert und somit auf vergleichbare Datengewinnung ausgerichtet ist. 
Ergänzend wurden die Empfehlungen von Mayring (2010, 55) in zwei Punkten berücksichtigt. Er schlägt vor, unverständliche Passagen sowie abgebrochene Sätzen mit Hilfe von drei Punkten (...) und Stockungen, Unterbrechungen oder Gedankensprünge durch einen Gedankenstrich (-) zu kennzeichnen. 
Für das Formatieren wurde außerdem der Hinweis von Gläser~\& Laudel (2010, 194) berücksichtigt, inhaltlich zusammengehörende Aussagen -- Sinneinheiten -- durch Absätze erkenntlich zu machen. Das heißt, das eine Antwort des Interviewpartners, vorausgesetzt ein neuer Gedanke kommt hinzu, mehrere Absätze umfassen kann. Eine bessere Lesbarkeit wurde außerdem durch das Kursivsetzen der Interviewfragen erreicht (vgl. Transkripte der Interviews im Anhang C). 

Anschließend wurden die Transkripte mit Hilfe der Qualitativen Inhaltsanalyse entsprechend der Vorgaben von Gläser~\& Laudel (2010) ausgewertet. Laut ihnen (2010, 200 ff.) ist das Ziel dieses Verfahrens eine vom Ursprungstext neue Informationsbasis zu schaffen, die nur noch die Aussagen enthält, die für die Beantwortung der Forschungsfrage relevant sind. Der Kern dieses Verfahrens ist die Extraktion, die Reduktion des Datenmaterials durch die Fokussierung auf die Forschungsfrage.  

In der Durchführung bedeutet Extraktion konkret, den Textabsatz zu lesen, diesen zu interpretieren und auf Grundlage der Interpretation zu entscheiden, welche der in ihm enthaltenen Informationen für die Beantwortung der Forschungsfrage relevant sind. Diese Informationen werden in zusammengefasster Form unter den entsprechenden Kategorien und Dimensionen eingetragen, welche ausgehend von den theoretischen Vorüberlegungen a priori entwickelt wurden. Das heißt, die Vorüberlegungen, welche Einflussfaktoren die Umsetzung von Inklusion im Kindergarten begünstigen oder auch behindern -- Gedanken, ohne die der Leitfragebogen nicht hätte erstellt werden können -- leiten die Extraktion an. Sie ergeben das Suchraster, welches auf den Text gelegt wird. Das Kategoriensystem hat durch die Vorüberlegungen einerseits erste Festlegungen erfahren, wird aber gleichzeitig aus dem Textmaterial heraus neu geformt. Das heißt, sollten im Text neue Informationen auftauchen, die zuvor nicht bedacht wurden, können die Dimensionen oder Kategorien jederzeit ergänzt werden. Dass das Kategoriensystem während der Extraktion an die Besonderheiten des Materials angepasst wird, erklärt das Prinzip der Offenheit dieses Vorgehens. 

Information können nicht immer eindeutig einer Auswertungskategorie zugeordnet werden. So konnte es in einigen Fällen nicht vermieden werden, dass in einem Absatz enthaltene Informationen mehrfach unter verschiedenen Kategorien abgelegt worden. Bei besonders uneindeutigen Einschätzungen wurden so genannte Extraktionsregeln formuliert. Diese machen transparent, wie ein Abgrenzungsproblem gelöst wurde und stellen sicher, dass während der gesamten Extraktion einheitlich und regelgeleitet und somit für den Leser nachvollziehbar und nicht willkürlich vorgegangen wurde. Diesen Anspruch erhebt die Qualitative Inhaltsanalyse als ein regelgeleitetes Vorgehen. 

Zudem wurden in der Auswertung so genannte Kausalketten berücksichtigt. Diese stellen Verbindungen zwischen den Kategorien her, welche für die Beantwortung der Forschungsfrage von Bedeutung sind, da von einer großen Verwobenheit der Aspekte auszugehen ist. Wenn in einem Absatz Kausalketten berichtet werden, erschwert das die Zuordnung erheblich. Deshalb ist die Tabelle um die Spalten „Ursache“ und „Wirkung“ ergänzt. Dort werden Querverweise zu anderen Kategorien hergestellt. Ein Beispiel hierfür: Individuelle Unterstützung der Kinder mit erhöhtem unterstützungsbedarf hängt von entsprechenden strukturellen Rahmenbedingungen ab, wie durch die Aussage der Interviewpartnerin deutlich wird (Quelle): „Wenn ein solches Kind im Morgenkreis auf einer Matte neben dran liegt, dann tut mir das in der Seele weh. Aber ich kann es auch nicht immer in den Arm nehmen, weil recht und links von mir Kinder sitzen, die mich auch brauchen [...]“ (\ref{C3_64}). 

\emph{F: „Was wäre in diesem Beispiel eine Lösung? Mehr Fachkräfte?“}

A: „Wenn vier oder fünf Erwachsene im Raum sind, werden die Kinder auch ganz kirre. Sie brauchen ein ruhiges Umfeld mit nicht so viel Gewusel, da sie zu hause schon so viel Unruhe haben. Deshalb sehe ich die Lösung in kleineren Gruppen“ (\ref{C3_65}).


     
 
