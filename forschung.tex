\part{Die empirische Studie: Unter Berücksichtigung welcher Aspekte ist Inklusion im Kindergarten umsetzbar?}
\chapter{Die qualitative Untersuchung}
\section{Forschungsfrage und Begründung des qualitativen Designs}
Das Erkenntnisinteresse der vorliegenden Arbeit besteht darin, Aspekte und Zusammenhänge aufzudecken, die Inklusion im Kindergartenalltag gelingen lassen beziehungsweise erschweren, sodass notwendige Bedingungen für die Umsetzung von Inklusion gesammelt werden können. Der Gewinn im Identifizieren solcher Mechanismen ist vielfältig. Erstens ist denkbar, dass die befragten Leitungen durch eine Bestandsaufnahme ihre inklusiven Prozesse kritisch hinterfragen und verbessern. Zweitens können weitere Kindergärten aus den empirischen Ergebnissen Schlüsse ziehen, die ihnen Orientierung auf dem Weg zur inklusiven Einrichtung geben. Drittens geben die Ergebnisse darüber Auskunft, was konkret von Seiten der Verantwortungsträger erwartet und gebraucht wird, um das Inklusionskonzept erfolgreich im Kindergarten etablieren zu können.

Die Forschungsfrage, \emph{Unter Berücksichtigung welcher Aspekte ist Inklusion im Kindergarten umsetzbar?}, zielt auf das Aufdecken von sozialen Mechanismen innerhalb der Strukturen des Kindergartens ab, zum Beispiel welche vorhandenen strukturellen Voraussetzungen Inklusion fördern. 
Die Forschungsfrage erfordert folglich eine mechanismusorientierte Strategie, die auf der Rekonstruktion von Fällen basiert, keine  standardisierte, die mittels der Aussagen von vielen individuellen Akteuren erhobenen wird. Als geeignete Vorgehensweise für die Beantwortung der Forschungsfrage wird das  
qualitative Forschungsdesign ausgewählt, da selbiges nicht nur der Rekonstruktion von Fällen oder Prozessen dient, sondern zudem auch immer dort empfohlen wird, „wo es um die Erschließung eines bislang wenig erforschten Wirklichkeitsbereichs“ geht (Flick, von Kardoff und Steinke 2000, 25) -- Inklusion im Untersuchungsfeld Kindergarten stellt einen solchen Wirklichkeitsbereich dar -- und wo es hinsichtlich der Theorieorientierung auf Entdeckung abzielt.
Laut Flick, von Kardoff und Steinke (2000, 45) werden aus den unmittelbar gesammelten Daten Theorien und Hypothesen entwickelt, die  an der Realität scheitern können und deshalb in einem weiteren Schritt überprüft werden müssen, um bestätigt werden zu können. Das Forschungsanliegen der Überprüfung kann nicht Inhalt dieser Arbeit sein, da dieses den Rahmen sprengen würde.  

Um solcherart komplexe Wissensbestände zu rekonstruieren empfehlen Meuser und Nagel (1997, 481) das Experteninterview als eine geeignete Methode. Dieser Empfehlung folgend fiel die Wahl der Methode zur Datengewinnung auf das Experteninterview. 


\section{Die Erhebungsmethode: Experteninterview}
Im Folgenden wird zunächst der Expertenbegriff diskutiert, um darzulegen, wer für das Vorhaben der vorliegenden Arbeit überhaupt als Experte angesprochen werden konnte.
Daran anschließend wird die Auswahl der befragten Experte aufgezeigt. 

\subsection{Der Expertenbegriff}
Gläser und Laudel (2010, 13) verweisen darauf, dass der Begriff ’Experteninterview’ in der sozialwissenschaftlichen Literatur in der Regel an die Expertenrolle des Interviewten im untersuchten sozialen Feld, das heißt, an seine gehobene berufliche Position, gebunden ist. Entgegen dieser weit verbreiteten Meinung findet sich bei Gläser und Laudel (2010, 11) der bereichernde Gedanke, dass Experten nicht zwingend in einer gehobenen Position zu sein haben, um über ein besonderes Expertenwissen zu verfügen. Jeder Mensch, der einen Erfahrungsbereich für sich erschließt, wird zum Experten. So wird der Musiker, der einen bestimmten Musikstil aufgreift und alles darüber in Erfahrung bringt oder der von einer seltenen Krankheit Betroffene zum Experten für diesen Musikstil oder jene Krankheit. Die Autoren verweisen darauf, dass schließlich jeder Mensch über besonderes Wissen verfügt, nämlich das Wissen über die sozialen Kontexte, in denen er agiert und an denen er unmittelbar Anteil hat. Ein Experte ist im Sinne von Gläser und Laudel (2010, 12) als „Quelle von Spezialwissen über die zu erforschenden sozialen Sachverhalte“ zu verstehen. „Experteninterviews [wiederum] sind eine Methode dieses Wissen zu erschließen.“ Ausgehend von diesem Expertenbegriff kommen für das Untersuchungsfeld Kindergarten sowohl die Leitung der Einrichtung als auch die Erzieherinnen als zu befragende Experten in Betracht. Beide verfügen über besonderes Wissen im Hinblick auf die Organisation, in der sie arbeiten und die eigenen Arbeitsprozesse.
 
\subsection{Die befragten Experten}
Die wenigsten Kindergärten weisen sich als inklusiv aus, das heißt, der Name der Einrichtung war bei der Expertensuche nicht leitend. Ein Ausschlusskriterium für die Auswahl der Experten war, dass die Einrichtung sowohl von Kindern mit besonderen Bedürfnissen als auch von Kindern ohne besondere Bedürfnisse besucht wird, sodass die Gruppenmischung der gesellschaftlichen Vielfalt entspricht und somit als inklusiv anzusehen ist. Um fündig zu werden, waren sozial schwierige Einzugsgebiete mit einem hohen Anteil an Familien mit einem Migrationshintergrund im Blickfeld, da die Kindergärten zumeist auf die soziale und kulturelle Vielfalt des Einzugsgebietes und somit auf die unterschiedlichen Bedarfslagen der Familien vor Ort Antworten zu finden versuchen und solche Antworten inklusive Prozesse anregen. Da bei diesen Einrichtungen der Anteil der sozial benachteiligten Familien hoch ist, kann nur bedingt von gesellschaftlicher Vielfalt gesprochen werden, da berücksichtigt werden muss, dass die sozial besser gestellten Familien in ’sozialen Brennpunkten’ als Minderheit vertreten sein können.    
Weiterhin wurde bei der Auswahl der Einrichtungen darauf Wert gelegt eine gewisse Trägerbreite abzubilden, da die inklusive Ausrichtung der Einrichtung erheblich von der Unterstützung und Steuerung, die sie durch den Träger erfährt, abhängt (vgl. Kapitel~\ref{sec:kitaSelbst} und~\ref{Strukturelle Rahmenbedingungen}), was nicht automatisch heißt, dass dies der Wahrnehmung der Träger entspricht. Es konnten in der Stadt Freiburg im Breisgau\footnote{Da einige Angebote zur Förderung der Chancengleichheit von der Stadt Freiburg oder von der AWO-Freiburg initiiert wurden, würde das Anonymisieren der Stadt einen großen Informationsverlust bedeuten.} drei Einrichtungen gefunden werden, deren Träger erstens die Diakonie, zweitens die Arbeiterwohlfahrt (AWO) und drittens die katholische Kirchengemeinde sind. Der Erstkontakt per Telefon erfolgte stets mit der Kindergartenleitung, die sich in allen drei Fällen als Interviewpartner zur Verfügung stellte. 

\section{Datenerhebung}
\subsection{Methode der Datenerhebung: Interviewleitfaden}
Um sich ein Bild vom Arbeitsfeld der Kindergartenleitung zu machen, sah die Gliederung des Interviewleitfadens vor, dass die Expertinnen gefragt wurden, wie sich die strukturellen Rahmenbedingungen der Einrichtung und das Konzept im Zusammenhang mit dem Inklusionsverständnis beschreiben lassen und welche Kinder die Einrichtung zum gegenwärtigen Zeitpunkt der Befragung besuchten. Die ausgewählten Leitfragen orientieren sich an den theoretischen Vorüberlegungen in Kapitel~\ref{sec:Wie} und berücksichtigen den stattgefundenen Inklusionsprozess, die damit verbundenen Herausforderungen an das Team, die Beteiligung der Eltern sowie die persönliche Haltung zu Grenzen der Inklusion und stattgefundenen Veränderungen (vgl. Anhang~\ref{Interviewleitfaden}).

\subsection{Durchführung der Datenerhebung}
Im Rahmen der Bachelorthesis wurden drei Experteninterviews im Zeitraum vom 15. bis 27.10.2012 durchgeführt. 
Die Anfragen erfolgten durch telefonische Kontaktaufnahme. Im Erstkontakt wurde kurz das Forschungsvorhaben erklärt sowie begründet, warum ein Interesse an der jeweiligen Einrichtung beziehungsweise an der Person vorliegt und wodurch die Aufmerksamkeit auf die Person gelenkt wurde, zum Beispiel durch Presseberichte oder die Empfehlung des Gutachters. 
Des weiteren wurde angeboten auf Wunsch den Interviewleitfaden zuzusenden, sodass die Fragen von der Leitung vorab eingesehen werden konnten und eine Vorbereitung auf den umfangreichen Leitfaden stattfinden konnte. Davon machten zwei von drei Expertinnen Gebrauch. Das hatte zur Folge, dass die Expertinnen unterschiedliche Voraussetzungen mitbrachten. Einerseits gab es die sehr gut vorbereitete Expertin, die Stichpunkte zu jeder Frage formuliert und die Fragen so verinnerlicht hatte, dass sie bei deren Beantwortung berücksichtigte, welche Antworten an welcher Stelle ihren Platz finden sollten. Andererseits gab es die Expertin, die unvorbereitet angetroffen, mit der Beantwortung einer Frage viele weitere Bezüge zu anderen Fragen herstellte. Die Durchführung der Interviews zeigte, dass die Auseinandersetzung mit den Fragen vorab Auswirkungen darauf hatte, wie reflektiert, aber auch spontan und damit verbunden emotional beteiligt geantwortet wurde.     
Die Vergleichbarkeit der Interviews wird laut Meuser und Nagel (2005, 81) durch die Nutzung des Leitfadens in der Interviewführung und den „gemeinsam geteilten institutionell-organisatorischen Kontext“ sicher gestellt, weshalb die erwähnten unterschiedlichen Ausgangsbedingungen keinen Einfluss auf die Vergleichbarkeit haben.

Der Ort der Befragung war der Arbeitsplatz der Expertinnen, konkret deren Büro. Die Befragungsdauer variierte zwischen 50 Minuten und zweieinhalb Stunden, wobei ohne Ausnahme alle Fragenkomplexe angesprochen wurden. Die unterschiedliche Interviewdauer ergab sich aus dem Verlauf des jeweiligen Interviews und der zusätzlichen Themen, die von Seiten der Experten angesprochen wurden. In dem zeitlich intensivsten Interview wurde die Aufzeichnung des Interviews verweigert, jedoch darauf hingewiesen, dass Bereitschaft besteht, ausreichend Zeit für das Anfertigen entsprechender Notizen zur Verfügung zu stellen, sodass das Gespräch anschließend in einem Gedächtnisprotokoll rekonstruiert wurde. Nach Anfertigung des Gedächtnisprotokolls wurde dieses der Expertin zugesandt, sodass für sie die Möglichkeit bestand Veränderungen vorzunehmen. Auch den anderen Experten wurden zur Überprüfung die vollständig transkribierten Interviews zugesandt. 
Die Experteninterviews wurden mit Hilfe des Aufnahmegeräts \emph{Zoom H2} technisch problemlos aufgezeichnet und anschließend transkribiert. Parallel zu den Tonaufnahmen wurden Notizen zu den Interviews angefertigt.  

Insgesamt stieß die Befragung bei allen Interviewten auf Neugierde und Interesse für das Thema.

\section{Das Auswertungsverfahren: Qualitative Inhaltsanalyse}

\paragraph{Transkription} Nach Abschluss der Interviewführung erfolgte als Vorbereitung der Qualitativen Inhaltsanalyse die Transkription entsprechend der Vorgaben von Meuser und Nagel (2005, 83). Das bedeutet, die Interviews wurden unter der Wahrung der Anonymität -- inhaltlich vollständig -- verschriftlicht und geglättet und somit von Zwischenlauten, Dialektfärbungen, überflüssigen Redundanzen und Floskeln befreit. Diese Methode fand Anwendung, weil sich die „Auswertung von Experteninterviews an thematischen Einheiten, an inhaltlich zusammengehörigen, über die Texte verstreute Passagen [und] nicht an der Sequenzialität von Äußerungen je Interview“ (Meuser und Nagel 2005, 81) orientiert und somit auf vergleichbare Datengewinnung ausgerichtet ist. 
Ergänzend wurden die Empfehlungen von Mayring (2010, 55) in zwei Punkten berücksichtigt. Er schlägt vor, unverständliche Passagen sowie abgebrochene Sätzen mit Hilfe von drei Punkten (...) und Stockungen, Unterbrechungen oder Gedankensprünge durch einen Gedankenstrich (--) zu kennzeichnen. 
Für das Formatieren wurde außerdem der Hinweis von Gläser und Laudel (2010, 194) berücksichtigt, inhaltlich zusammengehörende Aussagen -- Sinneinheiten -- durch Absätze erkenntlich zu machen. Das heißt, dass eine Antwort des Interviewpartners, vorausgesetzt ein neuer Gedanke kommt hinzu, mehrere Absätze umfassen kann. Eine bessere Lesbarkeit wurde außerdem durch das Kursivsetzen der Interviewfragen erreicht (vgl. Anhang~\ref{Transkripte}). 

\paragraph{Extraktion} Anschließend wurden die Transkripte mit Hilfe der Qualitativen Inhaltsanalyse entsprechend der Vorgaben von Gläser und Laudel (2010) ausgewertet. Als Ziel dieses Verfahrens wird von ihnen (2010, 200 ff.) benannt eine neue Informationsbasis zu schaffen, die nur noch die Aussagen enthält, die für die Beantwortung der Forschungsfrage relevant sind. Der Kern dieses Verfahrens ist die Extraktion, die Reduktion des Datenmaterials durch die Fokussierung auf die Forschungsfrage. In der Umsetzung bedeutet das konkret, den Textabsatz zu lesen, diesen auf der Grundlage der theoretischen Vorüberlegungen zu interpretieren und zu entscheiden, welche der in ihm enthaltenen Informationen für die Beantwortung der Forschungsfrage relevant sind. Diese Informationen werden in zusammengefasster Form unter entsprechenden Kategorien und Dimensionen abgelegt, welche durch die theoretischen Vorüberlegungen entwickelt wurden. Das heißt, die Vorüberlegungen, welche Einflussfaktoren die Umsetzung von Inklusion im Kindergarten begünstigen oder auch behindern -- Gedanken, ohne die der Interviewleitfaden nicht hätte erstellt werden können -- leiten die Extraktion an. Sie ergeben das Suchraster, welches auf den Text gelegt wird. Das Kategoriensystem hat durch die Vorüberlegungen einerseits erste Festlegungen erfahren, wird aber gleichzeitig aus dem Textmaterial heraus neu geformt, was das Prinzip der Offenheit bewahrt. Das Kategoriensystem wird während der Extraktion fortwährend an die Besonderheiten des Materials angepasst. Laut Gläser und Laudel (2010, 262 f.) ermöglicht die Offenheit des Auswertungsverfahrens, dass die empirischen Phänomene nicht unter theoretischen Annahmen subsumiert werden, sondern Widersprüche oder über die Theorie hinausgehende Inhalte berücksichtigt werden.

\paragraph{Kategoriensystem}
Das ’Kategoriensystem’ als Auswertungsraster beruht laut ihnen (2010, 206) auf einer Zusammenstellung der theoretisch angestellten Einflussfaktoren und Kausalmechanismen, die Inklusion bedingen und somit die Forschungsfrage repräsentieren. Um das Auswertungsraster zu konstruieren, wurden die fünf zentralen Aspekte aus Kapitel~\ref{sec:Wie} als sogenannte Auswertungskategorien übernommen, die Merkmalsausprägungen dieser zentralen Aspekte bilden im Kategoriensystem die sogenannten Dimensionen. 

Die Extraktionstabelle zur Auswertungskategorie 'Strukturelle Rahmenbedingungen' (vgl. Anhang~\ref{Kategoriensys}) soll als Illustration dienen. 
Die Definition im Theorieteil in Kapitel~\ref{Strukturelle Rahmenbedingungen} enthält die Dimensionen, die für das Kategoriensystem vorab festgelegt werden. 

\begin{quote}
Laut Tietze (2004, 407) werden unter [strukturelle Rahmenbedingungen] Aspekte zusammengefasst, die situationsunabhängig, zeitlich stabil und politisch regulierbar sind. Solche strukturellen Rahmenbedingungen können wiederum unterschieden werden in einerseits der Einrichtung zur Verfügung stehende materielle Ressourcen und andererseits personelle Ressourcen. Unter materiellen Ressourcen werden die vorhandenen Räume und deren Ausstattung, die Finanzierung des Personals, der Betreuungsschlüssel, die Gruppengröße und die Gruppen- und Personalstruktur sowie die vorgesehenen Zeiten für mittelbare pädagogische Arbeit zusammengefasst, unter personelle Ressourcen zählt die Qualifikation der Fachkräfte, festgelegt durch Curriculum und Standards in der Erzieherinnen-Ausbildung (vgl. Kapitel~\ref{subsec:Strukturquali}). 

Jerg (2011, 54) sieht die systemerneuernden strukturellen Voraussetzungen vor allem in Veränderungen hinsichtlich des Finanzierungsmodells, der Angebots- und Infrastruktur, verlässlicher Unterstützungsleistung sowie Durchlässigkeit und Demokratisierung der Strukturen. Zudem spielt der Träger bei der Umsetzung von Inklusion eine wesentliche Rolle. Im besten Falle löst er den politischen Willen zur Inklusion ein und schafft die dafür notwendigen Strukturen (vgl. Kapitel~\ref{Strukturelle Rahmenbedingungen}). 
\end{quote}

Ausgehend von diesen theoretischen Vorüberlegungen werden nach Anpassung an die Expertenaussagen folgende Dimensionen in das Kategoriensystem der Auswertungstabelle integriert:  
Räume und Ausstattung, Finanzierung, der Einfluss des Trägers, Personalstruktur, Qualifikation der Fachkräfte, Fachkraft-Kind-Relation, notwendige Voraussetzungen von regierungsamtlicher Seite, Blick auf die aktuelle politische Situation.
 
\paragraph{Extraktionsregeln und Kausalketten} Information können nicht immer eindeutig einer Auswertungskategorie zugeordnet werden. So konnte es in einigen Fällen nicht vermieden werden, dass in einem Absatz enthaltene Informationen mehrfach unter verschiedenen Kategorien abgelegt wurden. Um das Vorgehen bei Abgrenzungsproblemen transparent zu machen, wurden Extraktionsregeln formuliert. Diese stellen sicher, dass während der gesamten Extraktion einheitlich und somit für den Leser nachvollziehbar und nicht willkürlich vorgegangen wurde. 

Die für dir Auswertung festgelegten Extraktionsregeln sind:
\begin{enumerate}
\item Querverweise zu anderen Kategorien werden im Textfeld 'Ursachen' und 'Wirkungen' in eckige Klammern [] geschrieben.
\item Aussagen bezüglich politisch festgelegter Standards innerhalb der Erzieherinnen-Ausbildung werden unter der Kategorie 1 'Strukturelle Rahmenbedingungen' abgelegt. Sobald persönliche, nicht politisch regulierbare Aspekte wie die Haltung der Fachkräfte oder individuelle Kompetenzen erwähnt werden, wird der Fakt unter der Kategorie 2 'Die Rolle der pädagogischen Fachkraft' unter die entsprechende Dimension abgelegt.
\item Aussagen über die Gruppenzusammensetzung als eine eigentlich strukturelle Rahmenbedingung wird unter 'Bedarfslagen der Kinder' in der Auswertungskategorie 3 'Antworten auf die unterschiedlichen Bedarfslagen der Kinder' abgelegt.
\item Angebote, die beschrieben werden und die Beteiligung der Eltern beinhalten, werden unter Auswertungskategorie 3 'Antworten auf Bedarfslagen' abgelegt.
\item Die Bedarfe der Eltern werden von denen der Kinder separiert und  der Dimension 'Bedarfslagen der Eltern' in der Auswertungskategorie 4
'Partnerschaft mit den Eltern' zugeordnet.   
\end{enumerate}

Wie bereits erwähnt wurden in der Auswertung sogenannte Kausalketten berücksichtigt. Diese stellen Verbindungen zwischen den Kategorien her, welche für die Beantwortung der Forschungsfrage von Bedeutung sind, da von einer großen Verwobenheit der Aspekte auszugehen ist. Wenn in einem Absatz Kausalketten berichtet werden, erschwert das die Zuordnung erheblich. Deshalb ist das Kategoriensystem um die Spalten 'Ursache' und 'Wirkung' ergänzt. Dort werden Querverweise zu anderen Kategorien hergestellt oder auch innerhalb derselben Kategorie die Ursachen und Wirkungen vermerkt, die von dem Experten erwähnt wurden.

\paragraph{Aufbereitung der Daten}
Die Situationen in den Einrichtungen sind heterogen. Sie unterscheiden sich durch die unterschiedlichen Bedingungen, konkret durch die gegebenen Möglichkeiten und Anforderungen, mit denen der Kindergarten im jeweiligen Stadtteil konfrontiert ist. Zudem handelt es sich bei den Forschungsdaten um die individuelle Perspektive der Leitung, die ihre Ansicht von den wirkenden Bedingungen und Kausalmechanismen erklärt. Ihre Interpretation der Beobachtungen wird von ihrer individuellen Persönlichkeit, ihrer Haltung und ihrem Fokus beeinflusst. Die übergeordneten Werte und Ziele der Einrichtung, die einerseits durch die Träger, andererseits durch das Team unter der Führung der Leitung festgelegt werden, prägen wiederum die Entwicklung der Einrichtung, die beschritten wurde. All diese erwähnten Merkmale tragen zur Verschiedenheit der Kindergärten bei.   
Die Heterogenität der Kindergärten zeigte sich auch bei der Anpassung des Kategoriensystems an das Datenmaterial. Abhängig davon, welchen Aspekten die Leitung besondere Bedeutung zuschreibt und welche Schwerpunkte sie in ihrer Arbeit setzt, wurden die Dimensionen entsprechend unterschiedlich gewichtet. 
Schlussfolgernd kann festgehalten werden, dass nicht davon auszugehen ist, dass bei dem Daten\-material viele Gemeinsamkeiten gefunden
werden können. Ausgehend von dieser Erkenntnis wurde entschieden, den gemeinsam geteilten Wissensstand zwar, wenn vorhanden, zu berücksichtigten, bei der Auswertung jedoch den Einzelfall nachzuzeichnen, weshalb die Ansichten der Interviewpartnerinnen im Hinblick auf fördernde, hemmende, notwendige oder auch verhindernde Bedingungen für Inklusion gesammelt und in Bezug auf die jeweilige Einrichtung zusammengefasst dargestellt wurden. Anschließend wurden die Einzelfälle einer vergleichenden Analyse unterzogen, in der
Gemeinsamkeiten und Unterschiede zwischen den Fällen erklärt wurden.
Dieses an die Forschungsdaten angepasstes Vorgehen entspricht der Empfehlung von Gläser und Laudel (2009, 247), nämlich bei einer geringen Anzahl von Fällen zunächst den Kausalmechanismus jedes Falls zu identifizieren und anschließend diese Mechanismen, die gewirkt haben, vergleichend zu analysieren.  
Für unsere Forschungsfrage ist zum Beispiel von Relevanz herauszufinden, welche spezifischen Ausgangsbedingungen – Ursachen – dazu geführt haben, dass das Team entschieden hat, dass ein Kind mit besonderem Förderbedarf für die Einrichtung nicht länger tragbar sei – Wirkung – und was die Einrichtungen im Umgang mit solchen ’Grenzfällen’ voneinander unterscheidet.

\section{Forschungsergebnisse}
Die folgende Darstellung der Ergebnisse ist der Übersichtlichkeit halber in fünf Auswertungskategorien gegliedert.
 
\subsection{Auswertungskategorie 1: Gegebene strukturelle Voraussetzungen in den jeweiligen Einrichtungen}

\paragraph{Räume und Ausstattung}
Fördernde Bedingungen für die Umsetzung von Inklusion werden im Vorhandensein zusätzlicher Räume gesehen. Konkret wurden das Einrichten eines zusätzlichen Raums für Sprachförderangebote sowie ein Raum für die heilpädagogische Förderung als hilfreich empfunden. Auch wenn die Integrationshilfe, durch Heilpädagogen umgesetzt, das Ziel verfolgt, die Kinder mit besonderen Bedürfnissen in gemeinsame Gruppenprozesse einzubeziehen, wurde die Erfahrung gemacht, dass ein Schonraum als Rückzugmöglichkeit eine Bereicherung für das Kind darstellt, da sich ihm bei Bedarf im Eins-zu-Eins-Kontakt genähert, Nöte besser aufgefangen werden konnten und die Kinder den intensiven Kontakt zu einer Bezugsperson genießen würden. Ausreichend verfügbarer Raum wurde zudem als Antwort auf einen hohen Bewegungsdrang gesehen. Als Wunsch wurde formuliert, einen zusätzlichen Raum für Elterngespräche zur Verfügung zu haben, um Störungen wie das Klingeln des Telefons im Büro zu vermeiden. 
Für die Umsetzung von Inklusion wirkt demnach hemmend, wenn die baulichen Voraussetzungen nur eine begrenzte Anzahl an Räumen zuließen oder wenn der Umbau aus Kostengründen bisher nicht bewilligt wurde. 

Eine notwendige Voraussetzung für das Gelingen von Inklusion wurde im  barrierefreien Bewegen des Rollstuhls gesehen. Bauliche Voraussetzungen wie enge Flure oder Türen verhindern, dass Kinder, die auf einen Rollstuhl angewiesen sind, aufgenommen werden können. Als hemmende Bedingungen speziell für diese Kinder wurden außerdem die Erreichbarkeit des Turnraums nur über Treppen, das Fehlen einer Behindertentoilette und das Vorhandensein von Türschwellen benannt, letztere wurden durch das Einfügen eines Holzkeils zu kompensieren versucht, der jedoch beim Schließen der Gruppenraumtür wieder entfernt werden musste. 
Das Einrichten von Kochmöglichkeiten, um den Wunsch der Eltern nach einem Mittagessen für die Kinder in der Einrichtung zu beantworten; im Innenbereich Möglichkeiten zu schaffen, dass Kinder mit Wasser spielen können oder die Anschaffung von Materialien, die zur Förderung der Körperwahrnehmung eingesetzt werden können, wie zum Beispiel eine Therapieschaukel, wurden als Wünsche benannt, deren Umsetzung oder Anschaffung in Ermanglung finanzieller Ressourcen noch nicht erfolgte. 

Übereinstimmend wurde darauf hingewiesen, dass Inklusion danach verlangt, dass bestehender Baubestand entsprechend der Bedarfe verändert wird und dass finanzielle Mittel ausreichend zur Verfügung gestellt werden. 
 
\paragraph{Finanzierung}
Die Ergebnisse zeigen, dass Gelder nicht nur für die notwendigen räumlichen Veränderungen fehlen, sondern auch für die Dotierung von Leistungen, wie zum Beispiel die kontinuierliche Qualitätsentwicklung.  Auch dem Bedarf einer stellvertretenden Leitung aufseiten des diakonischen Kindergartens ist noch nicht entsprochen worden. 
Durch Spendeneinnahmen konnten fehlende finanzielle Mittel kompensiert und Antworten auf Bedarfslagen gefunden werden. Ausschließlich über Spenden konnten so der Ausbau des oberen Stockwerks, neue Projekte im Bereich Naturpädagogik, die Eltern-Kind-Gruppen und Eltern-Kind-Ausflüge beinhalten, Bildungsausflüge wie Theater- oder Museumsbesuche und Bildungsangebote wie musikalische Früherziehung, der Tanz- oder der Schwimmkurs finanziert werden. Nicht zuletzt konnte durch Spenden die Teilnahme am Mittagessen für Kinder, deren Familien den Elternbeitrag von einen Euro am Tag nicht leisten können, unterstützt werden.
Je etablierter Spenden aus dem Haus heraus oder durch die Unterstützung des Trägers akquiriert wurden, desto mehr Bedarfe konnten gedeckt werden. Die Kindergartenleitung der Diakonie verweist auf die Bedeutung des eigenen Fördervereins: „Alle Projekte, die wir haben, bezahlt der Förderverein. Sonst könnten wir diese Arbeit nicht machen.“ (C1,\ref{C1_37})

\paragraph{Einfluss des Trägers}
Die Aufgabe des Trägers als finanzieller 'Unterstützer' wurde durch die Kinderarmutskampagne 2012 der AWO-Freiburg: „Wenn ich groß bin, werde ich arm!“ transparent, wodurch dem Kindergarten für das Jahr 2012 Spendeneinnahmen in Höhe von 10000 Euro zur Verfügung gestellt wurden, sodass Familien, die sich das Mittagessen für ihre Kinder in der Einrichtung nicht leisten konnten, über einen Gutschein unterstützt wurden. Weitere Spenden wurden für die zuvor erwähnten Bildungsangebote und -ausflüge bereitgestellt. Durch den Austausch zwischen der Leitungs- und Trägerebene und der konkreten Nachfrage seitens der AWO: „Was braucht ihr jetzt vor Ort wieder, was für Bildungsausflüge wären wichtig?“ (C2,~\ref{C2_601}), wurde das Bedürfnis nach Unterstützung bei der Leitung erfüllt. Zudem gab sie zu verstehen, dass sie gern bei der AWO angestellt sei und Wertschätzung durch den Träger erfahren, was sie an das Team weitergeben würde.

Als fördernde Bedingungen für die Steuerung der pädagogischen Arbeit wurde die Möglichkeit benannt autonom Entscheidungen treffen zu können. 
Der zur Verfügung gestellte 'Spielraum' am Beispiel der Diakonie bemisst sich daran, dass die Fortbildungsthemen leitungsintern entschieden werden können: „Wir können uns den Themen widmen, die wirklich dran sind. Da steht dieser Träger hinter seinen Einrichtungen, weil die sagen, wir sind hier im Brennpunkt und wenn unsere Leute hier mit diesen Themen kommen, dann sind die dran.“ (C1,~\ref{C1_36}) Der Träger regelt lediglich den Finanzrahmen für die Fortbildungen. Unter solchen Voraussetzung konnte das Team bisher am Bedarf orientiert entscheiden, welchen Themen sie sich widmen wollen, was von Seiten der Leitung als Voraussetzung für konzeptionelle Weiterentwicklung und hohe fachliche Kompetenz gesehen wird. 
Die Offenheit des Trägers wurde auch am Beispiel der katholischen Kirchgemeinde erkennbar. Die Gruppengröße wurde, nachdem sich in einer Gruppe drei Kinder mit diagnostiziertem besonderen Förderbedarf befanden, von den üblichen 22 beziehungsweise 23 Kindern auf 20 reduziert. 

\paragraph{Fachkraft-Kind-Relation}
Die Personalbemessung richtet sich nach der Anzahl der Kinder und den Hauptbetreuungszeiten, das heißt, wenn eine Einrichtung verlängerte Öffnungszeiten bis 18 Uhr anbot, dann war der Personalschlüssel höher als bei Einrichtungen, deren Betreuungszeiten 13:30, 14:30 oder 15 Uhr endeten. Die Personalschlüssel differierten von Kindergarten zu Kindergarten zwischen 3,2 Fachkräften auf 20 Kinder; 2,5 Fachkräften auf 18 beziehungsweise 19 Kinder und 2 Fachkräften auf 22 beziehungsweise 23 Kinder. Ab einer Fachkraft-Kind-Relation von 3,2 Stellen auf 20 Kinder wurde die Situation als gut befunden, alle Relationen unter diesem Wert waren mit dem Bedarf nach 'mehr Personal' verbunden.     

\paragraph{Personalstruktur}
Zudem erfuhr das Personal eine Aufstockung, wenn die Einrichtung von einem hohen Anteil an Kindern mit Migrationshintergrund besucht wurde und diese Kinder wiederum einen erhöhten Förderbedarf hatten oder die Elternarbeit erschwert war, was zum Beispiel Unterstützung der Eltern bei der Antragstellung erforderlich machte. Unter Voraussetzung dieser Bedingungen gewährt die Stadt Freiburg seit 2011 eine sogenannte Migrationsanteilsaufstockung, die im Fall des Kindergartens der AWO zur Freistellung der Leitung um weitere 20\,\% führte, sodass für diese Leitung die Freistellung von der mittelbaren Arbeit am Kind 50\,\% betrug, während die anderen Leitungen 100\,\% freigestellt waren. 
Auf insgesamt 12 Gruppen in den drei Einrichtungen verteilten sich 33 Erzieherinnen und ein Erzieher. Zusätzlich sind Heilpädagogen als Integrationshilfe in den Gruppen. Wenn die Kinder in der Einrichtung kommen, die einen erhöhten Unterstützungsbedarf haben, wird abhängig von der Art der Behinderung nach § 35 a oder § 53 Integrationshilfe und bei entsprechendem Bedarf eine 'begleitende Hilfe' beantragt. Wird dem Kind diese zugesprochen, kommt die Heilpädagogin an zwei Tagen in der Woche jeweils zwei Stunden in die Einrichtung, um das jeweilige Kind innerhalb der Gruppe zu unterstützen. Bei Bewilligung wird die begleitende Hilfe mit 300 Euro monatlich finanziert, weshalb diese in im Kindergarten der AWO durch Vorpraktikanten oder junge Menschen im Freiwilligen Soziale Jahr umgesetzt wird. Unter dem Zeitaspekt kann das Kind auf diese Weise neben den Erzieherinnen eine tatsächliche Begleitung im Alltag erfahren, jedoch mit der Einschränkung versehen, dass die begleitende Hilfe durch qualifiziertes Personal angeleitet werden muss, was wiederum Mehraufwand bedeutet. Die Anleitung wurde in dem konkreten Fall von der Heilpädagogin übernommen. 

Im Bereich der Sprachförderung haben alle Einrichtungen qualifiziertes Personal vor"-zu"-wei"-sen. Der diakonische Kindergarten hat bedingt durch die Entwicklung zum Familiennetzwerk zusätzliche Fachkräfte mit spezifischer Qualifikation im Haus, wodurch der Personalschlüssel aufgewertet wird, konkret 'Bücherwurmfrauen' im Ehrenamt, die in der Bibliothek arbeiten und vorlesen, eine Musikpädagogin, zwei Logopäden, eine Sprachförderkraft, wodurch Logopädie und Sprachförderung als integrierte Hilfen angeboten werden können, Sportler im Ehrenamt, die die 'Ringergruppe' durchführen sowie eine Natur- und Umweltpädagogin für die Eltern-Kind-Gruppen im Bereich Naturpädagogik. Durch die unterschiedlichen Fachkräfte wird der interdisziplinäre Austausch angeregt, da in allen angebotenen Projekten eine Erzieherin dabei ist.  

\paragraph{Blick auf die aktuelle politische Situation sowie Schlussfolgerungen}

Die Schlussfolgerungen beziehen sich auf die Wahrnehmungen der aktuellen politischen Situation und zeigen, welche strukturellen Voraussetzungen für gelingende Inklusion, die von regierungsamtlicher Seite garantiert werden sollen, als notwendig angenommen werden.
 
„Deutschlandweit schätze ich das so ein, dass Kindergärten bereits viel Inklusionsarbeit leisten, das aber in der Schule steil abfällt. Die Kitas sind in ihrer Gesamtstruktur anders aufgestellt, sie sind näher an den Familien, bekommen Inklusion eher hin und wollen es eher hinbekommen.“ (C1,~\ref{C1_24}) Das gemeinsam benannte Ziel Inklusion in der Schule weiterzuführen, zeigt, dass die Bedeutung von Inklusion in ihrer gesellschaftlichen Tragweite erkannt wurde. Unter Weiterführung in der Schule wird zum Beispiel verstanden, dass sich die Schule für neue Strukturen öffnet, Bildungsangebote für Eltern einbezieht und Strategien entwickelt, Familien und deren Kinder zu stärken. Zudem wird darauf hingewiesen Inklusion als Trend kritisch zu begegnen und zu überprüfen, „wo Inklusion drauf steht, ist da auch Inklusion drin?“ (C1,~\ref{C1_27}). 
Da Inklusion im Stadtteil stattfindet, ist eine dezentrale Politik Voraussetzung, das heißt, dass Bildungsmöglichkeiten zur Partizipation aller Menschen, die in diesem Stadtteil wohnen, entwickelt werden müssen. Dafür bedarf es Konzepte, die von 'oben' auferlegt, aber nicht vorgegeben, sondern vom jeweiligen Kindergartenteam ausformuliert werden und Fragen zur Haltung einbeziehen.

Das Amt für Kinder, Jugend und Familie bewilligt den Förderbedarf und die entsprechenden Hilfen, Integrations- und begleitende Hilfen. In der Wahrnehmung der Kindergartenleitung der AWO arbeitet selbiges unter der Vorgabe genau zu überprüfen, ob der Bedarf nicht auch mit dem vorhandenen Personal bewältigt werden kann und somit möglichst ein"-zusparen. Der Umfang von zwei mal zwei Stunden in der Woche, in denen die Heilpädagogin die Integrationshilfe leistet, entspricht dem  bewilligten Höchstmaß. Hierbei überprüft das Amt in regelmäßigen Abständen, ob der Bedarf noch vorhanden ist oder reduziert werden kann. Die begleitende Hilfe, die zusätzlich zur Integrationshilfe kommt, wird nach Feststellung der Leitung ungern finanziert. Die Befunde rufen nach Verbesserung. Es wird nicht nur nach mehr Zeit und Unterstützung durch die Heilpädagogin verlangt, sondern darauf hingewiesen, dass als notwendige Voraussetzung für Inklusion angesehen wird, dass Heilpädagogen in jeder Einrichtung mit einem Stellendeputat vertreten sind. 

Die Situation in den Kindergärten kann in Bezug auf die finanzielle Ausstattung nicht befriedigen. Da vermutet wird, dass die durch Hilfen kompensierte Situation positive Auswirkungen auf das Gesamtsystem Kindergarten und das pädagogische Selbstverständnis sowie die Haltung der Fachkräfte hat, werden folgend die Daten in zusammengefasster Form wiederholt. 
Der katholische Kindergarten zeigt in Bezug auf den Personalschlüssel, die räumliche und finanzielle Ausstattung die niedrigsten Werte für Strukturqualität. Die anderen beiden Einrichtungen zeigen positivere Werte bedingt durch deren innere und äußere Kompensationsmöglichkeiten. Die Einrichtung der Diakonie verfügt über einen eigenen Förderverein und die Strukturen im Familiennetzwerk, wodurch zusätzliche Fachkräfte gewonnen werden können, die vielfältige Projektarbeit in Kleingruppen anbieten, weshalb die Stammgruppen entlastet und der Personalschlüssel aufgewertet wird. In der Einrichtung der AWO, die ohnehin die beste Personalbesetzung aufgrund der langen Öffnungszeiten vorzuweisen hat, erfolgt eine Kompensation durch die umfangreichen Spenden des Trägers zur Finanzierung von Bildungsangeboten. Die Kindergartenleitung des katholischen Kindergartens mit der vergleichsweise niedrigsten Strukturqualität äußert den Wunsch nach gesamtgesellschaftlicher Anerkennung der Rolle der Erzieherin in ihren Aufgaben und in ihrer Bedeutung für das Kind. Unter ihre Aufgaben, die sich angesichts der politischen Ideen zur inklusiven Erziehung und Bildung erweitert haben, zählt sie Entwicklungsberichte schreiben, Gespräche mit Eltern und Kooperationspartnern führen und übersetzen. Sie erlebt den Anspruch, dem Orientierungsplan entsprechen und darüber hinaus den Kindern mit besonderen Bedürfnissen gerecht werden zu müssen, als Druck von außen, der in Anbetracht der Fülle der Aufgaben mit Überforderung, dem Gefühl des Alleingelassen-Seins und dem Bedarf nach Unterstützung verbunden wird. Die kritische Wahrnehmung der Leitung spiegelt sich in dem Satz wieder: „Wir {[die Politiker]} schreiben uns das {[Inklusion]} auf die Fahne, aber nachgedacht haben wir nicht. Ach so, dafür brauchen wir Geld. Aber das haben wir nicht!“ (C3,~\ref{C3_53}) Konkret hinterfragt sie zum Beispiel, wer den Dolmetscher bezahlt, wenn die Eltern kein Deutsch sprechen. Die Regelgruppen dieser Einrichtung arbeiten am Nachmittag aufgrund der geringeren Kinderzahl verstärkt gruppenübergreifend, sodass der Personalschlüssel positiver ausfällt. Hierbei wird die Erfahrung gemacht, dass auf den Bedarf der Kinder besser eingegangen werden kann, sodass Eltern die Empfehlung ausgesprochen wird, das Kind, das nach Einschätzung der Mutter „noch viel lernen muss“ (C3,~\ref{C3_12}), in diese Nachmittagsbetreuung zu geben. Die Gruppensituation wird mitunter als nicht tragbar beschrieben, weshalb Kinder in andere Einrichtungen verwiesen werden.
Um das zu vermeiden, werden kleinere Gruppen als notwendige Bedingung benannt. Den Bedürfnissen des Kind mit ausgeprägten Verhaltensauffälligkeiten im Sinne eines Aufmerksamkeitsdefizitsyndroms hätte nach der Wahrnehmung der Kindergartenleitung bei einer Gruppengröße von acht Kindern und drei Fachkräften entsprochen werden können. 
Die Gruppengröße zu senken wird als wichtigstes Merkmal eingeschätzt, da ein erhöhter Personalschlüssel mit der Erfahrung verbunden wird, dass zu viele Erwachsene im Raum Unruhe bringen und das Bedürfnis der Kinder nach einer ruhigen Atmosphäre gefährden.

Inklusive Erziehungs- und Bildungsarbeit in jeder Einrichtung setzt voraus, dass die Rahmenbedingungen von vornherein verändert werden, sodass die Gruppensituation bewältigt werden kann und Kinder nicht in andere Einrichtungen empfohlen werden müssen. In allen benannten Fällen wird als Motiv Kinder 'wegzuschicken' Hilflosigkeit beziehungsweise Überforderung benannt, dem Kind mit besonderen Bedürfnissen nicht gerecht werden zu können. In der Einrichtung der Diakonie wird der Fall, dass ein Kind in eine andere Einrichtung verwiesen wird, nicht beschrieben, ebenso wenig wie Überforderung. 

Übereinstimmend wird als zentrale Voraussetzung für die Umsetzung von Inklusion die Gruppengröße und entsprechend die positivere Fachkraft-Kind-Relation hervorgehoben. 

\subsection{Auswertungskategorie 2: Die Rolle der pädagogischen Fachkraft}

\paragraph{Erforderliche Kompetenzen}
Fördernde Bedingungen für Inklusion liegen vor, wenn die Fachkräfte über die Kompetenz verfügen, sich selbst sowie die stattfindenden Prozesse im Alltag zu reflektieren. Das setzt voraus, dass die Erzieherin sich mit ihrer Persönlichkeit und ihrer eigenen Biografie auseinandersetzt, da unverarbeitete familiäre Inhalte auf andere Menschen projiziert werden und auf diese Weise Missverständnisse und schwer wieder aufzulösende Konflikte entstehen können. Die selbstreflexiven Auseinandersetzung mit der eigenen pädagogischen Arbeit wird von der Leitung unter dem Träger der Diakonie als Voraussetzung für die Weiterentwicklung der pädagogischen Qualität verstanden. 
Die Bereitschaft Konflikte zu lösen, in Zusammenarbeit mit den Eltern zu treten und kommunikationsfähig im Team zu sein werden neben den pädagogischen Kenntnissen als notwendige Kompetenzen genannt.

Geeignete Erzieherinnen werden durch Merkmale wie respektvoll, wert\-schätzend, einfühlsam, flexibel im Umgang mit den Kindern und deren Spontanität, 'mit beiden Beinen auf dem Boden stehend' und offen beschrieben.
Im Umgang mit den Eltern wird eine Metapher des Reiters auf dem Pferd und des Menschen, der nebenher läuft, beschrieben:  
„Um den da unten [Mutter oder Vater] zu erreichen, muss ich [die pädagogische Fachkraft] vom Pferd steigen, nicht ihn mit hinaufnehmen wollen.“ (C3,~\ref{C3_72}) 

\paragraph{Zusatzqualifikationen}

Die Forschungsergebnisse zeigen, dass Bildung als lebenslanger Prozess verstanden und somit der Bedarf an Fort- und Weiterbildung als selbstverständlich angesehen wird. Der gesehene Weiterbildungsbedarf ist an den Kindern und ihren Lebenslagen orientiert.
Zwischen der Bereitschaft zur Teilnahme an Fortbildungen und dem zur Verfügung gestellten finanziellen Rahmen wird ein Kausalzusammenhang beschrieben. Zudem wird gesagt, dass je mehr Personal zur Verfügung steht das Wissen desto spezialisierter im Team angelegt werden kann. Entsprechend der Interessen und Ressourcen der einzelnen Teammitglieder werden Überlegungen angestellt, wer sich in welchen Bereichen weiterbilden kann. Zudem wurde die Teilnahme des Gesamtteams an Fortbildungen positiv bewertet, da beim Transport von Informationen in das Team keine Verluste entstehen können. 

\paragraph{Pädagogisches Selbstverständnis} 
Die Einrichtungen heben sich in Bezug auf ihr Inklusions\-verständnis, die übergeordneten Ziele und Werte der Einrichtung, die Haltung, die benannten Leitungsaufgaben und die gesehenen Grenzen besonders deutlich voneinander ab, da das pädagogische Verständnis der Einrichtung noch zusätzlich mit dem individuellen Ansatz der jeweiligen Leitungsperson konfundiert ist. Deshalb werden diese Aspekte folgend im Zusammenhang und fallbezogen unter die Überschrift 'Pädagogisches Selbstverständnis' gesetzt.

Inklusion bedeutet für die Kindergartenleitung des diakonischen Kindergartens, dass die Einrichtung sich in einem Prozess der Veränderung befindet und dabei das von allen geteilte Ziel verfolgt Bildungsangebote zu ermöglichen, mit derer Hilfe die unterschiedlichen Themen der Kinder beantwortet werden können. Die Herausforderung werden dabei in den unterschiedlichen Bedarfen und Ausgangsbedingungen gesehen, die die Kinder mitbringen. 
Das angestrebte übergeordnete Ziel für alle Kinder, egal, aus welcher sozialen Lage sie kommen, ist die Persönlichkeitsbildung in Vorbereitung auf die Schule. Das bedeutet konkret, dass erstens das Sprechen der deutschen Sprache gefördert wird, sodass die Kinder in der Schule gut mitkommen können -- „Thema Sprache ist Nummer-Eins-Thema im Haus!“ (C1,~\ref{C1_20}) -- und zweitens, dass die Kindern im sozial-emotionalen Bereich so gestärkt werden, dass sie sich gegenseitig in ihrem schulischen Lernprozess unterstützen können. Dementsprechend lernen sie ihre Gefühle zu regulieren und zu wissen, was ihnen gut tut. 
Auch der Elternarbeit und -bildung wird von der Leitung zentrale Bedeutung zugeschrieben: „Sie können ja nicht sagen, ihr Kind bilden wir und sie lassen wir gerade mal links liegen. Wer mit den Kindern arbeitet, muss mit den Familien arbeiten.“ (C1,~\ref{C1_16}) Darunter wird verstanden eine Beziehung und Bindung zu den Eltern aufzubauen, die Vertrauen und ein Gemeinschaftsgefühl von 'miteinander unterwegs sein' ermöglicht. Der Wert der Gemeinschaft wird den Eltern bereits im Aufnahmegespräch sowie durch verschiedene Angebote im Kindergartenalltag stetig zu vermitteln versucht. Damit wird das Ziel verbunden, „Kinder und Eltern stark [zu] machen, auch die Mitarbeiter“ (C1,~\ref{C1_69}).
Die Haltung der Fachkräfte wird in Zusammenhang mit der Frage gebracht, ob diese sich den genannten gemeinsamen Zielen verpflichten wollen.
Dabei wird eine Haltung als notwendig erachtet, durch welche offen und ohne Überheblichkeit in Kontakt zueinander getreten werden kann, so dass jeder von der Fachkompetenz des anderen profitiert, da jede Perspektive auf das Kind als Bereicherung für die Arbeit angesehen wird.
Da die Haltung der Fachkräfte als entscheidendes Kriterium für einen gelingenden Inklusionsprozess angenommen wird, richten sich alle Fort- und Weiterbildungen daran aus. Die Persönlichkeitsentwicklung der Erzieherin wird ernst genommen und als Voraussetzung für pädagogische Qualitätsentwicklung angesehen. Die Erzieherin soll befähigt werden mit den komplexen Herausforderungen im Alltag authentisch umzugehen.
Die Aufgaben der Kindergartenleitung werden darin gesehen, die Bildungs- und Entwicklungsprozesse der ihr anvertrauten Fachkräfte und Familien zu fördern und gemeinsam mit ihnen zu ermitteln, was deren Bedarfe sind sowie Hilfestellungen zu geben, sodass Antworten gefunden werden können. Die Kindergartenleitung leitet die Fachkräfte über den Dialog zur Selbstreflexion an. „Wir sind alle im Werden. Wir entwickeln uns alle. [...] es geht nicht um funktionieren hier drin.“ (C1,~\ref{C1_55}) Der Dialog ist 'auf Augenhöhe' und beidseitig, dass heißt, die Leitung ist ebenfalls offen in Austausch zu treten und in Situationen, in denen sie an ihre Grenzen stößt, Unterstützung von ihren Kollegen anzufragen. 
Zudem hat die Leitung eine Beratungsfunktion für die Eltern. Sie steht als Ansprechpartnerin jederzeit zur Verfügung und stabilisiert in Krisen. Dass Eltern sich in der Einrichtung wohlfühlen, ist das Wichtigste, was es zu leisten gilt, denn nur unter dieser Voraussetzung besteht Offenheit seitens der Eltern sich einzubringen und an Bildungsangeboten teilzunehmen.
Grenzen der Inklusion werden bei Kindern gesehen, die so stark durch ein Handicap beeinträchtigt sind, dass sich die Leitung die Inklusion im Alltag massiv erschwert vorstellt. Ein Kind, das einen sehr hohen Pflegeaufwand benötigt, bedarf eines Extraraums und spezieller Pflegekräfte. Da die Bereitstellung solcher Bedingungen fehlen und die Leitung konstatiert, dass das Schaffen selbiger durch die Politik als nicht absehbar eingeschätzt werden muss, steht sie der Zuführung von Kindern mit schweren Handicaps skeptisch gegenüber. Eine Zuführung an sich wird aber generell mit positiver Wirkung für das betroffene Kind verbunden.

Das Inklusionsverständnis der Kindergartenleitung der AWO impliziert den un"-ein"-ge"-schränk"-ten Zugang zu Bildungsmöglichkeiten, was das Schaffen von Erfahrungsräumen voraussetzt, die die Kinder von sich aus nicht hätten. Zudem verweist sie auf das Leitbild des Trägers, das übergeordnete Werte wie Solidarität, Gleichheit und Gerechtigkeit vorgibt und verbindet sich mit dem von der AWO erteilten politischen Auftrag einen Beitrag zur Chancengleichheit zu leisten. In disem Zusammenhang verweist sie auf die Notwendigkeit, sich die Ziele im Team immer wieder bewusst zu machen. Bei der pädagogischen Arbeit werden die Interessen und Stärken in den Blick genommen, sodass die Ressourcenorientierung als ein wesentlicher Wert, der die Arbeit mit den Kindern anleitet, festgehalten werden muss. 
Mit dem Begriff Haltung verbindet die Leitung, dass die Idee der Inklusion vom Team getragen werden muss und dass Inklusion als Wert  beim Umgang miteinander und der Offenheit sich weiter zu bilden ansetzt. In der Wahrnehmung der Kindergartenleitung entsteht Inklusion in einer Atmosphäre gegenseitiger Wertschätzung. Diese herzustellen sieht die Leitung als ihre Hauptaufgabe an, konkret bedeutet das für sie, dafür Sorge zu tragen, dass sich alle wohl und angenommen fühlen sowie ein wertschätzender Umgang und eine gute Zusammenarbeit im Team herrschen. Das Wohlbefinden im Team wird als Voraussetzung für das der Eltern betrachtet und schließlich das der Kinder. 
Grenzen der Inklusion werden bei der Teilgabe gesehen und auch erlebt, nämlich im Bestreben alle Kinder zu integrieren, ohne dass einzelne das Gefühl haben in einer besonderen Situation zu sein.    
Die Leitung beschreibt, dass es trotz der Bemühungen Situationen gibt, in denen Kinder mit besonderen Bedürfnissen erfahren, dass sie im Vergleich zu den anderen Kindern eingeschränkt sind, zum Beispiel beim Waldausflug, wenn das Kind mit motorischer Einschränkung auf das Schieben im Buggy durch eine Betreuerin angewiesen ist oder beim Schwimmkurs. Ideen zur Veränderung dieser Situationen fehlen. Gefahren auszuschließen wie das Umkippen des Kinderwagens wird als Grund benannt, warum das Kind nicht über unwegsame Wege geschoben wird oder ein anderes Kind das Schieben des Buggys übernehmen kann. Als Grenzen werden im Zusammenhang mit der Teilhabe und Teilgabe die eigenen Barrieren im Kopf benannt. Es wird die Situation beschrieben, dass in Teambesprechungen viel Zeit investiert wird, um Überlegungen anzustellen wie das Angebot gestaltet werden kann, um die Ausgrenzung des Kindes zu vermeiden. Die Kinder jedoch werden in der kritischen Situation als unkompliziert wahrgenommen. So erklärt zum Beispiel ein Kind einem anderen Kind, das neu in der Gruppe ist: „Weißt du, der kann jetzt mit der Hand das nicht rüber machen, das musst du jetzt machen“ (C2,~\ref{C2_508}).
Die Frage, ob die Fachkräfte dem individuellen Bedürfnis jeden Kindes gerecht werden können, wird grundsätzlich in Zweifel gezogen. In diesem Zusammenhand wird beschrieben, dass die Bildungsthemen der Kindern aufgegriffen und zusammengefasst werden und sich somit Grenzen für die Kinder ergeben, die sich für Themen interessieren, die nicht von weiteren Kindern geteilt werden. Die Gründe, dem individuellen Bedürfnis einzelner Kindern nicht gerecht zu werden, werden in der Konstellation verschiedener Bedarfe, die zur gleichen Zeit auftreten, gesehen zum Beispiel wenn eine Mutter mit einem Anliegen in der Tür steht. Das Fazit lautet: „Da bräuchte man ja fast pro Kind eine pädagogische Fachkraft!“ (C2,~\ref{C2_54}) 
Als hemmende Bedingung für individuelle Unterstützung wird insbesondere Personalmangel bei Krankheitsausfall erlebt.

Inklusion verstanden als Prozess sowohl für den Kindergarten als auch für die Familien bedeutet für die Leitung des katholischen Kindergartens, dass Kinder nicht ausgegrenzt, sondern respektiert werden, am 'normalen' Leben teilhaben können und ihnen etwas zugetraut wird. Darüber hinaus bedeutet es für sie ein gegenseitiges Lernen zu ermöglichen und den nicht behinderten Kindern zu zeigen, dass es nicht selbstverständlich ist, dass diese ohne Handicap leben und dass die Kinder mit Handicap dafür andere Kompetenzen mitbringen. Als zentrale Aufgabe wird verstanden die Chancen der Kinder im Wohngebiet zu erhöhen und deren Potential wach zu kitzeln.
Die Haltung gegenüber Inklusion kommt in den folgenden Sätzen zum Ausdruck: „Wir sind offen für alle Kinder und haben die Position, wir probieren es aus und wenn wir es nur zwei Wochen probieren, aber wir probieren es. Erst im Tun, im Leben und Erleben sehen wir oft, was das bringt. Manchmal stoßen wir dann auch an unsere Grenzen und haben Angst, diesem Kind nicht gerecht werden zu können.“ (C3,~\ref{C3_28}) Grenzen der Inklusion werden in Ermanglung von Zeit und Zuwendung und abhängig von dem Grad der Behinderung gesehen.  
Die Leitungsaufgaben werden in der Funktion als Ansprechpartnerin und in der zwischenmenschlichen Kommunikation zwischen den Fachkräften, den Eltern und den Kindern sowie anderen Institutionen gesehen. Um fehlende Informationen zu beschaffen, wird Fachpersonal von außen gewonnen.
„Manchmal sagen meine Kollegen zu mir: ‚Du hast doch immer für alles Verständnis und sagst, wir schaffen das.’ Dann ist es besser, wenn jemand von außen kommt und aus der Praxis Beispiele bringt, sodass die Erzieherinnen sich verstanden fühlen und noch einmal eine andere Autorität zu Wort kommt.“ (C3,~\ref{C3_68}) 

\subsection{Auswertungskategorie 3: Antworten auf die individuellen Bedarfslagen}

\paragraph{Gruppenzusammensetzung und Bedarfe der Kinder}
Die bestehende Gruppenzusammensetzung wird von den Kindergartenleitungen durch Attribute wie 'hohe soziale und kulturelle Vielfalt' oder 'gute Mischung' beschrieben. Die Heterogenität lässt sich erstens dadurch definieren, dass sowohl Kinder aus bildungsorientierten als auch Kinder aus bildungsfernen Elternhäusern gemeinsam die Einrichtung besuchen, zweitens der Anteil an Kindern mit Migrationshintergrund zwischen 63 und 73 \,\% liegt und drittens, dass aufgrund der Einzugsgebiete, in denen soziale Problemlagen gehäuft vorkommen und der Kindergarten sich für diese unterschiedlichen Problemlagen öffnet, zum gegenwärtigen Zeitpunkt der Datenerhebung Kinder mit sehr unterschiedlichen Voraussetzungen die jeweiligen Einrichtung besuchten: 
Kinder mit diagnostiziertem Förderbedarf aufgrund einer seelischen oder geistigen Behinderung oder einer drohenden Behinderung, ein Kind, das an Krebs erkrankt war und sich zum gegenwärtigen Zeitpunkt in Remission befand und große Entwicklungsverzögerungen, insbesondere im sprachlichen Bereich zeigte, Kinder mit Verhaltensauffälligkeiten und Problemen im sozial-emotionalen, motorischen oder sprachlichen Bereich, drei Kinder mit diagnostizierter Lernbehinderung, zwei Kinder mit Hörbeeinträchtigung, die Hörgeräte trugen, ein Kind mit einer Sehbehinderung sowie Kinder mit Wahrnehmungsproblematiken, bei denen noch eine diagnostische Abklärung bevorstand.

\paragraph{Angebote}
Angebote werden vom bestehenden Bedarf der Kinder ermittelt.
Der Bedarf der Eltern nach einem gemeinsamen Mittagessen, dem aufgrund der räumlichen Bedingungen noch nicht entsprochen werden konnte, wurde im katholischen Kindergarten versucht durch ein gemeinsames Frühstück mit reichhaltigem Buffet einmal im Monat nachzukommen. Empathie für die Voraussetzungen, die das Kind mit Sehbehinderung mitbringt, konnte in Kooperation mit der Beratungsstelle erreicht werden, die den Kindern Brillen für mehrere Wochen zur Verfügung stellten, sodass die Kinder ohne Sehbeeinträchtigung durch das Tragen spielerisch erfahren konnten, dass ihr Blickwinkel eingeschränkt ist und wie leicht sie mit den Brillen stolpern können. 
Das elterliche Anschaffen eines Rollbrettes für den Sohn mit spastischer Lähmung, auf dieses er geschnallt werden und sich mit den Armen abstoßen konnte, gab den Impuls für alle Kinder Rollbretter anzuschaffen, sodass der Junge gemeinsam mit den anderen Kindern auf dem Hof umher fahren konnte. Die Kindergartenleitung erklärt anhand dieses Beispiels, dass die Kinder am Leben der anderen teilhaben und voneinander lernen konnten, auch die Erzieherinnen von den Ideen der Mutter. 

Eine zentrale pädagogische Aufgabe wird darin gesehen, den Erfahrungsraum für die Kinder zu erweitern, da Angebote wie der Besuch des Theaters oder des Münsterplatzes einigen Kindern aufgrund der häuslichen Situation vorenthalten bleiben, wenn nicht der Kindergarten diesbezüglich eine kompensatorische Aufgabe übernehmen würde. 
Da sich einige Familien Bildungsmaßnahmen dieser Art nicht leisten können, werden solcherart Angebote als wichtiger Beitrag zur Chancengleichheit verstanden.

\paragraph{Konzeptionelle Überlegungen in Bezug auf Inklusion}
Das Konzept des diakonischen Kindergarten ist niederschwellig angelegt. Das heißt, dass der Fokus bereits in der Eingewöhnungsphase auf die 'Stolpersteine' beim Kind gerichtet ist und beim Erkennen eines  Förderbedarfs das Kind sofort dem entsprechenden Projekt zugeführt wird. Diesem Konzept wird eine positive Wirkung zugesprochen, die in der folgenden Expertenaussage deutlich wird: „Dieses niederschwellige Konzept, das wir hier entwickelt haben, das greift wahnsinnig und den Eltern tut das gut, weil die fühlen sich gesehen und getragen. Das ist auch so. Die sind richtig mit uns unterwegs. Wir haben das ziemlich genau im Blick -- das spüren die auch.“ (C1,~\ref{C1_32}) 
Die Konzeptentwicklung basiert auf dem Prinzip der dialogischen Qualitätsentwicklung, das heißt, dass Eltern und Fachkräfte gemeinsam überlegen, welche Angebote gebraucht und installiert werden. Die Qualitätsentwicklung - die Entwicklung der Antworten auf unterschiedliche Bedarfslagen -- die in der sogenannten \emph{Zukunftswerkstatt} stattfindet, sieht vor, dass die Themen des Kindergartens gemeinsam, das heißt, unter Berücksichtigung der Eltern-,  Mitarbeiter-, gegebenenfalls der Träger- und der Gesetzes gebenden Perspektive, erörtert werden. Dabei spielt die Analyse eine bedeutende Rolle, deren Ziel es ist, den Bedarf genau bestimmen zu können.  Anschließend werden alle Positionen zu einem Thema aufgegriffen und in einem weiteren Schritt gegenüberstellt, um zu erörtern, wo der nächste Handlungsschritt gesehen wird. Mit Hilfe dieses Dialogverfahrens, wofür die Leitung die nötige Qualifikation mitbringt, wurde der Kindergarten entwickelt.\footnote{Die Leitung verweist vor Beginn der Interviewaufnahme auf das auf der Homepage der Einrichtung abgelegte Konzept, welches die einzelnen Angebote auflistet. Die Informationen dieses Dokuments werden bei der Auswertung berücksichtigt, das Dokument kann der Arbeit jedoch aufgrund der Wahrung der Anonymität nicht beigelegt werden. Die Beschreibung der Angebote oder Projekte durch die Experten wurde durch die zu beziehenden Informationen aus dem Konzept angereichert. Diese Informationen wurden mit einer Fußnote und dem Vermerk [Ergänzung aus dem Konzept des diakonischen Kindergartens] versehen. Gleiches gilt für das Konzeptpapier des Kindergarten der AWO, welches der Interviewerin während der Befragung überreicht wurde. Der Vermerk lautet entsprechend [Ergänzung aus dem Konzept des Kindergartens der AWO].} Die Einrichtung arbeitet zudem nach dem Situationsansatz vom Deutschen Jugendinstitut München. 

Die Leitungen der anderen beiden Einrichtungen berufen sich bei der Frage, ob es für die Kinder mit besonderen Bedürfnissen konzeptionelle Überlegungen gibt, auf die Umsetzung des vom Träger vorgegebenen Leitbildes. 
Die Leitung der AWO verweist auf die gesetzliche Vorgabe, dass, wenn Integrationshilfe beantragt wird, die Einrichtung -- durch Verankerung in der Konzeption -- gewährleisten muss, dass die Kinder entsprechend ihrer Bedarfe gefördert werden. Sie beschreibt, dass die Konzeption das Bild vom Kind, die Zusammenarbeit mit den Eltern oder auch die Frage beinhalten würde, welche Werte die Einrichtung vermitteln will und dass das Leitbild der AWO die Werte im Grunde genommen schon vorgeben würde. Dem fügt sie hinzu, dass das Konzept vor zehn Jahren vom Team erstellt und dabei der Gliederungspunkt „Integration von Kindern mit Behinderung“ unter der Überschrift „Umsetzung des Orientierungsplans“ (Konzept der Kindertagesstätte AWO) mit aufgenommen worden ist.
Der Kindergarten arbeitet ebenfalls nach einem situationsorientierten Ansatz und der pädagogischen Grundhaltung Maria Montessoris, dem Kind mit Wertschätzung und Achtung zu begegnen und Hilfe zur Selbsthilfe anzubieten.\footnote{Ergänzung aus dem Konzept des Kindergartens der AWO}

Die Kindergartenleitung des katholischen Kindergartens bezieht sich in ihren Aussagen ebenfalls auf das Leitbild, in dem kirchliche Werte wie Respekt voreinander und Gleichheit aller Personen verankert sind, denen zu entsprechen versucht wird. Eine differenzierte Sprachkonzeption wurde gemeinsam im Team erstellt, in der Gesamtkonzeption der katholischen Kirchgemeinde wurden Kinder mit Behinderung einbezogen, aber es liegt kein Konzept, vom Team ausgearbeitet, zur individuellen Unterstützung von Kindern mit besonderen Bedürfnissen vor. 

\paragraph{Sprachförderung}
Die Sprachförderung ist in jeder der untersuchten Einrichtungen als zentrale Zielsetzung verankert. 
Ein Großteil der Kinder, vor allem die, die zweisprachig aufwachsen und zusätzliche Sprachanreize benötigen oder jene, denen es an Mut fehlt zu sprechen, erhalten individuelle Sprachförderung in Kleingruppen. Ein solches mögliches Angebot, \emph{Singen--Bewegen--Sprechen}, findet in Kooperation mit der Musikschule Freiburg statt und wird vom  Bundesland Baden-Württemberg finanziert, an welchem Kinder des diakonischen Kindergartens teilnehmen. Auch das \emph{Rucksack}-Projekt ist seit 2011 in selbiger Einrichtung als Modellprojekt etabliert und hat zum Ziel die Sprachförderung noch besser in der Einrichtung zu integrieren, die Zweisprachigkeit der Kinder zu fördern und dabei die Eltern einzubeziehen. Einmal pro Woche treffen sich die russischsprachigen Eltern mit einer speziell dazu ausgebildeten gleichsprachigen Elternbegleitung, die pädagogische Themen mit ihnen bespricht und sie anleitet russische Lieder. Parallel werden diese Aktivitäten in der Gruppe für alle Kinder in deutscher Sprache durchgeführt. Die Leitung sieht die Bedeutung dieses Angebots in der Austauschmöglichkeit für die Eltern und im Erfahren von Wertschätzung für die Heimatsprache. Ihre Beobachtungen zeigen, dass teilnehmende Mütter selbstbewusster geworden und häufiger in der Einrichtung und bei Veranstaltungen präsent sind. Zudem werden die Kinder eingeladen in der Einrichtung auch russisch zu sprechen, wozu ihnen zum Beispiel im Stuhlkreis Raum gegeben wird.     
In der Einrichtung der Diakonie finden in Kooperation mit \emph{Südwind} in den vorhandenen Zusatzräumen Deutschkurse für Eltern an vier Vormittagen in der Woche statt. Gleichzeitig wird eine Kinderbetreuung für deren Kinder unter drei Jahren angeboten.   
Auch im katholischen Kindergarten wird individuelle Sprachförderung unter der Führung der Kindergartenleitung angeboten. Die Dauer und die Gruppengröße werden abhängig vom Förderschwerpunkt flexibel angepasst. Der Einsatz musikalischer Elemente und Bewegung wurden als 'Türöffner' erlebt, die sich vor allem bei gehemmten Kindern bewährt haben.

\paragraph{Kleingruppenprojekte}

Für die Jungen in der Einrichtung, die männliche Identifikationsfiguren suchen, wird die geschlechtsspezifische \emph{Ringergruppe} in Kooperation mit \emph{ProFamilia} angeboten, die vom Sportler und zwölffachen Weltmeister Adolf Seger ehrenamtlich trainiert wird. Das Projekt wird über den Elternbrief und im Haus bekannt gegeben und wird für Jungen und deren Väter ausgeschrieben, ist aber auch für Jungen offen, deren Väter zuhause nicht präsent sind. Um konstruktiv, das heißt, ohne dem anderen weh zu tun, mit Kraft und Aggressionen umgehen zu lernen, fließen beim Ringen das Kräftemessen, Nähe und Körperkontakt ein.\footnote{Ergänzung aus dem Konzept des diakonischen Kindergartens} Im Vorfeld backen die Kinder mit dem männlichen Erzieher Kuchen, das Projekt heißt \emph{Männer backen für Männer}, und im Anschluss findet das Männercafé statt.
Zudem werden einmal pro Woche Eltern-Kind-Gruppen im Bereich Naturpädagogik angeboten, die von der Natur- und Umweltpädagogin geleitet werden. Im Rahmen von Naturtagen findet handlungsorientierte Sprachförderung statt, Familienausflüge werden immer sonntags durchgeführt.\footnote{Ergänzung aus dem Konzept des diakonischen Kindergartens}

\paragraph{Schritte in Richtung Inklusion und wahrgenommene Veränderungen}

In der Einrichtung der AWO und der katholischen Kirchgemeinde wurde darauf hingewiesen, dass der Impuls von außen kam, der Schritte in Richtung Inklusion anregte. Die Leitung der Diakonie wiederum weist daraufhin, dass die Entwicklung eines Hauses stattgefunden habe, das sich Inklusion auf die Fahne geschrieben hat. „Dadurch, dass wir uns so aus uns selbst heraus entwickelt haben, weil wir das so wollten, funktioniert es natürlich auch super. Es funktioniert und es ist auch eine sehr schöne Arbeit.“ (C1,~\ref{C1_48}) Der Impuls ist hier also als eine intrinsische Handlung zu sehen.

Übereinstimmende Veränderungen werden darin wahrgenommen, dass Erziehung aufgrund des gruppenübergreifenden Austauschs und Denkens  eine Weite bekommen habe und nicht mehr auf die einzelne Gruppe begrenzt sei und Strukturen hinsichtlich der aufgebauten Kooperationen und der aufgehobenen Trennung zwischen Schule und Kindergarten gewachsen seien.  

Die Leitung der AWO benennt an beobachteten Veränderungen eine offenere Haltung gegenüber Inklusion im Team, eine ebenfalls erhöhte Fortbildungsbereitschaft, da die Notwendigkeit eingesehen und der finanzielle Rahmen geregelt wurde, eine erhöhte Rücksichtnahme unter den Kindern in Bezug auf Kinder mit besonderen Bedürfnissen, ein gewachsenes Bewusstsein bezüglich der Anpassung der Angebote an die  Bedarfe der Kinder sowie eine gewachsene Kooperation mit der benachbarten Schule.

Die wahrgenommenen Veränderungen beziehen sich bei der Leitung der Diakonie auf die Öffnung, das heißt, das Zulassen anderer Strukturen und neu hinzukommender Menschen. Zudem benennt sie, dass ein Prozess zu einer Ziel- und Ergebnisoffenheit stattgefunden habe. Darunter versteht sie, dass Antworten auf der Prozessebene gefunden und akzeptiert werden, ohne dass das Handeln auf das Erreichen eines Ziels ausgerichtet ist. Sie sagt, dass dieser Prozess auch zu Verunsicherung und Angriffen von außen und wiederum zu Abgrenzung und einem gestärkten Selbstbewusstsein geführt habe: „Es hat auch zu Abgrenzungen geführt, weil wir gesagt haben und wir machen das so, weil die Energie oder auch der Wille so zu arbeiten sich so verstärkt hat.“ (C1,~\ref{C1_65}). Weiterhin weist sie daraufhin, dass die Klärung von Konflikten heute besser gelingen würde und spricht sich Kompetenz zu Konflikte zwischen Kollegen und Eltern durch die Wahrung des Außenblicks in der Regel gut auflösen zu können und aufgrund der langjährigen Leitungstätigkeit zahlreiche Erfahrungen gesammelt zu haben, wie sie Eltern unterstützen kann. Zuletzt weist sie auf die Chancen der Kinder hin, die sich aufgrund des pointierten Blickes auf ihre Entwicklung erhöht haben sowie auf die Zufriedenheit unter den Mitarbeiter.

Auch die Leitung des katholischen Kindergartens beschreibt, dass die Arbeit für sie persönliche Bereicherung bedeuten würde, weil sie in den letzten Jahren viel habe lernen können und Wertschätzung erfahren habe: „In diesem Kindergarten werde ich ganz anders gebraucht!“ (C3,~\ref{C3_78}) 

\subsection{Auswertungskategorie 4: Partnerschaft mit den Eltern}

\paragraph{Bedarfslagen der Eltern}
Die Expertenaussagen zeigen, dass viele Familien aufgrund schlechter beruflicher Perspektive staatliche Unterstützung in Form von Arbeitslosengeld II oder Sozialhilfe beziehen und mit Themen wie Armut, Ausgrenzung und Beschämung konfrontiert sind. Die Kindergartenleitung der AWO konstatiert, dass bildungsferne Eltern im Vergleich zu bildungsorientierten Eltern weniger Interesse an der pädagogischen Arbeit im Kindergarten zeigen würden, wobei die fehlende elterliche Präsenz auch durch hohe familiäre Belastung durch beispielsweise mehrere Kinder erklärt wird.
In den Elternhäusern werden von der Kindergartenleitung der katholischen Kirchgemeinde fehlende Strukturen, eine Verarmung hinsichtlich der Lernanreize und Unruhe wahrgenommen. Der Zulauf zu Helfersystemen wie städtischen Beratungsstellen ist erschwert, wobei der Bedarf aufgezeigt wird Krisen wie Suchtkonflikte, Trennung und Scheidung sowie Todesfälle aufzufangen. 
Zudem beschreibt die Kindergartenleitung, dass elterliche Ängste und Sorgen bestünden, dass die Kinder zu wenig Zuwendung und Lernanreize bekommen, weil die Erzieherinnen aufgrund des Mehrbedarfs, bedingt durch die Kinder mit besonderen Bedürfnissen, nicht die nötige Zeit zur Verfügung haben würden. Gründe der Eltern sieht die Kindergartenleitung in dem gesellschaftlichen Erwartungsdruck, der mit Angst in Verbindung steht: „Mein Kind kommt zu kurz. Unser Kind fällt durch das Raster. Die Erzieherinnen haben nicht die nötige Zeit. Was passiert mit meinem Kind, wenn es nicht da 'hoch' kommt?“ (C3,~\ref{C3_44}). Demgegenüber stehen die von der Leitung der AWO beschriebenen finanziell gut situierten und bildungsinteressierten Eltern, die bewusst eine Einrichtung gewählt haben, in der Kinder unterschiedlicher sozialer Herkunft vertreten sind.

\paragraph{Elternabende}
Um die Ängste der Eltern ernstzunehmen, werden von der Leitung des katholischen Kindergartens Referentinnen eingeladen, konkret die Heilpädagogin und die Sprachtherapeutin, die von ihrer pädagogisch-therapeutischen Arbeit mit den Kindern berichteten und als Einblick eine Videoaufnahme des Kindes mit Down-Syndrom  zeigten. Die Leitung beschreibt folgende Resonanz: „Da gab es ein Aha-Erlebnis bei den Eltern -- es ist nicht selbstverständlich, dass wir gesund sind. Mir kann heute etwas passieren, dann bin ich behindert und will genauso geschätzt werden wie gestern noch.“ (C3,~\ref{C3_38}) Zudem wurde die Mutter, deren Kind eine spastische Lähmung hat, zu einem Elternabend eingeladen. Sie berichtete aus ihrem Leben, von ihren Erfahrungen und Ängsten. Auch hierbei stellte die Leitung eine positive Wirkung fest, die sie durch eine größere Offenheit der Eltern im Umgang mit der referierenden Mutter beschrieb. 
Als Antwort auf die unterschiedlichen kulturellen Hintergründe der Familien veranstaltete der diakonische Kindergarten einen Elternabend, in dem zum Austausch über die unterschiedlichen Feste in den jeweiligen Kulturen eingeladen wurde. 

\paragraph{Eltern- und Entwicklungsgespräche}
Im katholischen Kindergarten wurden die Elternabende zugunsten einer höheren Frequenz von Elterngesprächen reduziert, da in einem kleineren Rahmen die Entwicklung des einzelnen Kindes besser in den Blick genommen werden kann und auch die elterlichen Ängste besser beantwortet werden können. Diesen wird mit der Frage: „Auf welchen Gebieten haben Sie Angst, dass ihr Kind zu kurz kommt?“ (C3,~\ref{C3_63}) begegnet sowie mit dem Angebot, einen Vormittag in der Einrichtung zu hospitieren. 
Entwicklungsgespräche finden in regelmäßigen Abständen und bei Bedarf statt. Die Leitung des katholischen Kindergartens beschreibt, dass Eltern sehr unterschiedlich reagieren würden, wenn ihnen die Beobachtungen ihres Kindes mitgeteilt werden. 
Manche Eltern sagen, dass sie das beschriebene Verhalten von zuhause kennen würden, andere würden sehr erschrocken reagieren und die Angst äußern, dass ihr Kind aus der Einrichtung 'raus' muss. In einem solchen Fall wurde als sinnvoll erlebt den Eltern Zeit zu geben und im zeitlichen Abstand von einer Woche ein weiteres Elterngespräch durchzuführen. Entwicklungsgespräche finden unter Einsatz der Methoden zur Dokumentation kindlicher Entwicklung statt. 
Persönliche und bestärkende Elterngespräche werden von der Leitung des katholischen Kindergartens als Antwort auf elterliche Überforderung und Entmutigung gesehen, die Durchführung wird jedoch als anstrengend erfahren. 

\paragraph{Information und Transparenz}
Im Aufnahmegespräch wird zu einem Austausch angeregt und die Eltern werden informiert, dass Kinder mit besonderem Förderbedarf aufgenommen werden und der Anteil an Kindern mit Migrationshintergrund erhöht ist. Zudem werden zentrale Aspekte der Arbeitsweise erklärt, in der Einrichtung der AWO, dass die Fachkräfte einen ressourcen-orientierten Blick auf das Kind haben, in der Einrichtung der Diakonie, wie wichtig das 'Zusammenleben' und die 'Gemeinschaft' sind. „Wenn die aufgenommen werden, wird denen das natürlich auch gesagt, wie wir hier zusammenleben und wie wichtig die Gemeinschaft uns ist und die haben wirklich ganz schönes Vertrauen. Die kommen auch mit allem. Das wissen die auch, wenn etwas ist, dann darf man kommen. Wenn nicht hier, wo sonst.“ (C1,~\ref{C1_59})
Möglichkeiten die Eltern zu informieren sehen verschieden aus. Der Kindergarten der AWO bringt zum Beispiel ein Blatt Papier an der Tür zum Gruppenraum an, auf dem die Bilder und Namen der neuen Kinder, die in die Gruppe kommen, festgehalten werden.
Der Kindergarten der katholischen Kirchgemeinde nutzt ein Gruppentagebuch, das die Eltern täglich über die Aktivitäten, die in der Gruppe stattgefunden haben, informiert und Liedtexte oder Geschichten bereithält, die die Eltern kopieren können. 

\paragraph{Beteiligungsmöglichkeiten}
Eine Möglichkeit für Eltern sich zu beteiligen und auszutauschen ist das Elterncafé, das von den Kindergärten der Diakonie und der AWO eingerichtet wurde. 
Die Kindergartenleitung der AWO berichtet, dass das Elterncafé einmal im Monat stattfindet und sich unter pädagogischer Leitung verschiedenen Themen widmet. Das Thema zum Zeitpunkt der Befragung bezog sich, wie auch schon beim thematischen Elternabend, auf die kulturellen Hintergründe, die in der Einrichtung vertreten sind. Ein Elterncafé wurde zum Beispiel von den russischstämmigen Eltern gestaltet. Sie erzählten, wie sie nach Deutschland gekommen sind, berichteten Bedeutsames über ihre Kultur und Religion und bereiteten ein traditionelles Gericht vor. 
Die Bedeutung in solchen Treffen wird darin gesehen, dass Austauschmöglichkeiten für die Eltern bereitgestellt werden, wodurch das Bedürfnis einander zu verstehen und mitzufühlen beantwortet werden kann. Das Elterncafé ist auf Initiative der Eltern entstanden, die eine Möglichkeit des Austauschs suchten. 
In der Einrichtung unter dem Träger der Diakonie findet das Elterncafé aller 14 Tage statt. Auch hier finden thematische Gesprächsrunden mit kurzem fachlichen Input statt. Offen ist dieses nicht nur für die Eltern des Kindergartens, sondern auch für interessierte Eltern mit Kindern aus dem Stadtteil. 
Neben dem Elterncafé gibt es weitere Veranstaltungen, wie Elternnaturtage oder Feste, an denen auch Eltern teilnehmen können, die ihre Kinder nicht in der Einrichtung haben, sodass die Arbeit im Kindergarten in das Gemeinwesen hinaus strahlt. Selbiges wird auch durch die Veranstaltung von Flohmärkten oder das Einrichten eines eigenen Standes auf dem Weihnachtsmarkt erreicht.
Die Leitung des diakonischen Kindergartens betont eine emotional enge Interaktion mit der Familie und stellt entsprechenden Raum für Begegnung und Beziehungsaufbau zur Verfügung. Dem liegt die Haltung zugrunde: „Wer mit den Kindern arbeitet, muss mit den Familien arbeiten.“ (C1,~\ref{C1_16}) Den Eltern werden Beteiligungsmöglichkeiten in Eltern-Kind-Gruppen im Bereich Naturpädagogik, bei Familienausflüge am Wochenende, im Förderverein und Elternbeirat oder in der 'Zukunftswerkstatt' angeboten. Die Leitung spricht sich in Bezug auf die Elternarbeit Kompetenz zu, indem sie beschreibt, dass sie Konflikte zwischen Eltern und Kollegen durch die Wahrung des Außenblicks in der Regel gut auflösen können würde und aufgrund der langjährigen Leitungstätigkeit vielfältige Erfahrungen gesammelt habe, wie sie Eltern unterstützen kann.  
Die Leitung des katholischen Kindergartens benennt als elterliche Beteiligungsmöglichkeiten Feste feiern, Bastelaktionen, zum Beispiel Kinder und Eltern basteln gemeinsam Laternen oder Schultüten, das gemeinsame Frühstück zwei Mal im Jahr, was sehr dankbar von den Eltern angenommen wird, die mögliche Begleitung an Wandertagen oder Spieltagen. Die intendierte Wirkung ist dabei Eltern Anreize für mögliche gemeinsame Aktivitäten mit ihren Kindern zu geben. 
Die Kindergartenleitung unter dem Trägerdach der AWO beschreibt ein Projekt, das der Kindergarten in Kooperation mit der Hochschule durchführt -- \emph{Gesund aufwachsen in der Kita}. Hierbei wurden die Eltern in der Auswahl der Projekte von Anfang an einbezogen. So wurde entschieden, dass ein Waldausflug gemeinsam mit den Eltern stattfinden sollte, der an einem Freitagnachmittag zeitlich möglichst spät gelegt wurde, um einer großen Elternschaft die Teilnahme zu ermöglichen. Die Leitung berichtet von der Erfahrung, dass erstmals alle Elternteile, vollzählig erschienen seien und ihren Überlegungen, dass sie ihren Fokus noch mehr darauf zu richten habe, wie Eltern beteiligt werden können.  

\subsection{Auswertungskategorie 5: Kooperation}

\paragraph{Kooperation mit Fachdiensten und Institutionen}

Als notwendige Bedingung für Inklusion formulieren die Kindergartenleitungen übereinsimmend ein „breite Bündnis“ (C1,~\ref{C1_14}), das auf die gegebenen Bedürfnissen der Kinder in der Gruppe abgestimmt wird. So werden bei Gefährdungssituationen zum Beispiel Institutionen wie \emph{Wildwasser} oder \emph{Wendepunkt} zur Fallsupervision eingeladen, bei Suchtkonflikten wird der kommunale Suchtbeauftragte angefragt oder wenn Kinder mit einer Seh- oder Hörbeeinträchtigung in die Einrichtung kommen, werden Kontakte zur Beratungsstelle der Sehbehindertenschule beziehungsweise zum Bildungs- und Beratungszentrum für Hörgeschädigte hergestellt. Bei der Einrichtung der AWO besteht eine enge Kooperation mit der Frühförderstelle der AWO, da beide Institutionen unter einem Trägerdach zusammengefasst sind. Da die Heilpädagogin der Frühförderstelle in die Einrichtung kommt, wurde eine niedrige Hemmschwelle für das Anbahnen des Erstkontakts als fördernde Bedingung erlebt. Bei beobachteten  Bedarf des Kindes wird die Heilpädagogin gebeten innerhalb der Gruppensituation das Kind wahrzunehmen. Die Hemmschwelle der Eltern, Hilfen in Anspruch zu nehmen, kann durch die einfachen Zugangswege und die Erfahrung, dass viele Kinder innerhalb der Einrichtung heilpädagogische Förderung bekommen, abgebaut werden.
Auch im katholischen Kindergarten kommt ein bis zwei Mal im Jahr eine Fachkraft aus der Sprachheilberatungsstelle in die Einrichtung, um in Absprache mit den Eltern den Klärungsbedarf auf Seite der Erzieherinnen zu beantworten. Die niedrige Hemmschwelle wird auch hier als positiv erlebt, da die Eltern entlastet werden: „Die meisten unserer Eltern würden es nicht schaffen einen Termin einzuhalten.“ (C3,~\ref{C3_21}) 
Wenn die Kinder bereits mit bestehenden Therapeuten-Kontakten, häufig Ergotherapie oder Spieltherapie, in den Kindergarten kommen, werden diese Kontakte aufgenommen.     
Die außenstehenden Helfer in Verbindung zueinander zu bringen, wird von der Leitung des katholischen Kindergartens als schwierig erlebt, was zum Beispiel durch bestehende lange Wartezeiten für eine Entwicklungsdiagnostik zum Ausdruck gebracht wird. 
  
\paragraph{Kooperation mit der Schule}
Die Zusammenarbeit zwischen Kindergarten und Grundschule wird durch das Modellprojekt \emph{'Bildungshaus'} unterstützt, bei welchem der diakonische Kindergarten zu den Modellstandorten gehört. Das Bildungshaus-Projekt wurde 2007 vom Ministerium für Jugend und Sport Baden-Württemberg initiiert und wird durch das Zentrum für Neurowissenschaften und Lernen wissenschaftlich begleitet. 
Kinder im Alter von drei bis zehn Jahren lernen und spielen gemeinsam in der Schule und im Kindergarten, wodurch der Übergang in die Schule erleichtert und die Bildungsbiografie des Kindes nahtlos fortgeführt werden kann sowie Ängste und Vorbehalten seitens der Eltern gegenüber Schule abgebaut werden können.\footnote{Ergänzung aus dem Konzept des diakonischen Kindergartens}

Der Kontakt des Kindergartens der AWO zu der benachbarten Grundschule konnte durch gegenseitige Besuche intensiviert werden. So kommt die Musiklehrerin der Grundschule zwei Mal in der Woche in den Kindergarten, um mit den Kinder vor Ort zu arbeiten. Einmal pro Woche gehen die Kinder in die Grundschule und erfahren musikalische Früherziehung im Musiksaal. Die Kinder haben die Schule dadurch bereits kennenlernen und Vertrauen aufbauen können.
Als positive Veränderung wird von der Kindergartenleitung der AWO die aufgehobene Trennung zwischen Schule und Kindergarten genannt, die in gemeinsam besuchten Fortbildungen, Elternabenden, gegenseitigen Festeinladungen und der gemeinsamen Teilnahme am Modellprojekt \emph{'Schulreifes Kind'} ihren Ausdruck findet. Das Kooperationskonzept 'Schulreifes Kind' des Kultusministeriums Baden-Württemberg sieht vor, förderbedürftige Kinder in einer eigenen Gruppe in unterschiedlichen Bereichen rechtzeitig vor Schuleintritt zu fördern, sodass diese Kinder keine Zurückstellung erhalten müssen.\footnote{Ergänzung aus dem Konzept des Kindergartens der AWO} Im Team wurde der Wunsch benannt: „Eigentlich wäre es schön, wenn die Einrichtung daneben wäre, dass man wie so ein Kinderhaus hätte, wo die Übergänge noch besser wären.“ (C2,~\ref{C2_600}) 

\paragraph{Kooperation im Team}

In allen Einrichtungen gibt es Teamsitzungen, in denen das Verhalten von Kindern diskutiert wird. Die Unterschiede bestehen in dem zur Verfügung gestellten zeitlichen Rahmen. Die Einrichtung der Diakonie
misst diesem Aspekt die vergleichsweise größte Bedeutung bei und stellt dafür die meiste Zeit zur Verfügung. Die Leitung beschreibt, dass bedingt durch die vielfältige Projektarbeit der interdisziplinäre Austausch gewährleistet ist, der in der Gesamtteamsitzung für alle Beteiligten transparent gemacht wird. Die Kinder sind den Projektgruppen zugeordnet und die erwachsenen Verantwortlichen bringen ihre Beobachtungen, Wahrnehmungen und Erlebnisse ein, sodass bei jedem Gesamtteam-Treffen über die Kinder gesprochen wird, was den „pointierten“ (C1,~\ref{C1_12}) Blick auf deren Entwicklung ermöglicht. Neben diesem Treffen gibt es noch sogenannte Kleingruppenteams, in welchen die Fachkräfte von jeweils zwei Gruppen ein eigenes Team bilden, das sich wöchentlich trifft, um über die Kinder zu besprechen und Überlegungen anzustellen, ob die Förderungsangebote die individuellen Bedürfnisse der Kinder treffen. Zuletzt gibt es das Treffen, in dem das erwähnte Kleingruppenteam mit der Heilpädagogin zusammenkommt, um speziell die Kinder anzuschauen, die Integrationshilfe erhalten. 
Auch in dem Kindergarten der AWO finden Kleingruppentreffen statt, die jedoch, wenn Kinder mit besonderem Förderbedarf in der Gruppe sind, als nicht ausreichend beschrieben werden.    
Einen Freitag im Monat schließt die Einrichtung früher, sodass drei Stunden für Teambesprechungen zur Verfügung stehen. Diese Zeit steht für den kollegialen Austausch und zur Unterstützung der Selbstreflexion zur Verfügung.  

\section{Interpretation der Ergebnisse}
In diesem Kapitel erfolgt die Interpretation der Ergebnisse durch die Einordnung in die Theorie. Das bedeutet, laut Gläser und Laudel (2010, 263), dass die empirisch gefundenen Einflussfaktoren mit den in der Literatur beschriebenen abgeglichen werden. Die theoretischen Vorüberlegungen unter Kapitel~\ref{sec:Wie} werden dabei als Idealzustand von Inklusion ausgewiesen. Die zentrale Frage, die die Bewertung der Ergebnisse anleitet, ist somit, inwiefern die untersuchten Einrichtungen diesem Ideal entsprechen und worin Handlungsschritte in Richtung Inklusion gesehen werden können. 

Bevor die Bewertung der Ergebnisse fixiert wird, sollen kritische Anmerkung zu der gewählten Forschungsmethode dargelegt werden.
Aufgrund des begrenzten zeitlichen Rahmens der Bachelorthesis und der zuvor nicht absehbaren Länge der Interviews konnten nur drei Interviews durchgeführt werden. Durch die geringe Anzahl der Interviews können die Aussagen nicht verallgemeinert werden. Die von den Experten aufgezeigten Angebote, Projekte oder Methoden können anderen Einrichtungen als Anregungen dienen, auf andere Einrichtungen übertragbar sind sie aber sowieso nicht, da die Angebote an die örtlichen Rahmenbedingungen und bestehenden Bedarfe der Kinder anzupassen sind. 
Zudem erschwerte die hohe Spezifizität der Einrichtungen, die durch die ausführlichen Interviews und die intensive Auswertung hervorgehoben wird, die Suche nach dem gemeinsam geteilten Wissenstand. Wenn der zeitliche Rahmen für diese Arbeit weiter gesteckt gewesen wäre, hätten ergänzend die Erzieherinnen der Einrichtung befragt werden können, um deren Sichtweise auf die Phänomene zu erhalten. Die Differenzen hätten eine ganzheitlichere Abbildung der Forschungsfrage ermöglicht. Auch eine teilnehmende Beobachtung der Kinder hätte zu einer höheren Validität der Forschungsergebnisse beigetragen können.

Die Ergebnisse zeigen, dass die Sprachförderung als zentrale Zielsetzung in den Einrichtungen erfolgreich umgesetzt werden konnte. Ganzheitliche und mehrdimensionale Entwicklungsangebote wie Singen-Bewegen-Sprechen oder das Rucksack-Projekt geben Beispiel, dass auf Landesebene der Bedarf an Sprachförderung erkannt und durch entsprechende Angebote zu unterstützen versucht wird. Dem Förderbedarf hinsichtlich der Sprachentwicklung wird politische Aufmerksamkeit geschenkt, ebenso der Kooperation zwischen Kindergarten und Schule. 
Die Ergebnisse zeigen eindeutig, wo Deutschland in seiner Entwicklung hinsichtlich der Umsetzung von Inklusion steht, nämlich noch am Anfang.
Die Ziele, den Übergang von Kindergarten und Schule zu erleichtern und die Sprachförderung insbesondere bei Kindern mit Migrationshintergrund zu fördern, um Bildungschancen auszugleichen, sind im Bildungsplan\footnote{In Baden-Württemberg gilt die Bezeichnung Orientierungsplan, zuvor wurde aber die Verwendung des Begriffs Bildungsplan festgelegt, da diese Bezeichnung in der Fachliteratur zu finden ist.} verankert und somit im gesellschaftlichen Bewusstsein verankert. Die Strukturen der Einrichtung sind jedoch nicht an der Entwicklung des einzelnen Kindes ausgerichtet, sondern an der Mehrheit orientiert.  
So zieht die Leitung des Kindergartens der AWO grundsätzlich in Zweifel, dass individualisierte Förderung umgesetzt werden kann. Die Erfahrung, dass die Erzieherinnen nicht in der Lage seien jedem Kind gerecht zu werden, dürfen jedoch nicht losgelöst von der angewandten Methode betrachtet werden, nämlich die Interessengebiete der Kinder aufzugreifen und zusammenzufassen. Hierin zeigt sich das traditionelle Denken eines Systems, das auf die Schaffung gemeinsamer Lernsituationen ausgerichtet ist. Das könnte auch eine Erklärung für die Wahrnehmung der Leitung sein, dass egal wie sehr sich das Team bemühen würde, es immer Situation gibt, in denen die Kinder mit besonderen Bedürfnissen merken, dass sie in einer besonderen Situation sind. Es reicht nicht aus, die Kinder mit besonderen Bedürfnissen am Schwimmkurs oder beim Waldausflug teilhaben zu lassen -- hierbei werden die Kinder als 'Masse' verwaltet -- sondern es bedarf eines personenzentrierten Ansatzes, der passgenaue, an den individuellen Entwicklungen der Kinder ausgerichtete Angebote zu finden versucht.

Die Einrichtungen sind erwartungsgemäß sehr unterschiedliche Wege gegangen und haben unterschiedliche Schwerpunkte gesetzt.
Der Anstoß für den Inklusionsprozess kam im Fall des diakonischen Kindergartens von den Mitarbeitern selbst und nicht von außen durch Elternanfragen. Die bewusste Entscheidung für Inklusion, getragen von dem gesamten Team und unter Setzung gemeinsamer Ziele, die im Blick behalten werden, kann als 'sichere Basis' für einen gelingenden Inklusionsprozess betrachtet werden. Hinzu kommen gemeinsam erarbeitete konzeptionelle Überlegungen, die über die bloße Aufnahme des Aspektes 'Integration von Kindern mit Behinderung' weit hinausgehen und die Beschreibung individuell angepasster Angebotsstrukturen beinhalten. 
Die Konzepterarbeitung wird in der Literatur als Grundlage für einen interdisziplinären Austausch verstanden. Die Weiterentwicklung selbiger findet im diakonischen Kindergarten unter Beteiligung aller, auch der Elternperspektive, durch die Situationsanalyse, das Einnehmen unterschiedlicher Perspektiven auf die Kinder und ihrer Bedarfe sowie durch die gemeinsame Erarbeitung und Planung notwendiger Angebotsstrukturen statt. Die notwendigen Handlungsschritte werden also in erster Linie in einer differenzierten Konzepterarbeitung unter Beteiligung aller gesehen. Gemeinsam gefundene Werte und Ziele sind überhaupt erst einmal Voraussetzung dafür, ein Bewusstsein für Werte und Ziele im Praxisalltag zu entwickeln und den Blick darauf zu schärfen. Dementsprechend können Grenzen erkannt werden und wiederum neue Antworten gefunden werden, was als Qualitätsentwicklung fungiert.
Das steht scheinbar damit in Widerspruch, dass die Leitung erklärt, dass sich die Einrichtung zu einer Ergebnis- und Zieloffenheit entwickelt habe. Jedoch finden beziehen sich die Aussagen auf verschiedene Ebenen. Auf der Prozessebene werden Antworten zu finden versucht, die ausschließlich aus dem erkannten Bedarf und den Ideen zur Lösung resultieren. Trotzdem hat die Einrichtung eine Mission und  somit übergeordnete Werte und Ziele, die als 'Grundpfeiler' das System tragen. 

Die Kindergartenleitung des diakonischen Kindergartens bestätigt zudem den theoretisch angenommenen Einflussfaktor, dass die Prozessqualität entscheidend von den Persönlichkeitsmerkmalen der Erzieherin, insbesondere ihrer Zufriedenheit, bestimmt wird. Die Aussagen der Leitung, dass sich der Wille so zu arbeiten verstärkt habe, dass innere Wachstumsprozesse stattgefunden hätten und der eigene Standpunkt gefestigt werden konnte sowie dass die Arbeit Spaß mache, die Fachkräfte sehr kompetent seien, Konflikte in der Regel gut aufgelöst werden könnten und sich die gefundenen Strukturen für die Entwicklungsförderung der Kinder als wirksam erweisen würden, spiegeln eine hohe Kompetenzzuschreibung und das Gefühl von Selbstwirksamkeit wieder, was mit beruflicher Zufriedenheit einhergeht.
Als weitere fördernde Bedingungen für Inklusion werden angesehen, die Persönlichkeitsentwicklung der Erzieherinnen durch den Dialog im Team zu unterstützen und eine respektvolle, empathische Haltung auf Augenhöhe zu vertreten, die in dem Satz: „Wir sind alle im Werden!“ zum Ausdruck kommt. Auch sich die eigenen Barrieren im Kopf bewusst zu machen oder entstandene Situationen im Alltag hinsichtlich des Aspekts, dass eigene sensible Themen in das Kind, den Kollegen oder die Mutter projiziert wurden, zeugt von einem hohen Grad an Selbstreflexivität. 
Die ständig Weiterentwicklung der Person und ihrer inklusiven Grundhaltung werden als notwendiger beteiligter Teilprozess des Inklusionsprozess verstanden.
Bedingt durch die Entwicklung zum Familiennetzwerk kann der diakonische Kindergarten auch im Hinblick auf die Elternbeteiligung Orientierung geben. Die Haltung der Leitung, die unter dem Aspekt, dass entsprechende Wortwahl zur Bewusstseinsbildung beiträgt gelesen werden kann, spricht von dem Haus, das sich entwickelt habe, von der Gemeinschaft und davon miteinander unterwegs zu sein, was von einem gemeinsamen Weg mit den Eltern zeugt. Auch die Leitung der AWO hat darauf hingewiesen, dass sie den Bedarf sehen würde ihren Fokus noch mehr auf die Frage zu richten, wie Eltern enger beteiligt werden können statt zu sehen, dass das Erreichen mancher unmöglich ist. Das Bild vom Kind...     
 
„Allen Kindern gerecht werden?“ lautet der Titel dieser Arbeit. 
Die Ergebnisse haben gezeigt, dass die Kindergartenleitungen diesbezüglich Grenzen sehen. 
Die Leitung des katholischen Kindergartens macht die Grenzen am Grad der Behinderung fest und korrelierend dazu an der zur Verfügung stehenden Zeit sich speziell diesen Kindern zu widmen. Die Leitung der AWO sieht Grenzen bei der Umsetzung individualisierter Förderung. Die genannten Zweifel, dass Erzieherinnen in der Lage sind jedem Kind gerecht zu werden, dürfen jedoch nicht losgelöst von der angewandten Methode betrachtet werden, nämlich die Interessengebiete der Kinder zusammenzufassen. Hierin zeigt sich das traditionelle Denken eines Systems, das auf die Schaffung gemeinsamer Lernsituationen ausgerichtet ist. [Zudem bringt sie das Gelingen von individualisierter Unterstützung mit dem Aspekt des zur Verfügung stehenden Personals in Verbindung, in dem sie sagt, um individuell zu fördern, bräuchte es ja fast für jedes Kind eine Fachkraft.]
Die Leitung des diakonischen Kindergartens stellt sich die Integration von Kindern mit Handicaps, die einen sehr hohen Pflegebedarf mitbringen, im Alltag schwer umzusetzen vor, da strukturelle Rahmenbedingungen Voraussetzung sind, deren Bereitstellung durch die Politik in Zweifel gezogen werden. Grundsätzlich aber denkt sie, dass alle Kinder Vorteile aus der gemeinsamen Erziehung und Bildung ziehen würden.
Schlussfolgernd kann gezeigt werden, dass tatsächlich vorhandenen strukturelle Voraussetzungen wie der vorhandene Personalschlüssel, die Gruppengröße und bauliche Voraussetzungen mit den wahrgenommen Grenzen von Inklusion korrelieren. Der Komponente offene Haltung der Leitung kommt dabei eine weitere zentrale Bedeutung zu. 
Die Fachkräfte können nur offen sein, wenn sie ein Gefühl haben dem Bedarf durch die tatsächlich vorhandenen Ressourcen auch gerecht werden zu können. 
Jedoch entstehen Aufnahmebeschränkung allein durch die hinderliche bauliche Voraussetzungen wie zum Beispiel zu enge Flure für das Passieren mit dem Rollstuhl. 
Das Beispiel des eingefügten Holzkeils als Rampe, um die hohen Schwellen an den Türen zu kompensieren, verbunden mit dem Aspekt, dass diese zum Schließen der Türen wieder entfernt werden müssen, ist signifikant dafür, dass Fachkräfte kaum und nur unter einem erheblichen Mehraufwand in der Lage sind Defizite innerhalb der strukturellen Rahmenbedingungen zu kompensieren.

Streng genommen ist Inklusion bereist an der gedanklichen Auseinandersetzung gescheitert, welche Voraussetzungen ein Kind mitbringen muss beziehungsweise wie der Grad der Behinderung sein muss, damit die Einrichtung es aufnehmen kann. 
Denn genau diese Fragestellung macht den Paradigmenwechsel von Integration zu Inklusion aus: Der Blick wendet sich ab von der Fragestellung, wie ein Kind sein muss, damit es die Einrichtung besuchen kann hin zu der Frage, was an den gegebenen Strukturen und Konzepten geändert werden muss, damit das Kind in die Einrichtung kommen kann.

Genau an dieser Stelle zeichnet sich die Kluft zwischen Ideal und Wirklichkeit ab. Wenn die Leitung darauf hinweist, dass Kinder in andere Einrichtungen empfohlen werden, dann sollte die der Entscheidung zugrunde liegende Überforderung gesehen und ernstgenommen werden. Jeder dieser 'Grenzfälle' hätte laut Aussage der Expertinnen durch das Schaffen adäquater Rahmenbedingungen vermieden werden können. Die Einrichtung der Diakonie mit den bereist beschriebenen....   


Was also bedeuten die Befunde für die Politik, für die Praxis, für die Aus- und Weiterbildungen?

  




%2-3 Seiten zur Interpretation/Bewertung der Ergebnisse mit einem kleinen Ausblick und ein paar Sätzen zu Vor- und Nachteilen der angewandten Methode (geringe Anzahl von Interviews, hohe Einrichtungs-Spezifität, die durch die ausführlichen Interviews und die intensive Auswertung noch hervorgehoben wird)
%So eine Interpretation der Ergebnisse, wo man sich nochmal die Forschungsfragen vornimmt und schaut, welche man mit den Ergebnissen beantworten konnte und welche nicht,  und Schlussfolgerungen, was die Befunden jetzt für die Politik, Praxis, Aus- und Weiterbildung etc. bedeuten, das hält man kurz und knapp.

%Aufgrund des festgelegten zeitlichen Rahmens der Bachelorthesis und der ausführlichen Interviews, woraus eine intensive Auswertung resultierte, konnten nur drei Interviews durchgeführt werden. 
%Bedingt durch die geringe Anzahl der Interviews lassen sich die Aussagen nicht verallgemeinern. 
%Die hohe Einrichtungs-Spezifität, die durch die ausführlichen Interviews und die intensive Auswertung hervorgehoben wird, erschwerte die vergleichbare Datengewinnung/ die Suche nach einem gemeinsam geteilten Wissenstand. 

\chapter{Zusammenfassung und Ausblick}

 

  



  
 




   

 
 
 
 
