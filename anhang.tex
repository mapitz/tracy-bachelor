\chapter{Gesprächsleitfaden der Experteninterviews}
\paragraph{Strukturelle Rahmenbedingungen und Konzepte}
\begin{enumerate}
\item Können Sie etwas zur Organisation und zur Struktur Ihrer Einrichtung sagen?
\begin{itemize}
\item Welche Kinder besuchen Ihre Einrichtung? Welcher sozialen Herkunft sind diese Kinder? 
\item Werden Kinder mit besonderen Bedürfnissen betreut?
\item Personalschlüssel? Gruppengröße?
\item Wie ist das Team zusammengesetzt?
\item Spielen Heilpädagogen oder andere Fachleute mit Spezialwissen eine aktive Rolle im Gruppenalltag?
\item Gibt es Ansprechpartner, die bei Fragen einbezogen werden, wie die Entwicklung eines Kindes möglichst optimal zu unterstützen ist? 
\item Mit welchen Kooperationspartnern ist die Einrichtung vernetzt?
\end{itemize}
\item Gibt es für die Kinder mit besonderen Bedürfnissen konzeptionelle Überlegungen? Entsprechen diese Ihrem Inklusionsverständnis?
Was verstehen Sie unter Inklusion in Bezug auf den Kindergarten? 
\item Können Sie noch weitere Angaben über den Charakter des Konzepts hinsichtlich Inklusion machen?
\begin{itemize}
\item Wie wird der Verschiedenheit der Kinder Rechnung getragen?
\item Wie ist individuelle Unterstützung in Ihrem Konzept vorgesehen?
\item Wie wird die individuelle Arbeit am Kind in das Gruppengeschehen integriert?
\item Was sind gegebenenfalls Strategien, um Therapien in einer inklusiven Form anzubieten?
\end{itemize}
\item Was sind Ihrem Ermessen nach notwendige Voraussetzungen für das Gelingen von Inklusion, die von „oben“ garantiert werden sollten? 
\item Gibt es bildungspolitische Entscheidungen, die der Umsetzung von Inklusion im Weg stehen? 
Wenn ja, was sollte sich ändern? 
Was kann der Träger dazu beitragen? 
Wo sehen Sie Ihre besonderen Aufgaben als Leitung?
\end{enumerate}

\paragraph{Schritte in Richtung Inklusion}
\begin{enumerate}
\item Welche Schritte wurden konkret unternommen, um Inklusion einzuleiten?
\item Was waren und sind Stolpersteine auf dem Weg zur inklusiven Bildungseinrichtung? 
\end{enumerate}

\paragraph{Herausforderungen an das Team}
\begin{enumerate}
\item Welche Herausforderungen werden an das Team gestellt, wenn das Konzept in Richtung Inklusion umgestellt wird?
\item Was sind häufige Fragen der Erzieherinnen, die anzeigen, dass Beratungs- oder Unterstützungsbedarf besteht?
\item Gab es oder gibt es Ängste unter den Mitarbeitern? Welcher Art? Wie wird damit umgegangen?
\item Welche Unterstützung erfährt das Team? 
Gibt es eine gemeinsame Planung und Zusammenarbeit unter den Beteiligten? Gibt es Teamsitzungen, in denen das Verhalten von Kindern diskutiert wird? 
\item Brauchen Erzieherinnen Zusatzqualifikationen, um inklusiv zu arbeiten? 
In welchen Bereichen sind Kenntnisse Ihrer Meinung nach hilfreich?
\item Welches Handwerkszeug müssen ErzieherInnen mitbringen bzw. was macht gute Erzieherinnen aus?
\end{enumerate}

\paragraph{Partnerschaft mit den Eltern}
\begin{enumerate}
\item Welche Bedeutung hat die Elternarbeit im Hinblick auf Inklusion?
Wie werden die Eltern bei Entscheidungen eingebunden? 
\item Was wird in Ihrer Institution organisiert, um eine Beziehung zu den Eltern aller Kinder aufzubauen?
\item Kennen Sie die Einstellungen der Familien zu Fragen inklusiver Erziehung und wie unterstützen Sie den Dialog darüber?
\end{enumerate}

\paragraph{Persönliche Haltung und Sichtweise}
\begin{enumerate}
\item Haben Sie durch die Umsetzung von Inklusion Veränderungen bei sich feststellen können? Bei den pädagogischen Mitarbeitern? Bei den Kindern? 
\item Wie ist ihr Standpunkt zu der Aussage: „Allen Kindern gerecht werden“? Sehen Sie Grenzen der Inklusion?
\end{enumerate}


\chapter{Kategoriensystem}

% Beginn Querformat
\begin{landscape}
\begin{small}

% ============================
%         Tabelle 1       
% ============================
\begin{centering}
%\addcontentsline{toc}{table}{Tabelle1}
\begin{longtable}{p{2.2cm}p{8cm}p{6cm}p{3cm}p{2cm}} 
\caption{Extraktionstabelle zur Auswertungskategorie „Rolle der pädagogischen Fachkraft“}\\
% =========================== ANFANG =============================
% ========== Tabellenkopf für erste Seite der Tabelle ============
% ================================================================
\toprule
Dimension & Sachverhalt (Aussagen konzentriert) & Ursache & Wirkung & Quelle\\
\midrule
\endfirsthead
% ============================ ENDE ==============================

% =========================== ANFANG =============================
% ======= Tabellenkopf für jede weitere Seite der Tabelle ========
% ================================================================
\bottomrule
Dimension & Sachverhalt (Aussagen konzentriert) & Ursache & Wirkung & Quelle\\
\midrule
\endhead
% ============================= ENDE ==============================

% ====================== ANFANG Tabellenfuß =======================
\bottomrule
\endfoot
% ============================= ENDE ==============================

% ======================== Tabelleninhalt =========================
Haltung zu Inklusion & Wir sind offen für alle Kinder, haben die Position, wir probieren es aus und wenn wir es nur zwei Wochen probieren, aber wir probieren es. Erst im Tun, im Leben und Erleben sehen wir oft, was das bringt. Manchmal stoßen wir dann auch an unsere Grenzen und haben Angst, diesem Kind nicht gerecht werden zu können. & & & C3\ref{C3_28}\\

 & Grenzen sehe ich, wenn wir dem Kind nicht mehr gerecht werden, das heißt, in Ermanglung von Zeit und Zuwendung und abhängig von dem Grad einer Behinderung beziehungsweise einer Entwicklungsstörung. & & Siehe [Rahmenbedingungen] & C3\ref{C3_29}\\
 
Inklusions\-verständnis & impliziert, dass Kinder nicht ausgegrenzt, sondern respektiert werden, dass es selbstverständlich ist, dass sie am „normalen“ Leben teilhaben, dass man ihnen etwas zutraut. Dazu gehört für uns auch, Eltern in ihrer Erziehungsarbeit zu unterstützen und den nicht-behinderten Kindern zu zeigen, dass es nicht selbstverständlich ist, dass man ohne Handicap lebt und dass die Kinder mit Handicap dafür andere Dinge können, dass ein gegenseitiges Lernen möglich sein kann, wir müssen uns nur darauf einlassen. Inklusion ist ein Prozess, sowohl für die Familien als auch für den Kindergarten & & & C3\ref{C3_33}\\

Kompetenzen & halbjährige Weiterbildung in „Sprachförderung“ (Erzieherin und Leitung) & & Austausch im Team und Erstellung einer Sprachkonzeption & C3\ref{C3_18}\\ 

 & Kollegen sind zusätzlich geschult, in Sprachförderung oder Bewegungserziehung, Ressourcen der Fackräfte werden genutzt, d.h. Angebote und entsprechende Weiterbildung abhängig von den den Interessen der Erzieherinnen & & & C3\ref{C3_34}\\

Leitungsfunktion & Akquirieren von Spenden, um Theaterbesuche zu ermöglichen & & & C3\ref{C3_47}\\
 
& zwischenmenschlichen Kommunikation und Vernetzung zwischen allen Beteiligten: Eltern, Kinder, Kollegen und mit anderen Institutionen, um fehlende Informationen beschaffen & & & C3\ref{C3_51}\\

& Ansprechpartnerin, bei Erreichen persönlicher Grenzen Fachpersonal von außen dazu holen & Team überfordert, fühlen sich nicht verstanden: „Du [Leitung] hast doch immer für alles Verständnis und sagst, wir schaffen das.“ & Erzieherinnen fühlen sich verstanden, wenn jemand von außen eigene Erfahrungen aus der Praxis mit ihnen teilt und sie sich ernst genommen fühlen in ihrer Überforderung & C3\ref{C3_68}\\
\end{longtable}
\end{centering} 

% ============================
%         Tabelle 2       
% ============================
\begin{centering}
\begin{longtable}{p{2cm}p{11cm}p{3cm}p{3cm}p{2cm}} 
\caption{Extraktionstabelle zur Auswertungskategorie „Kooperation“}\\
\toprule
Dimension & Sachverhalt (Aussagen konzentriert) & Ursache & Wirkung & Quelle\\
\midrule
\endfirsthead
\toprule
Dimension & Sachverhalt (Aussagen konzentriert) & Ursache & Wirkung & Quelle\\
\midrule
\endhead
\bottomrule
\endfoot
Kooperation mit Fachdiensten und Institutionen &
Fachkraft der Sprachheilberatungsstelle kommt ein bis zwei Mal im Jahr, ihr werden in Absprache der Eltern Kinder zur Überprüfung vorgestellt: Braucht das Kind Logopädie? Reicht die Unterstützung im Kindergarten aus oder braucht es eine zusätzliche Unterstützung in anderen Entwicklungsbereichen? & „Die meisten unserer Eltern würden es nicht schaffen einen Termin einzuhalten.“ & Gegebenenfalls Empfehlung eines Sprachheilkindergarten oder einer -schule, wenn die Problematik weder durch Logopädie noch durch Unterstützung der Kinder im häuslichen Umfeld verbessert werde kann & C3\ref{C3_21}\\

 & AWO-Frühförderstelle und die dortigen Heilpädagogen, die als Integrationshilfe in die Gruppe kommen oder Kinder in der Beratungsstelle in Spieltherapien oder Ergotherapien betreuen & & & C3\ref{C3_22}\\
 
 & Erziehungsberatung vor Ort & & & C3\ref{C3_23}\\ 
 
 & Kontaktaufnahme zu den jeweiligen Therapeuten des Kindes, das sich vor Eintritt des Kindergartens bereist in Therapie, häufig Ergotherapie oder Spieltherapie, befindet & & & C3\ref{C3_25}\\
 
 & Bildungs- und Beratungszentrum für Hörgeschädigte in [Ortsname] und der Beratungsstelle in der Blindenschule in [Ortsname] & Weiterbildung
 & & C3\ref{C3_55}\\ 

 & Voraussetzung für Inklusion: Gute Vernetzung mit zusätzlichen Fachkräften, die außen stehenden helfenden Kräfte in Verbindung zu bringen wird als Glücksspiel erlebt & Offenheit der anderen Stellen,  lange Wartezeiten für Eltern bis genaue Entwicklungsdiagnostik erstellt 
& & C3\ref{C3_42}\\

\end{longtable}
\end{centering} 


% ============================
%         Tabelle 3       
% ============================
\begin{centering}
\begin{longtable}{p{2cm}p{11cm}p{3cm}p{3cm}p{2cm}} 
\caption{Extraktionstabelle zur Auswertungskategorie „Partnerschaft mit den Eltern“}\\
\toprule
Dimension & Sachverhalt (Aussagen konzentriert) & Ursache & Wirkung & Quelle\\
\midrule
\endfirsthead
\toprule
Dimension & Sachverhalt (Aussagen konzentriert) & Ursache & Wirkung & Quelle\\
\midrule
\endhead
\bottomrule
\endfoot

Ängste der Eltern & zu wenig Zuwendung und Lernanreize, Erzieherinnen  haben nicht die nötige Zeit & Angst nicht bestehen zu können: Was passiert mit meinem Kind, wenn mein Kind nicht da „hoch“ kommt? Erwartungsdruck der Gesellschaft; Werbung, die immerzu schöne und schlaue Menschen zeigt & & C3\ref{C3_44}\\
 
Elterngespräche & Eltern fragen: „Auf welchen Gebieten haben Sie Angst, dass Ihr Kind zu kurz kommt?“ Eltern einladen, einen Vormittag in der Gruppe zu erleben oder thematisieren in Elternabenden & & &C3\ref{C3_36}\\ 

 & wenn erhöhter Unterstützungsbedarf beobachtet wird & & Manche Eltern fühlen sich verstanden, weil sie das beschriebene Verhalten von zu hause kennen. Manche sind sehr erschrocken und reagieren mit Sätzen wie: „Mein Kind ist doch nicht dumm oder behindert. Muss mein Kind raus?“ Eltern Zeit geben, nach einer Woche nochmal Gespräch suchen & C3\ref{C3_26}\\

 & dazu anregen, sich in die Lage der Eltern mit Kind mit erhöhtem Unterstützungsbedarf zu versetzen, die Ängste aller Eltern ernst nehmen, auch die kritischen Eltern & & & C3\ref{C3_309}\\ 

Entwicklungsgespräche & unter Einsatz der Methoden zur Dokumentation der kindlichen Entwicklung & & & C3\ref{C3_35} \\

Eltern\-abende & Heilpädagogen oder Sprachtherapeuten einladen und die Zusammenarbeit vorstellen & & & C3\ref{C3_37}\\

 & Heilpädagogin und Sprachtherapeutin berichten von ihrer Arbeit mit den Kindern, Sprachtherapeutin hat mit Einwilligung der Eltern ein Video von dem Kind mit Down-Syndrom gezeigt & & Aha-Erlebnis bei den Eltern: es ist nicht selbstverständlich, dass wir gesund sind. Mir kann heute etwas passieren, dann bin ich behindert und will genauso geschätzt werden wie gestern noch & C3\ref{C3_38}\\ 


& [vor 16 Jahren] Mutter berichtet aus ihrem Leben mit dem Kind mit Spastik und von ihren Ängsten & & offenerer Umgang mit der betroffenen Mutter, Offenheit überträgt sich auf die Kinder & C3\ref{C3_54}\\

Angebote für Eltern & gemeinsames Frühstück mit Kindern und ihren Eltern zwei Mal im Jahr & & hohe Resonanz bei den Eltern: oft kommen 100 Eltern; sie genießen das sehr: „Jetzt mach mal jemand etwas für uns und wir dürfen uns hinsetzen.“ & C3\ref{C3_9}\\

Stolpersteine Eltern & fehlende Offenheit, Überzeugungsarbeit bei den Eltern leisten müssen & & & C3\ref{C3_56}
\\  

%Elternarbeit hoher Stellenwert viele Gespräche. Wir haben die Anzahl an Elternabenden für alle Eltern reduziert zu Gunsten einer höheren Frequenz von Kleingruppengesprächen und runden Tischen, da in diesem Rahmen viel besser Entwicklungsgespräche möglich sind. Gemeinsame Elternabende finden statt, wenn Referenten kommen.
%Entwicklungsgespräche sind nach jedem Geburtstag eines Kindes ein Muss oder auch, wenn sich gravierende Veränderungen beim Kind zeigen. Auch beim Bringen und Holen finden Gespräche mit den Eltern statt. Je nach Bedarf kann es drei, vier oder auch fünf Elterngespräche geben oder auch nur eins.\ref{C3_73}\\ 

Berücksichtigung der Elternwünsche & Elternbeirat gibt Wünsche weiter (manchmal trauen sich Eltern nicht, den direkten Weg zu gehen) & & &  C3\ref{C3_74}\\

Informieren & Gruppentagebuch, in dem die Eltern jeden Tag nachlesen können, was die Gruppe gemacht hat, beinhaltet auch Lieder oder Texte beigelegt, die die Eltern kopieren können. &
gemeinsame Aktionen wie Feste feiern, Kinder und Eltern basteln gemeinsam Laternen oder Schultüten, Spieltage an denen die Eltern dazukommen können, um zu sehen, welche Spiele es gibt oder Wandertage, an denen die Eltern die Gruppe begleiten können, um auch zu sehen, in den Wald kann ich mit meinem Kind jederzeit gehen, das kostet nichts. Diese Aktionen sind alle in denen normalen Gruppenalltag integriert und dienen dazu, die Kommunikation zwischen den Eltern und dem Kind zu fördern. Manche Eltern sagen dann beim gemeinsamen Laternen basteln: „Das habe ich noch nie gemacht. Ich wusste gar nicht, dass ich das kann.“&\ref{C3_75}
\\ 

%Ganz wichtig ist uns auch, dass die Eltern sich nicht zu schämen brauchen, wenn sie zusätzliche Hilfe benötigen, dass sie sich nicht schämen diese anzunehmen. Viele unserer Eltern haben einen großen Packen zu tragen und sind oft auch entmutigt. Das sind persönliche und bestärkende Gespräche sehr wichtig. Solche sind sehr anstrengend und verlangen viel Kraft.\ref{C3_76}\\
\end{longtable}
\end{centering} 

% ============================
%         Tabelle 4       
% ============================
\begin{centering}
\begin{longtable}{p{2cm}p{11cm}p{3cm}p{3cm}p{2cm}} 
\caption{Extraktionstabelle zur Auswertungskategorie „Antworten auf die unterschiedlichen Bedarfslagen“}\\
\toprule
Dimension & Sachverhalt (Aussagen konzentriert) & Ursache & Wirkung & Quelle\\
\midrule
\endfirsthead
\toprule
Dimension & Sachverhalt (Aussagen konzentriert) & Ursache & Wirkung & Quelle\\
\midrule
\endhead
\bottomrule
\endfoot
Dimensionen der Vielfalt und ihre Bedarfslage & Häufung von Familien mit hoher Arbeitslosigkeit und schlechter beruflicher Perspektive, viele Familien sind Harz IV Empfänger & & & C3\ref{C3_4}\\

 & viele Kinder mit Verhaltensauffälligkeiten und Probleme im sozial-emotionalen, motorischen oder sprachlichen Bereich & & & C3\ref{C3_5}\\

  & Bedarf an verlängerten Öffnungszeiten & Mangel an Lernmöglichkeiten und Anreizen zu hause -- Mütter sagen: „Ihr könnt meinem Kind mehr beibringen als wir zu hause.“ & & C3\ref{C3_6}\\  
 
 & Wunsch der Eltern: Entlastung durch gemeinsames warmes Mittagessen der Kinder im Kindergarten & fehlende Strukturen des gemeinsamen Kochens und Essens in der Familie & Siehe [Räumlichkeiten] & C3\ref{C3_7}\\ 
 
 & & & Einmal im Monat gemeinsames Frühstück aller Gruppen mit reichhaltigem Buffet & C3\ref{C3_8}\\
 
  & & & Zwei Mal im Jahr gemeinsames Frühstück mit Eltern & C3\ref{C3_9}\\
 
 & Von den 123 Kindern haben 90 plus/minus fünf einen Migrationshintergrund & & & C3\ref{C3_10}\\

 & drei Kinder mit diagnostizierter Lernbehinderung, zwei Kinder mit Hörbeeinträchtigung, beide tragen Hörgeräte, und Kinder mit Wahrnehmungsproblematiken, bei denen noch eine diagnostische Abklärung aussteht In der Vergangenheit: Kinder mit Down-Syndrom, Spastiken, oder starker Hörschädigung oder Sehbeeinträchtigung. & & & C3\ref{C3_27}\\
  
 Angebote & 40 Kinder erhalten individuelle Sprachförderung in Kleingruppen, es werden Kinder von den Erzieherinnen vorgeschlagen, die zweisprachig aufwachsen und noch mehr Sprachanreize in Deutsch benötigen; sich nicht trauen laut zu sprechen und in ihrem Selbstbewusstsein gestärkt werden sollen, die in der Grammatik und im Wortschatz Nachholbedarf haben & & & C3\ref{C3_14} und \ref{C3_16}\\ 

 & flexible Gruppengröße und Dauer je nach Thematik.
  Einsatz von Musik und Bewegung, Trommeln und Rhythmus $\rightarrow$ größere Gruppen; Wortschatzerweiterung sehr kleine Gruppen und kurze Einheiten, da die Aufmerksamkeitsspanne der Kinder kürzer & & Musik als „Türöffner“ für gehemmte Kinder & C3\ref{C3_17}\\
  
  & Kinder lieben es mal allein was zu machen, aber auch in Konkurrenz mit anderen zu treten. Wenn mehrere Kinder in Kleingruppen zusammen sind, sitzt der Motor quasi gegenüber und dann läuft es ganz von allein, dann muss ich nicht so viel motivieren, weil die Kinder, wenn sie sich aneinander messen können, aus sich selbst heraus motiviert sind.& & &\ref{C3_30}\\  
\end{longtable}
\end{centering} 
 
% ============================
%         Tabelle 5
% ============================
\begin{centering}
%\addcontentsline{toc}{table}{Tabelle 5}
\begin{longtable}{p{2.2cm}p{8cm}p{6cm}p{3cm}p{2cm}} 
\caption{Extraktionstabelle zur Auswertungskategorie „Strukturelle Rahmenbedingungen“}\\
\toprule
Dimension & Sachverhalt (Aussagen konzentriert) & Ursache & Wirkung & Quelle\\
\midrule
\endfirsthead
\toprule
Dimension & Sachverhalt (Aussagen konzentriert) & Ursache & Wirkung & Quelle\\
\midrule
\endhead
\bottomrule
\endfoot
Gruppengröße, Gruppenstruktur und Personalstruktur & 113 Kinder im Alter von drei bis sechs Jahren werden in fünf altersgemischten Gruppen betreut  & & C3\ref{C3_1}\\

  & während der Freispielzeit offene Türen, sonst Stammgruppen &  &  zusätzlicher Gang als Antwort auf einen hohen Bewegungsdrang & C3\ref{C3_3}\\
  
 & Regelgruppen arbeiten am Nachmittag verstärkt gruppenübergreifend, da Kinderzahl der Regelgruppen variiert (abhängig von den Schulferien und den Jahreszeiten) und so die Gruppen oft kleiner sind & & Erzieherinnen können gut auf den Bedarf reagieren, ein Angebot vom Morgen noch einmal wiederholen oder ein extra Angebot machen & C3\ref{C3_12}\\
 
 & Kleine Gruppen, um speziellen Kindern wirklich gerecht werden zu können & & & C3\ref{C3_29}\\
  
 & Gruppengröße von acht Kindern und drei Fachkräften als Antwort auf den Bedarf eines Kindes mit starkem ADHS & & dem individuellen Anspruch dieses Kindes hätte entsprochen und der Wechsel in eine andere Einrichtung hätte vermieden werden können & C3\ref{C3_40}\\
 
 & Voraussetzung für Inklusion: 
\emph{„Die Größe der Gruppe und der Personalschlüssel sind das A und O!“} & & & C3\ref{C3_41}\\

 & Zu große Gruppen führen zu Überforderung der Fachkraft und machen individuelle Unterstützung unmöglich & & & C3\ref{C3_64}\\

 & Lösung: nicht mehr Personal, sondern kleinere Gruppen gesehen. & Bedarf der Kinder: ruhiges Umfeld im Kindergarten als Ausgleich für Unruhe zu hause, zu viele Erwachsene gefährden die ruhige Atmosphäre & & C3\ref{C3_65}\\

Personalschlüssel und Personal & 22 oder 23 Kinder pro Gruppe werden von je zwei Fachkräften betreut  & & & C3\ref{C3_11}\\

 & Leitung 100~\% frei gestellt, bei Personalmangel durch Krankheit springt sie ein und Durchführung der Sprachförderung & sucht Kontakt zur "Basis" & & C3\ref{C3_13}\\
 
 & Beantragen einer zusätzlichen pflegerischen Kraft für Kinder, deren Selbstständigkeit eingeschränkt ist, die Hilfe beim Essen oder beim Gang zur Toilette benötigen; umgesetzt durch eine Frau, die aus der Gemeinde gewonnen werden konnte, die zuvor in einer Arztpraxis gearbeitet hat und einfühlsam war & & & C3\ref{C3_24}\\
 
 & Mehr Personal, um speziellen Kindern wirklich gerecht werden zu können & & & C3\ref{C3_29}\\
  
Betreuungszeiten & Eltern können nach Bedarf zwischen verschiedenen Gruppen mit jeweils verschiedenen Öffnungszeiten wählen: Regelgruppen, Betreuungszeit von 7.45 bis 12:30 Uhr, über die Mittagszeit zu hause, an drei Nachmittagen von 14:00 bis 16:00 Uhr; Gruppen mit verlängerten Öffnungszeiten wahlweise von 7:30 bis 13:30 Uhr oder von 8:30 bis 15:00 Uhr & & C3\ref{C3_2}\\  

Räume und Ausstattung & Möglichkeiten zum Kochen eines Mittagessen fehlen &  & Bedarf der Eltern noch nicht beantwortet & C3\ref{C3_7}\\ 

 & ausreichend Räumlichkeiten und Geld zur Verfügung haben für entsprechende Ausstattung: zur Förderung der Körperwahrnehmung: Bälle-Bad, Therapieschaukel, Erfahrungen mit Wasser (Stolperstein: keine Möglichkeiten im Innenbereich vorhanden) und
ruhige Atmosphäre für Elterngespräche gewünscht (Stolperstein Büro, wo das Telefon klingelt) & wenn man den Orientierungsplan umsetzen und den Kindern mit erhöhtem Unterstützungsbedarf gerecht werden will & & C3\ref{C3_49}\\

 & Voraussetzung für Inklusion Barrierefreiheit: breite Türen, so dass das Kind mit seinem Rollstuhl durch kann. 
Beim Umbau des Sanitärraumes wurde eine Behindertentoilette eingerichtet & & & C3\ref{C3_52}\\

 & Stolperstein, wenn Räumlichkeiten umgestaltet werden müssen & & & C3\ref{C3_506}\\

 & Stolperstein Barrierefreiheit: Turnraum ist nur über eine Treppen zu erreichen & & & C3\ref{C3_58}\\

 & Notwendigkeit: Freizügigkeit für Kinder im Rollstuhl, dass diese barrierefrei mit dem Rollstuhl geschoben werden können & & & C3\ref{C3_508}\\
 
Politik & gesamtgesellschaftliche Anerkennung der Erzieherinnen in ihrer Rolle und ihren Aufgaben gewünscht & & & C3\ref{C3_43}\\ 

 & Geld fehlt für die Umsetzung der politischen Ideen
Aufgaben der Erzieherin: Berichte schreiben, Gespräche führen, Übersetzen → Wer aber bezahlt den Dolmetscher? & & Wir fühlen uns allein gelassen, → Bedarf an Unterstützung, um die politischen Ideen umsetzen zu können & C3\ref{C3_405}\\

 & Stolperstein Finanzierung: „Wir [die Politik] schreiben uns das auf die Fahne, aber nachgedacht haben wir nicht. Achso, dafür brauchen wir Geld? Aber das haben wir nicht.“ & Kompensation durch Angebot Sprachförderung von Leitung getragen & C3\ref{C3_45}\\
 
 & Gesellschaft und Politik schaut anders auf die Kinder & & Dadurch können Erfolge erzielt werden, da frühes Ansetzen besser ist & C3\ref{C3_53}\\

 & Druck von außen, dem Orientierungsplan entsprechen und darüber hinaus noch den Kindern mit erhöhtem Förderbedarf gerecht werden zu müssen & & Angst und Überforderung & C3\ref{C3_62}\\
 
Träger & als offen erlebt, trägt vieles mit → Gruppenstärke wurde auf 20 reduziert, da in der Gruppe eine Häufung von Kindern mit diagnostiziertem besonderen Förderbedarf war (3 Kinder mit Integrationshilfe) & & & C3\ref{C3_50}\\
\end{longtable}
\end{centering} 

\end{small}
% Ende Querformat
\end{landscape}


\chapter{Transkripte der Experteninterviews}
\section{Kindergarten 1: Diakonie}
\begin{linenumbers*}
\emph{F: Können Sie etwas zur Organisation und zur Struktur Ihrer Einrichtung sagen, welche Kinder kommen her, aus welcher sozialen Lage?}\\
A: Bei uns ist eine sehr hohe soziale Vielfalt vertreten, also Kinder aus sehr belasteten, bildungsfernen Familien bis hin zu Kindern, die aus sehr gut organisierten Familien kommen. Viele Familien haben mit dem Thema Armut zu tun, Ausgrenzung, Beschämung -- und andere, die sind ganz gut integriert. \linelabel{C1_1}

Diese Mischung ist für unsere Arbeit sehr wichtig, dass wir keine Monokultur haben.\linelabel{C1_2}

\emph{F: Können Sie auch etwas zum Anteil der Kinder mit Migrationshintergrund sagen?}\\
A: Also es sind gut Zweidrittel oder sogar ein bisschen mehr im Moment. \linelabel{C1_3}

\emph{F: Betreuen Sie auch Kinder mit besonderen Bedürfnissen, also Kinder mit Behinderung?}\\
A: Ja, wir haben auch Kinder bei uns, die seelische Behinderungen haben, geistige Behinderungen -- körperlich Einschränkungen schon auch mal, aber jetzt nicht so gravierend -- schwerbehinderte, körperlich behinderte Kinder, die haben wir jetzt weniger.\\
\emph{F: Bedingt durch das Milieu sehen Sie wahrscheinlich den Zusammenhang, dass seelische Behinderungen auftreten, Sie haben jetzt kein Kind mit Diagnosen wie Down-Syndrom?}\\ 
A: Ja, wir haben es bei uns gehäuft -- also seelische Behinderungen sind sehr oft und geistige Behinderungen kommen auch nicht selten vor oder Kinder, die Autismus haben -- also das ist schon auch bei uns, ja.\linelabel{C1_4}

\emph{F: Können Sie noch etwas zum Personalschlüssel sagen und zu der Gruppengröße. Ich sehe gerade, sie haben vier Gruppen.}\\
A: Ja, wir haben vier Gruppen jetzt hier. Das sind ungefähr 18, 19 Kinder pro Gruppe. Ich habe pro Gruppe 2,5 Fachkräfte und wir haben durch die Angebote, die wir im Familiennetzwerk haben, aber zusätzliche Menschen, die hier die Arbeit mit uns gestalten, \linelabel{C1_5}

im Bereich Naturpädagogik, Heilpädagogik, im Sprachbereich, also Logopäden, auch noch eine Spracherzieherin ist bei uns. Dann haben wir im Ehrenamt \emph{Bücherwurmfrauen}, dann haben wir Sportler, die bei uns hier die \emph{Ringergruppe} machen.\\ 
\emph{F: Die kommen in die Einrichtung und sind im Gruppenalltag präsent und integriert?}\\ 
A: Nein, nein, nein, nein -- das sind bestimmte Projekte.
Das heißt, es sind Menschen, die eine gewisse Qualifikation haben, die wir hier für die Kinder brauchen \linelabel{C1_6}

und die dann in Kleingruppen diese Angebote durchführen. Die Kinder werden denen sozusagen im Alltag zugeführt -- die haben dann ihre Ringergruppe zum Beispiel. Da kommt dann der Adolf Seger, der war zwölf Mal Weltmeister, und die Papis laden wir dazu ein und dann wird gekämpft und im Vorfeld backen die schon mit unseren männlichen Erziehern Kuchen. \emph{Männer backen für Männer} heißt das Programm und nach dem Sport gibt es Männercafé. Das sind quasi Projekte, die wir mit anderen Menschen durchführen, die da Kompetenzen haben und sie aber in den Alltag der Kinder integrieren -- das sind aber jetzt keine Menschen, die hier Gruppenarbeit im klassischen Sinne übernehmen. Das ist das festangestellte Personal.\\
\emph{F: Ich wiederhole nochmal, ob ich es richtig verstanden habe. Sie nehmen Kinder raus oder sagen die sind geeignet für die Gruppe und die treffen sich dann.}\\
A: Die, die wollen -- wo die Eltern sich zum Beispiel anmelden. Die Ringergruppe ist für das ganze Haus und das wird dann bekannt gegeben und die melden sich dann und wir nehmen aber auch Jungs rein, die keine Papas haben, die mit können oder manchmal verschwinden die Väter ja auch und die kriegen dann die Kerle. Da geht es um ein geschlechtsspezifisches Projekt und da geht es wirklich darum, dass die auch männliche Identifikationsmöglichkeiten haben, weil das einfach verloren geht. Da gucken wir und so haben wir verschiedene Gruppen bei uns -- also die so funktionieren, die aber quasi in den Alltag der Kinder sich wirklich mühelos einbinden.\linelabel{C1_7}

\emph{F: Gibt es auch Ansprechpartner, die sie hinzuziehen oder befragen, wenn Sie Fragen haben, wie die Entwicklung zu unterstützen ist, also wenn sie spezifische Fragen zur Entwicklung von einem Kind haben. Rufen Sie dann jemanden an oder kooperieren sie dann mit jemandem?}\\
A: Ja, ich meine, wir sind natürlich alle sehr fachkompetent, deswegen sind wir auch Pädagogen. Wir sind ja hier keine Bäckerverkäuferinnen. Wir haben schon eine hohe Fachkompetenz und die Entwicklung der Kinder... \linelabel{C1_8}

Bei uns ist es so, dass es ein Dialog gibt im Team, jede Woche, wirklich wie mit der Lupe. 
Und wir haben auch Heilpädagogen und Sprachtherapeuten hier im Haus, also das heißt, die sind vor Ort und da gibt es natürlich durch das Interdisziplinäre einen regen Austausch. \linelabel{C1_9}

Und wenn dann besondere Situationen darüber hinaus für uns relevant sind -- das passiert meist in Gefährdungssituationen, dann laden wir die entsprechenden Beratungsstellen oder die Gruppen wie \emph{Wildwasser}, \emph{Wendepunkt}, wenn es um Missbrauch geht -- zur Fallsupervision ein. Das läuft dann über den Weg nochmal. \linelabel{C1_10}

Wir haben auch eine Musikpädagogin im Haus. Erstmal werden unsere verschiedenen...\linelabel{C1_11}

\emph{F: Was heißt im Haus? Sind die jeden Tag da?}
A: Eben so wie sie in den Projekten arbeiten -- aber dadurch, dass sie vor Ort sind, besteht der unmittelbare Austausch mit dem Fachpersonal.
Das heißt, es sind alles Kleingruppen und es ist auch immer jemand von den Erziehern dabei -- es wird nie etwas ohne eine pädagogische Fachkraft von uns stattfinden -- dadurch besteht schon der Austausch und das wird dann jede Woche im Team gebündelt -- durch den Austausch, die Wahrnehmung. Von daher haben sich die Chancen für die Kinder, die hier sind, unglaublich erhöht, weil der Blick auf ihre Entwicklung ist hier schon sehr pointiert -- das muss man schon sagen. \linelabel{C1_12}

Und was drüber hinausgeht, wie gesagt, manchmal gibt es auch Suchtthemen -- da gibt es dann den kommunalen Suchtbeauftragten, also die Leute holen wir uns dann ran. \linelabel{C1_13}

\emph{F: Mit welchen Kooperationspartnern ist die Einrichtung vernetzt?}
A: \emph{pro familia}, Schulen, Logopädische Praxis, mit der Heilpädagogik, dann die Ärzten hier im Quartier, die Polizei im Stadtteil, Hochschulen auch. Wir haben da ein recht breites Bündnis, kann man im Grunde genommen sagen.
\emph{Südwind},
mit denen machen wir die ganzen Deutschangebote hier für Eltern -- oben ist jetzt gerade ein Deutschkurs -- an vier Vormittagen für Frauen und eine Babygruppe haben wir -- da ist schon alles sehr gut vernetzt. \linelabel{C1_14}

\emph{F: Gibt es so etwas wie konzeptionelle Überlegungen für die Kinder mit besonderen Bedürfnissen?}\\
A: Die Antwort habe ich Ihnen quasi jetzt schon durch das, was ich erzählt habe, gegeben.\\

\emph{F: Oder können Sie sagen, was Sie verstehen unter Inklusion im Bezug auf Kindergarten?}
A: Ich denke, die Idee der Inklusion ist, dass sich die Einrichtung so auch verändert, um dem Kind quasi Bildungsangebote in der Einrichtung zu ermöglichen -- also das würde ich mal klipp und klar sagen. Darin liegt natürlich die Herausforderung, weil jeder Mensch kommt so, wie er ist und hier in [Stadtteil] haben wir sehr viele, sage ich mal, einmalige Menschen. Dieses Konzept, was Sie hier schon antreffen, ist schon ein Ergebnis davon -- also das, was wir hier unter Familiennetzwerk verstehen, ist schon die Bündnisarbeit, um diese Unterschiedlichkeit, oder die Themen, die hier auftreten, noch einmal individuell zu beantworten. 
Wir haben einen großen Fachbereich nur Elternbildung. Wir haben auch in der Naturpädagogik Eltern-Kind-Gruppen. Wir machen am Wochenende Familienausflüge mit Kindern. Es besteht eine ganz enge Interaktion in die Familie -- und das sind alles schon Entwicklungen von einer Kindertagesstätte im Rahmen -- um das schon zu beantworten, um zu sehen, hier gibt es ganz viele Themen, auch Bildungsthemen für Kinder und die Eltern.
Sie müssen ja die Eltern mit rein nehmen. Sie können ja nicht sagen, ihr Kind bilden wir und sie lassen wir gerade mal links liegen. Das geht nicht. Wer mit den Kindern arbeitet, muss mit den Familien arbeiten -- und da denke ich, hat dieses Haus, und das haben Sie ja auch schon in der Homepage gesehen, ganz interessante Antworten entwickelt. Und das sind schon... Das geht in die Richtung. \linelabel{C1_16} 

Die Entwicklung dieser Antworten findet bei und in der Zukunftswerkstatt statt. Es gibt eine Zukunftswerkstatt, und da werden die Themen in der Kita gemeinsam erörtert, mit dem Fachpersonal, mit den Eltern, mit dem Träger oder auch Stellen von außen und so wird quasi die pädagogische Qualität in der Einrichtung weiter verbessert.
\emph{F: Wie oft findet so etwas statt?}\\
A: Je nach Situation. Das kann mal sehr eng maschig sein, dann kann mal wieder ein bisschen Ruhe sein -- das schadet auch nichts.\\ 
\emph{F: Haben Sie einen festen Stamm, der da immer kommt?}\\
A: Wichtig ist, dass die Plätze besetzt sind: die Elternperspektive, die Mitarbeiterperspektive, Gesetz dem Fall auch die Trägerperspektive oder was wir immer auch machen, die Gesetzes gebende Perspektive, die Wissenschaften. Das ist ein ganz, ganz sehr fein gefächerte Art zu arbeiten, wo alle Positionen zu einem Thema erst einmal aufgegriffen werden und dann -- der zweite Schritt ist -- dann wird erörtert, wo ist der nächste Handlungsschritt -- und so hat sich in den letzten Jahren dieses Haus entwickelt. 

\emph{F: Ah, mit Hilfe dieser Werkstatt.}\\ 
A: Der Hintergrund ist eine dialogische Qualitätsentwicklung und ich habe diese Qualifikation -- das heißt, dass es ein Dialogverfahren gibt, wo man gemeinsam das verändert -- also nicht hinwegsetzt und sagt: „O.k., das ist jetzt gerade in, wir machen das!“ -- so läuft das hier niemals.\\
\emph{F: Demokratische Strukturen.}\\
A: Ja, basisdemokratisch, Eltern werden beteiligt und vor allem immer geguckt, was ist hier eigentlich los. Also es nützt ja nichts, wenn ich irgend etwas mache und es greift nicht. Die Analyse spielt eine ganz große Rolle.\linelabel{C1_17} 

\emph{F: Wie der Verschiedenheit der Kinder Rechnung getragen wird, haben Sie gesagt. Können Sie noch etwas dazu sagen, ob es Strategien gibt, um Therapien in einer inklusiven Form anzubieten, wenn die von Nöten sind: Logogpädie ...}\\
A: Nein, eine Logopädie ist eine Sprachtherapie, die findet eins zu eins statt. Wir haben zum Beispiel oben einen Extraraum nur Sprache.\linelabel{C1_18}  

\emph{F: Das heißt, die Kinder werden raus genommen und die Logopädin trifft die Kinder?}
A: Ja, genau. Die holt die ab, bringt die wieder runter. Manchmal dürfen die jemanden mitbringen -- und dann haben wir ja noch eine Sprachförderkraft extra, die ist auch da oben.
Also es sind zwei Logopäden im Haus, eine Sprachförderkraft 
und die ganzen Mitarbeiter sind auch geschult im Hinblick auf Sprachförderung. Dann haben wir die Bibliothek, die Bücherwurmfrauen im Ehrenamt, die die Bibliothek machen und die auch vorlesen.\linelabel{C1_20}

Die Angebote werden quasi auch vom Bedarf ermittelt. Das heißt, es gibt möglicherweise auf einen Entwicklungsbedarf mehrere Antworten. Gerade Thema Sprache ist Nummer Eins Thema im Haus. Da gibt es Kinder im Haus, die müssen eine Sprachtherapie bekommen, andere müssen Kleingruppenangebote bekommen. Die Basis ist Sprechen und Lesen.\linelabel{C1_21}

\emph{F: Habe ich das richtig verstanden, dass die im Haus sind, aber nicht in der Gruppe oder gibt es auch Therapien, die innerhalb der Gruppe stattfinden?}\\ 
A: Also die Heilpädagogik, wenn sie eine Integrationsmaßnahme haben, die wird auch in der Gruppe durchgeführt. Es sei denn es gibt ein Stolpersteinchen, dass man sagt, das geht jetzt gerade nicht. Dann müssen die Kinder mit hoch ins heilpädagogische Zimmer. Dann muss man die halt einfach mal raus nehmen.\\
\emph{F: Was heißt Stolpersteinchen?}\\ 
A: Ja, wenn das halt gerade nicht geht.\\ 
\emph{F: Wenn die Gruppe...}\\
A: Nein, wenn das Kind sich einfach nicht wohlfühlt. Manchen Kindern geht es einfach nicht gut, die brauchen mal jemanden alleine und da muss man oben im Therapiezimmer -- wir haben ein heilpädagogisches Therapiezimmer -- da muss man da gucken, dass man sich dem Kind nähert. Aber das Ziel ist natürlich, dass das Kind sich in die Gruppe integriert, aber wir machen das halt auch so. Manchmal geht das nicht und da muss man eine Rückzugmöglichkeit haben für die Kinder, dass man denkt, jetzt sind wir zwei mal nur und gucken, was ist jetzt zuerst einmal dran oder so, dass manchmal noch ein, zwei andere Kinder dazu geholt werden, -- dass man zuerst versucht, im kleinen Setting das herzustellen, bevor man da wieder runter geht. Da muss man dann gucken, was für ein Thema hat das Kind, wie ist es gerade. Manchmal müssen auch ein paar Sachen so mal raus und die Kinder genießen das natürlich auch. 
\linelabel{C1_22}

\emph{F: Was sind Ihrem Ermessen nach notwendige Voraussetzungen für das Gelingen von Inklusion, die von oben, das heißt durch regierungsamtliche Entscheidungen, garantiert werden sollten?}\\
A: Das ist ein Riesen-Thema. Zuerst einmal die Offenheit. Die müssten ein Haufen Finanzmittel zur Verfügung stellen. Man braucht auch Konzepte, gute Konzepte, weil ich denke, Inklusion ist auch eine Frage einer dezentralisierten Politik, weil sie einfach auch im Quartier stattfindet. Wenn ich verstehe, dass die Menschen, die in diesem Quartier leben, Zugang zu den Bildungsmöglichkeiten, und zwar alle, egal, wie sie sind, dann hat das nicht nur eine Veränderung für die Kita. Das hat ganz viel auch mit Schule zu tun. Wie öffnet sich die Schule und gibt es auch Bildungsangebote für Eltern? Zum Beispiel, die Mutter kann auch nicht Deutsch, gibt es eben einen Deutschkurs, wo man sagt, dein Kind da unten, du da oben, dein Baby nehmen wir auch noch. Das sind für mich Strategien, Angebote, um die Familien in Ihrer Eigenart zu stärken, dass wirklich wir das zuführen können, was die brauchen.\linelabel{C1_23}

Die Kitas sind überschaubare Zellen und ich schätze das so ein, Deutschland weit schätze ich das definitiv so ein, dass die schon ganz viel Inklusionsarbeit leisten und das aber steil in der Schule abfällt, weil die näher an den Familien dran sind und das eher hinkriegen und auch hinkriegen wollen. Die Kitas sind da in ihrer Gesamtstruktur anders aufgestellt. Ich denke, die kriegen das eher hin. Aber die muss man ganz klar unterstützen.\linelabel{C1_24} 

Wenn man sich mal überlegt, was heißt das für Raumprogramme. Sie können ja nicht alles abreißen. Man müsste den bestehenden Baubestand so verändern, dass es halt eben geht. Das heißt, wenn man das Ernst nehmen will, muss man richtig viel Geld in die Hand nehmen.\linelabel{C1_25}
  
Und dann gibt es ja auch Grenzen. Stellen Sie sich mal vor, ein Kind, was wirklich bettlägerig ist, gepflegt werden muss, schwerer Pflegefall, vielleicht auch noch beatmet oder Infusionen hat oder wie auch immer und würde so eine Bildungseinrichtung besuchen. Da müsste ja ein Extraraum her, wo die Pflege stattfinden kann und trotzdem eine Zuführung ist ins Soziale rein.\\ 
\emph{F: Da sehen Sie Grenzen?}\\
A: Ich sage nicht, dass es nicht möglich ist, aber das würde ich gern mal sehen. Ich glaube nicht, dass ich die nächsten drei Jahre -- so etwas sehe. Also ich denke, es gibt auch Kinder, die so stark mit einem Handicap zu tun haben, dass man sich das im Alltag eher massiv schwer vorstellen kann und dann würde das bedeuten -- so etwas müsste halt eingerichtet werden, da müssten spezielle Pflegekräfte ran -- Sie müssten mal überlegen, was das kostet!\\ 
\emph{F: Sie denken, es ist grundsätzlich möglich, wenn ich Sie richtig verstehe, aber es ist ein riesiger Kostenfaktor.}\\
A: Warum soll das nicht sein, theoretisch. Ich sperre mich nicht dagegen. Ich lasse das schon offen. Aber wenn Sie fragen von oben, da denke ich einfach, das würde ich gern mal sehen.\\
 \emph{F: Das ist spannend, weil Sie gerade die Brücke geschlagen haben, eben wie ist Ihr Standpunkt zu der Aussage Allen Kindern gerecht werden, sehen Sie Grenzen und Sie haben gesagt, ja, ich sehe Grenzen, aber im Hinblick auf finanzielle Ressourcen, die nicht zur Verfügung stehen.}\\ 
A: Ja, finanzielle, auch personelle Ressourcen. Ich meine, der Fachkräftemangel ist ja eklatant, das lässt sich ja überhaupt nicht weg diskutieren. Das ist ja ganz, ganz schwierig gerade. Vorstellbar von meiner Perspektive wäre das, dass Menschen oder auch Familien Zugang haben zu solchen Häusern, wo es einfach auch anders wuselt und lebt  --und das tut eher gut, ist eher gesund, als das es schadet, sag ich jetzt mal. Aber da kann ich mir schon vorstellen, dass das enorme Hürden sind. \linelabel{C1_26}

\emph{F: Gibt es bildungspolitische Entscheidungen, die der Umsetzung von Inklusion im Weg stehen?} 
A: Ich kann das jetzt im Detail nicht so sagen, beurteilen auch. Im Moment wollen die das ja irgendwie alle. Auf einmal heißt es ja dann auch, jetzt haben die hier eine inklusive Klasse und ich glaube, man muss schon aufpassen, wo Inklusion drauf steht, ist da auch Inklusion drin. Manchmal sind das ja auch Trends, wo ich dann manchmal denke, ich weiß jetzt auch nicht, wie das ist. Zum Beispiel neulich in einem Schulmodell, da haben sie das auch gehabt, eine Inklusionsklasse und haben die aber Inklusionsklasse genannt, war aber vom gesamten Schulablauf derart isoliert, mit der wollte keiner etwas zu tun haben [lachen]. Der Wunsch, der Vaters, des Gedankens.\linelabel{C1_27} 

Ich meine, das ist auch eine Frage der Haltung. Wollen die Fachkräfte das, können sie sich das vorstellen. Diese Öffnung, das kenne ich ja hier aus dem Prozess auch, dass es alles o.k. ist, wer kommt, jeder ist o.k. und dass man sich öffnet, weil man es alleine nicht schafft. Wir brauchen die anderen Fachkräfte, die uns unterstützen. Da muss man einfach aufmachen -- als Fachkraft auch.\\
\emph{F: Ich stelle jetzt die Frage, die da dazu gehört. Welche Herausforderungen werden an das Team gestellt, wenn das Konzept in Richtung Inklusion umgestellt wird?}\\
A: Die müssen sich öffnen. Die müssen lernen mit anderen zu kooperieren und diese Fachkompetenz der anderen auch aufzunehmen und das sich zu nutze zu machen. Im Grunde muss man sich das wie so eine Mannschaft, wie ein Team vorstellen, wo unterschiedliche Positionen sind und dass man die gut dynamisiert, um letztendlich das einzelne Kind zu unterstützen. Also Arroganz ist völlig fehl am Platz, Überheblichkeit auch. Weil jeder von seinem know how her einen Blick auf das Ganze geben kann. Das sollte gut zusammenwirken.\linelabel{C1_28}

\emph{F: Was sind häufige Fragen der Erzieher, die anzeigen, dass Beratungs- oder Unterstützungsbedarf besteht?}
A: Wenn im Grunde in der Kommunikation über die Entwicklung der Kinder Situationen entstehen, die man schwer verstehen kann. Wenn man jetzt erst einmal keine Erklärung für was hat. Das sind auch oft so die Grenzsituationen. Da geht es oft schon Richtung Kindeswohlgefährdung -- weil, wie gesagt, wir sind ja nicht fachfremd -- da muss man halt einfach noch mal einen anderen Beratungsbedarf... -- oder Akkutsituationen.\\ 
\emph{F: Können Sie ein Beispiel nennen?}\\
A: Wie? Mit den Beratungs...\\
\emph{F: Kindeswohlgefährdungen ist mir klar, aber Akkutsituation?}\\
A: Wenn ein massiver Konflikt ist, der mit einem Kind in der Gruppe -- der eskaliert zum Beispiel oder auch mit den Eltern -- das kann es schon auch mal geben, dass richtig sozusagen Grenzen gesetzt werden -- oder Kinder anhaltend nicht hören können, oder ein Verhalten umsetzen können, was die jetzt von ihnen verlangt, was eigentlich als sozial angemessen empfunden wird. Dann kommen die schon auch hier her und sagen: „Jetzt ist mal gerade Schicht im Schacht.“ Oder das Kind kommt hier her, oder wir gucken mit der Erzieherin, wie kann man das jetzt wieder hin kriegen -- oder letzte Woche war ein Gespräch mit einer Mutter, die hat dann zwei Kolleginnen ausgespielt miteinander. Das haben die aber erst gar nicht gemerkt. Ja, aber die Mutter hat dann so ganz geschickt [Handbewegungen] so gemacht. Ja, das muss man dann klären. Das ist so, dass wir das hier vor Ort -- oder dass ich das in der Regel gut hinkriege mit den Kollegen -- das schon. Aber da muss man dann auch immer den Außenblick... Dadurch, dass wir so unterschiedlich aufgestellt sind, können wir das auch leisten. Da haben wir uns auch sehr verbessert durch diese ganze Familiennetzwerkarbeit und diese dialogische Qualitätsentwicklung, was wir jetzt seit zwölf Jahren schon machen. \linelabel{C1_29}

\emph{F: In dem Fall, wenn sie dann nach einer Lösung suchen, sprechen Sie natürlich mit der Erzieherinnen, aber wenn Sie sagen, sie sind so gut vernetzt, wird dann jemand hinzu geholt?}\\
A: Na klar, wissen Sie, die die vor Ort sind, das läuft eh. Da brauchen Sie sich nicht darum kümmern. Das ist installiert, das läuft hier so. Das heißt, wenn die Projekte laufen, der redet mit dem über das, das läuft schon automatisch und kommt in diesem Gesamtteam zur Sprache für alle Ohren, verstehen Sie?\\
Gesamtteam ist immer Mittwochabend. Das heißt, was unterschiedlich so läuft, bündelt sich dann da, weil wir immer die Kinder besprechen. Wir reden ganz viel über unsere Kinder, die Kinder sind den Gruppen zugeordnet, und die anderen bringen dann ihre Beobachtungen, Wahrnehmungen, Erlebnisse oder was war dort mit ein. Das vervollständigt sich sozusagen dann immer und das ist im Fluss.\linelabel{C1_30}

\emph{F: Sie sagen, es gibt eine gemeinsame Planung und Zusammenarbeit unter den Beteiligten. Mittwochabend treffen sich alle und es gibt Fallsitzungen.}\\
A: Ja, das ist elementar. Zukunftswerkstatt ist elementar. Wir haben eine ganz hohe Dialogkultur, ohne die geht hier gar nichts. Sie würden nie jemanden hier treffen, der hier eine Einzelgeschichte macht, niemals -- das gibt es nicht.
\emph{F: Und es gibt Teamsitzungen, in denen das Verhalten von Kindern regelmäßig diskutiert wird.}\\
A: Haja, wir haben nicht nur das. Wir haben noch die Kleingruppenteams, wo sie das in ihrer Kleingruppe machen und es gibt noch das Team mit der Heilpädagogik -- also quasi noch mal das Kleingruppenteam mit der Heilpädagogik, wo die nochmal genau diese Kinder besprechen oder angucken. Das ist doppelt und dreifach hier 
und das jede Woche.\\
\emph{F: Ich habe die Strukturen noch nicht ganz verstanden. Können Sie es nochmal wiederholen. Also es gibt eine Gruppe Mittwoch, da treffen sich alle.}\\
A: Gesamtteam.\\        
\emph{F: Und was bedeutet Kleingruppe mit Heilpädagogik?}\\
A: Kleingruppe heißt bei uns, wir arbeiten ja hier in Zwillingsgruppenstrukturen, das heißt, das Team aus zwei Gruppen, die entweder rechts oder links hier im Haus angesiedelt sind, die haben nochmal so ein eigenes Team, wo die auch ihre Förderungen überlegen oder die tun Gruppen übergreifend... Wir haben hier geschlossene Gruppen, wir haben kein offenes Konzept - das heißt, die überlegen sich, welches Angebot machen wir für die zwei bis dreijährigen, was können wir für die Eltern machen. Zwei Gruppen arbeiten enger zusammen. Immer fester Rahmen, jede Woche gleich. Festgelegt ist auch dieses Heilpädagogikteam.\\
\emph{F: Alle Heilpädagogen vom Haus treffen sich?}\\
A: Nein, die Heilpädagogin trifft sich mit jeder einzelnen Gruppe. Das ist ganz differenziert.\\
\emph{F: Alles klar. Die geht immer wieder rein, jede Woche, und bespricht den Bedarf.} \linelabel{C1_31}

A: Ja, genau. Oder wenn man ein Kind hat -- wir nehmen jetzt gerade wieder neue Familien auf, da wird schon gleich auch geguckt, also wie gewöhnt der sich ein, wo sind Stolpersteinchen, dass wir die relativ schnell auf den Schirm kriegen. Das sind jetzt 23 Familien, die müssen sie auf dem Schirm haben. Natürlich am Anfang, dass die hier gut reinkommen. Das wir, jeder von seiner Seite, das unterstützen. Wenn wir merken Stolpersteinchen: Wie heißt das Stolpersteinchen, wer macht das -- zack, der kommt sofort dazu.
Dieses niederschwellige Konzept, das wir hier entwickelt haben, das greift wahnsinnig und den Eltern tut das gut, weil die fühlen sich gesehen und getragen. Das ist auch so. Die sind richtig mit uns unterwegs. Wir haben das ziemlich genau im Blick -- das spüren die auch. Das haben Sie jetzt verstanden?\linelabel{C1_32}

\emph{F: Ja, das habe ich jetzt verstanden. Ich habe überhaupt keinen Einblick, wie sich das konstituiert, aber ich habe jetzt eine Idee bekommen. Brauchen Erzieherinnen Zusatzqualifikation, um inklusiv zu arbeiten und wenn ja, in welchen Bereichen?}\\
A: Bildung, Fortbildung, Weiterbildung ist lebenslänglich. Entwicklung auch.\\ 
Wir reden hier über die Haltung. Die Haltung ist das entscheidende. Bei uns im Haus ist es auch so, dass unsere gesamten Fortbildungsangebote ausschließlich sich an der Haltung der Fachkräfte orientieren. Alles andere bringt es nicht. Das heißt, wenn es Entwicklungen gibt oder Konzepte gibt, oder Perspektiven oder fachliches Wissen gibt, was hier relevant ist oder was hier Einzug hält, dann geht es erst einmal darum, habe ich das verstanden und was macht das mit mir und wie bin ich in der Lage überhaupt als Erzieherin zu arbeiten und mich zu verstehen. Da gehört die eigene Persönlichkeitsentwicklung der Fachkräfte dazu, dass die auch angeguckt wird und ernst genommen wird. Und dann natürlich auch, dass man durch diese Fortbildungen sie befähigt, mit dieser Komplexität im Alltag -- auch authentisch mit den Kindern an ihrer Entwicklung -- zu arbeiten und das ist ein Riesen-Programm.\linelabel{C1_33} 

\emph{F: Können Sie noch etwas zum Träger sagen, was der leistet im Hinblick auf Inklusion. Der Träger ist ja der, der die Zusatzausbildung anberaumt, oder?}\\
A: Nein, das machen wir autonom. Die Kita ist sehr autonom. Der Diakonie-Verein geht davon aus, dass die Fortbildung in seinen Häusern stattfindet. Es gibt einen geregelten Finanzrahmen, der uns zur Verfügung steht und wir teilen sozusagen die Themen mit, aber das wird bei uns leitungsintern in den Häusern, konzeptionell wird das geregelt und veranschlagt.\\
 \emph{F: Also, ist die Verantwortung dafür bei Ihnen?}\\
A: Total bei den Leitungen, für alles. Der Diakonie Süd West, also unser Verein, das darf man nicht verwechseln mit Diakonie Stadt, die sind noch einmal anders. Es gibt den Diakonie-Verein Süd West, der ist hier verortet und Diakonie Stadt, die haben nochmal andere Kitas. Da haben die Leitungen eine hohe -- sind sehr autonom, gestalten die ganze Arbeit, das läuft quasi von hier, auch finanziell, also das sind sehr viele Aufgaben, die dann nochmal dazu kommen.\\ 
\emph{F: Würden Sie sich was vom Träger wünschen, was der unterstützen könnte?}\\
A: Ja, die unterstützen uns inhaltlich, in dem sie uns den Rahmen bieten und auch die notwendigen Spielwiesen, um einfach auch konzeptionell sich zu entwickeln und weiter zu entwickeln. Das ist ganz arg wichtig.\linelabel{C1_34}

Wir sind hier in einem Gebiet, in einem Ballungsgebiet, wo ja alles kommt, eine unglaubliche kulturelle, soziale Vielfalt, dann eben die schweren Themen des Lebens wie Armut, auch Kindeswohlgefährdungen.
Da können Sie keine starren Verwaltungsvorschriften gebrauchen, das geht gar nicht. \linelabel{C1_35}

\emph{F: Also Offenheit, viel Raum.}\\
A: Wir arbeiten sehr professionell, und ich denke, das hat damit auch etwas zu tun, dass wir uns den Themen zuwenden können, die wirklich immer dran sind, egal ob das jetzt gerade en vogue ist oder nicht. Da steht dieser Träger hinter seinen Einrichtungen, weil die sagen, wir sind hier im Brennpunkt und wenn unsere Leute hier mit diesen Themen kommen, dann sind die dran. \linelabel{C1_36}

\emph{F: Dann stellen die auch die finanziellen Mittel zur Verfügung?}
A: Ja, Finanzen sind natürlich auch begrenzt und da sind wir natürlich auch nicht besser gestellt wie andere. 
Aber unsere Kita hat einen eigenen Förderverein und dieser Förderverein bezahlt das alles, also alle Projekte, die wir haben, bezahlt der Förderverein. Sonst könnten wir diese Arbeit nicht machen. Das heißt, wir haben eine eigene Spendenaquise, die wir auch aus dem Haus raus managen. 
\emph{F: Das managen auch Sie?}
A: Auch mit. 
\emph{F: Das hat nichts mit dem Träger zu tun?}
A: Nein.\linelabel{C1_37}

\emph{F: Jetzt stelle ich trotzdem nochmal die Frage, obwohl Sie schon hundert Sachen gesagt haben. Wo sehen Sie Ihre besonderen Aufgaben als Leitung im Hinblick auf Inklusion? Vielleicht könnten Sie es noch mal bündeln.}\\ 
A: Die Aufgabe von Leitungen ist natürlich, dass sie die Bidlungs- und Entwicklungsprozesse von Menschen, die ihr anvertraut sind, fördert. Das sind ganz klar die Fachkräfte, die Kinder, die Familien und dass die eine Antwort finden, auf ihre Alltagssituation -- im Groben gesagt. Was das bedeutet, das ist ein ganz weites Spektrum -- und dass sie eben sicherstellt, dass genau das stattfindet, was die brauchen. Das herauszufinden, gemeinsam mit anderen nicht für, mit den anderen, und dass sie alles unterstützt, ihre Mitarbeiter, die Kinder und die Eltern, dass die unterstützt werden in dem -- dass Systeme oder Hilfestellungen -- das sehen Sie bei uns im Konzept sehr gut, wenn Sie den Flyer angucken, was es alles gibt, das, finde ich, muss die Leitung einfach auch machen. Damit es die Antworten gibt. Die Menschen, die hier sind, tun sich eher schwer in die Stadt zu laufen zur Beratungsstelle, das würden die nie machen. Man kommt hier her, man hat das, man bekommt das und geht wieder.\linelabel{C1_38}

\emph{F: Also haben Sie ganz klar eine Beratungsfunktion für die Eltern, für alle drei, die sie genannt haben.}\\
A: Ja, klar. Wir sind ja auch praktisch voll mit drin, wenn es darum geht, Krisen oder so -- voll, wir steigen voll mit ein. \linelabel{C1_39}

Also, wir können das auch machen. Wir sind lange genug vor Ort. Ich selbst bin über dreißig Jahre hier, bin jetzt fast 22 Jahre die Leitung hier. Da machen Sie schon auch Erfahrungen, wissen auch genau, wie sie die unterstützen können. \linelabel{C1_40}

Oder auch Krisen mit auszuhalten. Manches muss einfach auch ausgehalten werden.\\ 
\emph{F: Kann ich mir das jetzt so vorstellen, eine Krise zu hause und Sie sind dann da, um da...}\\
A: Ja, klar. Stellen Sie sich mal vor, jemand stirbt, ein Vater stirbt. Es wird sich getrennt. Es gibt Suchtkonflikte. Es gibt, weiß der Kuckuck, was. Dann muss natürlich geguckt werden, wie kann man das jetzt stabilisieren, wie kriegt man das jetzt auf den Weg und da haben wir, wie gesagt, über diese ganzen -- auch Elterncafés, was wir machen -- immer wieder gute Möglichkeiten, dass eben auch Eltern andere Eltern kennenlernen, sich stabilisieren können. \linelabel{C1_42}

Das Sozial-emotionale ist das Stärkste, was stattfinden muss. Wenn sich eine Mutter hier nicht wohlfühlt, können Sie gehen, brauchen Sie gar kein Bildungsangebot starten, würde die nie machen.\linelabel{C1_43}  

\emph{F: Ich habe ja jetzt überlegt, die Kita wurde ja nie wirklich inklusiv umgestaltet, die war ja eigentlich von Anfang an...}\\
A: Ja, die Kita gibt es seit 1991 und die hat von Anfang an mit diesen Themen... Natürlich war die schnell unterwegs. Wir hatten von Anfang an schon Angebote, die das zum Ausdruck gebracht haben und das, was man heute antrifft, ist eigentlich ein Entwicklungsprozess einer Einrichtung über einen sehr langen Zeitraum. Das können Sie ja nicht umschnipsen wie einen Lichtschalter. Das geht nicht. Diese Prozesse, auch dass die Strukturen sich so ändern, das braucht Zeit. \linelabel{C1_44}

\emph{F: Welche Schritt konkret unternommen worden, um Inklusion so umzusetzen, können Sie die nochmal gebündelt sagen?}
A: Die dialogische Qualitätsentwicklung, die Zukunftswerkstatt, das heißt, die Eltern oder Fachkräfte -- wir überlegen gemeinsam, welche Angebote entwickelt werden. Es geht im Grunde schon darum, dass man mit den anderen gemeinsam das erörtert, sie beteiligt und dieses Haus gestaltet -- \linelabel{C1_45} 

ich meine, Kita ist eine Baustelle. Malaguzzy, nehmen Sie den Pädagogen, Sie sind mit den anderen unterwegs -- mit diesen Familien, die hier um sie herum sind.\linelabel{C1_46} 

\emph{F: Was waren und sind Stolpersteine?}
A: Stolpersteine sind natürlich im Inneren eigentlich wenig. Manchmal können es Finanzmittel sein, Raumprogramme. Wir haben ja oben dieses Raumprogramm, diesen Speicherausbau gehabt. Das war ja ewig bis das ging\linelabel{C1_47} 

und nur auf Spenden. Wir leben heute noch von Spenden. Auch die Leitungstätigkeit, die gemacht wird in diesem Netzwerk, wird nicht dotiert. Es gibt nicht einen Euro für uns. Entweder wir machen es oder wir machen es eben nicht. Aber es ist keine bezahlte Leistung.          
\emph{F: Die Leitungsqualität?}
Ich werde als meine Kita-Leitung schon bezahlt, aber zum Beispiel die Netzwerkarbeit wird nicht bewertet, die Qualitätsentwicklung wird nicht bewertet. Das, was hier Stunden frisst, die gehen pi mal Daumen davon aus und dass...\\
\emph{F: Das Zusätzliche wird über den Förderverein getragen?}
A: Nein, gar nicht. 
Entweder wir machen es so, weil wir uns so entwickelt haben oder man macht es nicht und lässt es bleiben. Das sind Hürden, dass im Grunde so etwas total gewollt wird -- wir sind ja auch sehr bekannt, auch Deutschland weit sehr bekannt -- aber die Behörden, das heißt, die Verwaltung bringt es nicht fertig das zu regeln. Hier müsste eine Dotierung her, hier müsste mindestens noch eine stellvertretende Leitung her. Das ist alles nicht. Dadurch, dass wir uns so aus uns selbst heraus entwickelt haben, weil wir das so wollten, funktioniert es natürlich auch super. Es funktioniert und es ist auch eine sehr schöne Arbeit. Aber es ist nicht geregelt und wenn wir neue Projekte machen, die laufen über den Förderverein, läuft über Spenden. Die ganze Konstruktion Naturpädagogik läuft ausschließlich über Spenden bei uns.
Das muss man sich einfach nochmal klar machen.\linelabel{C1_48} 

Das, was man hier abruft, das hat nichts mit Standard zu tun, sondern dass ist eine Entwicklung von einem Haus, das sich das so auf die Fahne geschrieben hat.\linelabel{C1_49}
 
Ich habe auch sehr langjährige Mitarbeiter, die wollen immer nicht gehen [lachen]. Das steht auch dafür. \linelabel{C1_50}

Das ist einfach so ein Garant für die Prozesse hier im Arbeitsleben mit den Kindern und Familien. Das hat uns ganz stark geprägt. Das muss man schon auch wissen, wenn man darüber redet und das andere ist -- \linelabel{C1_51}

wir werden ja auch von den Hochschulen irre frequentiert -- alles eigentlich für was, wo wir nicht dafür dotiert werden [lachen]. Das muss man sich nochmal klar machen, das ist so.\linelabel{C1_52}

\emph{F: Welches Handwerkszeug müssen Erzieherinnen mitbringen beziehungsweise was macht eine gute Erzieherin aus, die in der Lage sind das zu leisten?} \linelabel{C1_53}

A: Sich zu reflektieren. Zu reflektieren und zu verstehen, warum der pädagogische Alltag jetzt so war. Ich meine, jede Situation -- wir erfinden die ja -- Pädagogik wird ja jeden Moment neu erfunden mit den Menschen. Manchmal gibt es Situationen, die laufen super und andere laufen nicht so gut. Und die Fähigkeit das zu reflektieren und zu sagen, was war jetzt daran, dass es gut gelaufen ist oder was war beteiligt, dass es nicht so gut gelaufen ist. Nur das hilft ja auch die Situation zu verstehen und mein eigenes erzieherisches Verhalten zu verfeinern und ich muss mich im Grunde mit meiner eigenen Persönlichkeit und Biografie auseinandersetzen. Je mehr ich mich geklärt habe, Selbstklärung ist ganz wichtig, nur dann bin ich auch in der Lage anhaltend gut mit andern oder diese Kinder zu begleiten. Also wenn ich nicht weiß, ist das jetzt das Kind vor mir oder ist es mein eigenes inneres Kind, ist es die Mutter vor mir oder mein Bild von der Mutter, das ich habe. Dann führt das zu Projektionen und auch zu schwierigen Situationen, dann kriegen Sie die fast nicht mehr aufgelöst so Situationen und viele in diesem Bereich erkranken zum Beispiel an Burn out. Das ist richtig gefährlich, wenn man das nicht macht. Weil die Komplexität, weil wir immer unser eigenes familiäres spiegeln. \linelabel{C1_54}

\emph{F: Wie unterstützen Sie das als Leitung?}\\
A: Nur über den Dialog. Das heißt, man muss wenn gesprochen wird, wenn wir auch diese Teamrunden haben, wenn es Situationen gibt, die man nicht verstehen kann, warum war das jetzt so, das machen wir ja da oben, also was war, wie kann man das verstehen, was versteht man nicht, wo gibt es den Druck. Das passiert ja alles dann da oben auch. Da sieht man schon, wo die Stolpersteine sind. Das haben Sie ziemlich schnell. 
\emph{F: Und dann unterstützen Sie die Erzieherinnen sozusagen ganz nah in ihrem Weg und es nicht so, dass Sie beim Einstellungsgespräch schon sagen, die und die Qualifikationen setze ich voraus?}\\
A: Nein, das lassen wir alles. Wir sind alle im Werden. Wir entwickeln uns alle. Ich meine, wenn ich diese Tür zu mache -- es geht nicht um funktionieren hier drin. Es geht um die Einstellung, es geht darum, will ich -- will ich damit unterwegs sein und will ich auch so werden in diesem Ding, das ist eine Grundvoraussetzung und darin werden die unterstützt und das tun wir auch untereinander, auch die mit mir. 
Wenn es zum Beispiel Situationen gibt, die für mich schwierig sind, das gibt es ja immer, obwohl nach dreißig Jahren, gibt es immer noch Situationen, die schwer auszuhalten sind, die sagen mir auch... Es ist beidseitig. Es ist auf Augenhöhe. Und es tut mir auch sehr gut, eben weil wir uns kennen. Wir wissen, wo jeder seine Stärken hat und wir wissen, wo die Schwachstellen sind. \linelabel{C1_55}

\emph{F: Ist das Team also schon sehr lange miteinander unterwegs und es gab nicht viele Neuerungen, oder?}\\
 A: Doch, es gibt immer wieder Kolleginnen... Wir erweitern uns ja auch. Aber das Hauptteam, wenn Sie durchgehen würden, würden Sie staunen, wie lange die alle schon da sind.   

\emph{F: Wie werden die Eltern bei Entscheidungen mit eingebunden?}\\
A: Über den Dialog natürlich. Das heißt, wir haben viele Beteiligungsmöglichkeiten. Das pädagogische ist die Zukunftswerkstatt, wo ich schon gesagt habe und ansonsten werde die auch durch Elternveranstaltungen, dann Elternbeirat ist so ein Gremium, Förderverein, wo man mitmachen kann. Mit allen Sachen werde die miteinbezogen und angesprochen auch und ständig eingeladen. Weil wir die Eltern-Kind-Gruppen haben, die Naturgruppe zum Beispiel, also die können eigentlich ständig hier irgendwie etwas mitmachen. 

\emph{F: Und die werden eingeladen über Aushänge?}

A: Ja, über Elternbrief werden die informiert und da können die da mitmachen und sich einbringen.

Wenn die aufgenommen werden, wird denen das natürlich auch gesagt, wie wir hier zusammenleben und wie wichtig die Gemeinschaft uns ist und die haben wirklich ganz schönes Vertrauen. Die kommen auch mit allem. Das wissen die auch, wenn etwas ist, dann darf man kommen. Wenn nicht hier, wo sonst. Das ist gut. Das nutzen die auch. 

\emph{F: Die klopfen und dann sind sie da...}\\
A: „Frau [Name], haben Sie Zeit?“ Zack sitzen sie schon. Aber auch in den Gruppen zu unseren Erziehern, total... Das ist uns ganz arg wichtig!
Elterncafé, Elternnaturtage, am Wochenende die Familienausflüge, das sind alles so Sachen, wo Begegnung und Beziehungsaufbau stattfindet von der Familie zum Haus.  
 
\emph{F: Haben Sie durch die Umsetzung von Inklusion, durch das ganze Werden, Veränderungen bei sich, bei den Kollegen festgestellt, bei den Kindern?}\\
A: Ja, klar, wir haben uns alle verändert.\\ 
\emph{F: Können Sie noch ein bisschen mehr dazu sagen?}\\
A: Ich denke, dass ein ganz wichtiger Schritt...also bei uns ging der Prozess ja 99 los, also schon relativ lang, und ich denke, was verändert wurde, war, sich zu öffnen für das Andere [lachen] und andere Strukturen zu entwickeln und zuzulassen, so dass andere mit rein kommen können ins Boot, dass Erziehung nicht so eine Gruppensache ist, sondern eine Weite kriegt -- durch unseren Ansatz. Wir arbeiten nach dem Situationsansatz vom Deutschen Jugendinstitut München, und da ist ja sowieso das Gemeinwesen und alles ist da drin und das hat sich dann noch einmal verstärkt. Wir sind ja im Gemeinwesen auch sehr aktiv. Die Einrichtung strahlt in das Gemeinwesen rein. Es dürfen ja auch Menschen an unseren Angeboten teilnehmen, die nicht ihre Kinder hier haben. Man kann einfach kommen.

\emph{F: Wie werden die erreicht?}
A: Durch hier [macht eine Geste, die für Sprechen steht]. Werbung machen wir keine. Aber die Eltern wissen das und die dürfen jemanden mitbringen. Das machen die auch. So gibt es verschiedene Angebote, die dadurch sehr regelmäßig genutzt werden. Unser Code ist, was wir wollen, dass immer eine Beziehungs- und Bindungsebene stattfindet. Auch bei den Eltern. Die bringt den und die bringt den, das ist uns alles recht. Aber jetzt Werbung. Wir sind ja nicht so ein bürgerliches Ding, wo wir sagen, wir haben eine Veranstaltung, 1000 Leute kommen. Das ist bei uns nicht. Wir arbeiten mit Benachteiligten. Wir setzen komplett auf diese Bindungs- und Beziehungsebene und gehen davon aus, dass die Menschen das schon mitkriegen. Also wer hier die Tür findet, die finden die, weil die Eltern untereinander das kommunizieren. 

\emph{F: Im Prozess gab es oder gibt es Ängste unter den Mitarbeitern, welche sind das und wie wird damit umgegangen? Ich kann mir vorstellen, sich zu öffnen...}

A: Am Anfang war vielleicht auch die Angst, dass man das nicht schaffen könnte. Klar, wir haben manchmal auch gedacht: „Oh Gott, wo führt das hin? Wo führt das wirklich hin?“ Das war vielleicht schon so diese Offenheit. Es war eine Veränderung zu einer Zieloffenheit, dass wir nicht mehr so nach einem Ziel gearbeitet haben und gesagt haben, wir müssen jetzt das Ziel erreichen, sondern wir machen mehr auf der Prozessebene und haben gedacht, wenn das die Antwort ist, dann ist es die Antwort. Wir haben es dann eher akzeptiert. Dann sind wir natürlich auch an Orte oder in Projekte rein gekommen, wo wir niemals gedacht hätten, dass wir da jemals landen würden. Verstehen Sie? Wenn man nicht denkt, das muss so sein -- das Ergebnis -- das haben wir gar nicht mehr. Das ist komplett weg jetzt. Aber das kann verunsichern [lachen], weil man denkt, Oh Gott, man mach jetzt was und weiß gar nicht, wo man da hinkommt oder wo man da landet. 

\emph{F: Ja, klar, diese Offenheit, Wahnsinn...}

A: Und dann war es natürlich auch so, dass es natürlich auch mit unserem Außen was gemacht hat. Was macht denn diese Kita eigentlich? Oder es gab Angriffe, wir sollen mal in den Gesetzesordner gucken, was wir machen, steht da nicht drin. Bis der Gesetzgeber -- Jahre später -- hat das ja durch den Orientierungsplan geregelt, aber wir waren acht bis zehn Jahre in diesem Vakuum drin, auch in unserer Außenwirkung. Wo wir gesagt haben, wir machen es jetzt so, weil wir die und die Gründe haben.

\emph{F: Schön.}

A: Ja, es war auch anstrengend. Es hat auch zu Abgrenzungen geführt, weil wir gesagt haben und wir machen das so, weil die Energie oder auch der Wille so zu arbeiten sich so verstärkt hat. Sie würden heute das nicht mehr umdrehen können. Wenn jetzt hier eine Leitung käme, die sagen würde, es ist mir alles zu viel, wir machen jetzt so, das würde so nicht weitergehen können.

\emph{F: Ja, es wird von allen getragen.}

A: Ja, ich denke, wer da einmal drin war, der kann auch nicht mehr changen. Das geht nicht. Also Sie können dann nicht in funktionorientierte Strukturen rein, das macht sie krank. Ich könnte das nicht mehr. Für mich wäre das... Das meine ich ganz im Ernst. Das ist unvorstellbar sich in so etwas zu bewegen. 

\emph{F: Weil es so eng ist.
Und die bildungspolitischen Entscheidungen, die da von oben kommen, sind die nicht eine Hürde? Zum Beispiel, dass eigentlich alle de gleichen Kompetenzen haben sollen bevor sie in die Schule kommen. Sind die nicht eine Hürde?} 

A: Ich meine, das ist ja ein Märchen. Was ist, um bestimmte Anforderungen muss man sich schon sehr darum kümmern. Wir wollen hier, dass alle Kinder, egal woher sie kommen, so gut Deutsch können, dass die einfach im Unterricht gut mitkommen. Das ist zum Beispiel ein klares Ziel. Wir wollen, dass die mit anderen Kindern so gut sozial und emotional unterwegs sein können, dass sie sich gegenseitig in ihrem Lernprozess unterstützen können. Deswegen machen wir ja auch das ganze Heilpädagogische, weil das sind die größten Störfaktoren. Wenn jemand mit sich nicht gut klar kommt und nur sich reibt und rasselt, das funktioniert nicht, da kann er nicht richtig lernen. Das heißt, wenn man das kann, auch sich selbst ein bisschen steuern kann, mit seiner Wut umgehen kann, nicht gleich drauf haut, Wums, das sind alles Sachen -- und das wollen wir, dass unsere Kinder das können. Die sollen sprechen können und die sollen sozial, emotional mit anderen gut unterwegs sein können und sie sollen einfach auch gut fühlen können, was tut mir gut, was nicht, Nein sagen, Ja sagen, weil wir denken, dass sind einfach die Grundkompetenzen. 
Und die sollen natürlich auch ein gewisses Wissen haben. Gerade über die Naturpädagogik lernen die wahnsinnig viel, haben richtiges Fachwissen und auch in anderen Bereichen, Mathematik und so Sachen, Zahlenlandprojekte, wir haben verschiedene Projekte hier im Haus, dass die das schon lernen können. Das ist uns einfach wichtig, dass die dann einen guten Start haben, ein gutes Standing für die Schule und dass sie den Anforderungen auch begegnen können und das ist natürlich für so eine Kita... Da haben wir ganz schön zu tun mit dieser Unterschiedlichkeit, wo die halt her kommen. 

Und das bedeutet ja nicht nur, dass jetzt die Kinder, sondern auch die Eltern da mit können. Wir haben auch noch diese Projekt Bildungshaus, wo man auch in Zusammenarbeit mit der Schule, die Kooperation verstärkt hat und die Kinder und Eltern mehr mit einbindet. Das nimmt ganz viel Ängste weg und stärkt die einfach. Kinder und Eltern stark machen, auch die Mitarbeiter, das ist hier Programm!

F: Die letzte Frage. Finden Sie das Formen der Isolationen eine temporäre Situation sind oder sich durchziehen. Wenn Sie ein Kind anschauen, gibt es das, dass es sich durchzieht oder dass es immer mal wieder... Wo liegen da Barrieren, was passiert, damit ein Kind sich so zurückzieht? 

A: Wenn eine Kind sich zurückzieht. Also erstmal denke ich, wird das natürlich wahrgenommen und dann... ein Kind, das isoliert ist hier im Alltag, das gibt es hier nicht. Es gibt eher Kinder, die sich mal zurückziehen und abschotten, aber die bleiben da nicht drin. 

\emph{F: Also es ist eine temporäre Situation, mit der situationsabhängig umgegangen wird.} 

A: Ja.    
   
\end{linenumbers*}

\section{Kindergarten 2: AWO}
\begin{linenumbers*}
\emph{F:  Können Sie etwas zur Organisation und zur Struktur Ihrer Einrichtung sagen, also welche Kinder kommen, aus welcher sozialen Lage? Wie viele werden in der Gruppe betreut, wie viele Gruppen und so weiter?} 

A: Wir haben hier bei uns in der Kita drei Gruppen, also 60 Kinder sind insgesamt in der Einrichtung und in jeder Gruppe sind 20 Kinder altersgemischt von drei bis sechs Jahren. Teilweise nehmen wir auch Kinder mit 2 Jahren und neun Monaten bereits auf, zur Eingewöhnung. 

Die Kinder sind aus ganz unterschiedlichen Herkünften. Wir haben sowohl Kinder aus Akademikerfamilien, Eltern, die sehr interessiert an unserer Arbeit sind, die sehr bildungsorientiert ist, denen auch wichtig ist, dass wir Bildungsangebote machen und wir haben Kinder, die aus sehr schwierigen Lebenslagen kommen, wo die Eltern eher als bildungsfern zu bezeichnen sind, wo die Eltern ganz wenig Interesse an unserer Arbeit teilweise haben. Man kann es natürlich nicht immer eins zu eins sagen, aber es ist schon auffallend, dass diese Eltern selber oft in schwierigen Lebensalgen aufgewachsen sind und diese Situation zieht sich dann weiter bis in die nächste Generation. 

In Zahlen ausgedrückt, habe ich jetzt mal geschaut, wir haben 40~\% der Kinder bekommen gerade das Mittagessen durch ein Bildungs- und Teilhabegesetz, also durch einen Gutschein bezahlt, 38,33~\% haben eine Übernahme vom Jugendamt, die Beiträge werden übernommen. Da spielt auch die Höhe der Miete eine Rolle, da wird das ja immer per Antrag genau ausgerechnet, was die Familien zur Verfügung haben, aufgrund dessen berechnet sich das dann. 63~\% von allen zahlen ermäßigten Beitrag, das ist auch noch wichtig, das heißt, es gibt Familien, die bekommen zwar keine Übernahme vom Amt für Kinder, Jugend und Soziales, aber zahlen hier nur einen ermäßigten Beitrag, weil ihr Einkommen bei einem Kind unter 2267 € Netto liegt, das ist die Grenze, bei zwei Kindern ist die Summe ein bisschen höher und der Rest verdient eben oft, oft ist die Schere sehr groß, so viel, dass die den erhöhten Beitrag zahlen.

\emph{F: Das heißt, Armut ist ein Thema.} 

A: Armut ist ein ganz großes Thema, wobei es bei uns sehr gemischt ist. Es gibt AWO-Einrichtungen, da gibt es gar keine Eltern, die einen regulären Beitrag zahlen, da haben alle einen ermäßigten Beitrag. Das ist bei uns, was ich ganz schön finde, noch eine gute Mischung, wie es in der Gesellschaft ja auch ist. 

Dann ist es so, dass wir 63~\% der Kinder haben, die einen Migrationshintergrund haben, wobei es bei uns so ist, dass es sehr viele Nationen sind. Es gibt keine Anhäufung aus einem Land. Es gibt Einrichtungen, da sind sehr viele italienische Kinder zum Beispiel und bei uns ist die größte Gruppe aus einem Land Kinder mit russischem Migrationshintergrund. Da haben wir derzeit acht Familien. Aber sonst ist es sehr verteilt, zwei Kinder aus Frankreich und so weiter. 

Die meisten Kinder wachsen zweisprachig aus dieser Gruppe auf. Es gibt aber auch Kinder, die wirklich dann nur russisch sprechen und dann mit keinem Wort Deutsch hier in unsere Einrichtung kommen. Aber wir sind immer wieder erstaunt, wie schnell die das lernen, weil natürlich hier die Sprache Deutsch ist, bei uns in der Kita, dadurch lernen die Kinder das natürlich auch. 

Dann haben wir bei uns in der Einrichtung auch Kinder mit besonderem Förderbedarf. Bei uns bekommen 26 Kinder Sprachförderung momentan, also zusätzliche Sprachförderung. Natürlich ist es so, dass auch alles, was wir im Alltag mit den Kindern tun, wird verbal begleitet, das heißt, das ist alles auch Sprachförderung. Aber wir haben noch eine zusätzliche Sprachförderung, die wurde bis Sommer von der Stadt Freiburg finanziert und jetzt gibt es eine neue Regelung vom Land Baden-Württemberg, dass es vom Land die Zuschüsse gibt und das ist auch noch gekoppelt mit einer Gruppe Singen-Bewegen-Sprechen, SBS heißt es, das läuft mit der Musikschule Freiburg, wird auch als Sprachförderung vom Land finanziert, und da konnte man sich entscheiden -- die Kinder können nicht an beidem teilnehmen, entweder Sprachförderung oder dieses SBS. Da haben wir momentan, wie gesagt, 26 Kinder Sprachförderung und elf Kinder, die an dem SBS teilnehmen.  

\emph{F: Das Projekt findet hier in der Kita statt?} 

A: In der Einrichtung, in der Kita. Genau. Teilweise innerhalb der Gruppe – also diese Sprachförderung, SBS, findet immer außerhalb der Gruppe statt, weil es die Kleingruppe in der Turnhalle, die machen ja viel mit Singen, Bewegung ist dabei, Musizieren und das ist dann schwierig in der Gruppe und Sprachförderung teilweise in der Gruppe, teilweise aber auch außerhalb der Gruppe, dass nochmal Themen vertieft werden, die in der Gruppe sowieso besprochen werden und dann wird nochmal Kleingruppenarbeit gemacht. Da wird auch mit Musik und Bewegung gearbeitet, mit Spielkärtchen und solchen Sachen. 

Dann haben wir in unserer Einrichtung derzeit fünf Kinder, die über Integrationshilfe -- also da haben wir eine Unterstützung in der Gruppe, zum Teil eine Heilpädagogin, zum Teil auch noch eine zusätzliche begleitende Hilfe. Das ist dann über den § 35 a oder § 53 folgende, je nach Art der Behinderung beantragen wir das dann, und da sind die Heilpädagogen dann auch bei und in der Einrichtung. 

\emph{F: Die ganze Woche, das heißt der Personalschlüssel ist wie, in den jeweiligen Gruppen?} 

A: Die Heilpädagogen, das hat ja nichts mit dem Personalschlüssel hier vor Ort zu tun, die kommen ja zusätzlich. Die sind dann über die Integrationshilfe, zwei mal zwei Stunden in der Woche finanziert in der Einrichtung und kommen dann zu den Kindern, die eben Integrationshilfe bekommen, begleiten die Kinder und leiten dann auch die begleitende Hilfe an. Bei uns ist die begleitende Hilfe in Form von FSJ-Praktikanten umgesetzt, damit die auch möglichst einen großen Anteil der Zeit hier sind und die Kinder auch unterstützen können. 

Da haben wir, wie gesagt, derzeit fünf Kinder, wobei ein weiteres Kind auch den Behindertenausweis, sage ich jetzt mal, hat, weil das hatte Krebs und ist jetzt in der Nachphase. Wir hoffen, dass da nicht noch irgend etwas nachfolgt. Das Kind hat auch ganz große Entwicklungsverzögerungen, insbesondere im sprachlichen Bereich, braucht auch Unterstützung. Da sind wir jetzt gerade dran zu schauen, ob das Kind vielleicht in eine Sprachheilschule geht ab Sommer, das ist jetzt sechs Jahre alt und da müssen wir jetzt gerade überlegen, was dann weiter passiert. 

Dann haben wir natürlich auch Kinder in der Gruppe, um jetzt nur nochmal den Punkt abzuschließen Kinder mit besonderen Bedürfnissen, die darüber hinaus noch heilpädagogischen Bedarf haben, also die sind jetzt keine Integrationskinder in dem Sinn, aber die erhalten dann Heilpädagogik, Logopädie oder solche Unterstützung, Ergotherapie.

\emph{F: Hier?} 

A: Nein, das ist außerhalb der Einrichtung. Da ist ja leider die gesetzliche Grundlage noch so, dass man es eben nicht in der Einrichtung machen kann. 

Wir arbeiten eng mit der AWO-Frühförderstelle zusammen, die haben ja eigentlich ihre Hauptstelle in der [Obstbaum]straße, aber dadurch, dass wir unter einem Trägerdach sind, ist das Schöne, die haben ja Außenstellen in den ganzen Kitas, weil sie sonst im Osten sind und die Kitas alle hier im Westen, von der Lage her, und dadurch haben wir dann eine niedrige Hemmschwelle. 

Und natürlich können wir die Kinder dann auch sehr gut empfehlen, wenn wir feststellen, das Kind hat noch so einen besonderen Förderbedarf, jetzt nicht im Sinne von Integrationshilfe, das ist das andere, aber es sollte vielleicht zusätzlich eine heilpädagogische Förderung erhalten, dann sprechen wir mit den Eltern, ob es in Ordnung ist, dass die Heilpädagogin, die sowieso vor Ort ist, das Kind einfach mal in der Gruppe anschaut und so können wir den ersten Kontakt schon mal herstellen und die Eltern haben dadurch eine einfachere Möglichkeit, weil der Zugangsweg ist einfacher, als wenn man sagen muss, gehen sie mal irgendwo hin. 

Da denken die gleich, oh je heilpädagogische Förderung, was ist da mit meinem Kind. Und so kann man das einfach -- die sehen ja auch, kennen die Leute vor Ort und für die Eltern, glaube ich, ist es auch noch so ein Weg, weil die sehen, dass viele Kinder da hingehen, dass es für sie nichts Besonderes in dem Sinne ist oder nichts Ausgrenzendes. Da haben wir schon den Vorteil, dass wir unter dem gleichen Trägerdach sind, dass da die Zugangswege vereinfacht sind und wir auch eine gute Zusammenarbeit haben.

\emph{F: Die Heilpädagogin, ist das die, die durch die Integrationshilfe rein kommt, oder ist sowieso eine vor Ort?}

A: Die, die über die Integrationshilfe rein kommt. 

\emph{F: Sagen Sie nochmal was zum Personalschlüssel, das fehlt mir noch, so dass ich das verstehen kann.}

A: Ja, genau. Wir haben insgesamt hier pro Gruppe 3,2 Stellen, was sehr gut ist, wobei wir natürlich auch eine lange Öffnungszeit haben, die sind natürlich auch im Schichtdienst, weil wir von 7:30 bis 18 Uhr geöffnet haben. Das muss man natürlich dazu sagen, die sind jetzt nicht die ganze Zeit da. Und dann ist es nach Hauptbetreuungszeiten gegliedert, wie der Personalschlüssel ist. 

Ich selber habe eine Freistellung von einer halben Stelle als Leitung 

und wir haben auch Teilzeitleute, das heißt insgesamt sind es zwölf Erzieher, die bei uns beschäftigt sind: zwei Erzieherinnen im Anerkennungsjahr, eine Kinderpflegerin, drei FSJ-Kräfte wie gesagt über die Integrationshilfe finanziert, eine Sprachförderfrau, die ist also Erzieherin und Heilpädagogin, die macht aber wirklich nur diese Sprachförderung, die ist stundenweise angestellt für die Sprachförderung und dann eine Reinigungskraft und eine Küchenhilfe. Das ist das derzeitige Personal.

\emph{F: Und Sie sind eine halbe Stelle in der Gruppe?}

A: Nein, ich bin noch in jeder Gruppe mit zehn Prozent drin 

und habe zwanzig Prozent Migrationsanteil. Das ist auch sehr gut. Das hat die Stadt Freiburg eingeführt seit letztem Jahr, dass Einrichtungen, die einen hohen Migrationsanteil haben, das bemisst sich auch genau an der Anzahl der Kinder -- und Kinder mit Migrationshintergrund, das allein reicht nicht als Kriterium, müssen auch einen erhöhten Förderbedarf haben oder die Elternarbeit muss erschwert sein. Also erschwert im Sinne von, dass die vielleicht aus schwierigen Familien kommen und dann einfach Unterstützung brauchen für die Antragstellung von Anträgen und verschiedenen Dingen. Dafür kriegt man Migrationsanteilsaufstockung vom Personal und das habe ich jetzt in diesem Fall übernommen, weil ich sowieso in der Arbeit mit Eltern und Kindern bin. 

\emph{F: Weil das sozusagen auch einen erhöhten Bedarf an Elternarbeit erfordert?}

A: Genau. Und in der Arbeit mit den Kindern. Wir haben jetzt auch ein Elterncafé eingerichtet. Aber da kommen wir ja nachher noch dazu. Genau. 

Aber das finde ich schon sehr gut, dass die Stadt da sehr viel tut. Das kann ich auch gerade lobend erwähnen, das war ja nicht immer so und wir sind jetzt sowieso auch sehr gut personell besetzt im Vergleich zu den Jahren vorher, weil es auch über den Bildungs- und Orientierungsplan eine Aufstockung gab, das man den umsetzt, und weil sich auch die Personalbemessung nach den Hauptbetreuungszeiten bemisst. Das heißt, eine Einrichtung, die einfach längere Öffnungszeiten hat, wo die Kinder auch länger da sind, bekommt dann mehr Personal, was ja auch gerechtfertigt ist. 

Dann Heilpädagogen habe ich schon gesagt, die Erzieherin und Heilpädagogin der Sprachheilschule ist bei uns in der Einrichtung, ansonsten über Integrationshilfe. 

Ansonsten sind wir natürlich auch vernetzt mit Heilpädagogen, es gibt ja auch Eltern, wenn die Kinder zu uns kommen, die haben vielleicht schon einen heilpädagogische Förderung, die Kinder, und da sagen wir natürlich nicht, ihr müsst jetzt das belassen und zur AWO, das ist ganz klar. Und zu diesen freien Praxen oder Heilpädagogen, wo auch immer die auch hingehen, haben wir natürlich auch Kontakt, dass die mal in die Einrichtung kommen, wir uns auch austauschen 
und wir sagen den Eltern auch immer schon im Aufnahmegespräch, dass es wichtig ist für die Vernetzung, dass wir eine Schweigerechtsentbindung bekommen, dass man sich austauschen kann, weil wir natürlich wissen müssen, wie können wir die Kinder noch in der Einrichtung unterstützen oder umgekehrt, dass die Heilpädagogen auch einen Einblick haben müssen, wie ist das Kind in der Einrichtung und die besuchen uns dann auch und hospitieren hier mal in der Gruppe. Aber da finde ich schon, hat sich viel getan.

\emph{F: Das heißt, sie holen die ganz am Anfang schon ein, die Schweigepflichtentbindung?}

A: Genau. Das klappt nicht immer. Man hat das so ein bisschen im Gespür. Es gibt Eltern, die haben da Ängste und das merkt man ja. Dann kann es auch sein, dass wir das jetzt nicht immer gleich beim ersten Aufnahmegespräch machen, dass man erst mal guckt oder insbesondere auch bei Kindern, wo wir feststellen -- es gibt ja Kinder, bei denen stellt man erst fest, wenn die in der Einrichtung sind, dass die einen Förderbedarf haben, dass die Eltern das gar nicht sagen im Vorfeld und dann ist es natürlich so, dass man erst mal schaut, sind das Eingewöhnungsschwierigkeiten oder zieht sich das durch oder ist da eine Entwicklungsverzögerung, was auch immer, und dann würde man mit den Eltern sprechen, dass man erst mal Vertrauen aufbaut. Aber das hat man ja so ein bisschen im Gespür, das ist -- aber was ich immer mache, dass ich im Aufnahmegespräch sage, wie wir arbeiten, dass wir vernetzt sind, dass wir Heilpädagogen hier vor Ort haben. Also das ist schon was, was wir gleich offen sagen, aber wie gesagt, im Einzelfall muss man immer gucken.
      
Und dann gibt es natürlich auch noch Ansprechpartner, die bei Fragen einbezogen werden, wenn es jetzt um die Entwicklung von Kindern geht. Zum einen haben wir eine enge Kooperation mit der Schule, die hier bei uns um die Ecke ist. 

Da haben wir auch seit Jahren ein gemeinsames Projekt. Das heißt \emph{Schulreifes Kind}, wo es um die Chancengleichheit geht, also dass Kinder keine Zurückstellung erhalten müssen, weil Kinder, die einen besonderen Förderbedarf haben, dann nochmal extra in einer Gruppe sind. Da kommt die Frau Singdrossel, ist momentan zwei bis drei Mal wöchentlich bei uns, was zum einen die Kooperation intensiviert hat und zum anderen aber auch die Möglichkeit bietet, dass die Kinder -- also wir machen es im Moment so, dass einmal die Woche die Kinder fest in der [dortigen] Schule sind und die Frau Singdrossel ist auch Musiklehrerin, die macht dann auch was mit Musik im Musiksaal und die Kinder kennen die Schule dadurch schon ganz gut. Die Hemmschwelle ist dadurch weg und das zweite Mal kommt Frau Singdrossel hier in die Einrichtung und dann ist oft auch noch ein drittes Mal, je nachdem wie viel Zeit sie hat. Es gibt natürlich auch so, wie soll sich sagen -- also es hängt auch an dem Engagement von der Frau Singdrossel, es gibt natürlich Wochen, da hat sie eigentlich gar keine Zeit, weil sie vertretungsmäßig einspringen muss und da kommt sie am Nachmittag, das ist natürlich -- da können wir froh sein, dass sie so engagiert ist, weil die Stunden hat sie eigentlich gar nicht zur Verfügung. 

\emph{F: Dreimal pro letzten Kindergartenjahr?}

A: Ja. Pro Woche. Das ist wichtig, weil es gibt eben Kindergärten, wo es wirklich nur ein paar Mal im Jahr ist und das ist bei uns schon sehr gut. Wobei wir natürlich auch diese zusätzliche Förderung haben durch das \emph{Schulreife Kind}-Projekt, das muss ich dazu sagen. Das sind eigentlich alle Einrichtungen, die sich damals beworben haben für das Projekt -- ich bin heute froh, dass wir das gemacht haben, weil wir ja nicht wussten, was kommt da auf uns zu und das auch kritisch beurteilt wurde zum Teil von Kollegen, aber wir sind jetzt im Nachhinein froh, dass wir das gemacht haben. 

Und dann natürlich weiterhin Vernetzungen zu Ärzten, Beratungsstellen je nachdem. 

Wir hatten jetzt auch eine Schulung für das gesamte Team von \emph{Wendepunkt} und \emph{pro familia}. Da gab es ja dieses \emph{Puk}-Projekt, weil wir gesagt haben -- sexuelle Entwicklung, Missbrauch vorbeugen -- wollten wir das im Team haben und an die können wir uns ja immer wenden, wenn es Fragen gibt im Alltag. 

Dann haben wir, jetzt komme ich zwar schon zum nächsten, aber egal, also wir haben auch zu der sprachheilpädagogischen Schule Kontakt, da kommen die mindestens einmal im Jahr und machen so eine Beratung. Das heißt, da können wir Kinder in Absprache mit den Eltern vorstellen, wo wir überlegen, ist das entwicklungsbedingt noch, kann man noch abwarten oder sollte man frühzeitig mit einer logopädischen Maßnahme beginnen. Oder die haben ja auch Spielgruppen in der Schule -- das ist ja nicht weit von uns die Lautenbachschule und da haben wir auch schon Kinder rein vermittelt. Das ist auch eine gute Sache. Von daher sind eigentlich viele Ansprechpartner. Ich könnte die Liste jetzt hier noch erweitern. 

Genauso die Kooperationspartner, mit denen die Einrichtung vernetzt ist -- haben wir auch sehr viele. Ich habe jetzt schon gesagt die Schule, auch Fachschulen für Sozialpädagogik, wir haben immer wieder Praktikanten hier, dann Hochschulen, wo jetzt durch die Studiengänge Praktikanten kommen zu uns, Ärzte, Beratungsstellen. Im Grunde habe ich es jetzt schon ein Stück weit benannt.
Also ich habe jetzt leider -- ich habe mir noch Schaubild aufgeschrieben, ich wollte das eigentlich mitbringen, ich habe es leider in einem Ordner zu hause vergessen. Ich hatte letztes Jahr so ein Schaubild gemacht, dass man einfach mal sieht -- das war schon beeindruckend, das war wie so ein Mindmapping-Bild -- mit wie vielen Partnern wir vernetzt sind. Wenn man sich mal die Mühe macht, das aufzuschreiben, dann sieht man, dass es sehr viele Leute sind, die man im Alltag gar nicht im Blick hat, was ja auch wichtig ist. Ich denke, wenn ich es jetzt vergleiche mit unserer Einrichtung vor fünfzehn Jahren, das war noch nicht so. Aber das ist immer mehr im Blick von uns, was ja die Arbeit auch erweitert. 

Es gibt ja auch immer mehr Förderbedarf, muss ich leider so sagen, dass man es allein gar nicht schafft. Ich bin froh, dass wir so eine Vernetzung haben und dass wir die Möglichkeit haben mit anderen zusammen zu arbeiten. Weil wir sonst den Kindern gar nicht gerecht werden könnten. 

Wir hatten zum Beispiel auch ein Kind mit einer Sehbehinderung, da haben wir in [Wagenstadt] mit der Beratungsstelle zusammen gearbeitet. Wir haben uns dann Brillen ausgeliehen, dass die anderen Kinder Brillen aufziehen konnten, um zu sehen, wie sieht das Kind, also wie ist der Blickwinkel eingeschränkt und das war auch ein gutes Erlebnis für die anderen Kinder, die sind in der ganzen Einrichtung gelaufen, haben gemerkt, oh da sehe ich ja gar nicht, oder wie leicht man dann Stolpern kann. 

\emph{F: In der Beratungsstelle in der Gartenstraße?}

A: Nein, das ist oben in der Sehbehindertenschule, da ist eine Beratungsstelle angeschlossen gewesen und das Kind -- und da haben die die Materialien ausgeliehen, die durften wir dann ein paar Wochen in der Kita behalten, das war ganz schön. 

So gucken wir halt immer, je nachdem was für Kinder gerade bei uns sind, das wir den Kontakt herstellen. 

Dann natürlich auch im Stadtteil haben wir eine große Vernetzung. Wir haben jetzt noch die Situation, dass wir in [unserem Stadtteil] an der Stadtteilgrenze sind. Das heißt, dass wir im Grunde zu beiden Stadtteilen eine Verbindung haben, was auch wichtig ist für unsere Kinder.

Oder in dem Netzwerk Bildung und Migration bin ich mit drin und wir haben auch durch \emph{LEIF} angestoßen -- ich weiß nicht, ob sie \emph{LEIF} kennen, das ist ja \emph{Lernen erleben in Freiburg} -- eine Projektgruppe, die vom Bund unterstützt wurde, dass die Bildung von verschiedenen Kommunen gefördert wird und die Vernetzung in den Städten  -- und da war ein Modellprojekt und da bin ich auch in verschiedenen Arbeitsgruppen, Sprache zum Beispiel, da ist auch die Frau Singdrossel von der Schule drin. 

Und über diese Gelder -- haben wir natürlich auch pragmatisch es genutzt, dass wir eine gemeinsame Fortbildung finanziert bekommen haben für das Team der Schule und das Team unserer Kita im Bereich Sprache. Das war zum ersten Mal, das wir zusammen mit Lehrern eine Fortbildung hatten, was wir schön fanden. Nur als Beispiel, da gibt es natürlich verschiedene Arbeitsgruppen. Ich habe jetzt sicherlich viel vergessen zu sagen, aber das ist eben eine große Vernetzung. 

\emph{F: Klar, und so ist auch die Schwelle zur Schule gering.}

A: Hmm (zustimmend). Doch das ist uns wichtig.

{F: Gibt es für die Kindern mit besonderen Bedürfnissen, die Sie vorhin benannt haben, so etwas wie konzeptionelle Überlegungen?}

A: Ja, ich habe auch die Konzeption für Sie noch mal hier, die können Sie gern mitnehmen. Sie ist zwar nicht auf dem ganz aktuellen Stand, weil sie schon ein paar Jahre alt ist, aber... Damals natürlich hat man das noch Integration genannt, sowohl von Kindern in besonderen Lebenslagen als auch von Kindern mit Behinderung. 

Wir haben ausgehend von dem Leitbild der AWO, das ist ja im Grunde schon, die Auszüge aus dem Leitbild der AWO, die konzeptionelle Vorüberlegung, da geht es darum, dass die Leitziele Solidarität, Freiheit, Gleichheit, Gerechtigkeit zeigt es ja schon in die Richtung. 

Bei uns ist Chancengleichheit ein ganz wichtiges Thema. 
Wir haben innerhalb der AWO im Grunde eine politische Arbeit. Sie haben mit Sicherheit die Plakate, die jetzt gerade in der Stadt sind, wenn ich groß bin, werde ich arm, gesehen. 

Das ist ein Punkt, dass wir da immer an Spenden kommen, also, dass zum einen die Kinder mit einem Mittagessen unterstützen können, also Familien, die sich sonst kein Mittagessen leisten könnten, die kann man praktisch dadurch unterstützend unter die Arme greifen. Auch über dieses Bildungs- und Teilhabegesetz hinaus, da hat man jetzt die Möglichkeit einen Anteil des Mittagsessens per Gutschein zu bekommen, wenn man von Armut betroffen ist, also unter einer Einkommensgrenze liegt. Aber es ist für manche Kinder oder Familien auch noch schwierig, diese ein Euro pro Tag, das sind dann zwanzig Euro im Monat zu bezahlen und für die Familien haben wir dann die Möglichkeit über Spenden, dass die Kinder dann am Mittagessen teilnehmen können. 

[Nachtrag: Über das Bildungs- und Teilhabegesetz können Familien, die Arbeitslosengeld II, Sozialhilfe oder Wohngeld/Kinderzuschlag erhalten, zusätzlich zu Gutscheinen für Bildungsangebote (kulturelle Teilhabe) auch einen Gutschein (=Zuschuss) für das Mittagessen erhalten, so dass diese Familien nur noch 1,- EURO pro Tag dazu zahlen müssen. 
Die AWO finanziert mit Spenden zusätzlich diese 1,- pro Tag, also 20,- EURO pro Monat (Restbetrag vom Mittagessen) für manche Familien, die in sehr schwierigen Lebenslagen sind, bzw. unterstützt Familien, die keinen Mittagessensgutschein erhalten, weil ihr Einkommen knapp über der "Grenze" liegt.]

\emph{F: Dafür macht sich die AWO stark -- für die Spenden?}

A: Macht sich die AWO stark, genau. Die bekommen jetzt -- konkret in [unserer Stadt] haben wir da sehr viele Spendeneinnahmen, also wir haben derzeit -- brauchen wir 10000 Euro pro Jahr, um Familien zu unterstützen, die in unseren Kitas sind, die dann eben am Mittagessen teilnehmen können und wir kriegen auch Spenden für Bildungsangebote, weil unser Ziel ist... 

Wir haben Familien, die wie gesagt viel Geld haben, aber wir haben sehr viele Familien, die das nicht haben. 

Das geht schon los, wenn man ein Museum besucht. Wir hatten jetzt ein Kunstprojekt seit einem halben Jahr und allein der Eintritt von 3,50 € für die Museumspädagogik ist für manche Familien schlicht weg zu viel und dafür hatten wir jetzt aus diesem Bildungsfond oder Spendentopf Geld dann entnehmen können oder abrufen können, um das dann zu finanzieren. Und genauso ist es jetzt mit... 

Wir haben sehr viele Dinge, zum Beispiel auch musikalische Früherziehung, wir haben eine Tanzgruppe hier, das macht eine ehrenamtliche Tanzlehrerin, wir haben einen Schwimmkurs, der wird auch von diesen Geldern finanziert. Man fragt sich vielleicht, was hat Schwimmen mit Bildung zu tun, aber das ist eben auch ein Punkt, dass es für die Kinder wichtig ist, dass sie vor der Schule schon schwimmen können sollten und da einfach auch Familien die Möglichkeit geben, dass sie das von hier durchführen, weil die würden das von sich aus nie machen. 

Oder manche Kinder kennen nicht mal den Marktplatz in Freiburg am Münsterplatz, wenn wir das nicht von hier aus machen würde. 

Auch dieser Punkt Erfahrungsräume für die Kinder zu erweitern, die das von Haus aus nicht können, das sehen wir auch als unsere Aufgabe und zwar für alle Kinder.

\emph{F: Würde das auch Ihrem Inklusionsverständnis entsprechen, was Sie gesagt haben, Erfahrungsräume für alle Kinder erweitern?}

A: Genau. Da geht es ja meiner Ansicht nach darum, dass es eine gemeinsame Bildung und Erziehung von Kindern mit und ohne Behinderung, mit und ohne Migrationshintergrund, geschlechtsunabhängig, also egal, was für einen ethischen oder kulturellen Hintergrund man hat, Kinder mit besonderen Bedürfnissen, dass die an allen Angeboten teilnehmen sollen, um da Zugang zu haben. 

Da denke ich, ist es unsere Aufgabe Bildungsangebote zu schaffen, also Erfahrungsräume, die die Kinder von sich aus nicht hätten zu ermöglichen und da auch eine uneingeschränkte Teilhabe zu haben. 

Das ist natürlich abhängig von den Rahmenbedingung, das ist ganz klar, das kann man sicherlich nicht immer optimal im Alltag immer hinkriegen, wie man es gern würde. Aber da kommen wir sicherlich gleich noch dazu. Aber unser Ziel ist es schon und das kann man natürlich auch nur mit Vernetzung hinkriegen und indem das ganze Team dahinter steht. 

Ich denke, für mich ist immer das Wichtigste am Thema Inklusion die Haltungsfrage, also sowohl die Atmosphäre in der Einrichtung mit Wertschätzung und Achtung zu begegnen als auch dass das Team dahinter steht. 

Ich denke, dass ist nichts, wo man von oben sagen kann, jetzt machen wir mal Inklusion, sondern für mich ist es etwas, was wachsen muss, was auch unterm Team untereinander... Ich finde, es geht schon los, wie man selber miteinander umgeht, wie man Konflikte löst und dass eben, wie gesagt, auch das ganze Team dahinter steht. 

\emph{F: Ich mache mal an diesem Punkt weiter. Welche Herausforderungen werden an das Team gestellt, ist eine Frage, wenn das Konzept in Richtung Inklusion umgestellt wird? Ich weiß nicht, ob man hier davon sprechen kann, es wurde umgestellt, sondern vielmehr ein Prozess über Jahre, der sich...}

A: Ja, würde ich so sehen. Es ist natürlich auch bei uns... Das ist schon vor zehn Jahren gewesen, da hatten wir ein Kind, eine Anfrage ein Kind aufzunehmen, das eine Behinderung hat, und wir waren ja immer eine Regeleinrichtung, wie gesagt momentan haben wir fünf beziehungsweise sechs Kindern, wenn man das Kind noch dazu zählt, das Krebs hatte, aber wir haben dadurch einfach die Frage -- wir standen vor der Frage, könnten wir das hier hinkriegen in der Einrichtung? 

Haben dann geklärt, was für Voraussetzungen -- was brauchen wir, um das Kind aufzunehmen und ich kann sagen, dass Team war gleich offen, was ich ganz schön fand. 

Natürlich gab es im Alltag immer noch Problem, dass man dann überlegt hat, wir wollen jetzt einen Ausflug machen, können wir das jetzt, wenn wir einen Buggy mitnehmen müssen für das Kind, das schlecht laufen kann, wie kriegen wir das jetzt hin? Da gab es schon immer wieder auch Problemsituationen im Alltag, wo man auch überlegt hat, wie kriegen wir es personell hin und wir waren damals noch viel schlechter besetzt. 

Also heute sind es ja traumhafte Zustände in Anführungszeichen. Das ist schon so. Aber ich denke, auch damals schon war klar, dass es erst mal eine Frage ist, das Team muss dahinter stehen -- also erst mal die Haltung, was ich eben schon gesagt habe, 

dann dass man Fortbildungen macht, sich weiterbildet, dass man da nicht stehen bleibt, finde ich auch ganz wichtig. Auch neue wissenschaftliche Erkenntnisse, egal ob jetzt im Bereich Bindung, Hirnforschung und so weiter -- also das ist immer so ein wichtiger Punkt. 

Konzeption, das haben wir ja damals sowieso gerade erarbeitet, da haben wir das eben mit rein genommen. Es war zum Teil natürlich auch durch die gesetzliche Vorgabe, wenn man Integrationshilfe beantragt, ist es ja auch so, dass es in der Konzeption auch verankert sein muss. Es muss auch klar sein, wie -- wird das Kind hier in der Einrichtung gefördert, kann man dem auch gerecht werden und wenn man dann einen Hilfeplan erstellt, dann steht da auch ganz konkret drin --  man kann ja nicht einfach nur ein Kind aufnehmen und mitlaufen lassen, sondern es muss ja auch gewährleistet sein, was passiert denn auch in der Einrichtung alles. 

\emph{F: Sie erstellen den Hilfeplan?} 

A: Nein, im Zusammenarbeit mit dem Jugendamt -- ist ja jemand dabei und die Heilpädagogin ist auch an dem runden Tisch. Also wenn man Integrationshilfe bekommt, ist es immer so, da sind die Eltern dabei, Heilpädagogin, Kita und eben Jugendamt und dann wird der Hilfeplan erstellt, weil dann jeder sagt, was braucht das Kind, welche Unterstützung im Alltag, was wäre wichtig für das Kind noch zusätzlich. 

\emph{F: das habe ich mich schon immer gefragt. Ich habe einiges darüber gelesen, aber mir war nie klar, wer den Hilfeplan nun wirklich schreibt.}

A: Federführend ist da das Jugendamt, die stellen da die Fragen -- aber die können es ja alleine nicht. Die kennen ja weder das Kind noch die Einrichtung. 

\emph{F: Die wollen den Bedarf ermitteln und...}

A: Ja, genau. Dadurch ist es auch immer gut das an einem runden Tisch zu machen, weil dann jeder was sagen kann: „In dem Fall ist das noch wichtig.“ Die Eltern wissen ja, wie das Kind im Alltag -- was das für einen Bedarf hat und die Heilpädagogin sieht es natürlich aus ihrer Sicht, was kann sie individuell noch für das Kind jetzt tun -- fördermäßig. 

Also das heißt Konzeption -- natürlich Bild vom Kind ist darin enthalten, Erziehungspartnerschaft, das ist uns auch immer ganz wichtig die Zusammenarbeit mit den Eltern. 

Dann -- für mich hat es auch immer noch was mit Werten zu tun. Also was für Werte man vermitteln will überhaupt in einer Einrichtung. Da kommen wir natürlich wieder auf die Zielsetzung oder Konzeption -- also gerade das Leitbild der AWO gibt es im Grunde schon vor. 

Das ist auch so ein Punkt, was man sich im Team auch immer wieder bewusst machen muss, wo soll eigentlich alles hin gehen und dann finde ich, kann man eigentlich gar nicht anders als daraufhin zu arbeiten. 
Ich würde jetzt nicht sagen, dass wir eine inklusive Einrichtung sind. Das ist immer ein Prozess oder ein Weg dahin und ich sage Ihnen auch gleich noch, wenn 
wir zu den Rahmenbedingungen kommen, wieso. 

Aber was Strukturen, Vernetzung anbelangt und da finde ich hat sich schon viel getan. 

Und dann ist es auch immer noch, weil Sie nach den Herausforderungen gerade gefragt haben, eine Zeitmanagementfrage ---- (Unterbrechung, Klopfen an der Tür). Ich finde so -- es ist ja auch die Frage, dass man Zeit haben muss für einen Austausch, einmal untereinander im Team -- Besprechungen haben muss, also das stellt noch mal höhere Anforderungen, wenn man Kinder mit besonderen Bedürfnissen hat und das ist zum einen 
meine Aufgabe als Leitung natürlich diese Zeit zur Verfügung zu stellen als auch das ins Team rein zu bringen manche Themen oder dann auch Zeit zu geben für Vernetzung, auch in Arbeitskreisen aktiv zu sein. 

Ich finde, wenn man nur als Einrichtung so isoliert arbeiten würde, wäre es auch schwierig, also dass man sich sowohl im Stadtteil als auch in Stadtteil übergreifenden Arbeitskreisen engagiert, dass man die neuen Themen überhaupt erhält, dass man weiter -- sich weiter bildet, Fortbildungen macht. 

\emph{F: Und die ins Team trägt.}

A: Die ins Team trägt oder eben auch -- also was ich immer mache, wir haben ja verschiedene Projekte, Arbeitsgruppen, da bin ich immer dabei und immer einer aus dem Team. Da ist immer noch einer mitverantwortlich. Erstens ist es dann einfacher das ins Team zu tragen und dann ist es auch – auf zwei lastet die Verantwortung, auf zwei Leuten (lachen). Das finde ich immer schöner.

\emph{F: Ist das eine feste Person oder...}

A: Immer eine feste Person. Wir haben jetzt zum Beispiel -- fürs Elterncafé ist einer zuständig und die ist dann auch mit mir in einem Arbeitskreis mit drin, wo es um Migration geht 
oder wir wollen jetzt irgendwann eine U3 Gruppe eröffnen, das ist so ein Projekt von uns, was wir hoffen, dass es irgendwann klappt. Da ist dann auch jemand, der die Fortbildung jetzt schon gemacht hat. Dann haben wir eine Gesamtteam-Fortbildung im U3 Bereich -- also solche Dinge. 

Ich habe jetzt selber an einer Antibarrier's Fortbildung teilgenommen -- ein ganzer Tag -- die vorurteilsbewusste Erziehung, kennen Sie ja und das hat mir sehr gut gefallen und ich habe jetzt -- da wir noch Gelder hatten für den Bildungsplan für Fortbildungen haben wir jetzt einen Tag am 2.11. – das ist jetzt dieser Brückentag, und da haben wir fürs ganze Team eine Fortbildung. 

Und das ist, glaube ich, schon die Aufgabe von mir als Leitung zu gucken, was brauchen wir und wir haben in letzter Zeit die Erfahrung gemacht, dass es mehr bringt, wenn das gesamte Team an einer Fortbildung teilnimmt als wenn jetzt einzelne an einzelnen Fortbildungen, weil das Transportieren in das Team ist, natürlich weil wir noch viele andere Sachen zu besprechen haben, Kinder, Fallbesprechungen und so weiter, nicht so einfach. Deshalb ist es gut -- natürlich kriegt man es nicht zu allen Themen und immer hin, aber das ab und zu zu Themen zu machen, ist eben sehr wichtig. 

\emph{F: Was sind Ihrem Ermessen nach notwendige Voraussetzungen für das Gelingen von Inklusion, die von oben garantiert werden sollten? So können wir dann vielleicht nochmal zu dem Träger kommen.}

A: Zum einen natürlich die Rahmenbedingungen, es fängt an bei der personellen Ausstattung, räumlichen Ausstattung, Material, technische Hilfsmittel, dann die bauliche Ausstattung, Zeit zu haben für Besprechungen, Fortbildungen zu ermöglichen, sowohl Zeit als auch Geld, was es dafür braucht. Dass es auch in der Schule fortgeführt wird, dass man sich vernetzten kann. 

Eigentlich bräuchten wir multiprofessionelle Teams, finde ich. So auf Dauer, dass auch wirklich Heilpädagogen fest angestellt sind in der Einrichtung, das wäre natürlich toll, oder Logopäden, aber ich glaube da ist es noch ein Weg dorthin, einfach auch aus finanziellen Gründen. 

Aber ich finde es schon gut, dass wir einfach vom Team aus mehr Personal haben, sodass man auch innerhalb des Teams gucken kann, in welchen Bereichen können sich einzelne nochmal spezialisieren, durch Fort- oder Weiterbildungen. 

Es ist natürlich schon so ein erster Ansatz, aber wie gesagt auf die Zukunft denke ich, wäre es gut da noch eine größere Professionalität reinzubringen, indem man Heilpädagogen fest in Team hätte. 

Dann einfach auch, was ich denke was wichtig ist, Struktur und Rituale in der Einrichtung. Ich persönlich bin zum Beispiel der Meinung, ich halte zwar auch etwas vom offenen Konzept für ältere Kinder, aber ich sehe so für die jüngeren Kinder oder gerade für die  Kinder aus besonderen Lebenslangen oder mit Behinderung ist es ganz wichtig so eine Struktur zu erleben, Familiengruppen, Bezug zu den Erzieherinnen zu haben, da bin ich ein bisschen kritisch dem offenen Konzept gegenüber, muss ich ganz ehrlich sagen. Wobei das sicher auch Vorteile hat für ältere Kinder im Bildungsbereich, aber wir haben es jetzt hier so, dass wir die Gruppen fest haben und trotzdem gucken – übergreifende Projekte, die Kinder dürfen sich gegenseitig Besuchen, aber das ist für mich ein wichtiger Punkt. Dann natürlich wieder, dass die Konzeption da sein sollte und die Haltung...

\emph{F: Die von oben quasi.}

A: Konzeption, da würde ich jetzt sagen, ich fände es wichtig, dass jede Einrichtung eine Konzeption haben muss, jetzt nicht von oben vorgeschrieben, so und so muss es sein, sondern, dass es eine gibt, die aber natürlich vom jeweiligen Team dann, geht meiner Ansicht nach gar nicht anders, formuliert oder ausgestaltet werden muss. 

Und die Haltung ist natürlich auch eine Frage. Klar ist die Frage, wie kann man das von oben – kann man natürlich nicht eindoktrinieren, aber ich finde zumindest muss das als Stichpunkt klar sein, dass man sich als Team Gedanken macht. 

Wobei jetzt gerade zum Thema Rahmenbedingungen, ist natürlich ein wichtiger Punkt – wir haben jetzt eine gute personelle Ausstattung, aber da finde ich muss dann natürlich auch im Alltag hingucken. Im Grunde müssten die Gruppen generell kleiner sein. Wir haben jetzt zum Beispiel in einer Gruppe die Situation, wir haben ein Kind, was eine Frühgeburt war, wo klar war, es hat Entwicklungsverzögerungen, haben wir aufgenommen, haben auch gleich Integrationshilfe beantragt, es hat auch alles gut funktioniert, hatten dann aber die Situation, dass wir ein zweites Kind, wovon wir es nicht wussten, also es wurde aufgenommen und es hat eigentlich fast die gleichen Schwierigkeiten in der Gruppe, kann auch ganz schwer Kontakt aufnehmen, hat also eine schwierige Ausgangssituation – also ich will jetzt nicht über das Kind, das würde zu weit führen -- aber -- und haben jetzt noch ein drittes Kind in der gleichen Gruppe, was jetzt gerade auch aufgenommen wurde. Manchmal ergibt es sich zufällig, dass man mehrere Kinder in einer Gruppe hat, die ganz besonders schwierige Lebenslagen, schwierige Ausgangssituation haben und dann wird es natürlich schwierig, da kann man noch so gut personell besetzt sein, weil die Kinder dann wenn sie in einer – noch dazu großen Gruppe, man muss ja den anderen Kindern auch noch gerecht werden – dann zusammen drin sind. Da muss man immer gucken, kann man das jetzt noch leisten, kann man sowohl den Kindern im einzelnen gerecht werden, die diesen Förderbedarf haben, als auch den anderen Kindern. Da finde ich, kommt man dann wirklich auch an seine Grenzen.

\emph{F: Und was würden Sie jetzt denken, was braucht es konkret um da besser...}

A: Genau, es ist natürlich jetzt nicht so einfach zu sagen. Da würde ich denken, generell wäre es toll, wenn man einfach kleinere Gruppen hätte. Weil man dann – da kann man noch so viel Personal haben, wenn die Gruppe so groß ist. Dann wird das halt schwierig. Und natürlich ist es auch immer schwer zu händeln, man muss natürlich in Einzelfall, kann es dann sein, dass man sagen kann, wir können das hier in der Gruppe nicht mehr leisten, so dass man noch ein weiteres Kind aufnimmt, dass man im Zweifel gucken muss – kann man da durch einen Tausch – dass das Kind in eine andere Gruppe vielleicht innerhalb der Einrichtung wechseln kann. 

Oder im Zweifelsfall – wir hatten auch schon eine Situation, wo es ganz schwierig war, auch die Zusammenarbeit mit den Eltern. Das kommt zum Glück selten vor, aber dass man – so hart es klingt – auch mal darüber nachdenken muss, ob das Kind in der Einrichtung richtig aufgehoben ist. Das klingt jetzt vielleicht so -- ja – man will einerseits Inklusion, aber man muss wirklich auch gucken – es gibt natürlich Einrichtungen, da werden dann mehr Kinder angemeldet, weil man weiß, die sind offen, die machen das und da wird das nicht gleich gesagt: „Nein, wir wollen keine Kinder mit Behinderungen aufnehmen, oder wir wollen nicht so viele Kinder Migrationshintergrund.“ Und dann ist es natürlich so, wenn in den Einrichtungen sich es dann sehr häuft, dass man viele Kinder aus schwierigen Lebenslagen hat, dann ist es schwierig im Alltag damit auch umzugehen und dem einzelnen Kind noch gerecht zu werden. 

\emph{F: Hatten Sie den Eindruck, dass diese Familie woanders hätten besser aufgehoben sein können.}

A: Also bei einem Kind, das war in einer anderen Einrichtung und ist dann zu uns gekommen, aber ich wusste nicht, dass diese Schwierigkeiten im Hintergrund sind. Natürlich macht man sich da Gedanken, wie es zustande kam, sage ich jetzt mal so... Deshalb denke ich, wäre es vielleicht generell besser, wenn die Gruppen einfach kleiner wären, also gerade wenn es vom Weg her so ist -- ich finde es gut, wenn wir Inklusion in jeder Einrichtung haben, aber dann ist wirklich auch die Frage, wie kann man die Bedingungen von vornherein schon verändern. Das ist für mich so der Punkt, dass man da nicht Kinder, die schon da sind, wegschicken muss, weil man es einfach nicht bewältigen kann im Alltag. 

\emph{F: Gibt es bildungspolitische Entscheidungen, die der Umsetzung von Inklusion im Weg stehen. Bildungspolitisch ist ja wieder von oben, gibt es quasi Auflagen, auch durch den Orientierungsplan, bei denen Sie das Gefühl haben, dass diese mit dem Wert Inklusion kollidieren.}

A: Ja, zum einen fand ich es sehr spät, dass überhaupt diese UN-Behindertenrechtskonvention überhaupt mal in Kraft gesetzt wurde, das ist mal das erste und dann ist es natürlich schwierig manchmal in der Umsetzung, wenn man Chancengleichheit verwirklichen möchte, dass man dann wirklich jedem Kind gerecht wird. Dazu brauch man eben Rahmenbedingungen. 

Der Bildungsplan, den finde ich generell gut und der gibt ja hier in Baden-Württemberg auch sehr viel Freiraum, was ich gut finde, dass es nicht so festgelegt ist, dass man schauen kann, aber es ist natürlich nicht immer so einfach im Alltag wirklich den individuellen Bedürfnissen des Kindes dann auch gerecht werden zu können. Da bräuchte man ja fast pro Kind eine pädagogische Fachkraft. Wir versuchen natürlich die Bildungsthemen der Kinder aufzugreifen, die auch individuell zu fördern, aber es ist schon mal so – man kann das, muss man ganz klar sagen, im Alltag nicht für jedes Einzelkind so hinkriegen. Natürlich versucht man das und kann es auch punktuell, aber – wir haben zum Beispiel immer Projektthemen, wo man auch versucht die Themen der Kinder zusammenzufassen, also dass man jetzt, was viele Kinder in der Gruppe beschäftigt, aufgreift, natürlich auch noch Themen von außen dazu nimmt, aber es gibt mit Sicherheit einzelne Kinder, die sich nur für ein Thema interessieren, denen man in dem Maße vielleicht nicht gerecht wird, wie es für dieses Kind gut wäre.  
Aber ich denke, das ist in der Gesellschaft ja immer das Problem, von dem her (lachen) – ob man das überhaupt lösen kann, ist die andere Frage.

\emph{F: Können Sie nochmal zusammentragen, wie Sie konkret der Verschiedenheit der einzelnen Kinder gerecht werden?}

A: O.k. Wir arbeiten hier nach den Bildungs- und Lerngeschichten von, ich weiß nicht, ob Sie das kennen, von Neuseeland und das heißt, wir beobachten die Kindern, wenn die sich jetzt zum Beispiel mit irgend einem Thema besonders beschäftigen, beobachten wir die, schreiben das auch auf und versuchen dann aufgrund dieser Beobachtungen den Kindern zum einen eine Rückmeldung zu geben – mit dem wir Lerngeschichten, über das, was ein Kind lernt oder womit es sich besonders beschäftigt, was das Kind besonders interessiert, dann dem Kind einmal eine Rückmeldung zu geben und das auch zu dokumentieren in diesen Portfolio-Ordnern, damit man so einen Ordner hat, dass das Kind dann sieht, wie die Bildungsentwicklung dann war. Aufgrund dieser Themen des Kindes versuchen wir dann dem Kind weitere Angebote zu machen. Also zum Beispiel, was weiß ich, wenn ein Kind jetzt zum Beispiel sich mit Steckerle gern beschäftigt und da sich ganz arg drauf konzentriert, dass wir dem Kind dann eine Rückmeldung geben und sucht, was kann man dem Kind an feinmotorischen Dingen weiteres zur Verfügung stellen. Also so jetzt mal -- ein kleines Beispiel.

\emph{F: Also interessenorientiert?}

A: Interessenorientiert und auch stärkenorientiert, das man die Stärken aufgreift und natürlich auch im Blick hat mit anderen Beobachtungsverfahren, wo hat das Kind vielleicht noch Schwächen, also dass man praktisch ansetzt an den Stärken, die Stärken stärkt und die Schwächen schwächt, sag ich jetzt mal so (lachen). 

Dass man dann auch versucht Dinge noch mit rein zu nehmen, wo das Kind noch eine besondere Förderung benötigen würde. Das ist natürlich ein sehr hoher Anspruch, das ist auch mit sehr viel Dokumentation, mit sehr viel Zeit verbunden. Dann gibt es noch kollegialen Austausch darüber und da ist natürlich, da bin ich mir sicher, auch wenn wir uns noch so bemühen, wird es immer Situation geben, wo wir Kindern doch nicht so gerecht werden, wo man im Alltag eben auch noch andere Kinder hat, wo man Eltern hat, die in der Tür stehen, die was Wichtiges mitteilen wollen und wir haben jetzt schon eine gute personelle Besetzung, ich denke, wir kriegen es besser hin als noch vor fünf oder zehn Jahren, aber es ist natürlich nicht optimal. Es gibt immer wieder auch Stresssituationen im Alltag oder es ist einer krank, wo man dann das nicht machen konnte, was man eigentlich geplant hat und so weiter. Deshalb denke ich optimal läuft es sicher noch nicht. 

Und genauso ist es natürlich auch, dass man dann bei Kindern, die nochmal einen besonderen Förderbedarf haben, dann auch besonders viel Zeit vielleicht auch braucht und von daher es nicht immer so hinkriegt, wie man gern würde. ...es gibt zum Beispiel auch Dinge, wir führen ja einen Schwimmkurs durch, da ist es zum Beispiel auch so, dass wir da ein Kind, was eben vom Körperlichen her nicht mitschwimmen konnte, da haben wir dann trotzdem versucht – also das Kind war dabei und wir hatten dann eine Begleitung auch im Wasser für das Kind, aber trotzdem – das Kind hat schon gemerkt, es kann nicht so mitmachen, wie die anderen Kinder und das ist schon mal so ein Problempunkt, eigentlich möchten wir Kinder alle integrieren, dass das einzelen nicht das Gefühl hat in einer besonderen Situation zu sein und das ist manchmal schon schwierig dann im Alltag. Also es gibt einfach Angebote, da merken das die Kinder einfach, da kann man es noch so gut versuchen hinzukriegen. Wir haben in dem Fall die Kinder mit dem Bus abgeholt, zum Schwimmbad gefahren, es waren eigentlich die Voraussetzungen ganz gut, man musste dann nicht lange laufen und so, aber trotzdem...

\emph{F: Die Teilhabe ist das Problem.}

A: Die Teilhabe ist das Problem, ja, genau. Es sind schon Situationen, wo man an seine Grenzen kommt, sage ich jetzt mal so.

\emph{F: Ich habe letztens ein Zitat gelesen, ich meine das passt da jetzt nicht ganz... Sie hat gesagt, dass sie vor zehn Jahren noch gedacht hat, dass doch die gesunden Kinder die anderen fragen sollen, also Kinder mit besonderen Bedürfnissen, fragen sollten, ob sie mitspielen wollen, aber dass sie merken, dass sie jetzt doch irgendwie die Situation so attraktiv gestalten müssen, dass da ein Kontakt hergestellt wird. Das Beispiel war, das fand ich sehr bewegend, dass die... Also es war ein Kind, das war motorisch auch eingeschränkt und es war ein warmer Tag im Sommer, die haben einen Pool draußen aufgebaut auf der Wiese und das wäre zu wild gewesen, auch zu laut, da haben sie einen kleinen daneben und 30 Korken ins Wasser und das war so anziehend, dass dann eben drei Kinder gekommen sind und mit dem Kind gespielt haben. Aber ich habe jetzt genau eine Idee, was sie meinen, dass es dann so ein Gefühl ist von Barriere und „Ich bin doch nicht ganz dabei.“} 

A: Genau. Oder ein Kind, das vielleicht dadurch, dass es nicht laufen kann, im Buggy geschoben werden muss, die anderen Kinder können beim Waldausflug dann rennen, können in den Wald rein rennen, Dinge holen und das Kind hat dann doch die Unterstützung von dem Erzieher, der dann mit dem Buggy hinterher fährt und kann halt doch nicht über jede Wurzel so schnell hinterher wie die anderen Kinder und das sind Situation, wo die Kinder dann auch spüren, sie können halt doch nicht so ganz ohne Hindernisse dabei sein. Natürlich bemüht man sich da, aber... Das tut einem manchmal dann auch Leid, aber man hat dann oft auch keine Idee, wie könnte man manche Dinge dann verändern.

\emph{F: Vielleicht wenn die großen Kinder schieben dürfen -- das Rennauto...}

A: Ja, klar, aber das ist wie gesagt dann auch -- da muss man ja dann wieder von der Sicherheit her Gefahren ausschließen, gerade im Wald, wenn das dann umkippen könnte der Buggy, da zwischen den Bäumen durch...

\emph{F: Ja, ja.}

A: Ich meine, das ist schon, da bemühen wir uns auch und da sind die Kinder auch von sich aus -- die haben das ganz gut im Blick. 
Ich kenne das auch wiederum anders herum, wo wir uns dann eine Stunde lang in der Teambesprechung Gedanken machen, wie könnte das Kind gut integriert werden oder wie wird es nicht ausgeschlossen -- also für die Kinder ist es dann oftmals kein Thema, die erklären das dann ganz locker, wenn dann ein Kind etwas fragt, was neu ist: "Weißt du, der kann jetzt mit der Hand das nicht rüber machen, das musst du jetzt machen." Das ist dann ganz unkompliziert und man selber macht sich Gedanken: "Oh, hoffentlich passiert da jetzt nichts, dass das Kind ausgegrenzt wird und für die Kinder ist es dann oftmals gar kein Thema. 

Also da merken wir dann natürlich auch, dass wir genauso unsere Barrieren im Kopf haben oder wir uns manchmal sogar zu viele Gedanken machen und manche Dinge selber sogar verkomplizieren. Aber auch das, denke ich, ist wieder so eine Frage, wo man bei Haltung ankommt und dann einfach auch merkt, dass ist eben das, was man selber mit rein 
bringt 

oder was ich auch immer ganz wichtig finde, so selbstreflexive Dinge, also das man sich auch im Team Gedanken macht, was kommt aus der eigenen Vergangenheit, dass man vielleicht sich selber auch klar machen muss, dass vieles ganz anders ist als man vielleicht im Kopf hat, gell? 

Aber deshalb finde ich es jetzt auch schön, dass wir uns jetzt diesen Tag nächste Woche auch haben, wo man sich noch einmal Gedanken machen kann und das ganze Team dann auch dabei ist. Das finde ich auch immer schön dann.

\emph{F: Sie haben jetzt auch die Frage, was macht einen guten Erzieher aus, welches Handwerkszeug müssen Erzieherinnen mitbringen, schon beantwortet, mit der Haltung, der Selbstreflexion.}

A: Ja, genau, Haltung, Wertschätzung, Achtung, Selbstreflexiviät, dann finde ich eben auch die Bereitschaft Konflikte zu lösen, auch Kommunikationsfähigkeit mit dem Team, gut auch miteinander umzugehen, natürlich genauso auch pädagogische Kenntnisse, wissenschaftliche Erkenntnisse gerade zu solchen Themen, ich habe ja vorhin schon das genannt mit der Hirnforschung oder Bindung und dann finde ich, ist auch immer ganz wichtig, die Zusammenarbeit mit den Eltern, dass man da auch bereit dazu ist und das auch durchführt und auch die Vernetzungsbereitschaft, dass man da auch auf einer Ebene arbeitet. 

Ich finde, da ist bei uns auch -- gewachsene Strukturen, schon viel passiert. Wir haben jetzt nicht mehr diese klassische Rollensituation: Schule Kindergarten zum Beispiel, das ist bei uns sehr gut auf einer Ebene. Wir haben gemeinsame Fortbildungen, die Grundschulen, also die macht die Kooperation und ist Direktorin dort kommt auch zu unseren Weihnachtsfeiern und umgekehrt. Also das ist eigentlich fast schon wie ein gemeinsames Team. Wir haben schon gesagt, eigentlich wäre es schön, wenn die Einrichtung daneben wäre, dass man wie so ein Kinderhaus hätte, wo die Übergänge noch besser wären. Das hat ja auch viel mit Inklusion zu tun. Also da ist bei uns schon viel passiert. Ich finde da sind wir schon auf einem guten Weg, auch wenn es noch viel gibt, was wir dazu lernen können. 

\emph{F: Was kann der Träger konkret beitragen, damit das, was Sie jetzt gerade angesprochen haben, was als wünschenswert angesehen wird, umgesetzt wird oder haben Sie das Gefühl, der Träger macht eigentlich schon sehr viel?}

A: Da habe ich ein gutes Gefühl, bei der AWO speziell -- wir haben zum einen eine gute Unterstützung, wir haben auf Leitungsebene auch Besprechungen, wo man sich gegenseitig austauscht, wir erhalten auch Zeit dafür oder eben jetzt gerade... Es gibt natürlich auch Geld über die Orientierungs- und Bildungsplan für Fortbildungen -- da das zur Verfügung gestellt bekommen. Wir haben auch mehr Freiheit in dem Sinne als dass wir jetzt nichts von oben vorgegeben haben, sondern dass es auch klar ist, dass jedes einzelne Team je nachdem in welcher Situation man sich befindet, in welchem Stadtteil, das wir da die einzelnen Familien brauchen, aber da haben wir eine sehr gute Unterstützung und können da auch... Natürlich immer im Rahmen, die AWO hat natürlich auch keine großen finanziellen Mittel, das ist das andere, das ist ganz klar, aber durch solche Situationen, das man dann auch eine Unterstützung von oben bekommt für Bildungsausflüge, Mittagessen, Spenden uns so weiter finde ich schon, dass unsere Arbeit gut unterstützt wird und wir auch gute Möglichkeiten haben Einfluss zu nehmen auch zu sagen, wir werden auch gefragt, was braucht ihr jetzt vor Ort wieder, was für Bildungsausflüge wären wichtig 

und was ich auch schön finde, es ist auch insofern entlastend, jetzt fand zum Beispiel am Dienstagabend ein Abend statt, der hieß Dankeschön-Abend für die Spender und da haben wir dann auch unsere Projekte vorgestellt, das bedeutet auch mehr Arbeit für uns, aber es ist auch entlastend, dass das so zentral gesteuert ist, das heißt, dass die Spender eingeladen werden, dass eine Feier gemacht wird, da haben wir diese Organisation nicht in jeder Einrichtung nochmal einzeln. Natürlich bedankt man sich auch immer nochmal einzeln, aber... Das fand ich jetzt ganz schön diesen Abend und da konnten wir dann auch unsere Arbeit darstellen, die Bildungsprojekte vorstellen, und darüber sprechen, auch berichten aus dem Alltag.

\emph{F: Und die anderen fünf Einrichtungen haben das auch gemacht?}

A: Genau, die waren alle dort vor Ort. Da denke ich, das sind ganz gute Möglichkeiten.

\emph{F: Wow, auch des Austauschs.}

A: Ja, genau. Von daher arbeite ich auch gern bei der AWO, auch wenn es jetzt natürlich -- wir verdienen hier nicht so viel (lachen), es gibt sicher Einrichtungen, wo man mehr verdient, aber ich finde es auch wichtig, dass man sich auch wohlfühlt, dass es auch eine Wertschätzung gibt von Seiten der Geschäftsführung. Darüber bin ich persönlich schon sehr froh. 

Das gibt man dann natürlich auch an das Team weiter und ich finde es jetzt es so, dass die Atmosphäre das wichtigste in einer Einrichtung ist, also Haltung, Wertschätzung sind ja immer so die Schlagwörter, aber ich bin mir sicher, dass wenn wir jetzt als Team nicht so eine gute Zusammenarbeit hätten, könnten wir das wiederum nicht an die Familien weiter geben und wenn die sich hier nicht gut aufgehoben fühlen, dann würden sie sicher auch Kinder wiederum nicht anmelden. Das ist so ein Rattenschwanz. 

Das finde ich auch das Wichtigste überhaupt bei Inklusion, dass sich alle auch angenommen, sich auch wohl fühlen, das heißt nicht Friede Freude Eierkuchen, natürlich gibt es auch Konflikte, es gibt auch unterschiedliche Meinungen, aber die Frage ist immer, wie geht man damit um. Es beginnt auch schon da, wie werden Praktikanten behandelt, sind die jetzt Leute, die nur die Fenster putzen müssen, oder sind die auch integriert, sie nehmen Teil an Gesprächen. 

\emph{F: Da sehen Sie sicher auch ihre besondere Aufgabe als Leitung.}

A: Genau. Das sehe ich so als meine Hauptaufgabe. 

\emph{F: Welche Schritte wurden konkret unternommen, um Inklusion einzuleiten? Sie haben gesagt vor zehn Jahren ist das erste Kind aufgenommen wurden und dann haben Sie auch immer geschaut, wie ist jetzt der Bedarf von dem Kind und wie können wir ihm gerecht werden, durch Projekte und Vernetzung. Wollen Sie das noch ergänzen?}

A: Wir haben Fortbildungen im Team, habe ich auch schon gesagt, Vernetzung mit der AWO, ich bin im Netzwerk Bildung und Migration. 

Genau, was vielleicht noch wichtig ist, wir haben jetzt in den letzten Jahren unsere Mitarbeiter in Arbeitsgemeinschaften verstärkt, 

haben ein Elterncafé eingerichtet, dass die Eltern auch untereinander Kontakt haben und auch die Möglichkeit des Austauschs, da können wir auch über verschiedene Themen sprechen. 
da haben wir es jetzt auch so, dass jedes Land sich vorstellen kann. Wir hatten jetzt zum Beispiel einen Nachmittag, da wurde erst was berichtet, auch ein Film gezeigt, da hatten die russischen Familien gesagt, wie sind sie überhaupt nach Russland damals gekommen, wieso ist jetzt diese Rückbewegung, dann haben sie aus ihrer Kultur vieles vorgestellt und haben am Schluss etwas gemeinsames zum Essen gemacht.

\emph{F: Und da waren die Familien und die Kinder dabei?}

A: Da waren die Familien und Kinder dabei und eben auch andere. Das war ein ganz schöner Austausch, da haben auch viele berichtet, bei uns ist ja eigentlich ganz ähnlich, auch wenn das eine ganz andere Religion, religiöser Hintergrund ist. 

Dann haben wir hier das \emph{Rucksack}-Projekt fest installiert als eine von drei Modelleinrichtungen in Freiburg, ich weiß nicht, ob sie das schon mal gehört haben, also das ist eigentlich dafür da, dass die Zweisprachigkeit der Kinder gefördert wird, das heißt bei uns ist es jetzt, weil wir die größte Gruppe mit Migrationshintergrund aus Russland haben, sind es jetzt die russischen Familien, die sich einmal pro Woche treffen mit einer speziell dazu ausgebildeten russisch sprechenden Elternbegleiterin und die bespricht dann zum einen pädagogische Themen mit den Eltern, da ist natürlich auch wieder eine Austauschmöglichkeit und dann werden Spiele, Lieder in russischer Sprache durchgeführt, die parallel dann mit den Kindern in deutscher Sprache bei uns hier in der Einrichtung durchgeführt werden, so dass die Kinder zum einen die russische Sprache zu hause, also dass die Eltern sich natürlich auch mal Zeit nehmen mit ihren Kindern, und das gleiche in deutsche Sprache. Also Sprachförderung für die Kinder, aber auch der Austausch für die Eltern und ich finde, dass ist dann gleichzeitig auch noch eine Wertschätzung der Heimatsprache und wir haben festgestellt, dass die Mütter, es sind leider hauptsächlich Mütter, weil die Väter arbeiten, das war natürlich mal anders gedacht (lachen), 
aber die Mütter sind dadurch auch viel selbstbewusster geworden, also kommen auch häufiger in die Einrichtung, nehmen auch an anderen Veranstaltungen teil und auch bei den Kinder merkt man es, dass die merken, meine Mama ist jetzt hier, wir dürfen jetzt hier auch russisch sprechen und dann haben wir das jetzt auch in den Stuhlkreis integriert, zum Beispiel beim Begrüßungslied, dass die Kinder dann auch in russischer Sprache singen dürfen. Das läuft ganz gut, das ist jetzt seit März 2011 drin das Projekt und ist eben auch durch den \emph{LEIF}-Arbeitskreis angestoßen wurden und man hat eben geschaut, wie kann man Sprachförderung noch besser integrieren in die Einrichtung und die Eltern auch ganz anders miteinbeziehen. Und durch das \emph{Rucksack}-Projekt ist auch das Elterncafé entstanden, weil die anderen Eltern gesagt haben, wir wollen auch so ein Elterncafé, warum nur für die Eltern mit russischer Sprache, was ich auch richtig fand. Dadurch haben wir jetzt das Elterncafé noch installiert für alle, wo aber die Familien aus Russland dann auch noch teilnehmen. Das machen wir nur einmal im Monat, einfach deshalb, weil wir kein zusätzliches Personal dafür haben und damit wir es auch langfristig beibehalten können. Man muss ja auch immer gucken, was können wir alles bewältigen mit dem vorhandenen Personal, um die anderen Aufgaben nicht streichen zu müssen, sondern weiter zu führen. 

\emph{F:  Sie haben jetzt alles schon genannt, auch was eine Beziehung zu den Eltern aufbaut.  
Hier ist ja die Frage, was wird von der Institution organisiert, um eine Beziehung aufzubauen.}

A:  Genau, das ist so das Ding mit dem Elterncafé. Dann haben wir natürlich auch Waldausflüge mit Kindern und Eltern, Fest feiern, Flohmarkt hatten wir jetzt gerade. Das ist auch auf Elternwunsch entstanden, dass die Eltern die Möglichkeit haben ihre Sachen zu verkaufen und wir gleichzeitig... Das ist natürlich Öffentlichkeitsarbeit, die Familien haben einen Ort sich zu begeben und für uns ist es gleichzeitig auch noch eine Einnahmequelle. Dann haben wir immer einen Stand auf dem Weihnachtsmarkt. 

\emph{F: Ah, die Eltern verkaufen und zahlen dann...}

A:  Die Eltern können verkaufen, die zahlen drei Euro pro Meter an uns, dann haben wir eine Einnahmequelle, bringen einen Kuchen mit, dann wird der Kuchen verkauft und so kann man eigentlich allen helfen. 

\emph{F: Kennen Sie die Einstellung der Eltern zu Fragen inklusiver Erziehung und wie unterstützen Sie den Dialog darüber?}

A: Also das geht zum einen bereits im Anmeldegespräch los, da berichte ich ja den Eltern, wie unsere Arbeit ist, da kommt man ja automatisch drauf, dass wir auch Kinder mit Behinderung aufnehmen, wie hoch der Migrationshintergrund ist, wie wir damit umgehen, wenn Kinder besonderen Förderbedarf haben oder Entwicklungsdefizite oder auch wie unsere Arbeitsweise ist, dass wir an den Stärken ansetzen. So kommt man automatisch in das Gespräch darüber 

und die Eltern erzählen ja dann auch, dass sie das sehr gut finden, 
manche sagen, dass sie bewusst unsere Einrichtung ausgewählt haben. 
Das fand ich auch ganz erstaunlich, wir haben auch Eltern aus Herdern, die sehr viel Geld haben, die sagen, wir wollen bewusst nicht in eine Einrichtung, wo nur die gleichen Kinder von der sozialen Herkunft sind, sondern sie haben bewusst ihr Kind jetzt hier angemeldet. 
Und dadurch kommt man natürlich in den Dialog und dann auch an Elternabenden, in dem Elterncafé tauscht man sich darüber aus. 

Oder wir haben zum Beispiel letztes Jahr vor Weihnachten einen Elternabend gehabt, dass man sich über Feste in den jeweiligen Kulturen ausgetauscht hat, wie feiert man die einzelnen Feste. Tür und Angelgespräche, eigentlich ergibt es sich ja im Alltag, wenn die Eltern noch mitkriegen, da ist jetzt ein neues Kind... Wir machen das zum Beispiel auch, dass wir ein Blatt an die Tür hängen mit Bildern von den neuen Kindern, welche Kinder jetzt gekommen sind, und dann kriegen die Eltern das auch mit und begrüßen zum teil: „Ah, du bist jetzt der sowieso, der jetzt in die Einrichtung kommt.“ 
Also  so auch immer zu gucken, das transparent zu machen. Natürlich ist – wir haben genauso auch die andere Seite, dass es an einem Elternabend manche Eltern einfach nicht kommen können, weil sie berufstätig sind und wir da – manche Eltern erreicht man, egal was man anbietet, einfach schwierig.  Zum Teil liegt es wirklich auch daran, dass sie kein Interesse haben, zum Teil sagen sie einfach auch ganz klar es ist ihnen zu viel. Sie haben drei Kinder und sind noch berufstätig und dann schaffen sie es zum Teil dann nicht an irgendwelchen Dingen teilzunehmen, das ist die andere Seite. Aber es gibt auch immer wieder Punkte – wir hatten jetzt zum Beispiel mit der evangelischen Hochschule dieses Projekt „Gesund aufwachsen in der Kita“ - da hängt noch gerade der Flyer und da haben wir die Eltern eigentlich von Anfang an mit einbezogen in der Auswahl des Projektes, was wir da genau machen und haben dann Waldausflüge hier installiert und da hatten wir an einem Freitagnachmittag – haben wir dann relativ spät gelegt, also so dass es für die Kinder gerade noch machbar war, dass sie nicht zu müde sind. Und da hatten wir wirklich das erste Mal, dass alle Eltern von der Einrichtung – das fanden wir ganz toll – teilgenommen haben. Alle 60, zum Teil auch doppelt teilgenommen haben.

\emph{F: Die wurden auch bei der Entscheidung von Anfang an mit einbezogen...}

A: Da ist schon was dran, dass man sich so Gedanken machen muss, wie kann man doch noch mehr Dinge mit den Eltern gemeinsam – ich meine klar, es wird auch dann immer wieder Situationen geben, wo jemand einfach krank ist oder sonst etwas, aber das war jetzt auffallend, dass dann wirklich alle 60 Familien auch da waren. Und wenn auch einer etwas später nach der Arbeit dazu kam, aber er kam. Und das fanden wir eben ganz schön.

\emph{F: Das ist ja auch ein ganz positives Feedback.
Meinen Sie, dass Erzieherinnen Zusatzqualifikationen brauchen, um inklusiv zu arbeiten?}

A: Ja, würde ich schon so sehen. Dass man ja eigentlich immer Zusatzqualifikationen braucht für – ja ich glaube das lebenslange Lernen, wie es immer heißt, ist schon wichtig. Egal ob jetzt für  mich oder für alle anderen aus dem Team. Und jetzt wie gesagt Antibarrier's steht jetzt gerade vor der Tür,  deshalb fällt mir das natürlich wieder ein. Dann ändern sich ja auch ständig gesetzliche Grundlagen, einfach so banale Dinge wie jetzt diese Integrationshilfe – man muss ja immer auf dem laufenden sein, um zu wissen auch was kann ich mir an zusätzlicher Hilfe an zusätzlichen personellen Ressourcen in die Einrichtung reinholen, dafür ist das ja wichtig. Dann auch – ja es gibt ja immer wieder neue entwicklungspsychologische Erkenntnisse – dass man da am laufenden bleibt, oder jetzt wie die sexuelle Entwicklung, das war auch so ein Thema, was wir bisher einfach noch nicht so als Thema hatten, da haben wir ja jetzt gerade dieses Projekt gehabt. Da hatten wir auch Elternabende – zwei – zu diesem Thema. Hatten auch ein Theaterstück, um das Thema mit den Kindern aufgreifen zu können. Natürlich jetzt nicht so bewusst genannt, sondern das war so präventiv, aber das fand ich auch eine gute Sache. Es gibt schon im Alltag immer wieder Unsicherheiten, gerade jetzt auch – Thema sexuale Entwicklung – wo man glaube ich – wo es gut ist, wenn man sich als Gesamtteam damit auseinandersetzt. So denke ich – es treten ja immer wieder auch neue Dinge auf, oder es werden Kinder angemeldet mit Situationen, mit denen wir bisher noch nicht so in Berührung kamen, und dann ist es auch wichtig, dass man da einfach sich auch weiterbildet. Von daher glaube ich, dass man das immer sagen kann, dass man ständig Zusatzqualifikationen braucht. Oder auch das Spezialisieren, dass man sich ja auch aufteilt, weil nicht jeder alles wissen kann. Das man – so meine ich das mit den multiprofessionellen Teams auch – dass man sich da auch entsprechend weiterbildet.

\emph{F: Erwarten Sie das von ihren Erzieherinnen – klar dass sie ein Offenheit haben – aber wahrscheinlich sehen Sie auch da wieder ihre Rolle, als die Transportierende.}

A: Ich würde es nicht so als Zwang formulieren, aber ich mach es dann immer so, dass ich mich im Moment dann auch freue, dass wir so viel Geld haben – es gab Zeiten da hatten wir 50 Euro pro Jahr pro Erzieherin und da musste man alles andere aus eigener Tasche zahlen und das dann die Bereitschaft nicht so groß ist, Fortbildungen zu machen ist klar. Aber jetzt im Moment haben wir 3000 Euro gerade bekommen.

\emph{F: Pro Erzieherin?}

A: Nein, für das ganze Team. Aber das ist für uns schon schön und wir haben das jetzt eben so gemacht, dass wir – oder ich habe jetzt versucht, dass wir Gesamtteamfortbildungen, je nach Zeitmöglichkeit organisiert haben, wie jetzt diesen \emph{Antibarriers}-Tag, wie jetzt dieser Sexualentwicklung. Und dann darüber hinaus haben aber trotzdem noch einzelne Erzieherinnen die Möglichkeit an Fortbildungen teilzunehmen. Und da war dann auch jetzt Geld dafür da. Und das ist natürlich dann motivierender und da haben sich auch alle drüber gefreut. Aber ich stelle auch fest, dass so die Fortbildungsbereitschaft auch zugenommen hat. Ich weiß nicht, ob es jetzt nur an dem Geld liegt, ich glaube schon auch, dass man so die Einsicht hat, oder dass man selber auch merkt, man braucht das. Das war wirklich noch vor 15 Jahren ganz anders. Da war es eher so: „Oh je, schon wieder Fortbildungen,” wenn man dann irgendwas im Team eingebracht hat. Das hat sich dann schon sehr gewandelt. Zum Glück, muss ich sagen. Auch was ältere Erzieherinnen anbelangt, da ist man ja manchmal schon so, dass man da jetzt auch noch eine Fortbildung ans Wochenende dranhängen, oder so. Aber da ist schon eine andere Offenheit jetzt da.

\emph{F: Und wahrscheinlich auch, was die gemeinsame Zusammenarbeit und Planung im Team betrifft.}

A: Genau. Da hat sich natürlich auch durch den Orientierungsplan sehr viel verändert. Man kocht nicht mehr so sein eigenes Süppchen finde ich, was die einzelnen Gruppen anbelangt. Wie machen jetzt mehr gruppenübergreifend auch indem wir uns auch austauschen darüber – wir überlegen ja immer, wen wir die einzelnen Bildungsbereiche anschauen – hatten wir jetzt auch gerade mit Begleitung von einem Fortbildungsinstitut, Impulse heißt das, dass wir unser Räume angeguckt haben, überlegt haben, was für Angebote haben wir zu den einzelnen Bildungsbereichen, was müssen wir noch ergänzen, können die Kinder bei uns einen Flaschenzug, zum Beispiel, beobachten, das war jetzt was, was wir noch nicht hatten – solche Dinge. Und dann uns auch zusammen überlegen – es muss ja auch nicht jede Gruppe alles haben, man kann das ja auch durch gegenseitige Besuche – aber das finde ich dadurch auch schön ,dass man dann auch mehr so gruppenübergreifend denkt. Trotzdem – wie gesagt, wie ich schon sagte – bin ich für die Struktur der Familiengruppen, aber das man trotzdem mehr so guckt, wie kann man miteinander arbeiten und man kann ja auch das gleiche Projekt in den Gruppen durchführen und einzelne Angebote dann gruppenübergreifend machen – solche Dinge.

\emph{F: Können sie nochmal was zu den Strukturen sagen: Wann gemeinsame Treffen stattfinden, ob die fallbezogen sind, das Verhalten der Kinder besprochen wird.}

A: Wir haben – zum einen hat jede Gruppe eine Gruppenbesprechung in jeder Woche – das ist immer pro Woche eine Stunde. Und da ist es so, dass da ein Teil dieser Zeit für Fallbesprechungen aus der Gruppe ist, speziell auch diese Bildungs- und Lerngeschichten betreffend, dass man sich darüber austauscht, aber auch Vorbereitung von Elterngesprächen, dass man da Entwicklungsberichte schreibt, etc. und dann ist ein Teil natürlich organisatorisch, was man plant für Angebote etc. Wenn die Zeit nicht ausreicht, was immer wieder auch ist, es gibt natürlich darüber hinaus auch Anleitungsgespräche mit Praktikanten oder wenn jetzt zum Beispiel ein Kind, was integrativ in der Gruppe betreut wird, da reicht die Zeit nie – natürlich ist das sehr wenig eine Stunde pro Woche. Dann finden immer noch Gespräche dann innerhalb der Gruppe statt, da nehmen dann nicht alle Erzieherinnen teil, da muss man sich einfach ein bisschen aufteilen, aber dann – dass  dann eine Erzieherin nochmal mit einer Heilpädagogin, dann bespricht.

\emph{F: Und Sie sind dann nicht dabei?}

A: Ich bin teilweise dabei, es gibt immer Situationen – jetzt zum Beispiel wenn Integrationshilfe beantragt werden muss, ein Entwicklungshilfebericht geschrieben wird oder ein runder Tisch vorbereitet wird, dann bin ich schon auch dabei, weil ich ja dann auch bei dem runden Tisch dabei bin. Aber ich bin nicht an allen dabei. Und dann gibt es, wie gesagt die Anleitergespräche noch, oder Einzelgespräche mit Heilpädagogen oder auch mit Eltern, das ist ja sowieso unabhängig davon dann. Und dann gibt es die Gesamtteambesprechung. Da ist schon mal eine pro Woche, die anderthalb Stunden geht, da werden dann auch Vorhaben, die das ganze - also organisatorische Dinge – die das ganze Team betreffen besprochen aber auch kollegiale Beratung gruppenübergreifend. Es gibt dann ost auch Situationen, wo man dann Kinder bespricht, einfach um auch nochmal den Blick von außen zu haben, weil man in der Gruppe oft dann Dinge schon ausprobiert hat, oder dann einfach noch die Beratung – ob andere noch andere Ideen haben – oder auch natürlich zur Information, was die anderen Gruppen auch wissen müssen, diese Dinge. Und dann haben wir einen Freitag im Monat, wo wir bereits um halb eins schließen. Und von eins bis vier haben wir drei Stunden für Teambesprechungen und da haben wir dann auch nochmal drei Stunden Zeit, wo wir dann auch entweder Teamfortbildungen oder auch eine Heilpädagogin einladen, wo Fallbesprechungen laufen können.

\emph{F: An einem Tag schließen sie bereits halb eins.} 

A: Genau, ein Freitag im Monat. Das ist wichtig. Aber das ist schon auch eine wichtige Zeit für uns, da legen wir dann oft auch Besprechungen - zum Beispiel die Heilpädagogin, die die Sprachförderung hier bei uns durchführt, die kommt dann auch an diesen Freitag mal dazu. Da können wir die Kinder einfach – da haben wir mehr Zeit. Die Zeit ist halt doch nicht so groß. Dann haben wir aber darüber hinaus noch pro Gruppe Orientierungsplanzeit. Das heißt, dass wird in der Einrichtung verbracht, die Erzieherinnen haben dann Zeit Kinder zu beobachten, Dokumentationen zu schreiben, Lerngeschichten zu schreiben. Und auch diese Zeit auch teilweise für den Austausch – also wenn die Gruppenbesprechung nicht ausreicht, kann man dann auch noch, diese Zeit dazu nutzen sich auszutauschen. 

\emph{F: Das klingt sehr intensiv.}

A: Ist aber auch erst in den letzten Jahren – dass wir einfach mehr Zeit, über den Bildungsplan – das muss man ganz klar sagen. Wir hatten – vor fünf Jahren war die Situation noch völlig anders. Ich finde jetzt haben wir schon eine richtige Luxussituation, wobei es immer noch oft nicht ausreicht, aber das muss man auch ganz klar sagen, dass sich da viel verändert hat.

\emph{F: Dass die Zeiten honoriert werden und auch Freistellungen passieren.}

A: Freistellung passiert – dass wir einfach personell besser bestückt sind. Früher hätte man dass gar nicht leisten können. Das sehen sie schon aufgrund des Personalschlüssels von 3,2 Stellen pro Gruppe. Früher wenn das nur zwei Stellen pro Gruppe waren, dann ging das einfach nicht, dann war man im Früh- und Spätdienst, musste gucken, dass man Überlappungszeiten hatte und da war das sehr schwierig. Und da hat man trotzdem die Besprechungszeiten abends  - das haben wir jetzt ja auch noch, aber es ist so – wir haben insgesamt einfach mehr Zeit, durch den Orientierungsplan, was ich auch wichtig und richtig finde, aber das hatten wir eben lange Zeit nicht. Von dem her hat sich da schon viel verbessert. 

\emph{F: Was sind häufige Fragen der Erzieherin, die anzeigen, dass Beratungs- oder Unterstützungsbedarf besteht?}

A: Man merkt natürlich zum einen die Überlastung im Alltag, also gerade wenn Situation sind, dass Leute krank sind oder dass jetzt in einer Gruppe gerade mehrere Kinder mit einem Förderbedarf sind. Dann kommen natürlich Fragen, wie: "Können wir den Kindern gerecht werden? Wie können wir den Förderbedarf der Kinder noch erfüllen?" Oder dass es auch ganz klar benannt wird, also dass gesagt wird: "Ich habe das Gefühl, es geht nicht, wenn man in der Gruppe ist und nur noch ein weiterer dabei ist und dann hat man eben drei Kinder, die man permanent beobachten muss." Dann gucken wir, wenn das so benannt wird, wir haben zum Glück eine Offenheit im Team, wenn das benannt wird, dass wir jetzt nichts hinten rum besprechen, dann gucken wir, wenn jetzt gerade eine extreme Krankheitsausfallsituation ist, dass man Entlastung dadurch gibt, dass Erzieher der anderen Gruppe rüber gehen zur Vertretung oder, was wir auch schon gemacht haben, dass wir Kinder aus der Gruppe raus genommen und Kleingruppen gebildet haben oder dass ein Teil der Gruppe im Garten war oder das auch mal Angebote verschoben werden, damit die personelle Ausstattung in der Gruppe besser ist. Manchmal ist es auch eine gute Entlastung, dass ein Kind mal raus genommen wird aus der Gruppe. 

\emph{F: Haben Sie denn den Eindruck, dass die Integrationshilfe noch mehr Stunden...}

A: Ja, klar, das ist auf jeden Fall ein Punkt, der wichtig wäre, dass die Heilpädagogin noch mehr Stunden in der Woche... Wobei das ein Punkt ist -- das ist auch nicht immer so leicht, wir müssen ja immer gucken -- ich glaube, dass vom Jugendamt auch die Vorgabe ist möglichst einzusparen und da sind wir schon immer froh, wenn wir sowohl die heilpädagogische Hilfe als auch begleitende Hilfe -- das ist auch immer nicht so einfach. Die begleitende Hilfe wird nicht so gern finanziert, also was heißt nicht so gern, das liegt nicht an den Mitarbeitern, die haben natürlich auch die Vorgabe genau hinzugucken, wird das auch gebraucht oder kann man das nicht auch mit dem vorhandenen Personal bewältigen. 

\emph{F: Die begleitende Hilfe ist die in diesem Schlüssel 3,2 dabei?} 

A: Die FSJ ist zusätzlich, die wird natürlich auch nicht hoch bezahlt. Das sind ja dann 300 €, die zur Verfügung sind. Wenn wir diese 300 € für eine Erzieherin einsetzen würden, die würde dann ein paar Stunden kommen, das können Sie sich ausrechnen, für einen Hungerlohn, das würde gar nicht gehen. Das geht nur, wenn man Vorpraktikanten oder FSJler einsetzt, die natürlich wieder gut angeleitet werden müssen, wo natürlich auch wieder ein Arbeitsaufwand da ist. Das muss man natürlich auch sehen, das heißt, es reicht auch nicht, die Heilpädagogin müsste viel häufiger kommen.

\emph{F: Es ist schon ein Sparmodell.}

A: Ja, klar.  
Denn die muss ja mit dem Kind arbeiten und die Elterngespräche durchführen, die Gespräche mit den Erziehern und auch noch die Anleitung von den FSJlern, also das geht gar nicht ohne das Personal.

\emph{F: Und das macht die in wie viel Stunden pro Woche?}

A: Zwei Mal zwei Stunden in der Woche. Das ist schon das Höchstmaß, was wir je bewilligt bekommen. Auch da wird immer geguckt, ob man das nicht irgendwann runter fahren kann. Das ist eigentlich die politische Schiene -- das habe ich vorhin noch gar nicht gesagt -- ich freu mich immer, wenn das kommt, man wird da schon bescheiden, sage ich jetzt mal -- aber das ist schon so, dass an der Integrationshilfe -- das wäre schon so ein Punkt, wo mehr...

\emph{F: Wow, die arme Heilpädagogin (beide lachen).}

A: Deshalb habe ich vorhin auch gesagt, es müssen Heilpädagogen fest im Team sein und zwar in jeder Kita. erst dann, finde ich, kann man richtig von Inklusion sprechen, weil die Zeit... Das kann man vergessen. Man guckt dann, wie kann man es intern regeln. Wir haben jetzt dadurch, dass die Heilpädagogin für verschiedene Kinder da ist, ist sie natürlich auch mehr Zeit da und wenn dann mal ein Kind krank ist, geht sie nicht weg, sondern guckt, wo kann sie dann noch mit einer Erzieherin sprechen in der Zeit. Aber (lachen) trotzdem finde ich, wäre es gut, wenn fest Heilpädagogen in jeder Einrichtung mit einem Stellendeputat drin wären. 

\emph{F: Haben Sie erlebt, dass es Ängste gibt oder gab bei Mitarbeitern, so auf diesem Weg in Richtung Öffnung und mehr Kinder mit Förderbedarf aufnehmen?}

A: Ja, klar zum einen geht es ja um die Sicherheit der Kinder. Die Unfallgefahr ist häufig auch höher, wenn es jetzt Kinder mit Behinderung sind. Dann haben wir natürlich auch dadurch bedingt, dass unsere Kita schon 60 Jahre alt ist, ganz klar räumliche Stolperfallen, wir haben zum Beispiel Türschwellen und bei dem Ausgang in den Garten -- was einfach auch bisher nicht bewilligt wurde, das mal durch einen Umbau zu verändern -- da müsste man Türen verändern und so weiter. Das ist auch wieder eine Kostenfrage. Auch was unseren Eigentümer dieses Gebäudes, das ist der Bauverein, anbelangt, wobei die schon großzügig sind und uns viel spenden. Aber das sind natürlich schon Punkte, die dann immer nicht so einfach sind. Wir hatten mal ein Kind, das schlecht laufen konnte und das Laufen eigentlich bei uns erst gelernt hat, also was dann gekrabbelt ist und da hatten wir immer die Problematik, ich guck jetzt gerade geradeaus auf die Tür der Mäusegruppe, da war immer so eine Schwelle, da haben wir dann überlegt, wir haben dann teilweise so ein Holzstück hingelegt, dass das Kind darüber fahren konnte, aber es musste immer weg geräumt werden, um die Türen zu schließen und im Winter war es dann auch schwierig. Also da gibt es schon allein solche Barrieren, wo man sich dann Gedanken macht. Dann natürlich auch der Wunsch nach Unterstützung und Beratung bei personellen Engpässen, die Angst, dass man alleine nicht mit der Situation klar kommt. Was wir jetzt noch nicht hatten, ein Kind, was schwerst mehrfach behindert ist und jetzt mit der Sonde ernährt werden muss. Solche Dinge hatten wir bisher noch nicht. Aber da kann ich mir vorstellen, da braucht man natürlich auch Unterstützung und noch ganz andere technische Hilfsmittel oder vielleicht auch noch eine andere Ausbildung. 

\emph{F: Also sehen Sie da Ihre Grenzen?}

A: Da würde ich jetzt die Grenze sehen aufgrund unserer baulichen Voraussetzungen. Wir haben auch wenig Zusatzräume, es gibt Einrichtungen, die haben einfach noch mehr Schlafräume, mehr zusätzliche Räume, das wäre bei uns schon... Allein das Durchkommen mit dem Rollstuhl in dem Flur ist bei uns schon schwierig, gerade im Winter, wenn die Kinder sich alle umziehen, muss man da schon immer schauen vom Platz -- ist sowieso schon eng. 

\emph{F: Also die Rahmenbedingungen müssten gewährleistet sein, um allen Kindern gerecht zu werden?}

A: Genau. Ich denke, es ist eine Frage des Wies, also wenn wir wie jetzt... Wir haben ja auch eine AWO-Einrichtung in der Nähe, die nehmen schwerst mehrfach behinderte Kinder auf, da haben wir schon oft gesagt, man könnte ja jede Einrichtung so einrichten, dass es technische Hilfsmittel gibt, dass Heilpädagogen fest angestellt sind, dass Krankengymnastik vor Ort passiert. Wenn jede Einrichtung diese Möglichkeiten hätte, multiprofessionelle Teams hätte, kleinere Gruppen. 

\emph{F: Machen die das da?}

A: Die machen das drüben. Die haben dann nur diese Kinder und auch noch -- normal ist sowieso so ein Begriff, was ist normal, aber die haben praktisch... Das ist eine Einrichtung für Kinder mit Behinderung. 

\emph{F: Eine Sondereinrichtung.}

A: Quasi eine Sondereinrichtung, genau.
Deshalb denke ich, die sagen auch immer, im Grunde müssten wir uns zusammenschließen, dass wir sowohl Kinder mit Behinderung als auch Kinder ohne Behinderung in der Einrichtung haben.

\emph{F: Klar, dann haben wir erst die Vielfalt.}

A: Dann haben wir erst die Vielfalt, genau. Ich denke es ist machbar, aber man müsste halt grundsätzlich alle Strukturen verändern. Angefangen von der Gruppengröße, mutliprofessionelle Teams, Krankengymnastik in der Einrichtung, räumliche, bauliche Veränderungen. Es steht und fällt wie immer mit den Rahmenbedingungen. 
Aber auch wenn wir jetzt so gut dastehen wie noch nie, ist es immer noch begrenzt im Hinblick darauf, dass es noch Dinge gibt, die noch nicht vorhanden sind und von daher können wir nicht sagen, egal, wir nehmen auf, komme, was wolle (lachen).

\emph{F: Haben Sie durch die Umsetzung von Inklusion Veränderungen bei sich feststellen können oder bei den Kollegen oder bei den Kindern?} 

A: Ja, ich würde sagen zum einen eine offenere Haltung dem Thema gegenüber, Vernetzung ist auf jeden Fall in den letzten Jahren -- also dass wir da eine viel größere Vernetzung haben, sicher auch das Anpassen von Angeboten, dass man sich vor jedem pädagogischen Angebot überlegt, wie können wir das gestalten, wie können alle Kinder daran teilnehmen, wie können auch Kinder mit Migrationshintergrund... Das man praktisch Angebote hat, die nicht immer nur Sprache im Vordergrund haben, wir haben jetzt auch ein  Zahlenprojekt eingeführt, dass die Kinder, die nicht gut Deutsch sprechen, sich anhand er Zahlen, dass sie da ihre Stärke haben und das fördert ja alle Kinder. Und die Rücksichtnahme ist sicherlich größer geworden, auch bei den Kindern ist das normal in den Alltag integriert, was ich schön finde. Gespräch, Austausch darüber. Ich würde schon sagen, auch die Zusammenarbeit ist intensiviert worden, nicht mehr so sein eigenes Süppchen kocht. Auch was die Gesamteinrichtung anbelangt, viel mehr in Arbeitskreisen integriert wird, was natürlich auch zusätzliche Arbeit bedeutet, aber auch da hat sich was getan. Als ich angefangen habe, habe ich eine Freistellung von 20~\% gehabt, jetzt habe ich 50 plus die Migrationsanteile. Da hat sich auch sehr viel verbessert. 

\emph{F: Sie haben vorhin gesagt, dass manchmal auch jemand in die Gruppe kommt, um die Therapie zu machen. Ich glaube, das war die Sprachförderung. Ansonsten gibt es Strategien, um Therapien in inklusiver Form anzubieten oder ist es eigentlich so, wie sie gesagt haben, dass die Kinder zur Logopädie gehen und zur Heilpädagogik.}

A: Also es ist ja zum Teil gesetzlich vorgegeben. Wo wir es ein Stück weit haben, ist mit diesen Außenstellen in den AWO-Einrichtungen von der Frühförderstelle, aber Logopädie zum Beispiel -- da hatte ich auch mal von der Logopädin gehört, die dürfen das wohl gar nicht in der Einrichtung machen, wenn die ein Rezept haben, müssen die das in ihrer Praxis durchführen. Wir haben bisher noch keine Situation gehabt, dass es in der Einrichtung gemacht wurde. Ich weiß nicht, ob es an den Krankenkassen liegt und ob es immer noch so ist -- da habe ich jetzt auch, das muss ich ehrlich sagen, gar nicht nochmal nachgefragt, nachdem ich diese Informationen hatte. Also, wenn wir jetzt Kinder haben, wo wir sagen, da wäre es gut, eine logopädische Unterstützung, die gehen dann immer mit den Eltern in die Praxis. Sonst, finde ich haben wir es schon ganz gut mit der AWO-Frühförderstelle, das die hier vor Ort ist und der Kontakt auch gut ist.

\emph{F: Und die geht rein in die Gruppe?}

A: Die geht rein in die Gruppe. 
\end{linenumbers*}
       
\section{Kindergarten 3: Katholische Kirchengemeinde}
\begin{linenumbers*}
\emph{F: Können Sie etwas zur Organisation und zur Struktur Ihrer Einrichtung sagen?}
A: Der katholische Kindergarten [Name] in [Stadtteil] wird von 123 Kindern besucht. 113 Kinder im Alter von drei bis sechs Jahren werden in fünf altersgemischten Gruppen betreut.
Darüber hinaus gibt es seit zweieinhalb Jahren eine Kleinkindgruppe mit zehn Plätzen.\linelabel{C3_1}
 
Die Familien können je nach Bedarf zwischen verschiedenen Gruppen mit jeweils verschiedenen Öffnungszeiten wählen. Es gibt Regelgruppen und Gruppen mit verlängerten Öffnungszeiten. Die Kinder der Regelgruppen, von 7.45 bis 12:30 Uhr, gehen über die Mittagszeit nach hause und werden an drei Nachmittagen von 14:00 bis 16:00 Uhr betreut, die Gruppen mit verlängerten Öffnungszeiten wahlweise von 7:30 bis 13:30 Uhr oder von 8:30 bis 15:00 Uhr. \linelabel{C3_2} 

Die Kinder werden in Stammgruppen betreut, während der Freispielzeit werden die Türen geöffnet, so dass zusätzlich der Gang genutzt werden kann als Antwort auf einen hohen Bewegungsdrang. \linelabel{C3_3}

Der Stadtteil ist geprägt durch eine hohe Arbeitslosigkeit, viele Familien sind Harz IV Empfänger. Die wenigsten unserer Kinder kommen aus gut situierten Familien. \linelabel{C3_4}

Viele Kinder zeigen in erhöhtem Maße Verhaltensauffälligkeiten und haben Probleme im sozial-emotionalen, motorischen oder sprachlichen Bereich. \linelabel{C3_5}

Obwohl wegen der beruflichen Situation, meist ist ein Elternteil zu hause, kein Bedarf an verlängerten Öffnungszeiten besteht, 
ist es für die Kinder und deren Eltern wichtig, dieses Betreuungsangebot zu nutzen. Die Eltern sehen, was ihr Kind im Kindergarten lernt und auch, dass sie diese Lernmöglichkeiten und Anreize zu hause nicht bieten können.
Es gibt Mütter, die sagen: „Ihr könnt meinem Kind mehr beibringen als wir zu hause.“ \linelabel{C3_6}
 
Bei den Familien fehlen oft die Strukturen des gemeinsamen Kochens und Essens, weshalb der Wunsch nach einem gemeinsamen warmen Essen für die Kinder wiederholt formuliert wurde. Dieser Nachfrage wurde bisher noch nicht nachgekommen, da die Räumlichkeiten fehlen. Auf diese Frage soll aber in den nächsten Jahren eine Antwort gefunden werden. Themen bezogen wird im Kindergarten ab und zu gekocht. \linelabel{C3_7} 

Einmal im Monat gibt es ein gemeinsames Frühstück mit reichhaltigem Buffet für alle Kinder der Einrichtung. Hierfür wird eine Liste ausgehängt, auf der sich die Eltern für das Mitbringen von Butter, Quark, Wurst, mit Vermerk „halal“, eintragen können. Dann stellt am Morgen jede Gruppe einen Tisch auf den großen Gang und dort wird gemeinsam gefrühstückt. 
Zudem findet zwei Mal im Jahr ein gemeinsames Frühstück für alle Kinder mit Eltern statt. Zu diesem kommen oft 100 Eltern. Sie genießen das sehr: „Jetzt mach mal jemand etwas für uns und wir dürfen uns hinsetzen.“ \linelabel{C3_9}

Von den 123 Kindern haben schätzungsweise 90 plus/minus fünf einen Migrationshintergrund. \linelabel{C3_10}

In den fünf Gruppen werden 22 oder 23 Kinder von je zwei Fachkräften betreut. In den Gruppen mit verlängerter Öffnungszeit ist der Betreuungsschlüssel höher berechnet. \linelabel{C3_11}

Die Regelgruppen arbeiten am Nachmittag verstärkt gruppenübergreifend, so dass die Erzieherinnen unterschiedliche Angebote anbieten können. 
In der Nachmittagsbetreuung variiert die Kinderzahl der Regelgruppen, zum Beispiel wenn die Schule bereits Ferien hat oder auch jahreszeitlich bedingt.  
In den kleineren Gruppen können die Erzieherinnen gut auf den Bedarf reagieren, ein Angebot vom Morgen noch einmal wiederholen oder ein extra Angebot machen. 
Heute morgen hat eine Mutter zu mir gesagt: „Gell, mein Kinder muss noch viel lernen?“ Daraufhin habe ich ihr geantwortet: „Es wäre gut, wenn sie ihr Kind regelmäßig nachmittags bringen, so dass es zusätzlich etwas Gutes bekommt.“ \linelabel{C3_12}

\emph{F: Sind Sie als Leiterin freigestellt?}	

A: Ich bin zu 100~\% frei gestellt, wenn jedoch Not an der Frau ist, das heißt, wenn jemand krank ist, springe ich ein. \linelabel{C3_13}
 
Darüber hinaus biete ich jeden Mittwoch Sprachförderung in Kleingruppen an. Nach Rücksprache mit den Kolleginnen werden Kinder ausgewählt, die zweisprachig aufwachsen und die noch mehr Sprachanreize in Deutsch benötigen oder andere, die sich nicht trauen laut zu sprechen, deren Selbstbewusstsein gestärkt werden soll oder aber Kinder, die in der Grammatik und im Wortschatz Nachholbedarf haben. Auch deutschsprachige Kinder, die Unterstützung brauchen, sind mit dabei. \linelabel{C3_14}

Die Sprachförderung übernehme ich auch, weil ich den Kontakt zur Basis, zu den Kindern, suche. \linelabel{C3_15}

Insgesamt bekommen ungefähr 40 Kinder zusätzliche Sprachförderung, die sind auf mich und meine Kollegin aufgeteilt. \linelabel{C3_16}

Die Anzahl der Kinder und die Dauer der Kleingruppen variiert in Abhängigkeit dessen, ob ich mit Musik und Bewegung arbeite oder beispielsweise Wortschatzerweiterung einplane. Dann sind die Gruppen sehr klein und die Einheiten umfassen nur um die 15 Minuten, da die Aufmerksamkeit der Kinder nicht mehr da ist. Trommeln und Rhythmus, Musik allgemein, ist ein Türöffner für alle Kinder, auch für solche, die sich wenig zutrauen und gehemmt sind zu sprechen. \linelabel{C3_17}

\emph{Haben Sie eine zusätzliche Ausbildung für Sprachförderung gemacht?}
 
A: Ja, eine halbjährige Weiterbildung habe ich und meine Kollegin gemacht. Dann habe wir unsere Erfahrungen im Team eingebracht und wir haben gemeinsam eine Sprachkonzeption erstellt.\linelabel{C3_18}

Die Einrichtung kooperiert mit der Sprachheilberatungsstelle. Ein bis zwei Mal im Jahr kommt eine Fachkraft von dort zu uns in den Kindergarten. Wir überlegen im Vorfeld, welche Kinder wir, mit dem Einverständnis der Eltern, vorstellen. Fragen, wie: Braucht das Kind Logopädie, reicht die Unterstützung im Kindergarten oder braucht ein Kind zusätzliche Unterstützung in anderen Entwicklungsbereichen. In diesem Jahr haben wir 22 Kinder vorgestellt. 
Die Eltern sind von Anfang an in diesen Prozess mit eingebunden und schätzen dieses Angebot sehr.
 Die meisten unserer Eltern würden es nicht schaffen einen Termin einzuhalten. Manche Termine finden mit den Kindern und den Eltern gemeinsam statt, andere wiederum nur mit dem Kind oder auch mit einer Erzieherin, je nach Situation und Kind. In jedem Fall erfolgt eine Rückmeldung ans Team und an die Eltern, ob das Kind noch an anderer Stelle vorgestellt werden sollte, weil die Gesamtentwicklung auffällig ist oder eine Wahrnehmungsproblematik vorliegt. 
Gegebenenfalls kann es vorkommen, dass ein Sprachheilkindergarten oder -schule empfohlen wird, wenn die Problematik weder durch Logopädie noch durch Unterstützung der Kinder im häuslichen Umfeld verbessert werde kann. \linelabel{C3_21}

Weitere Kooperationspartner sind die AWO-Frühförderstelle und die dortigen Heilpädagogen, die als Integrationshilfe in die Gruppe kommen oder Kinder in der Beratungsstelle in Spieltherapien oder Ergotherapien betreuen, \linelabel{C3_22}

und die Erziehungsberatung vor Ort. \linelabel{C3_23}

Betreuen wir Kinder, zum Beispiel mit Down-Syndrom, deren Selbstständigkeit zum Teil eingeschränkt ist, die Hilfe beim Essen oder beim Gang zur Toilette benötigen, beantragen wir eine pflegerische Kraft. Bei unserem Kind mit Down-Syndrom, das lange Zeit brauchte, bis es trocken war, kam eine Frau aus der Gemeinde zu uns, die zuvor in einer Arztpraxis gearbeitet. Mit ihr haben wir gute Erfahrung gemacht. Sie hatte ein Händchen für Kinder und konnte sich gut in die Kinder hinein versetzen. \linelabel{C3_24}

Manchmal kommen aber auch Kinder in die Einrichtung, die bereits Ergotherapie oder Spieltherapie bekommen, dann stellen wir auch einen Kontakt zu den jeweiligen Therapeuten her.\linelabel{C3_25} 

Wenn ein Kind in die Einrichtung kommt, dann geben wir diesem erst einmal viel Zeit anzukommen. Später kristallisiert sich heraus, ob es unruhig ist oder Probleme hat Kontakte zu knüpfen oder nicht sprechen will. Dann erfolgt ein Gespräch mit den Eltern. Manche Eltern fühlen sich verstanden, weil sie das beschriebene Verhalten von zu hause kennen. Manche sind sehr erschrocken und reagieren mit Sätzen wie: „Mein Kind ist doch nicht dumm oder behindert. Muss mein Kind raus?“ Dann ist es oft gut Zeit ins Land gehen zu lassen und nach einer Woche nochmal das Gespräch zu suchen. \linelabel{C3_26}

Derzeit betreuen wir drei Kinder mit diagnostizierter Lernbehinderung, zwei Kinder mit Hörbeeinträchtigung, beide tragen Hörgeräte, und Kinder mit Wahrnehmungsproblematiken, bei denen noch eine diagnostische Abklärung aussteht. In der Vergangenheit hatten wir Kinder mit offensichtlichen Behinderungen, wie Down-Syndrom, Spastiken, oder starker Hörschädigung oder Sehbeeinträchtigung. \linelabel{C3_27}

Wir sind offen für alle Kinder und haben die Position, wir probieren es aus und wenn wir es nur zwei Wochen probieren, aber wir probieren es. Erst im Tun, im Leben und Erleben sehen wir oft, was das bringt. Manchmal stoßen wir dann auch an unsere Grenzen und haben Angst, diesem Kind nicht gerecht werden zu können. So zum Beispiel bei einem Kind mit einer schweren Sehbehinderung, das haben wir dann in eine Einrichtung für Blinde empfohlen. \linelabel{C3_28}

\emph{F: Wo genau sehen Sie Ihre Grenzen?}

A: Grenzen sehe ich, wenn wir dem Kind nicht mehr gerecht werden, das heißt, in Ermanglung von Zeit und Zuwendung und abhängig von dem Grad einer Behinderung beziehungsweise einer Entwicklungsstörung. Wir werden dem Kind nicht gerecht, den anderen Kinder auch nicht und kommen auch selber an unsere Grenzen. Um speziellen Kindern wirklich gerecht werden zu können, braucht es meist kleinere Gruppen und mehr Personal.\linelabel{C3_29} 

Kinder lieben es mal allein was zu machen, aber auch in Konkurrenz mit anderen zu treten. Wenn mehrere Kinder in Kleingruppen zusammen sind, sitzt der Motor quasi gegenüber und dann läuft es ganz von allein, dann muss ich nicht so viel motivieren, weil die Kinder, wenn sie sich aneinander messen können, aus sich selbst heraus motiviert sind.\linelabel{C3_30}

Bis zur Bewilligung einer Maßnahme für Kinder mit Sonderförderung sind mehrere Schritte erforderlich. 
Beobachten die Erzieherinnen, dass bei einem Kind Entwicklungsverzögerungen vorliegen könnten, bringen sie diese Beobachtung zunächst mit in die Teamsitzung. Wir überlegen gemeinsam, was zu tun ist und ob Kolleginnen, bei denen das Kind zum Beispiel bei einem Projekt dabei ist, die selben Beobachtungen machen. Dann stellen wir den Kontakt zu den Eltern her und überprüfen, ob die Beobachtungen auch zu hause gemacht werden.   
Dann schreiben wir einen Entwicklungsbericht über das Kind, in den unsere Beobachtungen einfließen. Das geschieht im Team. Dieser wird dem Kinderarzt übergeben mit der Bitte um Überprüfung. Dieser stellt den Eltern gegebenen Fall ein so genanntes Formblatt A aus, mit selbigem und dem Entwicklungsbericht gehen die Eltern zunächst auf das Amt für Kinder, Jugend und Soziales. Dann ruft die Fallbeauftragte vom Amt bei uns an und ein runder Tisch wird anberaumt, an dem zumeist fünf oder sechs Personen sitzen: die Fallbeauftragte, die Eltern, die Kollegin aus der Gruppe, ich als Leitung und eventuell eine Heilpädagogin von der AWO, die als Integrationskraft infrage kommt. 
Beim runden Tisch wird ein Hilfeplan mit den wichtigsten Zielen für eine gute Entwicklung des Kindes festgelegt. Dabei wird darauf geachtet, dass es keine Überforderung für das einzelne Kind gibt, weil zu viele Ziele in wenigen Monaten erreicht werden müssen. Ich bin froh, dass wir heute so etwas haben, dass formuliert wird, das sind die zwei wichtigsten Ziele und wenn wir die erreicht haben, könnten wir eventuell noch das dritte dazu nehmen. \linelabel{C3_31}

\emph{F: Gibt es für die Kinder mit besonderen Bedürfnissen konzeptionelle Überlegungen? Entsprechen diese Ihrem Inklusionsverständnis?
Was verstehen Sie unter Inklusion in Bezug auf den Kindergarten?} 

A: In unserem Leitbild sind kirchliche Werte verankert wie Respekt voreinander und Gleichheit aller Personen, denen wir entsprechen. Eine Konzeption für Kinder mit besonderen Bedürfnissen liegt noch nicht gebündelt vor. In unserer Gesamtkonzeption werden Kinder mit Behinderung einbezogen, aber wir haben das noch nicht zusammengefasst.\linelabel{C3_32} 

Unter Inklusion verstehe ich, dass die Kinder nicht ausgegrenzt, sondern respektiert werden, dass es selbstverständlich ist, dass sie am „normalen“ Leben teilhaben, auch dass man ihnen etwas zutraut. Wir staunen oft, was Kinder mit Behinderungen, wie zum Beispiel Down-Syndrom, alles können. 
Dazu gehört für uns auch, Eltern in ihrer Erziehungsarbeit zu unterstützen und den nicht-behinderten Kindern zu zeigen, dass es nicht selbstverständlich ist, dass man ohne Handicap lebt und dass die Kinder mit Handicap dafür andere Dinge können, dass ein gegenseitiges Lernen möglich sein kann, wir müssen uns nur darauf einlassen. Inklusion ist ein Prozess, sowohl für die Familien als auch für den Kindergarten. \linelabel{C3_33}

\emph{F: Wie wird der Verschiedenheit der Kinder Rechnung getragen un wie ist die individuelle Unterstützung in ihrem Konzept vorgesehen?}

A: Es gibt Kleingruppenprojekte wie die Sprachförderung oder Gruppen zur Bewegungserziehung.
Wir nutzen die Ressourcen der Kollegen. So gibt es zwei Kollegen, die sich dem Angebot Bewegungserziehung verschrieben haben, was immer donnerstags stattfindet. Sie bereiten Programme für eine bestimmte Altersgruppe vor und sagen: „Heute vormittag brauche ich zehn der kleinen Kinder, wählt fünf aus der [Gruppenname] und fünf aus der [Gruppenname] aus. \linelabel{C3_34}
  
Wir nutzen Beobachtungsbögen und die Entwicklungsgrenzsteine sowie legen Portfolio-Ordner an, zur Klärung der Fragen, wo steht jedes Kind und was braucht es? Diese Instrumente werden auch in die Entwicklungsgespräche mit den Eltern einbezogen. \linelabel{C3_35}

\emph{F: Kennen Sie die Einstellungen der Familien zu Fragen inklusiver Erziehung und wie unterstützen Sie den Dialog darüber?}

A: Viele Eltern denken: „Mein Kind kommt zu kurz. Unser Kind fällt durch das Raster. Die Erzieherinnen haben nicht die nötige Zeit.“ Dahinter verbirgt sich oft die Angst nicht bestehen zu können, die Angst, was passiert mit meinem Kind, wenn mein Kind nicht da „hoch“ kommt, kein Abitur hat... Es sind die Erwartungen der Gesellschaft, denen sich die Eltern verschreiben. \linelabel{C3_44} 

Den Dialog darüber unterstütze ich, indem ich frage: „Auf welchen Gebieten haben Sie Angst, dass Ihr Kind zu kurz kommt?“ Dann werden oft Zuwendung und Lernanreize genannt. Ich lade Eltern auch ein, einen Vormittag in der Gruppe zu erleben oder spreche im Elternabend darüber. \linelabel{C3_36} 

Bei Elternabenden laden wir zum Beispiel Heilpädagogen oder Sprachtherapeuten ein und stellen die Zusammenarbeit mit diesen Partnern vor.  \linelabel{C3_37}

Einmal habe ich die Heilpädagogin und die Sprachtherapeutin gebeten von ihrer Arbeit mit den Kindern im Elternabend zu berichten. Die Sprachtherapeutin hat mit Einwilligung der Eltern ein Video von dem Kind mit Down-Syndrom gezeigt. Da gab es ein Aha-Erlebnis bei den Eltern -- es ist nicht selbstverständlich, dass wir gesund sind. Mir kann heute etwas passieren, dann bin ich behindert und will genauso geschätzt werden wie gestern noch.\linelabel{C3_38}

Ich sage auch zu den Eltern, versetzen Sie sich doch mal in die Lage der anderen Eltern. Es gibt aber auch kritische Eltern... Wir versuchen die Ängste aller Eltern ernstzunehmen. \linelabel{C3_309}

Bei einem Kind mit starkem ADHS haben wir nach einem halben Jahr entschieden, dass wir es nicht tragen können. Das Kind war sehr unruhig und hat seine Grenzen sehr schlecht einschätzen können, so dass er sich verletzt hat und andere Kinder.  \linelabel{C3_39}
	
\emph{F: Unter welchen Bedingungen hätten Sie das gewährleisten können?}

A: Eine Gruppengröße von acht Kindern und drei Fachkräfte. Dieses Kind hat immer eine Fachkraft gebunden, diese hat das Kind beobachten müssen, um dann schnell da sein zu können und einzuschreiten, damit auch der Lernprozess gewährleistet ist und das Kind versteht, wenn ich das mache, könnte das und das passieren. Den Eltern fiel es sehr schwer, die Entscheidung zu akzeptieren. Sie haben gesagt: „Ihr gebt euch solche Mühe!“  \linelabel{C3_40}

\emph{F: Was sind Ihrem Ermessen nach notwendige Voraussetzungen für das Gelingen von Inklusion, die von „oben“ garantiert werden sollten?} 

A: Kleine Gruppen und ein höherer Personalschlüssel. Die Größe der Gruppe und der Personalschlüssel sind das A und O. \linelabel{C3_41} 

Gute Vernetzung mit zusätzlichen Fachkräften. Dafür braucht es auch die Offenheit von den anderen. Oft müssen Eltern lange Wartezeiten in Kauf nehmen, bis die entsprechenden Stellen eine genaue Entwicklungsdiagnostik stellen können. Die außen stehenden helfenden Kräfte in Verbindung zu bringen, darin sehe ich eine wesentliche Voraussetzung für das Gelingen von Inklusion. Zurzeit ist es ein Glücksspiel.  \linelabel{C3_42}

Außerdem fehlt die gesamtgesellschaftliche Anerkennung von Erzieherinnen. Oft hört man entwertende Bemerkungen, wie die spielen ja nur oder Basteltanten. Der tiefere Sinn und die fördernden Aspekte, die sich hinter solchen, von der Erzieherin initiierten, Aktivitäten verbergen, können gar nicht alle aufgezählt werden.  \linelabel{C3_43}

\emph{F: Gibt es bildungspolitische Entscheidungen, die der Umsetzung von Inklusion im Weg stehen?} 

A: Wenn ich sehe, was die Politik will, ist der Ansatz nicht schlecht. Aber wie soll die Basis das ausführen? Die Ideen sind da, aber an der konkreten Umsetzung fehlt es, das Geld fehlt. Wenn es an die Umsetzung kommt, fühlen wir uns manchmal allein gelassen. Wir sollen Berichte schreiben, Gespräche führen, Übersetzen. Wer aber bezahlt den Dolmetscher? \linelabel{C3_405}

Es gibt viele Wenn und Aber. „Wir schreiben uns das auf die Fahne, aber nachgedacht haben wir nicht. Ach so, dafür brauchen wir Geld? Aber das haben wir nicht.“ 
Deshalb bin ich auch als Leitung rein, um gezielt Sprachförderung anzubieten, weil ich den Bedarf gesehen habe und auch, weil ich den Kontakt zu der Basis suche. Dann sagen die Eltern, Sie kennen mein Kind doch auch ein bisschen.  \linelabel{C3_45}

Ich denke, bei Bildungsmaßnahmen wird erst bei einer bestimmten Schicht angefangen. Viele Familien können sich Bildungsmaßnahme gar nicht leisten. Dass dadurch eine Elite herangezogen wird, finde ich bedenklich.\linelabel{C3_46} 

In einem solchen Wohngebiet wie dem unseren die Chancen der Kinder zu erhöhen und deren Potential wach zu kitzeln, darum sollte es gehen. Da geht es gar nicht um die Kinder mit Behinderung.\linelabel{C3_406}  

Die meisten unserer Kinder würden bis zur Pubertät nicht so etwas wie ein Theater sehen. Etwas hochwertiges und exklusives wie ein Theaterbesuch und das damit verbundene Ambiente verstehe ich als Bildungsmaßnahme in Richtung Chancengleichheit. Die Kinder staunen und sagen: „Da wollte ich ja schon immer mal hin.“ Um Theaterbesuche zu ermöglichen, spreche ich Sponsoren an, Stiftungen. \linelabel{C3_47} 

\emph{F: Sie übernehmen das Akquirieren von Spenden?}

A: Ja.  \linelabel{C3_48}

\emph{F: Was sollte sich ändern?} 

A: Räumlichkeiten und Geld sollten zur Verfügung gestellt werden. Ein Bälle-Bad, welches für die Körperwahrnehmung eingesetzt werden kann, braucht Platz, genauso eine Therapieschaukel. Um die Sensibilität zu stärken, ist auch das Spielen mit Wasser ein geeignetes Medium, aber im Innenbereich gibt es bei uns keine Möglichkeiten zur Umsetzung. 
Auch eine ruhige Atmosphäre für Elterngespräche würde ich mir wünschen. Nicht im Büro, wo das Telefon klingelt. 
Bei den Räumen sehe ich Engpässe. Wenn man den Orientierungsplan umsetzen und den Kindern mit erhöhtem Unterstützungsbedarf gerecht werden will, braucht man Räume ohne Ende.  \linelabel{C3_49} 

\emph{F: Was kann der Träger dazu beitragen?} 

A: Der Träger ist sehr offen und trägt vieles mit. Einmal hatten wir in einer Gruppe drei Kinder mit diagnostizierten besonderen Bedürfnissen. Alle drei hatten eine Integrationshilfe. Daraufhin wurde die Kinderzahl in der Gruppe auf 20 reduziert. Die Kosten hat der Träger übernommen, nicht die Eltern.  \linelabel{C3_50}

\emph{F: Wo sehen Sie Ihre besonderen Aufgaben als Leitung?} 

A: In der zwischenmenschlichen Kommunikation zwischen allen Beteiligten: Eltern, Kinder, Kollegen und in der Vernetzung mit anderen Institutionen, diese in Teamsitzungen zu holen und so fehlende Informationen zu beschaffen. \linelabel{C3_51} 

\emph{F: Welche Schritte wurden konkret unternommen, um Inklusion einzuleiten?}

A: Nicht wir waren die Initiativträger. Der Anstoß kam von außen. Wir wurden vor 16 Jahren von der Patin eines Kindes angefragt, die ihre eigenen Kinder bereits in der Einrichtung hatte, ob wir uns vorstellen könnten, ihr Patenkind aufzunehmen, welches eine spastische Lähmung hat und im Rollstuhl sitzt und dann sind wir in die Überlegung gegangen. Was brauchen wir? Breite Türen, so dass das Kind mit seinem Rollstuhl durch kann. Haben wir das? Ja. Als dann der Umbau des Sanitärraumes bevor stand, wurde eine Behindertentoilette eingerichtet. 
Außerdem braucht es Erzieherinnen, die bereit sind, sich auf so etwas einzulassen. Letztlich haben sich dann zwei Gruppen dazu bereit erklärt. So hat bei uns die Inklusion Einzug gehalten und ist jetzt Normalität. \linelabel{C3_52} 

Auch zuvor gab es schon Kinder mit erhöhtem Förderbedarf, aber politisch und gesellschaftlich ist zu dieser Zeit gerade etwas aufgebrochen. Die Gesellschaft und Politik schaut anders hin. Dadurch können Erfolge erzielt werden, da frühes Ansetzen besser ist. Wenn ich mir jetzt Fotos von damals anschaue, dann denke ich, wenn das Kind damals keine Wahrnehmungsstörung hatte... \linelabel{C3_53} 

Den Elternabend, als wir vor 16 Jahren unser erstes Kind mit dieser spastischen Lähmung aufgenommen haben, werde ich nicht vergessen. Damals haben wir die Mutter gebeten aus ihrem Leben mit dem Kind zu berichten. Zuerst hat sie abgelehnt, da sie die Erfahrung gemacht hatte, dass die Menschen um sie einen Bogen machen. Dann hat sie sich aber doch dafür entschieden. Es ist so viel aufgebrochen, als diese Mutter aus ihrem Leben und von ihren Ängsten berichtet hat. Bei den Eltern hat sich etwas getan, sie sind anders auf die Mutter zugegangen und diese Offenheit hat sich auch auf die Kinder übertragen. \linelabel{C3_54} 

 Weiterbildungsmöglichkeiten sind uns auch sehr wichtig. 
Um uns in Bezug auf unsere Kinder mit Hör- und Sehbehinderung weiterzubilden, kooperieren wir mit dem Bildungs- und Beratungszentrum für Hörgeschädigte in [Ortsname] und der Beratungsstelle in der Blindenschule in [Ortsname]. \linelabel{C3_55}  

\emph{F: Was waren und sind Stolpersteine auf dem Weg zur inklusiven Bildungseinrichtung?} 

A: Die Antwort ergibt sich aus dem vorher Gesagten: fehlende Offenheit, Überzeugungsarbeit bei den Eltern leisten müssen.\linelabel{C3_56}  

Räumlichkeiten sind auch Stolpersteine, wenn diese umgestaltet werden müssen.\linelabel{C3_506}
 
Die Mutter hat ein Rollbrett für das Kind mit der Spastik angeschafft, darauf haben wir das Kind geschnallt. So haben wir einen Blick dafür bekommen, was es alles gibt. Später haben wir auch für die anderen Kinder solche Rollbretter angeschafft und der konnte dann mit den anderen Kindern zusammen draußen sein herum fahren, seine Arme und Hände konnte er bewegen, von der Spastik waren vor allem die Beine betroffen, und so bekamen die gesunden Kinder einen Einblick und konnten sich voneinander lernen.\linelabel{C3_57} 
 
Aber der Turnraum, der nur mit Treppen zu erreichen ist, stellt dann schon wieder ein Hindernis dar.\linelabel{C3_58}

Ebenso die Freizügigkeit für Kinder im Rollstuhl, dass diese barrierefrei mit dem Rollstuhl geschoben werden können. 
Das Kind hat später angefangen zu robben...\linelabel{C3_508}

\emph{F: Welche Herausforderungen werden an das Team gestellt, wenn das Konzept in Richtung Inklusion umgestellt wird?}

A: Es braucht Kraft und kostet Nerven. Zum Beispiel, wenn ein Kind selbst essen lernen soll und hat Schwierigkeiten beim Kauen, dann läuft alles runter, das heißt auch immer wieder umziehen -- das ist mühsam. Sich nicht entmutigen lassen, ist da wichtig und auch mal fünf Minuten raus gehen können, um neue Kraft zu gewinnen und mit einem Lächeln zurück zu kommen, auch das ist o.k. \linelabel{C3_59} 

\emph{F: Was sind häufige Fragen der Erzieherinnen, die anzeigen, dass Beratungs- oder Unterstützungsbedarf besteht?}

A: „Darf ich das Kind im Gesamtteam vorstellen?“ „Kannst du mal dazu kommen, in die Gruppe?“ „Die Eltern machen sich Sorgen, meinst du, wir bekommen das hin?“ \linelabel{C3_60} 

Wir dürfen ja keine Diagnosen stellen, da sind die Grenzen der Fachlichkeit, aber unsere Beobachtungen sind wichtig. Wir nutzen die Ravensburger Beobachtungsbögen, auch um Defizite in den Blick zu bekommen. Dann bitten wir um Überprüfung der Gesamtentwicklung des Kindes und holen uns so Unterstützung. \linelabel{C3_61}  

\emph{F: Gab es oder gibt es Ängste unter den Mitarbeitern? Welcher Art? Wie wird damit umgegangen?}

A: Ja, die Sorge zu wenig Zeit für das einzelne Kind zu haben. Der Druck von außen, dem Orientierungsplan entsprechen und darüber hinaus noch den Kindern mit erhöhtem Förderbedarf gerecht werden zu müssen.
Wir würden gern, aber die Angst es nicht richtig zu machen, steht im Raum. \linelabel{C3_62} 

Ein Mädchen mit Down-Syndrom sollte in die Einrichtung kommen. Davor ist schon ganz viel Elternarbeit passiert, das war zeitaufwändig. Als sie dann kam, konnte sie weder sitzen noch gehen. Sie war ein Jahr da und hat auch in diesem Jahr nicht sitzen und laufen gelernt. Hinzu kamen häufige Klinikbesuche, weil sie Probleme mit der Atmung hatte. Es kam immer mehr hinzu. Wir haben uns gefragt: „Sind wir jetzt dafür verantwortlich, dass wir dem Kind das Sitzen beibringen?“ Die Eltern hätten das gut gefunden. Sie hatten den Wunsch. Wir haben dann entschieden, dass eine entsprechende Sondereinrichtung das beste für das Kind ist. \linelabel{C3_63} 

Wenn ein solches Kind im Morgenkreis auf einer Matte neben dran liegt, dann tut mir das in der Seele weh. Aber ich kann es auch nicht immer in den Arm nehmen, weil recht und links von mir Kinder sitzen, die mich auch brauchen, um sitzen...\linelabel{C3_64} 

\emph{F: Was wäre in diesem Beispiel eine Lösung? Mehr Fachkräfte?}

A: Wenn vier oder fünf Erwachsene im Raum sind, werden die Kinder auch ganz kirre. Sie brauchen ein ruhiges Umfeld mit nicht so viel Gewusel, da sie zu hause schon so viel Unruhe haben. Deshalb sehe ich die Lösung in kleineren Gruppen. \linelabel{C3_65} 

\emph{F: Welche Unterstützung erfährt das Team? 
Gibt es eine gemeinsame Planung und Zusammenarbeit unter den Beteiligten? Gibt es Teamsitzungen, in denen das Verhalten von Kindern diskutiert wird?}
 
A: Jeden dritten Mittwoch können die Erzieherinnen Kinder aus ihrer Gruppe vorstellen, bei denen sie Klärungsbedarf haben. Das Zeitfenster hierfür ist offen, der Bedarf wird vorher von mir erfasst. Darüber hinaus können auch Zeitfenster an anderen Tagen eingeräumt werden, wenn die Kollegen nicht bis zum dritten Mittwoch warten wollen. An jedem vierten Mittwoch berichtet jede Gruppe über das Gruppengeschehen.\linelabel{C3_66} 

Zudem nehmen wir an Fortbildungen für das gesamte Team teil. Der Träger bietet ein Angebot an Fortbildungsmöglichkeiten im Fortbildungskatalog der Caritas an und ist darüber hinaus offen für unsere Wünsche.\linelabel{C3_67} 

Als Leiterin stehe ich als Ansprechpartnerin zur Verfügung und wenn ich an meine Grenzen komme, hole ich Fachpersonal dazu. Manchmal sagen meine Kollegen zu mir: „Du hast doch immer für alles Verständnis und sagst, wir schaffen das.“ Dann ist es besser, wenn jemand von außen kommt, und aus der Praxis Beispiele bringt, so dass die Erzieherinnen sich verstanden fühlen und noch einmal eine andere Autorität zu Wort kommt. \linelabel{C3_68} 

\emph{F: Brauchen Erzieherinnen Zusatzqualifikationen, um inklusiv zu arbeiten?} 

A: Ja, aber vor allem das Herz dazu! Offenheit, Liebe und dann Zusatzqualifikationen.\linelabel{C3_69} 

\emph{F: In welchen Bereichen sind Kenntnisse Ihrer Meinung nach hilfreich?}

A: Wir sind kein Arzt, aber Grundkenntnisse über die Behinderung sind von Nöten, wie sich der Entwicklungsweg und -prozess gestaltet, welche besonderen Bedürfnisse diese Kinder haben und was ich denen zumuten kann, auch um die Schwellenangst abzubauen.\linelabel{C3_70} 

\emph{F: Welches Handwerkszeug müssen Erzieherinnen mitbringen beziehungsweise was macht gute Erzieherinnen aus?}

A: Empathie, mit beiden Beinen auf dem Boden stehen, Spontanität und Flexibilität, Kinder sind jeden Tag anders, das erfordert ein hohes Maß an Flexibilität.\linelabel{C3_71} 

Erzieherin müssen sich auf die Eltern, die Kinder und die schwierigen Lebenssituationen einlassen und sie so nehmen, wie sie sind. Ich erinnere mich an eine Geschichte mit einem Reiter auf einem Pferd und einem Menschen, der nebenher läuft. Um den da unten zu erreichen, muss ich vom Pferd steigen, nicht ihn mit hinauf nehmen wollen, Absteigen ist besser. Oft denken Erzieherinnen, in solch einer Situation war ich auch mal, da bin ich diesen und jenen Schritt gegangen. Dabei vergessen sie aber, dass sie andere Kompetenzen an der Hand haben und andere Startvoraussetzungen durch ihre eigene Biografie mitbringen.\linelabel{C3_72}  

\emph{F: Welche Bedeutung hat die Elternarbeit im Hinblick auf Inklusion?
Wie werden die Eltern bei Entscheidungen eingebunden?} 

A: Überhaupt hat die Elternarbeit einen sehr hohen Stellenwert, nicht nur in Bezug auf Inklusion. Es gibt sehr viele Gespräche. Wir haben die Anzahl an Elternabenden für alle Eltern reduziert zu Gunsten einer höheren Frequenz von Kleingruppengesprächen und runden Tischen, da in diesem Rahmen viel besser Entwicklungsgespräche möglich sind. Gemeinsame Elternabende finden statt, wenn Referenten kommen.
Entwicklungsgespräche sind nach jedem Geburtstag eines Kindes ein Muss oder auch, wenn sich gravierende Veränderungen beim Kind zeigen. Auch beim Bringen und Holen finden Gespräche mit den Eltern statt. Je nach Bedarf kann es drei, vier oder auch fünf Elterngespräche geben oder auch nur eins.\linelabel{C3_73} 

Der Elternbeirat ist ein weiteres Gremium, wodurch Eltern in ihren Entscheidungen mit eingebunden werden. Der Elternbeirat gibt dann die Wünsche weiter, manchmal trauen sich die Eltern auch nicht persönlich auf mich zu zukommen.\linelabel{C3_74} 

\emph{F: Was wird in Ihrer Institution organisiert, um eine Beziehung zu den Eltern aller Kinder aufzubauen?}

A: Neben den Elterngesprächen gibt es ein Gruppentagebuch, in dem die Eltern jeden Tag nachlesen können, was die Gruppe gemacht hat. Dort werden auch manchmal Lieder oder Texte beigelegt, die die Eltern kopieren können. Zudem gibt es gemeinsame Aktionen wie Feste feiern, Kinder und Eltern basteln gemeinsam Laternen oder Schultüten, Spieltage an denen die Eltern dazukommen können, um zu sehen, welche Spiele es gibt oder Wandertage, an denen die Eltern die Gruppe begleiten können, um auch zu sehen, in den Wald kann ich mit meinem Kind jederzeit gehen, das kostet nichts. Diese Aktionen sind alle in denen normalen Gruppenalltag integriert und dienen dazu, die Kommunikation zwischen den Eltern und dem Kind zu fördern. Manche Eltern sagen dann beim gemeinsamen Laternen basteln: „Das habe ich noch nie gemacht. Ich wusste gar nicht, dass ich das kann.“\linelabel{C3_75} 

Ganz wichtig ist uns auch, dass die Eltern sich nicht zu schämen brauchen, wenn sie zusätzliche Hilfe benötigen, dass sie sich nicht schämen diese anzunehmen. Viele unserer Eltern haben einen großen Packen zu tragen und sind oft auch entmutigt. Das sind persönliche und bestärkende Gespräche sehr wichtig. Solche sind sehr anstrengend und verlangen viel Kraft.\linelabel{C3_76} 

\emph{F: Haben Sie durch die Umsetzung von Inklusion Veränderungen bei sich feststellen können? Bei den pädagogischen Mitarbeitern? Bei den Kindern?} 

A: Bereicherung! Auch in der Auseinandersetzung mit sich selbst, dass ich möchte, dass man mit mir so umgeht. Dafür muss ich auch etwas tun, das Ganze positiv sehen. Das ist ein großes Ziel Kinder nicht von vornherein auszugrenzen, sondern als Bereicherung anzusehen, auch in der täglichen Arbeit mit den anderen Kindern. Das Leben ist nicht nur Schlaraffenland. \linelabel{C3_77} 

Ich stelle nach diesem Interview fest, es ist gar nicht wenig, was wir hier machen, in diesem Kindergarten werde ich ganz anders gebraucht,  ich habe in dieser Einrichtung so viel gelernt. \linelabel{C3_78}  
\end{linenumbers*}

