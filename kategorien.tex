\chapter{Kategoriensystem}
\label{Kategoriensys}

% Beginn Querformat
\begin{landscape}
\begin{small}

% ===================== Tabelle 1 ============================== %
\begin{centering}
\begin{longtable}{p{2cm}p{8.5cm}p{4cm}p{4cm}p{1.5cm}}
\caption{Extraktion der gegebenen strukturellen Voraussetzungen in den jeweiligen Einrichtungen}\\
\tabhead
\dimension{Räume und Ausstattung}
& Extraraum nur Sprache & & & C1,~\ref{C1_18}\\
\cmidrule{2 - 5}
& heilpädagogisches Zimmer als Rückzugmöglichkeit & hohe Bedürftigkeit des Kindes, braucht jemanden für sich alleine, so dass Belastungen aufgefangen werden können & Kinder genießen Eins-zu-Eins-Kontakt & C1,~\ref{C1_22}\\
\cmidrule{2 - 5}
& während Freispielzeit Gruppenraumtüren offen & & zusätzlicher Gang als Antwort auf einen hohen Bewegungsdrang & C3,~\ref{C3_3}\\
\cmidrule{2 - 5}
& aufgrund baulicher Voraussetzungen wenig Zusatzräume & & & C2,~\ref {C2_802}\\
\cmidrule{2 - 5}
& bestehender Baubestand müsste entsprechend verändert werden & & „Wenn man das Ernst nehmen will, muss man richtig viel Geld in die Hand nehmen.“ & C1,~\ref{C1_25}\\
\cmidrule{2 - 5}
& Speicherausbau langwierig & & & C1,~\ref{C1_47}\\
\cmidrule{2 - 5}
& hemmend: Räumlichkeiten, die umgestaltet werden müssen & & & C3,~\ref{C3_506}\\
\cmidrule{2 - 5}
& hemmend: Türschwellen in den Gruppenräumen, ungeeignete Türen  & alter Baubestand, Umbau bisher nicht bewilligt, Kostenfrage & Türschwellen problematisch für Kind, das krabbelte oder Rollstühle $\rightarrow$ zu kompensieren versucht durch Holzkeil als Rampe, Holzkeil musste zum Schließen der Tür wieder entfernt werden & C2,~\ref {C2_78}\\
\cmidrule{2 - 5}
& enger Flur hinderlich für Rollstuhl & & Grenzen der Inklusion für Kinder im Rollstuhl & C2,~\ref{C2_80}\\
\cmidrule{2 - 5}
& Türen müssen breit genug sein für Rollstuhl, Behindertentoilette & & & C3,~\ref{C3_52}\\
\cmidrule{2 - 5}
& Turnraum nur über Treppe zu erreichen & & & C3,~\ref{C3_58}\\
\cmidrule{2 - 5}
& Freizügigkeit für Kinder im Rollstuhl für barrierefreies Schieben & & & C3,~\ref{C3_508}\\
\cmidrule{2 - 5}
& fehlende Möglichkeiten zum Kochen & & Bedarf der Eltern noch nicht beantwortet & C3,~\ref{C3_7}\\
\cmidrule{2 - 5} 
& ausreichend Räumlichkeiten und finanzielle Ressourcen für Ausstattung zur Förderung der Körperwahrnehmung: Bälle-Bad, Therapieschaukel, Erfahrungen mit Wasser im Innenbereich, ruhige Atmosphäre für Elterngespräche gewünscht (im Büro Unterbrechungen durch Telefonanrufe)\vspace{1.5em}& & & C3,~\ref{C3_49}\\

\newpage
\dimension{Finanzierung}
& Finanzen begrenzt, Einrichtung nicht besser gestellt als andere, 
eigener Förderverein, Spendenaquise, die aus dem Haus heraus gemanagt wird, wodurch  alle Projekte finanziert werden & & „Sonst könnten wir diese Arbeit nicht machen.“ & C1,~\ref{C1_37}\\
\cmidrule{2 - 5}
& Speicherausbau komplett über Spenden finanziert & & & C1,~\ref {C1_47}\\
\cmidrule{2 - 5}
& neue Projekte laufen über den Förderverein, wie die Naturpädagogik & & & C1,~\ref{C1_408}\\
\cmidrule{2 - 5}
& Leitungstätigkeit im Netzwerk und Qualitätsentwicklung nicht dotiert → Hürden: die Verwaltung bringt es nicht fertig die Dotierung zu regeln, Bedarf auch an einer stellvertretenden Leitung & Netzwerkarbeit aus Überzeugung: „Entweder wir machen es so, weil wir uns so entwickelt haben oder man macht es nicht und lässt es bleiben.“ & Einrichtung Deutschland weit bekannt, Arbeit gewollt, von den Hochschulen oft kontaktiert & C1,~\ref{C1_48}\\
\cmidrule{2 - 5}
& Akquirieren von Spenden, um Theaterbesuche zu ermöglichen & & & C3,~\ref{C3_47}\\
\cmidrule{2 - 5}
& Kindergarten erhält Spendeneinnahmen (10.000 Euro pro Jahr), so dass Kinder, deren Familien sich das Mittagessen nicht leisten können, über einen Gutschein unterstützt werden & & & C2,~\ref{C2_29}\\ 
\cmidrule{2 - 5}
& durch Spendeneinnahmen der AWO werden Bildungsangebote abgedeckt & Familien, die nur sehr wenig Geld haben, die Möglichkeit gegeben, dass Bildungsangebote von der Kita aus durchgeführt werden & Eintritt Museumsbesuch, musikalische Früherziehung, Tanzkurs mit ehrenamtlicher Tanzlehrerin, Schwimmkurs & C2,~\ref{C2_30}\\
\dimension{Der Einfluss des Trägers}
& geregelter Finanzrahmen, der für Fortbildungen zur Verfügung steht, 
Themen werden autonom festgelegt, Träger (Diakonie) wird informiert, viel Spielraum, Träger unterstützt inhaltlich, in dem er den Rahmen bietet und auch die notwendigen Spielwiesen, um sich konzeptionell weiter zu entwickeln & & Leitungen sind sehr autonom, gestalten die ganze Arbeit & C1,~\ref{C1_34}\\
\cmidrule{2 - 5}
& keine starren Verwaltungsvorschriften gebraucht & Ballungsgebiet (kulturelle, soziale Vielfalt, schwere Themen des Lebens wie Armut, Kindeswohlgefährdungen) & & C1,~\ref{C1_35}\\
\cmidrule{2 - 5}
& Wir können uns den Themen widmen, die wirklich dran sind: „Da steht dieser Träger hinter seinen Einrichtungen, weil die sagen, wir sind hier im Brennpunkt und wenn unsere Leute hier mit diesen Themen kommen, dann sind die dran.“ & & Siehe [Schritte in Richtung Inklusion und wahrgenommene Veränderungen] & C1,~\ref{C1_36}\\
\cmidrule{2 - 5}
& AWO Aktion: „Wenn ich groß bin, werde ich arm!“ & & Siehe [Finanzierung] & C2,~\ref{C2_29}\\ 
\cmidrule{2 - 5}
& Bedürfnis nach Unterstützung durch die AWO erfüllt, werden gefragt: „Was braucht ihr jetzt vor Ort, was für Bildungsausflüge wären wichtig?“  
Austausch zwischen Leitungsebene und Trägerebene, Zeit für Austausch wird zur Verfügung gestellt, Möglichkeiten Einfluss zu nehmen & & & C2,~\ref {C2_601}\\
\cmidrule{2 - 5}
& von der AWO organisierter Dankeschön-Abend für die Spender – die sieben Einrichtungen der AWO haben ihre Bildungsprojekte vorgestellt & einerseits mehr Arbeit, andererseits entlastend, da zentral gesteuert & & C2,~\ref{C2_602}\\
%\cmidrule{2 - 5}
& Leitung arbeitet gern bei der AWO trotz ihrer Angabe vergleichsweise wenig zu verdienen, erfährt Wertschätzung von Seiten der Geschäftsführung & & gibt Wertschätzung an das Team weiter & C2,~\ref{C2_603}\\
\cmidrule{2 - 5}
& Träger (katholische Kirchgemeinde) als offen erlebt, trägt vieles mit → Gruppenstärke wurde auf 20 reduziert, da in der Gruppe eine Häufung von Kindern mit diagnostiziertem besonderen Förderbedarf war (3 Kinder mit Integrationshilfe)\vspace{0.5em} & & & C3,~\ref{C3_50}\\

\dimension{Fachkraft-Kind-Relation}
& 18, 19 Kinder pro Gruppe, pro Gruppe 2,5 Fachkräfte
zusätzliche Fachkräfte durch Familiennetzwerkarbeit & & & C1,~\ref{C1_5}\\
\cmidrule{2 - 5}
& Fachkräftemangel eklatant & & & C1,~\ref{C1_26}\\
\cmidrule{2 - 5}
& in jeder Gruppe sind 20 Kinder altersgemischt & & & C2,~\ref{C2_1}\\
\cmidrule{2 - 5}
& pro Gruppe 3,2 Stellen, (verlängerte Öffnungszeiten von 7:30 bis 18 Uhr $\rightarrow$ Schichtdienst,  Personalbemessung bemisst sich nach den Hauptbetreuungszeiten & & & C2,~\ref{C2_11}\\
\cmidrule{2 - 5}
& personell gut ausgestattet & & Siehe [Qualifikation] & C2,~\ref{C2_46}\\ 
\cmidrule{2 - 5}
& 22 oder 23 Kinder pro Gruppe werden von je zwei Fachkräften betreut & & & C3,~\ref{C3_11}\\
\cmidrule{2 - 5}
& verschiedene Gruppen für unterschiedliche Bedarfe: Regelgruppen und Gruppen mit verlängerten Öffnungszeiten $\rightarrow$ Regelgruppen von 7:45 bis 12:30 Uhr, gehen über die Mittagszeit nach hause und werden an drei Nachmittagen von 14:00 bis 16:00 Uhr betreut, die Gruppen mit verlängerten Öffnungszeiten wahlweise von 7:30 bis 13:30 Uhr oder von 8:30 bis 15:00 Uhr betreut & & & C3,~\ref{C3_2}\\
\cmidrule{2 - 5}
& mehr Personal gebraucht, um speziellen Kindern wirklich gerecht werden zu können\vspace{0.5em} & & & C3,~\ref{C3_29}\\

\dimension{Personalstruktur}
& durch Netzwerkarbeit sind zusätzliche Fachkräfte mit spezifischer Qualifikation im Haus, im Bereich Naturpädagogik, Heilpädagogik, im Sprachbereich, also Logopäden, auch noch eine Spracherzieherin, im Ehrenamt \emph{Bücherwurmfrauen}, ehrenamtliche Sportler für die \emph{Ringergruppe} & & bieten Kleingruppenprojekte an $\rightarrow$ Siehe [Angebote] & C1,~\ref{C1_6}\\
\cmidrule{2 - 5}
& Heilpädagogen und Sprachtherapeuten & & interdisziplinärer Austausch & C1,~\ref{C1_9}\\
\cmidrule{2 - 5}
& eine Musikpädagogin im Haus & & & C1,~\ref{C1_11}\\
\cmidrule{2 - 5}
& alle Mitarbeiter sind im Hinblick auf Sprachförderung geschult, zwei Logopäden im Haus, eine Sprachförderkraft, Bücherwurmfrauen im Ehrenamt in der Bibliothek, lesen vor & Siehe {[Übergeordnete Werte und Ziele]} & & C1,~\ref{C1_20}\\
\cmidrule{2 - 5}
& Heilpädagogen über die Integrationshilfe zusätzlich, zwei mal zwei Stunden in der Woche finanziert in der Einrichtung, Heilpädagogen kommen zu den Kindern, die Integrationshilfe bekommen, begleiten die Kinder und leiten dann auch die begleitende Hilfe an, die in Form von FSJ-Praktikanten umgesetzt ist, damit die auch möglichst einen großen Anteil der Zeit hier sind, je nach Art der Behinderung beantragen wir die Integrationshilfe, ggf. auch die begleitende Hilfe über den § 35 a oder § 53 folgende & & & C2,~\ref{C2_7}\\
\cmidrule{2 - 5}
& Leitung freigestellt zu 50\% & & & C2,~\ref{C2_12}\\ 
\cmidrule{2 - 5}
& insgesamt zwölf Erzieher: zwei Erzieherinnen im Anerkennungsjahr, eine Kinderpflegerin, drei FSJ-Kräfte, über die Integrationshilfe finanziert, eine Sprachförderfrau, die ist also Erzieherin und Heilpädagogin, stundenweise für Sprachförderung angestellt & & & C2,~\ref{C2_13}\\ 
\cmidrule{2 - 5}
& Leitung in jeder Gruppe mit zehn Prozent drin, zwanzig Prozent Migrationsanteilsaufstockung $\rightarrow$ 
Stadt Freiburg eingeführt seit letztem Jahr: Einrichtungen bekommen personelle Aufstockung, wenn hoher Migrationsanteil und diese Kinder wiederum einen erhöhten Förderbedarf haben oder aus schwierigen Familien kommen $\rightarrow$ Elternarbeit erschwert & & Siehe {[Blick auf die aktuelle politische Situation]} & C2,~\ref{C2_14}\\ 
\cmidrule{2 - 5}
& Heilpädagogen über Integrationshilfe, die Erzieherin und Heilpädagogin der Sprachheilschule ist bei uns in der Einrichtung & & & C2,~\ref{C2_15}\\
\cmidrule{2 - 5}
& für die Zukunft wäre es gut noch eine größere Professionalität reinzubringen, indem Heilpädagogen fest in Team sind & & & C2,~\ref{C2_47}\\ 
\cmidrule{2 - 5}
& Leitung 100~\% frei gestellt, bei Personalmangel durch Krankheit springt sie ein und Durchführung der Sprachförderung & sucht Kontakt zur 'Basis' & & C3,~\ref{C3_13}\\
\cmidrule{2 - 5}
& halbjährige Weiterbildung in Sprachförderung für Erzieherin und Leitung & & Austausch im Team und Erstellung einer Sprachkonzeption & C3,~\ref{C3_18}\\ 
\cmidrule{2 - 5}
& Kollegen sind zusätzlich geschult in Sprachförderung oder Bewegungserziehung & Siehe {[Personalstruktur]} & & C3,~\ref{C3_34}\\
\cmidrule{2 - 5}
& Beantragen einer zusätzlichen pflegerischen Kraft für Kinder, deren Selbstständigkeit eingeschränkt ist, die Hilfe beim Essen oder beim Gang zur Toilette benötigen; umgesetzt durch eine Frau, die aus der Gemeinde gewonnen werden konnte, die zuvor in einer Arztpraxis gearbeitet hat und einfühlsam war\vspace{0.5em}& & & C3,~\ref{C3_24}\\

\dimension{Blick auf die aktuelle politische Situation}
& Deutschland weit schätze ich das so ein, dass Kindergärten bereits viel Inklusionsarbeit leisten, das aber in der Schule steil abfällt. Die Kitas sind in ihrer Gesamtstruktur anders aufgestellt, sie sind näher an den Familien, bekommen Inklusion eher hin und wollen es eher hin bekommen. & & & C1,~\ref{C1_24}\\ 
\cmidrule{2 - 5}
& Inklusion ist ein Trend – aufpassen, wo Inklusion drauf steht, ist da auch Inklusion drin & & & C1,~\ref{C1_27}\\
\cmidrule{2 - 5}
& personell sehr gut besetzt im Vergleich zu den Jahren vorher, weil es über den Bildungs- und Orientierungsplan eine Aufstockung gab, und weil Einrichtungen, die längere Öffnungszeiten haben, mehr Personal bekommen – die Stadt tut sehr viel, das kann ich lobend erwähnen & & & C2,~\ref{C2_14}\\
\cmidrule{2 - 5}
& Jugendamt Vorgabe einzusparen und genau zu überprüfen, wird Hilfe gebraucht; Heilpädagogin zwei mal zwei Stunden in der Woche in der Einrichtung (Höchstmaß);
begleitende Hilfe mit 300 € nicht gern finanziert (durch Vorpraktikanten oder FSJler umgesetzt) → „Ich freu mich immer, wenn das kommt, man wird da schon bescheiden.“
& & hoher Arbeitsaufwand, da Anleitung erforderlich;
Aufgaben der Heilpädagogen (mit dem Kind arbeiten, Elterngespräche durchführen, die Gespräche mit den Erziehern, Anleitung von den FSJlern) schwer umsetzbar $\rightarrow$ Siehe [notwendige Voraussetzungen von regierungsamtlicher Seite] & C2,~\ref{C2_76}\\ 
\cmidrule{2 - 5}
& Regelgruppen arbeiten am Nachmittag verstärkt gruppenübergreifend, da Kinderzahl der Regelgruppen variiert und so die Gruppen oft kleiner sind & & Erzieherinnen können gut auf den Bedarf reagieren, ein Angebot vom Morgen noch einmal wiederholen oder ein extra Angebot machen → Anregung an die Mutter, ihr Kind in die Nachmittagsbetreuung zu bringen, da individuelle Förderung besser umsetzbar in Kleingruppen & C3,~\ref{C3_12}\\
\cmidrule{2 - 5}
& „Wir {[die Politik]} schreiben uns das auf die Fahne, aber nachgedacht haben wir nicht. Ach so, dafür brauchen wir Geld? Aber das haben wir nicht.“ & & deshalb  Sprachförderung von Leitung getragen & C3,~\ref{C3_45}\\
\cmidrule{2 - 5}
& Gesellschaft und Politik schaut anders auf die Kinder & & Dadurch können Erfolge erzielt werden, da frühes Ansetzen besser ist & C3,~\ref{C3_53}\\
\cmidrule{2 - 5}
& Druck von außen, dem Orientierungsplan entsprechen und darüber hinaus noch den Kindern mit erhöhtem Förderbedarf gerecht werden zu müssen & & Angst und Überforderung & C3,~\ref{C3_62}\\
\cmidrule{2 - 5} 
& Wunsch nach gesamtgesellschaftliche Anerkennung der Erzieherinnen in ihrer Rolle und ihren Aufgaben  & & & C3,~\ref{C3_43}\\ 
\cmidrule{2 - 5}
& Berichte schreiben, Gespräche führen, übersetzen → Wer aber bezahlt den Dolmetscher? mangelt an Geld für die Umsetzung der politischen Ideen & & „Wir fühlen uns allein gelassen!“ $\rightarrow$ Bedarf an Unterstützung, um die politischen Ideen umsetzen zu können\vspace{0.5em}& C3,~\ref{C3_405}\\

\dimension{notwendige Voraussetzungen von regierungsamtlicher Seite}
& Offenheit, Finanzmittel, gute Konzepte & Inklusion ist eine Frage einer dezentralisierten Politik, findet im Quartier statt, [Inklusion: wenn Zugang zu den Bildungsmöglichkeiten für alle Menschen im Quartier gewährt wird] $\rightarrow$ bedeutet auch Veränderungen für die Schulen -- wie öffnet sich die Schule, Bildungsangebote für Eltern, Strategien, um die Familien in Ihrer Eigenart zu stärken?  & & C1,~\ref{C1_23}\\
\cmidrule{2 - 5}
& personelle Ausstattung, räumlichen und bauliche Ausstattung, Zeit für Besprechungen, Zeit und Geld für Fortbildungen, Weiterführung in der Schule, Vernetzung, multiprofessionelle Teams, Heilpädagogen fest angestellt sind oder Logopäden & & & C2,~\ref{C2_45}\\
\cmidrule{2 - 5}
& von oben auferlegt, dass jede Einrichtung eine Konzeption haben muss, nicht von oben vorgeschrieben, sonder die vom jeweiligen Team formuliert und ausgestaltet wird & & & C2,~\ref{C2_49}\\ 
\cmidrule{2 - 5}
& Auseinandersetzung im Team zur Frage der Haltung sollte von oben vorausgesetzt, Haltung an sich aber nicht vorgegeben werden & & & C2,~\ref{C2_50}\\ 
\cmidrule{2 - 5}
& kleinere Gruppen, Inklusion in jeder Einrichtung, aber das voraus, dass Bedingungen von vornherein verändert werden & & dass man da nicht Kinder, die schon da sind, wegschicken muss, weil man es einfach nicht bewältigen kann im Alltag & C2,~\ref{C2_53}\\ 
\cmidrule{2 - 5}
& Heilpädagogen in jeder Einrichtung mit einem Stellendeputat vertreten & Zeit unzureichend & Voraussetzung für Inklusion & C2,~\ref{C2_76}\\
\cmidrule{2 - 5}
& Inklusion machbar, aber alle Strukturen müssten grundsätzlich verändert werden, steht und fällt mit den Rahmenbedingungen, wie technische Hilfsmittel, Heilpädagogen fest angestellt, Krankengymnastik in der Einrichtung, multiprofessionelle Teams, kleinere Gruppen & & Voraussetzung für Inklusion & C2,~\ref{C2_81}\\
\cmidrule{2 - 5}
& Zeitmanagementfrage, Zeit haben für Austausch, höhere Anforderungen, wenn Kinder mit besonderen Bedürfnissen in der Einrichtung, Zeit für Vernetzung, aktiv sein in Arbeitskreisen & & & C2,~\ref{C2_42}\\ 
\cmidrule{2 - 5}
& Kleine Gruppen, um speziellen Kindern wirklich gerecht werden zu können & & & C3,~\ref{C3_29}\\
\cmidrule{2 - 5}
& Gruppengröße von acht Kindern und drei Fachkräften als Antwort auf den Bedarf eines Kindes mit starkem ADHS & & dem individuellen Anspruch dieses Kindes hätte entsprochen und der Wechsel in eine andere Einrichtung hätte vermieden werden können & C3,~\ref{C3_40}\\
\cmidrule{2 - 5}
& \emph{„Die Größe der Gruppe und der Personalschlüssel sind das A und O!“} & & & C3,~\ref{C3_41}\\
\cmidrule{2 - 5}
& Zu große Gruppen führen zu Überforderung der Fachkraft und machen individuelle Unterstützung unmöglich & & & C3,~\ref{C3_64}\\
\cmidrule{2 - 5}
& Lösung wird in kleineren Gruppen und nicht in mehr Personal gesehen & Bedarf der Kinder: ruhiges Umfeld im Kindergarten als Ausgleich für Unruhe zu hause, zu viele Erwachsene gefährden ruhige Atmosphäre & & C3,~\ref{C3_65}\\
\end{longtable}
\end{centering}

\newpage
% ===============    Tabelle 2    ============== %
\begin{centering}
\begin{longtable}{p{2cm}p{8.5cm}p{4cm}p{4cm}p{1.5cm}}
\caption{Extraktion der Auswertungskategorie ’Die Rolle der pädagogischen Fachkraft’}\\
\tabhead
\dimension{Inklusionsverständnis}
& Einrichtung verändert sich entsprechend, um dem Kind Bildungsangebote in der Einrichtung zu ermöglichen. In der Unterschiedlichkeit, in den unterschiedlichen Themen, liegt die Herausforderung. Konzept entwickelt, das die unterschiedlichen Themen, die hier auftreten, individuell beantwortet & & & C1,~\ref{C1_15}\\
\cmidrule{2 - 5}
& Zugang zu allen Bildungsangeboten, unabhängig von Behinderung, Migrationshintergrund, Geschlecht, ethisch-kultureller Hintergrund, uneingeschränkte Teilhabe aller Kinder durch  Bildungsangebote/Erfahrungsräume, die Kinder von sich aus nicht hätten & & & C2,~\ref{C2_32}\\
\cmidrule{2 - 5}
& Kinder werden nicht ausgegrenzt, sondern respektiert; es ist selbstverständlich, dass sie am 'normalen' Leben teilhaben, ihnen wird etwas zugetraut; Eltern in ihrer Erziehungsarbeit  unterstützen, den 'nicht-behinderten' Kindern zeigen, dass es nicht selbstverständlich ist, dass man ohne Handicap lebt und dass die Kinder mit Handicap dafür andere Dinge können, dass ein gegenseitiges Lernen möglich sein kann, wir müssen uns nur darauf einlassen; Inklusion ist ein Prozess, sowohl für die Familien als auch für den Kindergarten\vspace{0.5em} & & & C3,~\ref{C3_33}\\

\newpage
\dimension{Übergeordnete Werte und Ziele}
& „Sie können ja nicht sagen, ihr Kind bilden wir und sie lassen wir gerade mal links liegen. Wer mit den Kindern arbeitet, muss mit den Familien arbeiten.“ & & & C1,~\ref{C1_16}\\
\cmidrule{2 - 5}
& „Thema Sprache ist Nummer eins Thema im Haus!“ & & & C1,~\ref{C1_20}\\
\cmidrule{2 - 5}
& Sie sind mit den anderen, auch den Familien unterwegs & & & C1,~\ref{C1_46}\\
\cmidrule{2 - 5}
& „Unser Code ist, was wir wollen, dass immer eine Beziehungs- und Bindungsebene stattfindet“
& & & C1,~\ref{C1_63}\\
\cmidrule{2 - 5}
& Wir wollen hier, dass alle Kinder, egal woher sie kommen, so gut Deutsch können, dass sie im Unterricht gut mitkommen, dass sie mit anderen Kindern so gut sozial und emotional unterwegs sein können, dass sie sich gegenseitig in ihrem Lernprozess unterstützen können → Voraussetzungen dafür sind die Kompetenzen, sich selbst steuern und mit der eigenen Wut umgehen, ja und nein sagen und gut fühlen können, was tut mir gut → gutes 'Standing' für die Schule & & hohe Anforderung aufgrund der unterschiedlichen sozialen Hintergründe & C1,~\ref{C1_67}\\
\cmidrule{2 - 5}
& „Kinder und Eltern stark machen, auch die Mitarbeiter, das ist hier Programm!“ & & & C1,~\ref{C1_69}\\
\cmidrule{2 - 5}
& uneingeschränkte Teilhabe ist unser Ziel, Ziel ist abhängig von Rahmenbedingungen, Vernetzung, und der Bereitschaft des Teams & Grenzen im Alltag nicht immer optimal umsetzbar & & C2,~\ref{C2_33}\\
\cmidrule{2 - 5}
& Leitbild der AWO gibt Zielsetzung vor, Ziele im Team immer wieder bewusst machen, wo soll es eigentlich alles hin gehen & & & C2,~\ref{C2_40}\\
%\cmidrule{2 - 5}
& Leitbild der AWO mit Leitzielen wie Solidarität, Freiheit, Gleichheit, Gerechtigkeit & & politische Arbeit → Beitrag zur Chancengleichheit: „ Wir haben innerhalb der AWO im Grunde eine politische Arbeit.“ & C2,~\ref{C2_28}\\ 
\cmidrule{2 - 5}
& interessenorientiert und auch stärkenorientiert: an den Stärken ansetzen, die Stärken stärken und die Schwächen schwächen & & & C2,~\ref{C2_55}\\ 
\cmidrule{2 - 5}
& in unserem Leitbild sind kirchliche Werte verankert wie Respekt voreinander und Gleichheit aller Personen, denen wir entsprechen & & & C3,~\ref{C3_32}\\ 
\cmidrule{2 - 5}
& im Wohngebiet wie dem unseren die Chancen der Kinder zu erhöhen und deren Potential wach zu kitzeln, darum sollte es gehen\vspace{0.5em} & & Siehe [Angebote] & C3,~\ref{C3_406}\\ 

\dimension{Haltung}
& Inklusion ist eine Frage der Haltung: Wollen die Fachkräfte Inklusion, können sie sich das vorstellen? → Prozess der Öffnung hat stattgefunden, jeder ist o.k., der kommt, sich öffnen, weil man es alleine nicht schafft → „Wir brauchen die anderen Fachkräfte, die uns unterstützen. Da muss man einfach aufmachen -- als Fachkraft auch.“ & & & C1,~\ref{C1_208}\\
\cmidrule{2 - 5}
& Fachkräfte müssen lernen sich zu öffnen, mit anderen kooperieren und die Fachkompetenz der anderen für sich zu nutzen, Arroganz und Überheblichkeit sind unakzeptabel & jede Perspektive ist für die Arbeit als Ganzes von Bedeutung & & C1,~\ref{C1_28}\\
\cmidrule{2 - 5}
& „Wir reden hier über die Haltung. Die Haltung ist das entscheidende.“
Bildung, Fortbildung, Weiterbildung, Entwicklung ist lebenslänglich, alle Fortbildungsangebote orientieren sich ausschließlich an der Haltung der Fachkräfte, d.h. es wird gefragt, was habe ich verstanden und was macht das mit mir, wie bin ich in der Lage als Erzieherin zu arbeiten und mich zu verstehen & & Persönlichkeitsentwick"=lung der Fachkräfte ernstnehmen, Erzieherinnen befähigen mit dieser Komplexität im Alltag im Hinblick auf die Kinder und ihre Entwicklung – authentisch – umzugehen & C1,~\ref{C1_33}\\ 
\cmidrule{2 - 5}
& „Ich denke, für mich ist immer das Wichtigste am Thema Inklusion die Haltungsfrage.“
Atmosphäre in der Einrichtung, einander mit Wertschätzung und Achtung begegnen, dass das Team dahinter steht; Inklusion kann nicht oben diktiert werden, sondern muss wachsen und beginnt beim Umgang miteinander, bei Konfliktlösung & & & C2,~\ref{C2_34}\\ 
\cmidrule{2 - 5}
& Inklusion ist zuerst einmal eine Frage, ob das Team dahinter steht, sich weiterbildet, an Fortbildungen teilnimmt & & & C2,~\ref{C2_37}\\
\cmidrule{2 - 5}
& Wir sind offen für alle Kinder, haben die Position, wir probieren es aus und wenn wir es nur zwei Wochen probieren, aber wir probieren es. Erst im Tun, im Leben und Erleben sehen wir oft, was das bringt.\vspace{0.5em}& & Siehe [Grenzen der Inklusion] & C3,~\ref{C3_28}\\

\dimension{Grenzen der Inklusion}
& Manchmal stoßen wir dann auch an unsere Grenzen und haben Angst, diesem Kind nicht gerecht werden zu können & & & C3,~\ref{C3_28}\\
%\cmidrule{2 - 5}
& wenn Kind schwerer Pflegefall, beatmet wird oder Infusionen bekommt, Ich denke, es gibt auch Kinder, die so stark mit einem Handicap zu tun haben, dass man sich das im Alltag eher massiv schwer vorstellen kann & & Extraraum für Pflege, spezielle Pflegekräfte – Kostenfrage, Grenzen liegen bei finanziellen und personellen Ressourcen – der Fachkräftemangel ist ja eklatant -- vorstellbar von meiner Perspektive wäre das $\rightarrow$ durch die Zuführung wird positive Wirkung für das Kind und die Familie vermutet & C1,~\ref{C1_26}\\
\cmidrule{2 - 5}
& Was wir jetzt noch nicht hatten, ein Kind, was schwerst mehrfach behindert ist und jetzt mit der Sonde ernährt werden muss → da braucht man natürlich auch Unterstützung und noch ganz andere technische Hilfsmittel oder vielleicht auch noch eine andere Ausbildung & & & C2,~\ref{C2_801}\\ 
\cmidrule{2 - 5}
& Kinder gehen locker und natürlich mit Handicaps um, während die Erwachsenen sich eine Stunde lang in der Teambesprechung Gedanken darüber machen, wie vermieden werden kann, dass sich das Kind ausgeschlossen fühlt → Ein Kind erklärt einem anderen Kind, das neu ist: "Weißt du, der kann jetzt mit der Hand das nicht rüber machen, das musst du jetzt machen." 
& & Barrieren im Kopf werden wahrgenommen $\rightarrow$ Siehe {[Kooperation im Team]} & C2,~\ref{C2_508}\\
\cmidrule{2 - 5}
& Grenzen sehe ich, wenn wir dem Kind nicht mehr gerecht werden in
Ermanglung von Zeit und Zuwendung und abhängig von dem Grad einer Behinderung  & & Siehe {[notwendige Voraussetzungen von regierungsamtlicher Seite]} & C3,~\ref{C3_29}\\
\cmidrule{2 - 5}
& auch wenn wir uns noch so bemühen, wird es immer Situation geben, wo wir Kindern doch nicht so gerecht werden, da im Alltag noch andere Kinder, Eltern, die in der Tür stehen, die was Wichtiges mitteilen wollen, Stresssituationen im Alltag bedingt durch Krankheit & & & C2,~\ref{C2_55}\\
\cmidrule{2 - 5}
& Teilgabe nicht gewährleistet bei Schwimmkurs trotz Teilhabe $\rightarrow$ Wunsch, alle Kinder zu integrieren, ohne dass das einzelne das Gefühl hat in einer besonderen Situation zu sein, es gibt Angebote, da merken die Kinder das einfach, da kann man es noch so gut versuchen & & & C2,~\ref{C2_506}\\
\cmidrule{2 - 5}
& es gibt Situationen, in denen Kinder spüren, dass sie nicht ohne Hindernisse dabei sein können → Beispiel: Ausflug im Wald, Kind mit besonderen Bedürfnissen wird im Buggy geschoben, die anderen Kinder können in den Wald hinein rein rennen, Dinge holen, Kind fährt hinterher und kann nicht über jede Wurzel so schnell hinterher  & Ideen zur Veränderung fehlen, Sicherheitsbedürfnis → Buggy darf nicht umkippen & & C2,~\ref{C2_507}\\ 
\cmidrule{2 - 5}
& individuelle Förderung nicht für jedes Einzelkind im Alltag umsetzbar, wir versuchen die Bildungsthemen der Kinder aufzugreifen und zusammenzufassen, einzelnen Kindern, die sich nur für ein Thema interessieren, wird man nicht gerecht & da bräuchte man ja fast pro Kind eine pädagogische Fachkraft, Frage, ob man das überhaupt lösen kann & & C2,~\ref{C2_54}\\ 
\cmidrule{2 - 5}
& bedingt durch die Offenheit Kinder aus schwierigen Lebenslagen, mit Migartionshintergrund und Behinderung gehäuft & schwierig mit der Häufung umzugehen und dem einzelnen Kind noch gerecht zu werden & & C2,~\ref{C2_52}\\ 
\cmidrule{2 - 5}
& innerhalb der Entwicklung der Kinder entstehen Situationen, die man schwer verstehen kann, wenn Verhalten eines Kindes nicht erklärt werden kann, oft Grenzsituationen oder Akkutsituationen in Richtung Kindeswohlgefährdung & & & C1,~\ref{C1_29}\\
\cmidrule{2 - 5}
& Überlastung durch Krankheitsausfall, 
drei Kinder mit besonderen Bedürfnissen in einer Gruppe, drei Kinder, die permanent beobachtet werden müssen, Gruppe bei Krankheitsausfall für eine Fachkraft untragbar & & Fragen: Wie können wir den Förderbedarf der Kinder noch erfüllen? $\rightarrow$ zwei Fachkräfte  zu wenig, Lösungen: Erzieher der anderen Gruppe wechseln, Kinder aus der Gruppe rausnehmen, Kleingruppen bilden, Angebote verschieben & C2,~\ref{C2_75}\\ 
\cmidrule{2 - 5}
& Wunsch nach Unterstützung bei personellen Engpässen & Angst, dass man alleine nicht mit der Situation klar kommt\vspace{0.5em}  & & C2,~\ref{C2_79}\\

\newpage
\dimension{Erforderliche Kompetenzen der Fachkräfte}
& die Fähigkeit zu reflektieren und zu verstehen, was die pädagogischen Situationen hat gelingen lassen und was nicht, das setzt voraus, dass sich die Erzieherin mit ihrer eigenen Persönlichkeit und Biografie auseinandersetzt, 'Selbstklärung' & & erzieherisches Verhalten kann verfeinert werden, Kinder können anhaltend gut begleitet werden & C1,~\ref{C1_504}\\
\cmidrule{2 - 5}
& unterscheiden können, ist es die Mutter vor mir oder mein Bild von der Mutter, ist das jetzt das Kind vor mir oder ist es mein eigenes inneres Kind? & Komplexität, wir projizieren eigene familiäre Inhalte auf andere & Projektionen führen zu schwierigen Situationen, die fast nicht mehr aufgelöst werden können, wenn keine Auseinandersetzung stattfindet, kann Burn out die Folge sein & C1,~\ref{C1_54}\\
\cmidrule{2 - 5}
& Haltung, Wertschätzung, Achtung, Selbstreflexivität, Bereitschaft Konflikte zu lösen,  Kommunikationsfähigkeit im Team, pädagogische Kenntnisse, wissenschaftliche Erkenntnisse (Hirnforschung, Bindung) Bereitschaft für die Zusammenarbeit mit den Eltern,   Vernetzungsbereitschaft & & & C2,~\ref{C2_509}\\
\cmidrule{2 - 5}
& Empathie, mit beiden Beinen auf dem Boden stehen, Spontanität und Flexibilität, Kinder sind jeden Tag anders, das erfordert ein hohes Maß an Flexibilität & & &C3,~\ref{C3_71}\\ 
\cmidrule{2 - 5}
& Erzieherin müssen sich auf die Eltern, die Kinder und die schwierigen Lebenssituationen einlassen und sie nehmen, wie sie sind $\rightarrow$ Geschichte mit einem Reiter auf einem Pferd & & & C3,~\ref{C3_72}\\

\newpage
\dimension{Zusatzqualifikation}
& Bildung, Fortbildung, Weiterbildung, Entwicklung ist lebenslänglich, Fortbildungen orientieren sich an der Haltung & & & C1,~\ref{C1_33}\\
\cmidrule{2 - 5}
& lebenslanges Lernen wichtig, um Wissen zu aktualisieren → fortwährender Bedarf an Zusatzqualifikation & Wissen um Gesetzesänderungen ist Voraussetzung um zusätzliche Hilfe in die Einrichtung zu holen; Kinder werden angemeldet, die sich in Situationen befinden, die neu sind & Weiterbildungsbedarf an den Kindern und ihren Lebenslagen orientiert & C2,~\ref{C2_606}\\
\cmidrule{2 - 5}
& Bereitschaft für Fortbildungen hängt von dem zur Verfügung gestellten finanziellen Rahmen ab
 & & & C2,~\ref{C2_67}\\
\cmidrule{2 - 5}
& Überlegungen, in welchen Bereichen können sich einzelne Fachkräfte durch Fort- oder Weiterbildungen spezialisieren für mehr Professionalität & personell gut ausgestattet & & C2,~\ref{C2_46}\\ 
\cmidrule{2 - 5}
& Anlegen von spezialisierten Wissen im Team, Wissen wird aufgeteilt, weil nicht jeder alles wissen kann → Bedeutung multiprofessionelles Team & & & C2,~\ref{C2_66}\\
\cmidrule{2 - 5}
& aktueller Bedarf: Auseinandersetzung mit der sexuellen Entwicklung der Kinder & Unsicherheit im Team & Siehe {[Elternabend]} & C2,~\ref{C2_606}\\
\cmidrule{2 - 5}
& Fortbildungen für Gesamtteam & Aufgabe der Leitung Bedarf des Teams zu ermitteln $\rightarrow$ Fortbildungen für das gesamte Team bringen mehr, weil beim Transportieren der Informationen ins Team vieles verloren geht & & C2,~\ref{C2_44}\\
\cmidrule{2 - 5}
& Ressourcen der Fachkräfte werden genutzt, d.h. Angebote und entsprechende Weiterbildung abhängig von den den Interessen der Erzieherinnen\vspace{0.5em} & & & C3,~\ref{C3_34}\\

\dimension{Aufgaben als Leitung im Hinblick auf Inklusion}
& Bildungs- und Entwicklungsprozesse der Menschen (Fachkräfte, Kinder, Familien), die ihr anvertraut sind, fördern, Antwort finden auf deren Alltagssituation, sicherstellen, dass genau das stattfindet, was sie brauchen. gemeinsam ’mit’ anderen herausfinden, was gebracht wird und durch Hilfestellungen und Systeme alles unterstützen, damit Antworten gefunden werden & & & C1,~\ref{C1_38}\\
\cmidrule{2 - 5}
& Beratungsfunktion für die Eltern & niederschwelliges Angebot – Siehe [Bedarfslagen der Eltern] & & C1,~\ref{C1_39}\\
\cmidrule{2 - 5}
& Sozial-emotionale ist das Stärkste, was stattfinden muss. Wenn sich eine Mutter hier nicht wohlfühlt, können Sie gehen, brauchen Sie gar kein Bildungsangebot starten, würde die nie machen & & & C1,~\ref{C1_43}\\
\cmidrule{2 - 5}
& Selbstreflexion über Dialog unterstützen, bei Situation, die nicht verstanden werden, es geht nicht um funktionieren hier drin – „Wir sind alle im Werden!“ -- beidseitig, auf Augenhöhe, bei Situationen, die für mich schwer auszuhalten sind, dann sagen die mir auch → „Und es tut mir auch sehr gut, eben weil wir uns kennen. Wir wissen, wo jeder seine Stärken hat und wir wissen, wo die Schwachstellen sind.“ & & & C1,~\ref {C1_55}\\
\cmidrule{2 - 5}
& das wichtigste in der Einrichtung ist Atmosphäre, Haltung, Wertschätzung → das Wichtigste überhaupt bei Inklusion, dass sich alle angenommen und wohl fühlen →  darin besteht die Hauptaufgabe der Leitung & &
wenn wir jetzt als Team nicht so eine gute Zusammenarbeit hätten, könnten wir das wiederum nicht an die Familien weiter geben und wenn die sich hier nicht gut aufgehoben fühlen, dann würden sie ihre Kinder nicht anmelden & C2,~\ref{C2_604}\\
\cmidrule{2 - 5}
& zwischenmenschlichen Kommunikation und Vernetzung zwischen allen Beteiligten, um fehlende Informationen zu beschaffen & & & C3,~\ref{C3_51}\\
\cmidrule{2 - 5}
& Ansprechpartnerin, bei Erreichen persönlicher Grenzen Fachpersonal von außen dazu holen & Team überfordert, fühlen sich nicht verstanden: „Du {[Leitung]} hast doch immer für alles Verständnis und sagst, wir schaffen das.“ & Erzieherinnen fühlen sich verstanden, wenn jemand von außen eigene Erfahrungen aus der Praxis mit ihnen teilt und sie sich ernst genommen fühlen in ihrer Überforderung & C3,~\ref{C3_68}\\
\end{longtable}
\end{centering}

% ===============    Tabelle 3    ============== %
\begin{centering}
\begin{longtable}{p{2cm}p{8.5cm}p{4cm}p{4cm}p{1.5cm}}
\caption{Extraktion zur Auswertungskategorie 'Antworten auf die unterschiedlichen Bedarfslagen'}\\
\tabhead
\dimension{Gruppenzusammensetzung -- Bedarfslagen der Kinder}
& hohe soziale Vielfalt: Kinder aus sehr belasteten, bildungsfernen Familien, Kinder aus sehr gut organisierten Familien & & & C1,~\ref{C1_1001}\\
\cmidrule{2 - 5}
& keine Monokultur, Mischung ist für unsere Arbeit sehr wichtig & & & C1,~\ref{C1_2}\\
\cmidrule{2 - 5}
& Kinder aus unterschiedlichen Herkünften: aus Akademikerfamilien, Eltern, die sehr interessiert an unserer Arbeit und sehr bildungsorientiert sind, Kinder aus sehr schwierigen Lebenslagen, wo die Eltern als bildungsfern zu bezeichnen sind, wo die Eltern teilweise wenig Interesse an unserer Arbeit haben & schwierige Elternbiografie → Weitergabe in die nächste Generation & & C2,~\ref{C2_2}\\ 
\cmidrule{2 - 5}
& Thema Armut, gute Mischung, wie es in der Gesellschaft auch ist, 40~\% der Kinder bekommen gerade das Mittagessen durch ein Bildungs- und Teilhabegesetz, also durch einen Gutschein bezahlt, bei 38,33~\% werden die Beiträge vom Jugendamt übernommen, 63~\% von allen zahlen ermäßigten Beitrag & & & C2,~\ref{C2_3}\\
\cmidrule{2 - 5}
& mehr als Zweidrittel der Kinder mit Migrationshintergrund & & & C1,~\ref{C1_3}\\
\cmidrule{2 - 5}
& von den 123 Kindern haben 90 plus/minus fünf einen Migrationshintergrund & & & C3,~\ref{C3_10}\\
\cmidrule{2 - 5}
& 63~\% der Kinder haben einen Migrationshintergrund, viele Nationen sind vertreten, keine Anhäufung aus einem Land, größte Gruppe acht Familien mit russischem Migrationshintergrund, die meisten Kinder wachsen zweisprachig auf, es gibt Kinder, die beim Eintritt in die Einrichtung nur russisch sprechen & & & C2,~\ref{C2_4}\\ 
\cmidrule{2 - 5}
& seelische und geistige Behinderungen gehäuft, etwas seltener Kinder mit Autismus & soziale Vielfalt im Ballungsgebiet & & C1,~\ref{C1_4}\\
\cmidrule{2 - 5}
& derzeit fünf Kinder über Integrationshilfe, ein weiteres Kind hatte Krebs und zeigt große Entwicklungsverzögerungen, insbesondere sprachliche & & Überlegungen, ob das Kind ab Sommer in eine Sprachheilschule geht, da sechs Jahre & C2,~\ref{C2_8}\\
\cmidrule{2 - 5}
& Kinder mit besonderen Bedürfnissen, die auch heilpädagogischer Unterstützung bedürfen oder Logopädie brauchen & & & C2,~\ref{C2_9}\\
\cmidrule{2 - 5}
& Kind mit einer Sehbehinderung & & Siehe [Kooperation mit Fachdiensten und anderen Institutionen] und [Angebote] & C2,~\ref{C2_24}\\
\cmidrule{2 - 5}
& Kinder aus schwierigen Lebenslagen gehäuft & bedingt durch die Offenheit der Einrichtung werden entsprechend viele Kinder angemeldet & Siehe [Grenzen der Inklusion] & C2,~\ref{C2_52}\\ 
\cmidrule{2 - 5}
& Unfallgefahr häufig erhöht bei Kindern mit Behinderung & & Siehe [Grenzen der Inklusion] & C2,~\ref{C2_77}\\ 
\cmidrule{2 - 5}
& viele Kinder mit Verhaltensauffälligkeiten und Probleme im sozial-emotionalen, motorischen oder sprachlichen Bereich & & & C3,~\ref{C3_5}\\
\cmidrule{2 - 5}
& drei Kinder mit diagnostizierter Lernbehinderung, zwei Kinder mit Hörbeeinträchtigung, beide tragen Hörgeräte, und Kinder mit Wahrnehmungsproblematiken, bei denen noch eine diagnostische Abklärung aussteht, in der Vergangenheit: Kinder mit Down-Syndrom, Spastiken, oder starker Hörschädigung oder Sehbeeinträchtigung & & & C3,~\ref{C3_27}\\

\dimension{Angebote}
& Angebote werden vom Bedarf ermittelt, d.h. es gibt möglicherweise auf einen Entwicklungsbedarf mehrere Antworten → Bsp.: es gibt Kinder, die müssen eine Sprachtherapie bekommen, andere brauchen Kleingruppenangebote & & & C1,~\ref{C1_21}\\
\cmidrule{2 - 5}
& das Anpassen von Angeboten auf die Bedarfe, vor jedem pädagogischen Angebot wird überlegt, wie alle Kinder teilnehmen können --> Bsp:  Zahlenprojekt, Angebot, bei dem Sprache nicht im Vordergrund steht & & & C2,~\ref{C2_82}\\
\cmidrule{2 - 5}
& Können wir den geplanten Ausflug machen, wenn wir einen Buggy  mitnehmen müssen? Wie viel Personal haben wir zur Verfügung? & & & C2,~\ref{C2_36}\\
\cmidrule{2 - 5}
& Einmal im Monat gemeinsames Frühstück aller Gruppen mit reichhaltigem Buffet & Bedarf der Eltern nach gemeinsamen Essen der Kinder & & C3,~\ref{C3_8}\\
\cmidrule{2 - 5}
& Brillen für ein paar Wochen ausgeliehen & & Selbsterfahrung für die Kinder ohne Sehbehinderung, haben durch das Tragen erkannt, wie deren Blickwinkel eingeschränkt ist und wie leicht sie stolpern können & C2,~\ref{C2_24}\\ 
\cmidrule{2 - 5}
& Impuls der Mutter eines Jungen mit Spastik ein Rollbrett für ihn anzuschaffen, worauf er geschnallt wurde und mit den Armen sich bewegen konnte, Einrichtung hat Rollbretter für alle angeschafft → der Junge konnte dann mit den anderen Kindern zusammen draußen sein und herum fahren & & Einblick, voneinander lernen & C3,~\ref{C3_57}\\ 
\cmidrule{2 - 5}
& Aufgabe Erfahrungsräume für die Kinder erweitern & Manche Kinder kennen nicht mal den Marktplatz in Freiburg am Münsterplatz, wenn wir das nicht von hier aus machen würden & & C2,~\ref{C2_31}\\ 
\cmidrule{2 - 5}
& Theaterbesuch und das damit verbundene Ambiente verstehe ich als Bildungsmaßnahme in Richtung Chancengleichheit & Die meisten unserer Kinder würden bis zur Pubertät nicht so etwas wie ein Theater sehen. & Die Kinder staunen und sagen: „Da wollte ich ja schon immer mal hin.“ & C3,~\ref{C3_47}\\ 
\cmidrule{2 - 5}
& Viele Familien können sich Bildungsmaßnahme gar nicht leisten. Dass dadurch eine Elite herangezogen wird, finde ich bedenklich. & & & C3,~\ref{C3_46}\\ 

\dimension{Konzept}
& niederschwelliges Konzept: beim Eingewöhnen neuer Kinder wird nach Stolpersteinen geschaut, wer macht das und das Kind kommt sofort dazu & & Den Eltern tut das Konzept gut, sie fühlen sich gesehen und getragen. „Die sind richtig mit uns unterwegs. Wir haben das ziemlich genau im Blick -- das spüren die auch.“ & C1,~\ref{C1_32}\\
\cmidrule{2 - 5}
& dialogische Qualitätsentwicklung, Zukunftswerkstatt, Eltern und Fachkräfte überlegen gemeinsam, welche Angebote entwickelt werden, Eltern beteiligen, gemeinsam Fragen erörtern und davon ausgehend Haus gestalten & & & C1,~\ref{C1_45}\\
\cmidrule{2 - 5}
& in der Zukunftswerkstatt findet die Entwicklung der Antworten auf unterschiedliche Bedarfslagen statt, so hat sich in den letzten Jahren dieses Haus entwickelt 
Themen der Kita werden gemeinsam erörtert unter Berücksichtigung der Elternperspektive, der Mitarbeiterperspektive, ggf. der Trägerperspektive, der Gesetzes gebenden Perspektive, der Wissenschaften
1. alle Positionen zu einem Thema werden aufgegriffen
2. erörtert, wo ist der nächste Handlungsschritt 
dialogische Qualitätsentwicklung, d.h. Dialogverfahren, wo man gemeinsam verändert, 
basisdemokratisch, Eltern werden beteiligt, Analyse spielt eine große Rolle, schauen, was ist hier eigentlich los, sonst greift es nicht & Qualifikation Leitung & & C1,~\ref{C1_17}\\
\cmidrule{2 - 5}
& Situationsansatz vom Deutschen Jugendinstitut München & & & C1,~\ref{C1_62}\\
\cmidrule{2 - 5}
& kritisch gegenüber dem offenen Konzept & Struktur und Rituale wichtig für Kinder aus besonderen Lebenslangen oder mit Behinderung, um Bezug zu den Erzieherinnen zu haben & & C2,~\ref{C2_48}\\ 
\cmidrule{2 - 5}
& Hilfeplan erstellen im Zusammenarbeit mit dem Jugendamt (federführend, wollen den Bedarf ermitteln und stellen somit die Fragen, kennen aber das Kind nicht, weshalb sie auf die Einschätzung der Eltern und der Fachkräfte angewiesen sind), den Eltern, Heilpädagogin, die Kita, gemeinsame Einschätzung, was im Alltag des Kindes an zusätzlicher Unterstützung wichtig ist
runder Tisch, jeder kann seine Perspektive beitragen & & & C2,~\ref{C2_39}\\ 
\cmidrule{2 - 5}
& Bildungs- und Lerngeschichten von Neuseeland, das heißt, wir beobachten die Kindern, um herauszufinden, mit welchen Themen sie sich von sich aus im besonderen Maße auseinandersetzen, finden adäquate Angebote/ Lerngeschichten als Rückmeldung auf den Bedarf, dokumentieren Bildungsentwicklung in Portfolio-Ordnern, mit Hilfe anderer Beobachtungsverfahren Schwächen der Kinder in den Blick bekommen und besondere Förderangebote finden, die kompensatorische Funktion haben & & Siehe [Übergeordnete Werte und Ziele] hoher Anspruch, zeitaufwendig, da viel Dokumentation und kollegialer Austausch erforderlich  → Siehe {[Grenzen der Inklusion]}& C2,~\ref{C2_55}\\ 
\cmidrule{2 - 5}
& vor etwa zehn Jahren in die Konzeption aufgenommen
gesetzliche Vorgabe, wenn Integrationshilfe beantragt wird, muss in der Konzeption verankert sein, wie das Kind mit besonderen Bedürfnissen in der Einrichtung gefördert wird, die Einrichtung muss gewährleisten dass das Kind entsprechend des Bedarfs gefördert wird und nicht nur mitläuft – Transparenz, was leistet die Einrichtung & & & C2,~\ref{C2_38}\\
\cmidrule{2 - 5}
& Konzeption beinhaltet das Bild vom Kind, die Zusammenarbeit mit den Eltern, die Frage, welche Werte wir als Einrichtung vermitteln wollen & & Siehe {[Übergeordnete Werte und Ziele]} & C2,~\ref{C2_40}\\ 
\cmidrule{2 - 5}
& Konzeption ausgehend von dem Leitbild der AWO mit Leitzielen wie Solidarität, Freiheit, Gleichheit, Gerechtigkeit & politische Arbeit → Beitrag zur Chancengleichheit: „ Wir haben innerhalb der AWO im Grunde eine politische Arbeit.“ & & C2,~\ref{C2_28}\\ 
\cmidrule{2 - 5}
& nach halbjähriger Weiterbildung in Sprachförderung, Erfahrungen im Team eingebracht und gemeinsam Sprachkonzeption erstellt & & & C3,~\ref{C3_18}\\
\cmidrule{2 - 5}
& Konzeption für Kinder mit besonderen Bedürfnissen liegt noch nicht gebündelt vor, in unserer Gesamtkonzeption werden Kinder mit Behinderung einbezogen, aber wir haben das noch nicht zusammengefasst\vspace{0.5em} & & & C3,~\ref{C3_32}\\ 

\dimension{Kleingruppenprojekte}
& Ringergruppe mit Adolf Seger (12 Mal Weltmeister) 
Gruppe für das ganze Haus, wird bekannt gegeben, auf Anmeldung nehmen Jungen und ihre Väter teil, auch für Jungen, die keine Väter haben oder deren Väter in der Familie nicht präsent sind, gemeinsam wird gekämpft, im Vorfeld werden mit männlichen Erziehern Kuchen gebacken: \emph{Männer backen für Männer} heißt das Programm, nach dem Sport gibt es Männercafé & 
& geschlechtsspezifisches Projekt → männliche Identifikationsmöglichkeiten & C1,~\ref{C1_7}\\
\cmidrule{2 - 5}
& über die Naturpädagogik lernen die wahnsinnig viel, haben richtiges Fachwissen & & & C1,~\ref{C1_67}\\
\cmidrule{2 - 5}
& Kinder lieben es mal allein was zu machen, aber auch in Konkurrenz mit anderen zu treten. Wenn mehrere Kinder in Kleingruppen zusammen sind, sitzt der Motor quasi gegenüber und dann läuft es ganz von allein, weil die Kinder, wenn sie sich aneinander messen können, aus sich selbst heraus motiviert sind\vspace{0.5em} & & &C3,~\ref{C3_30}\\ 

\dimension{Sprachförderung}
& in Kooperation mit \emph{Südwind} erfolgen Deutschangebote für Eltern, d.h. Deutschkurse werden an vier Vormittagen für Frauen im Haus angeboten, gleichzeitig Babygruppe, so dass Betreuung abgedeckt ist & & & C1,~\ref{C1_14}\\
\cmidrule{2 - 5}
& 26 Kinder bekommen zusätzliche Sprachförderung, elf Kinder nehmen an einer Gruppe Singen-Bewegen-Sprechen (SBS) teil, vom Land finanziert, läuft mit der Musikschule Freiburg\newline
Alltag mit den Kindern wird verbal begleitet,
zusätzliche Sprachförderung teilweise innerhalb der Gruppe, teilweise außerhalb, Themen vertiefend, die in der Gruppe sowieso besprochen werden und dann wird nochmal Kleingruppenarbeit gemacht SBS findet als Kleingruppe in der Turnhalle statt (viel mit Singen, Bewegung, Musizieren) & & & C2,~\ref{C2_5}\\ 
\cmidrule{2 - 5}
& das \emph{Rucksack}-Projekt und ist durch den \emph{LEIF}-Arbeitskreis angestoßen wurden mit dem Ziel die Sprachförderung noch besser in die Einrichtung zu integrieren, die Zweisprachigkeit der Kinder zu fördern und die Eltern einzubeziehen 
das \emph{Rucksack}-Projekt ist seit März 2011 (als eine von drei Modelleinrichtungen in Freiburg) installiert\newline
die russischen Familien treffen sich einmal pro Woche mit einer speziell dazu ausgebildeten russisch sprechenden Elternbegleiterin\newline
es werden pädagogische Themen mit den Eltern besprochen, Lieder in russischer Sprache, parallel mit den Kindern in der Gruppe in deutscher Sprache durchgeführt  & & Sprachförderung für die Kinder und gleichzeitig Austauschmöglichkeit für die Eltern, regt die Eltern dazu an, sich zu hause Zeit für ihre Kinder Zeit zu nehmen, Wertschätzung der Heimatsprache → höheres Selbstbewusstsein der Mütter, mehr Präsenz in der Einrichtung, Kinder erleben, dass sie in der Einrichtung auch russisch sprechen dürfen, zum Beispiel im Stuhlkreis → Siehe [Elterncafé] & C2,~\ref{C2_507}\\
\cmidrule{2 - 5}
& 40 Kinder erhalten individuelle Sprachförderung in Kleingruppen, es werden Kinder von den Erzieherinnen vorgeschlagen, die zweisprachig aufwachsen und noch mehr Sprachanreize in Deutsch benötigen; sich nicht trauen laut zu sprechen und in ihrem Selbstbewusstsein gestärkt werden sollen, die in der Grammatik und im Wortschatz Nachholbedarf haben & & & C3,~\ref{C3_14}\\ 
\cmidrule{2 - 5}
& flexible Gruppengröße und Dauer je nach Thematik.
Einsatz von Musik und Bewegung, Trommeln und Rhythmus $\rightarrow$ größere Gruppen; Wortschatzerweiterung sehr kleine Gruppen und kurze Einheiten, da die Aufmerksamkeitsspanne der Kinder kürzer\vspace{0.5em} & & Musik als „Türöffner“ für gehemmte Kinder & C3,~\ref{C3_17}\\

\dimension{Schritte in Richtung Inklusion und wahrgenommene Veränderungen}
& [seit 1999] sich zu öffnen für das Andere, andere Strukturen zu entwickeln und zuzulassen, so dass andere mit rein kommen können, dass Erziehung nicht begrenzt auf die Gruppe ist, sondern eine Weite bekommt & & & C1,~\ref{C1_61}\\
\cmidrule{2 - 5}
& Prozess zu einer Zieloffenheit und Ergebnisoffenheit, nicht mehr auf das Erreichen eines Zieles hin arbeiten, sondern auf der Prozessebene Antworten finden und gefundene Antworten akzeptieren & &  Verunsicherung: „Aber das kann verunsichern, weil man denkt, Oh Gott, man macht jetzt was und weiß gar nicht, wo man da hinkommt.“; Angriffe von außen; Abgrenzung gegenüber Angriffen durch Begründen des eigenen Standpunkts; Energie und Wille so zu arbeiten haben sich verstärkt, Prozess nicht mehr umkehrbar, weil inneres Wachstum stattgefunden hat  &  C1,~\ref{C1_66}\\  
\cmidrule{2 - 5}
& seit 12 Jahren Familiennetzwerkarbeit und dialogische Qualitätsentwicklung & & haben uns in Bezug auf die Klärung von Konflikten sehr verbessert & C1,~\ref{C1_29}\\
\cmidrule{2 - 5}
& in den Kleingruppen ist immer jemand von den Erziehern dabei & & interdisziplinäre Austausch in den Kleingruppen, der dann wiederum jede Woche im Team gebündelt wird → die Chancen für die Kinder, die hier sind, haben sich erhöht, weil der Blick auf ihre Entwicklung sehr pointiert ist & C1,~\ref{C1_12}\\
\cmidrule{2 - 5}
& Entwicklung von einem Haus, das sich Inklusion auf die Fahne geschrieben hat, das, was man hier abruft, das hat nichts mit Standard zu tun. & & Siehe [Schritte in Richtung Inklusion und wahrgenommene Veränderungen] & C1,~\ref{C1_50}\\
\cmidrule{2 - 5}
& hohe Fachkompetenz & & & C1,~\ref{C1_8}\\
\cmidrule{2 - 5}
& Leitung spricht sich Kompetenz zu in Konflikten zwischen Eltern und Kollegen den Außenblick zu wahren und Konflikte in der Regel gut auflösen zu können & & & C1,~\ref{C1_29}\\
\cmidrule{2 - 5}
& Aufgrund der langjährigen Leitungstätigkeit (seit fast 22 Jahren) zahlreiche Erfahrungen gemacht, wie sie Eltern unterstützen kann & & & C1,~\ref{C1_40}\\
\cmidrule{2 - 5}
& Wir haben uns aus uns selbst heraus entwickelt, weil wir das wollten & & „Dadurch [...] funktioniert es natürlich auch super. Es funktioniert und es ist auch eine sehr schöne Arbeit.“ & C1,~\ref{C1_48}\\
\cmidrule{2 - 5}
& langjährige Mitarbeiter, die nicht gehen wollen & & & C1,~\ref{C1_50}\\
\cmidrule{2 - 5}
& das Hauptteam ist schon lange in der Einrichtung & große Zufriedenheit unter den Mitarbeitern & & C1,~\ref{C1_56}\\
\cmidrule{2 - 5}
&[vor zehn Jahren] Anfrage ein Kind aufzunehmen, das eine Behinderung hatte, standen vor der Frage, könnten wir das als Regeleinrichtung hinkriegen und haben die Voraussetzungen geklärt, was wir brauchen, um das Kind aufzunehmen & das Team war offen & & C2,~\ref{C2_35}\\
\cmidrule{2 - 5}
& gewachsene Strukturen – Trennung von Schule und Kindergarten aufgehoben, wie ein gemeinsames Team & & & C2,~\ref{C2_600}\\ 
\cmidrule{2 - 5}
& Fortbildungsbereitschaft hat zugenommen & finanziell geregelt und Einsicht, weil Notwendigkeit erkannt & & C2,~\ref{C2_69}\\ 
\cmidrule{2 - 5}
& gruppenübergreifendes Denken durch Austausch trotz Stammgruppen & Veränderungen angestoßen durch den Orientierungsplan & zusammen überlegen, was für Angebote haben wir zu den einzelnen Bildungsbereichen, was müssen wir noch ergänzen (Bsp.: Flaschenzug), nicht jede Gruppe muss alles haben, gegenseitige Besuche & C2,~\ref{C2_70}\\ 
\cmidrule{2 - 5}
& offenere Haltung Inklusion gegenüber, gewachsene Vernetzung, das Anpassen von Angeboten auf die Bedarfe, 
Rücksichtnahme größer, auch bei den Kindern, Zusammenarbeit ist intensiviert, Gesamteinrichtung mehr in Arbeitskreisen integriert & & & C2,~\ref{C2_82}\\ 
\cmidrule{2 - 5}
& [vor 16 Jahren] Anstoß von außen Anfrage: Kind im Rollstuhl → 
Überlegungen: Was brauchen wir? Breite Türen, Behindertentoilette, 
Erzieherinnen, die bereit sind, sich auf so etwas einzulassen, zwei Gruppen haben sich für die Aufnahme bereit erklärt → „So hat bei uns die Inklusion Einzug gehalten und ist jetzt Normalität.“ & & & C3,~\ref{C3_52}\\
\cmidrule{2 - 5}
& Bereicherung in der Auseinandersetzung mit sich selbst, Bereicherung in den Kindern sehen & & & C3,~\ref{C3_77}\\
\cmidrule{2 - 5}
& „Ich stelle nach diesem Interview fest, es ist gar nicht wenig, was wir hier machen, in diesem Kindergarten werde ich ganz anders gebraucht; ich habe in dieser Einrichtung so viel gelernt.“& & & C3,~\ref{C3_78}\\ 
\end{longtable}
\end{centering}

% ===============    Tabelle 4    ============== %
\begin{centering}
\begin{longtable}{p{2cm}p{8.5cm}p{4cm}p{4cm}p{1.5cm}}
\caption{Extraktion zur Auswertungskategorie ’Partnerschaft mit den Eltern’}\\
\tabhead
\dimension{Bedarfslage der Eltern}
& Themen wie Armut, Ausgrenzung, Beschämung & & & C1,~\ref{C1_1}\\
\cmidrule{2 - 5}
& Häufung von Familien mit hoher Arbeitslosigkeit und schlechter beruflicher Perspektive, viele Familien sind Harz IV Empfänger & & & C3,~\ref{C3_4}\\
\cmidrule{2 - 5}
& die Menschen, die hier sind tun sich schwer zur Beratungsstelle in die Stadt zu gehen, das würden die nie machen & & & C1,~\ref{C1_39}\\
\cmidrule{2 - 5}
& Krisen wie Trennung, Todesfälle, Suchtkonflikte aushalten und stabilisieren & &  Elterncafé $\rightarrow$ Eltern die Möglichkeit gegeben andere Eltern kennenzulernen & C1,~\ref{C1_42}\\
\cmidrule{2 - 5}
& Kinder aus aus Akademikerfamilien, Eltern, die sehr interessiert an unserer Arbeit und sehr bildungsorientiert sind, Kinder aus sehr schwierigen Lebenslagen, wo die Eltern als bildungsfern zu bezeichnen sind, wo die Eltern teilweise wenig Interesse an unserer Arbeit haben & schwierige Elternbiografie → Weitergabe in die nächste Generation & & C2,~\ref{C2_2}\\ 
\cmidrule{2 - 5}
& Das Erreichen mancher gestaltet sich, egal, was man anbietet, schwierig & Eltern können am Elternabend nicht kommen, weil sie berufstätig sind, kein Interesse haben, sagen, es sei ihnen zu viel, weil sie drei Kinder haben und noch berufstätig sind & & C2,~\ref{C2_64}\\
\cmidrule{2 - 5}
& Eltern mit viel Geld erzählen, dass sie bewusst keine Einrichtung gewählt haben, wo nur die Kinder gleicher sozialer Herkunft sind & & & C2,~\ref{C2_60000}\\
\cmidrule{2 - 5}
& Mangel an Lernmöglichkeiten und Anreizen zu hause -- Mütter sagen: „Ihr könnt meinem Kind mehr beibringen als wir zu hause.“ & & Bedarf an verlängerten Öffnungszeiten & C3,~\ref{C3_6}\\
\cmidrule{2 - 5}
& fehlende Strukturen des gemeinsamen Kochens und Essens in der Familie & & & C3,~\ref{C3_7}\\
\cmidrule{2 - 5} 
& Kinder brauchen ein ruhiges Umfeld mit nicht so viel 'Gewusel', da sie zu hause schon so viel Unruhe haben & & & C3,~\ref{C3_65}\\
\cmidrule{2 - 5}
& Ängste, dass Kinder zu wenig Zuwendung und Lernanreize bekommen, Erzieherinnen haben nicht die nötige Zeit & Angst nicht bestehen zu können: Was passiert mit meinem Kind, wenn mein Kind nicht da 'hoch' kommt? Erwartungsdruck der Gesellschaft; Werbung, die immerzu schöne und schlaue Menschen zeigt & & C3,~\ref{C3_44}\\
%\cmidrule{2 - 5}
& fehlende Offenheit, Überzeugungsarbeit bei den Eltern leisten müssen\vspace{0.5em} & & & C3,~\ref{C3_56}\\

\dimension{Elternabende}
& Elternabend letztes Jahr vor Weihnachten, Austausch über die Feste in den jeweiligen Kulturen & & & C2,~\ref{C2_62}\\
\cmidrule{2 - 5}
& Anzahl an Elternabenden für alle Eltern reduziert zu Gunsten einer höheren Frequenz von Kleingruppengesprächen und runden Tischen, da in diesem Rahmen viel besser Entwicklungsgespräche möglich sind, gemeinsame Elternabende wenn Referenten kommen & & & C3,~\ref{C3_703}\\ 
\cmidrule{2 - 5}
& Heilpädagogen oder Sprachtherapeuten einladen und die Zusammenarbeit vorstellen & & & C3,~\ref{C3_37}\\
\cmidrule{2 - 5}
& Heilpädagogin und Sprachtherapeutin berichten von ihrer Arbeit mit den Kindern, Sprachtherapeutin hat mit Einwilligung der Eltern ein Video von dem Kind mit Down-Syndrom gezeigt & & Aha-Erlebnis bei den Eltern: es ist nicht selbstverständlich, dass wir gesund sind. Mir kann heute etwas passieren, dann bin ich behindert und will genauso geschätzt werden wie gestern noch & C3,~\ref{C3_38}\\ 
\cmidrule{2 - 5}
& {[vor 16 Jahren]} Mutter berichtet aus ihrem Leben mit dem Kind mit Spastik und von ihren Ängsten & & offenerer Umgang mit der betroffenen Mutter, Offenheit überträgt sich auf die Kinder & C3,~\ref{C3_54}\\
\cmidrule{2 - 5}
&Eltern- und Entwicklungsgespräche & Eltern fragen: „Auf welchen Gebieten haben Sie Angst, dass Ihr Kind zu kurz kommt?“ Eltern einladen, einen Vormittag in der Gruppe zu erleben oder thematisieren in Elternabenden & &C3,~\ref{C3_36}\\ 
\cmidrule{2 - 5}
& wenn erhöhter Unterstützungsbedarf beobachtet wird & & Manche Eltern fühlen sich verstanden, weil sie das beschriebene Verhalten von zu hause kennen. Manche sind sehr erschrocken und reagieren mit Sätzen wie: „Mein Kind ist doch nicht dumm oder behindert. Muss mein Kind raus?“ Eltern Zeit geben, nach einer Woche nochmal Gespräch suchen & C3,~\ref{C3_26}\\
\cmidrule{2 - 5}
& dazu anregen, sich in die Lage der Eltern mit Kind mit erhöhtem Unterstützungsbedarf zu versetzen, die Ängste aller Eltern ernst nehmen, auch die kritischen Eltern & & & C3,~\ref{C3_309}\\ 
\cmidrule{2 - 5}
& Entwicklungsgespräche unter Einsatz der Methoden zur Dokumentation der kindlichen Entwicklung & & & C3,~\ref{C3_35} \\
\cmidrule{2 - 5}
& Ganz wichtig ist uns auch, dass die Eltern sich nicht zu schämen brauchen, wenn sie zusätzliche Hilfe benötigen, dass sie sich nicht schämen diese anzunehmen. Viele unserer Eltern haben einen großen Packen zu tragen und sind oft auch entmutigt. Das sind persönliche und bestärkende Gespräche sehr wichtig. Solche sind sehr anstrengend und verlangen viel Kraft &&& C3,~\ref{C3_76}\\
\cmidrule{2 - 5}
& Entwicklungsgespräche sind nach jedem Geburtstag eines Kindes ein Muss oder auch, wenn sich gravierende Veränderungen beim Kind zeigen. Auch beim Bringen und Holen finden Gespräche mit den Eltern statt. Je nach Bedarf kann es drei, vier oder auch fünf Elterngespräche geben oder auch nur eins.\vspace{0.5em} &&& C3,~\ref{C3_73}\\ 

\dimension{Information und Transparenz}
& Wenn neue Kinder kommen, ein Blatt an die Tür hängen mit Bildern von den neuen Kindern und deren Namen & & & C2,~\ref{C2_63}\\
\cmidrule{2 - 5}
& Gruppentagebuch & & & C3,~\ref{C3_75}\\ 
\cmidrule{2 - 5}
& im Anmeldegespräch werden Eltern informiert über die Arbeitsweise, Ansetzen an deren Stärken, über die Aufnahme und den Umgang von Kindern mit Behinderung, Anteil an Kindern mit Migrationshintergrund & & & C2,~\ref{C2_60}\\
\cmidrule{2 - 5}
& im Aufnahmegespräch wird mitgeteilt, wie wir hier zusammenleben und wie wichtig uns die Gemeinschaft ist & & Vertrauen → die kommen mit allem, Eltern wissen das auch, wenn etwas ist, dann darf man kommen & C1,~\ref{C1_59}\\ 

\dimension{Beteiligungsmöglichkeiten}
& Zukunftswerkstatt, Elternveranstaltungen, Elternbeirat, Förderverein   & & & C1,~\ref{C1_57}\\
\cmidrule{2 - 5}
& werden mit allem einbezogen, angesprochen und eingeladen über den Elternbrief
bedingt durch die Eltern-Kind-Gruppen können die eigentlich ständig hier irgendwie etwas mitmachen & & & C1,~\ref{C1_58}\\
\cmidrule{2 - 5}
& Fachbereich nur Elternbildung: Eltern-Kind-Gruppen in der Naturpädagogik, Familienausflüge am Wochenende & Bildungsangebot für Kinder und Eltern, enge Interaktion in die Familie $\rightarrow$ Siehe [Übergeordnete Werte und Ziele] & & C1,~\ref{C1_16}\\
\cmidrule{2 - 5}
& Elterncafé, Elternnaturtage, Familienausflüge am Wochenende & & Begegnung und Beziehungsaufbau von der Familie zum Haus & C1,~\ref{C1_60}\\
\cmidrule{2 - 5}
& Angebote, an denen auch Menschen teilnehmen können, die ihre Kinder nicht in der Einrichtung haben, Eltern kommunizieren das untereinander & Siehe [Übergeordnete Werte und Ziele] & Aktive Arbeit im Gemeinwesen & C1,~\ref{C1_63}\\ 
\cmidrule{2 - 5}
& Elterncafé, Waldausflüge mit Kindern und Eltern, Feste feiern, Flohmarkt, Stand auf dem Weihnachtsmarkt & & Be"=zieh"=ungs"=auf"=bau in die Familie & C1,~\ref{C1_58}\\
\cmidrule{2 - 5}
& Flohmarkt ist auf Elternwunsch entstanden → alle profitieren: Eltern können ihre Sachen verkaufen, zahlen drei Euro pro Meter und Kuchen, der verkauft wird, Einnahmequelle für Kindergarten & & Ort für Begegnung, Öffentlichkeitsarbeit & C2,~\ref{C2_59}\\
\cmidrule{2 - 5}
& Dialog auch an Elternabenden, Elterncafé & & & C2,~\ref{C2_61}\\
\cmidrule{2 - 5}
& Projekt gemeinsam mit der evangelischen Hochschule: \emph{’Gesund aufwachsen in der Kita’} Eltern in der Auswahl des Projektes von Anfang an einbezogen, gemeinsam entschieden, was wir  in diesem Projekt genau machen → Waldausflüge installiert an einem Freitagnachmittag, Treffen so spät gelegt, dass es für die Kinder gerade noch machbar war (nicht zu müde) & & zum ersten Mal sind alle 60 Eltern der Einrichtung erschienen, teilweise auch Mutter und Vater → Erfolgserlebnis verhilft dazu zu realisieren: „Da ist schon was dran, dass man sich so Gedanken machen muss, wie kann man doch noch mehr Dinge mit den Eltern gemeinsam...“\vspace{0.5em} & C2,~\ref{C2_65}\\

\dimension{Berücksichtigung der Elternwünsche}
& Elternbeirat gibt Wünsche weiter (manchmal trauen sich Eltern nicht, den direkten Weg zu gehen) & & & C3,~\ref{C3_74}\\
\cmidrule{2 - 5}
& gemeinsames Frühstück mit Kindern und ihren Eltern zwei Mal im Jahr & & hohe Resonanz bei den Eltern: oft kommen 100 Eltern; sie genießen das sehr: „Jetzt mach mal jemand etwas für uns und wir dürfen uns hinsetzen.“ & C3,~\ref{C3_9}\\
\cmidrule{2 - 5}
& gemeinsame Aktionen wie Feste feiern, Kinder und Eltern basteln -gemeinsam Laternen oder Schultüten, Spieltage an denen die Eltern dazukommen können, um zu sehen, welche Spiele es gibt oder Wandertage, an denen die Eltern die Gruppe begleiten können, um auch zu sehen, in den Wald kann ich mit meinem Kind jederzeit gehen, das kostet nichts. Diese Aktionen sind alle in denen normalen Gruppenalltag integriert und dienen dazu, die Kommunikation zwischen den Eltern und dem Kind zu fördern. Manche Eltern sagen dann beim gemeinsamen Laternen basteln: „Das habe ich noch nie gemacht. Ich wusste gar nicht, dass ich das kann.“& & & C3,~\ref{C3_75}\\
\end{longtable}
\end{centering}

% ===============    Tabelle 5    ============== %
\begin{centering}
\begin{longtable}{p{2cm}p{8.5cm}p{4cm}p{4cm}p{1.5cm}}
\caption{Extraktion zur Auswertungskategorie Kooperation}\\
\tabhead
\dimension{mit Fachdiensten und anderen Institutionen} 
& je nachdem was für Kinder gerade bei uns sind, werden Kontakte hergestellt& & & C2,~\ref{C2_25}\\
%\cmidrule{2 - 5}
&Vernetzung im Stadtteil & & & C2,~\ref{C2_26}\\
\cmidrule{2 - 5}
& in Gefährdungssituationen laden wir die entsprechenden Beratungsstellen oder die Gruppen wie \emph{Wildwasser} oder wenn es um Missbrauch geht  \emph{Wendepunkt} zur Fallsupervision ein & & & C1,~\ref{C1_10}\\
\cmidrule{2 - 5}
& Kooperation mit der Beratungsstelle der Sehbehindertenschule & Kind mit Sehbehinderung & & C2,~\ref{C2_24}\\ 
\cmidrule{2 - 5}
& bei Suchtthemen-- kommunalen Suchtbeauftragten, die Leute holen wir uns dann ran & & & C1,~\ref{C1_13}\\
\cmidrule{2 - 5}
& breites Bündnis: \emph{pro familia}, Schulen, Logopädische Praxis, mit der Heilpädagogik, dann die Ärzten hier im Quartier, die Polizei im Stadtteil, Hochschulen, \emph{Südwind} & & Siehe [Antworten auf Bedarfslagen] & C1,~\ref{C1_14}\\
\cmidrule{2 - 5}
& enge Kooperation mit der AWO-Frühförderstelle, da unter einem Trägerdach & & gute Zusammenarbeit, einfache Zugangswege zu Erstkontakt, da Heilpädagogin vor Ort, sie kann bei beobachtetem Förderbedarf das Kind in der Gruppe anschauen → Eltern erleben, das Förderung ist nichts Besonderes, nichts Ausgrenzendes, da gehen viele Kinder hin & C2,~\ref{C2_10}\\
\cmidrule{2 - 5}
& Fachkraft der Sprachheilberatungsstelle kommt ein bis zwei Mal im Jahr, ihr werden in Absprache der Eltern Kinder zur Überprüfung vorgestellt: Braucht das Kind Logopädie? Reicht die Unterstützung im Kindergarten aus oder braucht es eine zusätzliche Unterstützung in anderen Entwicklungsbereichen? & „Die meisten unserer Eltern würden es nicht schaffen einen Termin einzuhalten.“ & Gegebenenfalls Empfehlung eines Sprachheilkindergarten oder einer -schule, wenn die Problematik weder durch Logopädie noch durch Unterstützung der Kinder im häuslichen Umfeld verbessert werde kann & C3,~\ref{C3_21}\\
\cmidrule{2 - 5}
& Kooperation mit Heilpädagogen aus Praxen, wenn Kinder, die in die Einrichtung kommen, schon einen heilpädagogische Förderung bekommen → Austausch, hospitieren in der Gruppe, um einen Einblick zu bekommen & & & C2,~\ref{C2_16}\\ 
\cmidrule{2 - 5}
& zu Ärzten, Beratungsstellen & & & C2,~\ref{C2_19}\\ 
\cmidrule{2 - 5}
& Kontakt zu der sprachheilpädagogischen Schule, mindestens einmal im Jahr im Haus für Beratung $\rightarrow$ Kinder werden in Absprache mit den Eltern zur Abklärung vorgestellt, ob logopädischen Maßnahme zu empfehlen ist. 
Kinder in deren Spielgruppen in der Schule vermittelt & & & C2,~\ref{C2_21}\\ 
\cmidrule{2 - 5}
& Fachschulen für Sozialpädagogik, wir haben immer wieder Praktikanten hier, dann Hochschulen, wo jetzt durch die Studiengänge Praktikanten kommen zu uns, Ärzte, Beratungsstellen & & erweitert die Arbeit & C2,~\ref{C2_22}\\ 
%\cmidrule{2 - 5}
& was Vernetzung anbelangt, da finde ich hat sich schon viel getan & & & C2,~\ref{C2_41}\\ 
\cmidrule{2 - 5}
& in Arbeitskreisen im Stadtteil oder in Stadtteil übergreifenden engagieren, es gibt verschiedene Arbeitsgruppen, an denen die Leitung teilnimmt und jeweils eine weitere Fachkraft → Bsp. Verantwortliche für das Elterncafé ist zusammen mit der Leitung im Arbeitskreis Migration & neue Themen erhalten, weiterbilden, Entlastung für Leitung durch geteilte Verantwortung, Transport ins Team vereinfacht & Spezialisierung & C2,~\ref{C2_43}\\ 
\cmidrule{2 - 5}
& AWO-Frühförderstelle und die dortigen Heilpädagogen, die als Integrationshilfe in die Gruppe kommen oder Kinder in der Beratungsstelle in Spieltherapien oder Ergotherapien betreuen & & & C3,~\ref{C3_22}\\
\cmidrule{2 - 5}
& Erziehungsberatung vor Ort & & & C3,~\ref{C3_23}\\ 
\cmidrule{2 - 5}
& Kontaktaufnahme zu den jeweiligen Therapeuten des Kindes, das sich vor Eintritt des Kindergartens bereist in Therapie, häufig Ergotherapie oder Spieltherapie, befindet & & & C3,~\ref{C3_25}\\
\cmidrule{2 - 5}
& Bildungs- und Beratungszentrum für Hörgeschädigte in {[Ortsname]} und der
Beratungsstelle in der Blindenschule in {[Ortsname]} & Weiterbildung
& & C3,~\ref{C3_55}\\ 
\cmidrule{2 - 5}
& Voraussetzung für Inklusion: Gute Vernetzung mit zusätzlichen Fachkräften, die außen stehenden helfenden Kräfte in Verbindung zu bringen wird als Glücksspiel erlebt & Offenheit der anderen Stellen, lange Wartezeiten für Eltern bis genaue Entwicklungsdiagnostik erstellt 
& & C3,~\ref{C3_42}\\

\dimension{Kooperation mit der Schule}
& Zusammenarbeit mit der Schule durch Projekt \emph{Bildungshaus} verstärkt, Kinder und Eltern werden mehr mit eingebunden & & nimmt elterliche Ängste und stärkt Kinder und Eltern & C1,~\ref{C1_68}\\ 
\cmidrule{2 - 5}
& enge Kooperation mit der Schule, die hier bei uns um die Ecke ist 
gemeinsames Projekt: \emph{Schulreifes Kind}, → Chancengleichheit, also dass Kinder keine Zurückstellung erhalten müssen, weil Kinder mit besonderem Förderbedarf in einer Gruppe extra gefördert werden.
Kontakt zur dortigen Musiklehrerin zwei bis drei Mal wöchentlich 
einmal pro Woche gehen die Kinder in die Schule, Musiklehrerin macht Musik im Musiksaal, das zweite und ggf. dritte Mal kommt sie in die Einrichtung (es gibt natürlich Wochen, da muss sie vertretungsmäßig einspringen, dann kommt sie am Nachmittag, da können wir froh sein, dass sie so engagiert ist, weil die Stunden hat sie eigentlich gar nicht zur Verfügung) & & Kinder haben die Schule bereits kennengelernt und die Hemmschwelle ist ihnen genommen & C2,~\ref{C2_18}\\ 
\cmidrule{2 - 5}
& gewachsene Strukturen – Trennung von Schule und Kindergarten aufgehoben, wie ein gemeinsames Team: gemeinsame Fortbildungen, Direktorin der Grundschule kommt zu unseren Weihnachtsfeiern und umgekehrt $\rightarrow$ „Eigentlich wäre es schön, wenn die Einrichtung daneben wäre, dass man wie so ein Kinderhaus hätte, wo die Übergänge noch besser wären.“\vspace{0.5em} & & & C2,~\ref{C2_600}\\ 

\dimension{Kooperation im Team}
& Im wöchentlichen Dialog im Team ist der Blick auf die Entwicklung der Kinder wie mit der Lupe -- Kompetenz-Zuschreibung & & & C1,~\ref {C1_8}\\
\cmidrule{2 - 5}
& durch die Projekte ist der interdisziplinäre Austausch installiert, wird in der Gesamtteamsitzung gebündelt und transparent für alle gemacht, die Kinder sind den Gruppen zugeordnet, die erwachsenen Verantwortlichen bringen ihre Beobachtungen und Erlebnisse, es wird immer über die Kinder gesprochen & & & C1,~\ref{C1_30}\\
\cmidrule{2 - 5}
& jede Woche: Kleingruppenteams, zwei Gruppen arbeiten enger zusammen in Zwillingsgruppenstrukturen, diese bilden ein eigenes Team, in dem Förderungen und Kinder besprochen werden sowie
Kleingruppenteam mit der Heilpädagogik, wo nochmal pointiert die Kinder, die Integrationshilfe erhalten, angeschaut werden & & & C1,~\ref{C1_31}\\
\cmidrule{2 - 5}
& jede Gruppe pro Woche eine Stunde Gruppenbesprechung, ein Teil dieser Zeit steht für  Fallbesprechungen zur Verfügung & & eine Stunde pro Woche oft unzureichend, vor allem wenn Kinder mit besonderen Bedürfnissen in der Gruppe, mehr Bedarf an Zeit auch für Anleitungsgespräche mit Praktikanten &  C2,~\ref{C2_700}\\
\cmidrule{2 - 5}
& Gesamtteambesprechung pro Woche anderthalb Stunden, beinhaltet kollegiale Beratung $\rightarrow$ Außenblick auf schwierige Situationen, neue Ideen und Impulse in Situationen, in denen schon viel ausprobiert wurde & & & C2,~\ref{C2_71}\\
\cmidrule{2 - 5}
& ein Freitag im Monat drei Stunden Teambesprechung oder gemeinsame Teamfortbildungen, Einrichtung schließt früher $\rightarrow$
Besprechung mit Fokus auf die Kinder & & & C2,~\ref{C2_72}\\
\cmidrule{2 - 5}
& Austausch im Team zur Unterstützung von Selbstreflexion, was kommt aus der eigenen Vergangenheit, Realität abgleichen mit dem Bild im Kopf & & & C2,~\ref {C2_508}\\
\end{longtable}
\end{centering}

% Ende Querformat und kleine Schrift
\end{small}
\end{landscape}