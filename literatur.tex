\chapter*{Literaturverzeichnis}
\addcontentsline{toc}{chapter}{Literaturverzeichnis}

Ahnert, L. (2010): Wieviel Mutter braucht ein Kind? Bindung – Bildung – Betreuung: öffentlich und privat. Heidelberg: Spektrum.

Albers, T. (2011): Mittendrin statt nur dabei. Inklusion in Krippe und Kindergarten. München: Reinhardt.

Becker-Stoll, F. (2009): Wie lernen Kinder in den ersten Lebensjahren? – Entwicklungspsychologische und bindungstheoretische Grundlagen. In: Becker-Stoll, F.~\& Nagel, B. (Hrsg.): Bildung und Erziehung in Deutschland. Pädagogik für Kinder von 0 bis 10 Jahren. Berlin: Cornelsen, S. 46-54.

Becker-Stoll, F.~\& Nagel, B. (Hrsg.) (2009): Bildung und Erziehung in Deutschland. Pädagogik für Kinder von 0 bis 10 Jahren. Berlin: Cornelsen. 

Behr, I. (2009): Aspekte inklusiver Qualität in Kindertageseinrichtungen aus Sicht 4- bis 6-jähriger Kinder mit und ohne besondere Bedürfnisse – eine Pilotstudie. Berlin: Dr. Köster.

Belmont, B., Pawlowska, A.~\& Vérillon, A. (2010):
Partnerschaft mit den Eltern. In: Kron, M., Papke, B.~\& Windisch, M. (Hrsg.): Zusammen aufwachsen. Schritte zur frühen inklusiven Bildung und Erziehung. Bad Heilbrunn: Klinkhardt, S. 68-75.

Bock-Famulla, K.~\& Lange, J. (2011): Länderreport frühkindlicher Bildungssysteme 2011. Transparenz schaffen – Governance stärken.
Gütersloh: Bertelsmann.

Booth, T., Ainscow, M.~\& Kingston D. (2006): Index für Inklusion (Tageseinrichtungen für Kinder). Lernen, Partizipation und Spiel in der inklusiven Kindertageseinrichtungen entwickeln. 4. Auflage, Frankfurt am Main: GEW. 

Erhardt, K.~\& Grüber, K. (2011): Teilhabe von Menschen mit geistiger Behinderung am Leben in der Kommune. Freiburg: Lambertus.

Esch, K., Klaudy, E., Micheel, B.~\& Stöbe-Blossey, S. (2006): Qualitätskonzepte in der Kindertagesbetreuung. Ein Überblick. Wiesbaden: Verlag für Sozialwissenschaften.

Fritzsche, R., Schastock, A.~\& Schöler, J. (Hrsg.) (2005): Ein Kindergarten für alle. Kinder mit und ohne Behinderung spielen und lernen gemeinsam. 2. Auflage, Weinheim: Beltz. 

Garai, D., Kerekes, V., Schiffer, C., Tamás,
K., Trócsányi, Z., Weiszburg, J.~\& Zászkaliczky, P. (2010):
Die Rolle der Fachkräfte in der inklusiven Bildung und
Erziehung. In: Kron, M., Papke, B.~\& Windisch, M. (Hrsg.): Zusammen aufwachsen. Schritte zur frühen inklusiven Bildung und Erziehung. Bad Heilbrunn: Klinkhardt, S. 46-53.

Gastinger, S.~\& Winkler, J. (2009): Gesetzestexte für Soziale Arbeit. Studienausgabe I: Kinder-, Jugend- und Familienhilfe. Freiburg im Breisgau: Lambertus.

Gastinger, S.~\& Winkler, J. (2009a): Gesetzestexte für Soziale Arbeit. Studienausgabe II: Soziale Sicherung. Freiburg im Breisgau: Lambertus.

Göransson, K. (2010): Unterschiedliche Perspektiven – unterschiedliches Verständnis von Inklusion. In: Kron, M., Papke, B.~\& Windisch, M. (Hrsg.): Zusammen aufwachsen. Schritte zur frühen inklusiven Bildung und Erziehung. Bad Heilbrunn: Klinkhardt, S. 17-22. 

Haefke, S. ~\& Mattke, U. (2011): Ein Kindergarten für Alle. Anwendung und Evaluation des Index für Inklusion in einer Regelkindertagesstätte. In: Teilhabe, Heft 3, S. 134-138.

Hammes-Di Bernando, E.~\& Schreiner, S. (Hrsg.) (2011): Diversität. Ressource und Herausforderung für die Pädagogik der frühen Kindheit. Weimar: das netz. 

Hansen, R. Knauer, R.~\& Sturzenhecker, B. (2011): Partizipation in Kindertageseinrichtungen. So gelingt Demokratiebildung mit Kindern! Weimar: das netz.

Haug, P. (2011): Inklusion als Herausforderung der Politik im internationalen Kontext. In: Kreuzer, M.~\& Ytterhus, B. (Hrsg.): „Dabeisein ist nicht alles“ -- Inklusion und Zusammenleben im Kindergarten. 2. Auflage, München: Ernst Reinhard, S. 36-52.   

Heimlich, U.~\& Behr, I. (2005): Integrative Qualität im Dialog entwickeln – auf dem Weg zur inklusiven Kindertageseinrichtun. Münster: Lit. 

%Heinze (2011, 10)

Hess, S. (2011): Befähigung zur Zusammenarbeit mit Eltern -- Professionalisierung von Pädagoginnen zur Unterstützung von Familien mit behinderten Kindern und Familien in sozialer Benachteiligung. In: Zeitschrift für Heilpädagogik, Nr. 9, Bad Heilbrunn: Klinkhardt, S. 346-354.  

Honig, M.-S., Joos, M.~\& Schreiber, N.: (2004): Was ist ein guter Kindergarten? Theoretische und empirische Analysen zum Qualitätsbegriff in der Pädagogik. Weinheim: Juventa.

%Hugoth, M. (2009): Skript der Vorlesung im Sommersemester 2009. 

Hugoth, M. (2010): Organisation von Kindertageseinrichtungen. Studienbrief der Hamburger Fern-Hochschule.

Hugoth, M. (2011): Inklusion – mal mit der „christlichen Brille“ betrachtet. In: Mitglieder-Info, Heft 1, S. 5-16.

Hugoth, M.~\& Fritz, A. (2010): Herausforderungen und Perspektiven. Studienbrief der Hamburger Fern-Hochschule.

Jerg, J. (2011): „Ein Kindergarten für alle“ und „Inklusion von Anfang an“. Entgrenzung als Herausforderung für eine inklusive Gestaltung von Kindertagesstätten. In Schulz, C.~\& Stammer, M. (Hrsg.): Von der Kinder- und Jugendhilfe zur Frühkindlichen Bildung. Mulitperspektivische Zugänge zu einer aktuellen Herausforderung. Stuttgart: Evangelische Gesellschaft, S. 46-62. 

Jeske, K. (1997): Mit den Eltern und nicht für die Eltern. Zusammenarbeit von Eltern und Erzieherinnen in Kindertageseinrichtungen. Grafschaft: Vektor.

Karlsson, M. (2010): Die Qualifikation der pädagogischen Fachkräfte -- ein entscheidender Aspekt der Qualität von Kindertageseinrichtungen. In: Kron, M., Papke, B.~\& Windisch, M. (Hrsg.): Zusammen aufwachsen. Schritte zur frühen inklusiven Bildung und Erziehung. Bad Heilbrunn: Klinkhardt, S. 76-79. 

Klein, F. (2010): Inklusive Erziehuns- und Bildungsarbeit in der Kita. Heilpädagogische Grundlagen und Praxishilfen. Troisdorf: Bildungsverlag EINS.

%Kobelt Neuhaus, D. (2008): Heterogenität als Motor für Bildungsprozesse -- Kinder mit Behinderung beteiligen und mitnehmen. In: Wagner, P.: Handbuch Kinderwelten. Freiburg: Herder.

Köhler, H. (2009): Normal ist die Verschiedenheit. Maximen einer gelingenden Integration. In: erziehungskunst, 11/2009, S. 5-10.

Köpcke-Duttler, A. (2011): Mehr als nur ein Trend. Ethische und rechtliche Begründung der Inklusion. In: Mitglieder-Info, Heft 1, S. 17-23.

Krenz (2011, 5)

Kreuzer, M. (2011): Zur Beteiligung von Kindern im Gruppenalltag von Kindergärten – Ein Überblick zu Ergebnissen deutscher Integrationsprojekte. In: Kreuzer, M.~\& Ytterhus, B. (Hrsg.): „Dabeisein ist nicht alles“ – Inklusion und Zusammenleben im Kindergarten. 2. Auflage, München: Ernst Reinhardt, S. 22-33.

Kron, M. (2002): Gemeinsame Erziehung von Kindern mit und ohne Behinderung im Elementarbereich. In: Eberwein, H.~\& Knauer, S. (Hrsg.): Integrationspädagogik. Kinder mit und ohne Beeinträchtigung lernen gemeinsam; 6.Auflage, Weinheim: Beltz, S. 178-190. 

Kron, M. (2006): 25 Jahre Integration im Elementarbereich – ein Blick zurück, ein Blick nach vorn. In: online-Zeitschrift für Inklusion, Nr.~1. Einsehbar unter: http://www.inklusion-online.net/index.php/inklusion/article/view/16/16, Stand: 10.05.2012.

Kron, M. Papke, B.~\& Windisch, M. (Hrsg.) (2010): Zusammen aufwachsen. Schritte zur frühen inklusiven Bildung und Erziehung. Bad Heilbrunn: Klinkhardt.

Kunkel, P.-C. (2006): Sozialgesetzbuch VIII. Kinder- und Jugendhilfe. Lehr- und Praxiskommentar. 3. Auflage. Baden-Baden: Nomos. 

Kunkel, P.-C. (2010): Jugendhilferecht. Systematische Darstellung für Studium und Praxis. 6. Auflage. Baden-Baden: Nomos. 

Largo, R. H. (2010): Lernen geht anders. Bildung und Erziehung vom Kind her denken. Hamburg: Köber-Stiftung. 

Largo, R. H. (2011): Kinderjahre. Die Individualität des Kindes als erzieherische Herausforderung. 22. Auflage, München: Piper.

Liegle, L. (2006): Soll der Kindergarten die Kinder auf das Lernen in der Schule vorbereiten? In: Friedrich-Ebert-Stiftung (Hrsg.): Droht der Kindergarten zu verschulen? Dokumentation zur Tagung in Stuttgart, S. 5-13.

Meuser, M.~\& Nagel, U. (2005): ExpertInneninterviews – vielfach erprobt, wenig bedacht. Ein Beitrag zur qualitativen Methodendiskussion. In: Bogner, A., Littig, B.~\& Menz, W: Das Experteninterview. Theorie, Methode, Anwendung. 2. Auflage, Wiesbaden: Verlag für Sozialwissenschaften.

Ministerium für Kultus, Jugend und Sport Baden-Württemberg (2011): Orientierungsplan für Bildung und Erziehung für die baden-württembergischen Kindergärten. Weinheim: Beltz. Einsehbar unter: http://www.kultusportal-bw.de, Stand: 10.03.2012.

Nagel, B. (2009): Kindorientierte Bildung: Entwicklung des Systems der Tageseinrichtungen für Kinder in Deutschland. In: Becker-Stoll, F.~\& Nagel, B. (Hrsg.): Bildung und Erziehung in Deutschland. Pädagogik für Kinder von 0 bis 10 Jahren. Berlin: Cornelsen, S. 12-26.

Nagel, B. (2009a): Bildungs- und Erziehungspläne – Perspektiven für die Weiterentwicklung der Kindertagesbetreuung. In: Becker-Stoll, F.~\& Nagel, B. (Hrsg.): Bildung und Erziehung in Deutschland. Pädagogik für Kinder von 0 bis 10 Jahren. Berlin: Cornelsen, S. 194-207.

Ostner, I. (2008): Ökonomisierung der Lebenswelt durch aktivierende Familienpolitik? In: Evers, A.~\& Heinze, R. (Hrsg.): Sozialpolitik. Ökonomisierung und Entgrenzung. Wiesbaden: VS-Verlag für Sozialwissenschaften, S. 49-66.

Papke, B. (2010): Bildung und Bildungspläne in der Elementarpädagogik – Chancen für Inklusion, In: online-Zeitschrift für Inklusion, Nr. 3. Einsehbar unter: http://www.inklusion-online.net/index.php/inklusion/article/view/66/71, Stand: 05.11.2012. 

Prenzel, M., Carstensen, C., Frey, A., Drechsel, B.~\& Rönnebeck, S. (2007): PISA 2006 – Eine Einführung in die Studie. In: PISA Konsortium Deutschland (Hrsg.): PISA '06 – die Ergebnisse der dritten internationalen Vergleichsstudie. Münster: Waxmann, S. 31-56.

%Roth, H.-J.~\& Terhart, H. (2009): Migrationshintergrund -- (k)ein frühes Risiko? In: Heinen, N.~\& Kissgen, R.: Frühe Risiken und Frühe Hilfen
%Hans Weiß, H. : Kinderarmut als Entwicklungsrisiko. In: siehe oben

Sander, A. (2004): Konzepte einer inklusiven Pädagogik. In: Verband Sonderpädagogik (Hrsg.): Zeitschrift für Heilpädagogik, Nr. 5, Bad Heilbrunn: Klinkhardt, S. 240-244.

%Sarimski, K. (2011): Soziale Kontakte behinderter Kinder in integartiven Gruppen -- eine explorative Studie im Elementarbereich. In: Zeitschrift für Heilpädagogik, Nr. 1, Bad Heilbrunn: Klinkhardt, S. 4-?.

Sarimski, K. (2012): Behinderte Kinder in inklusiven Kindertagesstätten. Stuttgart: Kohlhammer. 

Sarimski, K.~\& Schaumburg, M. (2010): Soziale Partizipation in der Freizeit von 3 bis 6-jährigen Kindern mit und ohne Behinderung -- eine vergleichende Elternbefragung. In: Zeitschrift für Heilpädagogik, Nr. 4, Bad Heilbrunn: Klinkhardt, S. 124-129.

%Schädler, J.~\& Dorrance, C. (2011): Barometer of Inclusive Education – Konzept, methodisches Vorgehen und Zusammenfassung der Forschungsergebnisse ausgewählter europäischer Länder. In: Zeitschrift für Inklusion, Nr. 4.

%Schlecht, D., Förster, C., Wellner, B.~\& Mörth, A. (2008): KITA – Wie gut sind wir? Skalen zur Einschätzung der pädagogischen Qualität nach internationalen Standards unter Einbeziehung aller Bildungspläne in Deutschland. Berlin: Cornelsen. 

Schulz, C.~\& Stammer, H. (Hrsg.) (2011): Von der Kinder- und Jugendhilfe zur Frühkindlichen Bildung. Mulitperspektivische Zugänge zu einer aktuellen Herausforderung. Stuttgart: Evangelische Gesellschaft.

Schuster, K., Viernickel, S.~\& Weltzien, D. (2006): Bildungsmanagement: Methoden und Instrumente der Umsetzung pädagogischer Konzepte. Studienbuch 11 zum Bildungs- und Sozialmanagement. Remagen: Ibus.

Serrano, A. M.~\& Afonso, J. L. (2010): Individualisierte Unterstützungsstrategien in der inklusiven Bildung und Erziehung. In: Kron, M. Papke, B.~\& Windisch, M. (Hrsg.): Zusammen aufwachsen. Schritte zur frühen inklusiven Bildung und Erziehung. Bad Heilbrunn: Klinkhardt, S. 62-67.

Spaermann (2011): Die Menschwürde kennt keine Kompromisse. Als Gastreferent beim Psychosomatischen Dienstagskolloquium "Seele–Körper–Geist" des Universitätsklinikums Freiburg am 7. November 2011.

Speck-Hamdan, A. (2011):Diversität - Herausforderungen und Chancen für die Pädagogik der frühen Kindheit. Ein Überblick. In: Hammes-Di Bernando, E.~\& Schreiner, S. (Hrsg.): Diversität. Ressource und Herausforderung für die Pädagogik der frühen Kindheit. Weimar: das netz, S. 14-23.

Tietze, W. (2004): Notwendigkeit und Perspektiven von Qualitätsentwicklung und Qualitätssicherung in Kindertageseinrichtungen. In: Wehrmann, I. (Hrsg.): Kindergärten und ihre Zukunft. Weinheim: Beltz, S. 406-419.

Tietze, W., Dittrich, I., Grenner, K. Groot-Wilken, B., Sommerfeld, V.~\& Viernickel, S. (2004): Pädagogische Qualität entwickeln. Praktische Anleitung und Methodenbausteine für Bildung, Betreuung und Erziehung in Tageseinrichtungen für Kinder von 0-6 Jahren. Weinheim: Beltz.

Tietze, W., Becker-Stoll, F., Bensel, J., Eckhardt, A. G., Haug-Schnabel, G., Kalicki, B., Keller, H.~\& Leyendecker, B. (Hrsg.) (in Vorbereitung): NUBBEK -- Nationale Untersuchung zur Bildung, Betreuung und Erziehung in der frühen Kindheit.
Fragestellungen und Ergebnisse im Überblick. Forschungsbericht. Weimar: das netz. Vorläufige Broschüre (2012) online abrufbar unter: http://www.nubbek.de, Stand:  9.11.2012.

Viernickel, S.~\& Schwarz, S. (2009): Wie viele Kinder pro Erzieherin? Die Expertise „Schlüssel zu guter Bildung, Erziehung und Betreuung“. Das Leitungsheft Kindergarten heute, Nr. 3, S. 14-16.

Volkert, W. (2008): Die Kindertagesstätte als Bildungseinrichtung. Neue Konzepte zur Professionalisierung in der Pädagogik der frühen Kindheit. Wiesbaden: Verlag für Sozialwissenschaften.

%Weber, E. (1999): Pädagogik. Eine Einführung. Band I: Grundfragen und Grundbegriffe. Teil 3: Pädagogische Grundvorgänge und Zielvorstellungen -- Erziehung und Gesellschaft~/ Politik. 8. Auflage. Donauwörth: ?

%Wocken, H. (2011): Über die Entkernung der Behindertenrechtskonvention. Ein deutsches Trauerspiel in 14 Akten. In: Online-Zeitschrift für Inklusion, Nr. 4. Einsehbar unter: http://www.inklusion-online.net. Stand: 25.09.2012. 

%Weiß, H. (2010): Kinder in Armut als Herausforderung für eine inklusive Perspektive. In: Online-Zeitschrift für Inklusion, Nr. 4. Einsehbar unter: http://www.inklusion-online.net. Stand: 05.11.2012. 
