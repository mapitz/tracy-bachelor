% Funktioniert leider nicht
%\RequirePackage[ngerman=ngerman-x-20080601]{hyphsubst}
%\RequirePackage{fix-cm} % macht, dass die Überschriften bei Verwendung von cm-super und fontenc T1 (EC-Fonts) fett sind
%
% Headerdatei der Bachelorarbeit
%
%

\documentclass[
a4paper,
12pt,
%twoside,
%openright,
parskip,
%draft,
ngerman,
chapterprefix,
]{scrreprt}

%Direkt Eingabe von ü,ä,ö,ß ermöglichen
%Sollte immer zuerst geladen werden.
\usepackage[utf8]{inputenc}

%Kapitel anschreiben als Kapitel
\usepackage{moreverb}

%Textgestaltung
\usepackage[T1]{fontenc} 	% Für korrekte Trennung von Wörtern mit Umlaut
\usepackage[ngerman]{babel} 	% Deutsche Trennregeln
\usepackage{lmodern}	 		% Verbesserter Standardfont als alternative zu cm-super
\usepackage{textcomp}		% zum Darstellen des µs im Text
\usepackage{verbatim}		% Darstellung wie Eingabe
\usepackage[babel,german=quotes]{csquotes} 	% Deutsche Anführungszeichen verwenden
\usepackage{color}

%Textsatz
\usepackage{microtype}  % verbesserter optischer Randausgleich
\setlength{\parindent}{0pt} % Einzug bei Absätzen
\setlength{\parskip}{5pt}	% vertikaler Abstand zwischen Absätzen

%Schusterjungen und Hurenkinder
\clubpenalty = 10000		% Diese drei Einstellungen verhindern diese gänzlich!
\widowpenalty = 10000		% Sind mit vorsicht zu genießen!
\displaywidowpenalty = 10000

%Zum Einbinden von Grafiken
\usepackage{graphicx}		%immer benötigtes Paket
\usepackage{subfigure}		%mehrere Bilder in einer Floatumgebung
%\usepackage{floatflt}		%gleitende Bilder mit Textumlauf am Seitenrand

%Chemie / Formeln 
%\usepackage{chemarrow}		%Beschriftung von Reachts-Links-Pfeil oben und unten
%\usepackage{amssymb,amsmath}%Zum Splitten von Formeln
%\usepackage{chemfig} 		%Darstellen von Ladungen über dem Atom

%Für Tabellen
\usepackage{array}
\usepackage{booktabs} %schönere Tabellen (toprule, midrule, ...)
\usepackage{multirow} %zeilen verbinden
\usepackage{supertabular}
\usepackage{longtable}% Seitenübergreifende tabellen

% Für Seiten im Querformati
\usepackage{pdflscape}

% Für Zeilennummerierung
% Der Modus ist running, das heißt, dass die Zeilennummern unabhängig von der Seitenzahl immer weiter gezählt werden.
\usepackage[running]{lineno}

%Für Hyperlinks (\url und \href)
%Sollte als letztes geladen werden
\usepackage{hyperref}

% Einbinden von PDFs
\usepackage{pdfpages}

% Neue Kommandos
% ==============

% Zitatumgebung
\newenvironment{zitat}{\small\begin{quote}}{\end{quote}}

% Tabellenkopf für Extraktionstabellen
\newcommand{\tabhead}{\toprule
\textbf{Dimension} & \textbf{Sachverhalt\newline(Aussagen konzentriert)} & \textbf{Ursache} & \textbf{Wirkung} & \textbf{Quelle}\\
\midrule
\endfirsthead
\toprule
\textbf{Dimension} & \textbf{Sachverhalt\newline(Aussagen konzentriert)} & \textbf{Ursache} & \textbf{Wirkung} & \textbf{Quelle}\\
\midrule
\endhead
\noalign{\vskip-2.5pt}
\bottomrule
\endfoot}
% Formatierung der Dimension
\newcommand{\dimension}[1]{\multicolumn{5}{l}{\emph{#1}}\\ \midrule}