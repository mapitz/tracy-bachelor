\chapter{Gesprächsleitfaden der Experteninterviews}\label{Interviewleitfaden}
\paragraph{Strukturelle Rahmenbedingungen und das Inklusionskonzept}
\begin{enumerate}
\item Können Sie etwas zur Organisation und zur Struktur Ihrer Einrichtung sagen?
\begin{itemize}
\item Welche Kinder besuchen Ihre Einrichtung? Welcher sozialen Herkunft sind diese Kinder? 
\item Werden Kinder mit besonderen Bedürfnissen betreut?
\item Personalschlüssel? Gruppengröße?
\item Wie ist das Team zusammengesetzt?
\item Spielen Heilpädagogen oder andere Fachleute mit Spezialwissen eine aktive Rolle im Gruppenalltag?
\item Gibt es Ansprechpartner, die bei Fragen einbezogen werden, wie die Entwicklung eines Kindes möglichst optimal zu unterstützen ist? 
\item Mit welchen Kooperationspartnern ist die Einrichtung vernetzt?
\end{itemize}
\item Gibt es für die Kinder mit besonderen Bedürfnissen konzeptionelle Überlegungen? Entsprechen diese Ihrem Inklusionsverständnis?
Was verstehen Sie unter Inklusion in Bezug auf den Kindergarten? 
\item Können Sie noch weitere Angaben über den Charakter des Konzepts hinsichtlich Inklusion machen?
\begin{itemize}
\item Wie wird der Verschiedenheit der Kinder Rechnung getragen?
\item Wie ist individuelle Unterstützung in Ihrem Konzept vorgesehen?
\item Wie wird die individuelle Arbeit am Kind in das Gruppengeschehen integriert?
\item Was sind gegebenenfalls Strategien, um Therapien in einer inklusiven Form anzubieten?
\end{itemize}
\item Was sind Ihrem Ermessen nach notwendige Voraussetzungen für das Gelingen von Inklusion, die von „oben“ garantiert werden sollten? 
\item Gibt es bildungspolitische Entscheidungen, die der Umsetzung von Inklusion im Weg stehen? 
Wenn ja, was sollte sich ändern? 
Was kann der Träger dazu beitragen? 
Wo sehen Sie Ihre besonderen Aufgaben als Leitung?
\end{enumerate}

\paragraph{Schritte in Richtung Inklusion}
\begin{enumerate}
\item Welche Schritte wurden konkret unternommen, um Inklusion einzuleiten?
\item Was waren und sind Stolpersteine auf dem Weg zur inklusiven Bildungseinrichtung? 
\end{enumerate}

\paragraph{Herausforderungen an das Team}
\begin{enumerate}
\item Welche Herausforderungen werden an das Team gestellt, wenn das Konzept in Richtung Inklusion umgestellt wird?
\item Was sind häufige Fragen der Erzieherinnen, die anzeigen, dass Beratungs- oder Unterstützungsbedarf besteht?
\item Gab es oder gibt es Ängste unter den Mitarbeitern? Welcher Art? Wie wird damit umgegangen?
\item Welche Unterstützung erfährt das Team? 
Gibt es eine gemeinsame Planung und Zusammenarbeit unter den Beteiligten? Gibt es Teamsitzungen, in denen das Verhalten von Kindern diskutiert wird? 
\item Brauchen Erzieherinnen Zusatzqualifikationen, um inklusiv zu arbeiten? 
In welchen Bereichen sind Kenntnisse Ihrer Meinung nach hilfreich?
\item Welches Handwerkszeug müssen ErzieherInnen mitbringen bzw. was macht gute Erzieherinnen aus?
\end{enumerate}

\paragraph{Partnerschaft mit den Eltern}
\begin{enumerate}
\item Welche Bedeutung hat die Elternarbeit im Hinblick auf Inklusion?
Wie werden die Eltern bei Entscheidungen eingebunden? 
\item Was wird in Ihrer Institution organisiert, um eine Beziehung zu den Eltern aller Kinder aufzubauen?
\item Kennen Sie die Einstellungen der Familien zu Fragen inklusiver Erziehung und wie unterstützen Sie den Dialog darüber?
\end{enumerate}

\paragraph{Persönliche Haltung zu Veränderungen und Grenzen der Inklusion}
\begin{enumerate}
\item Haben Sie durch die Umsetzung von Inklusion Veränderungen bei sich feststellen können? Bei den pädagogischen Mitarbeitern? Bei den Kindern? 
\item Wie ist ihr Standpunkt zu der Aussage: „Allen Kindern gerecht werden“? Sehen Sie Grenzen der Inklusion?
\end{enumerate}